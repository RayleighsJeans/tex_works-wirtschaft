\documentclass{article}

\usepackage{amsmath, amsfonts}
\usepackage{graphicx} 
\usepackage{lmodern}
\usepackage{longtable}


\begin{document}
\begin{longtable}{|l|l|} %Beginn Tabelle
	\hline
	\textsc{Variable} & \textsc{Bedeutung}\\
	\hline
	\endfirsthead
	
	\hline
	\textsc{Variable} & \textsc{Bedeutung}\\
	\hline
	\endhead
	
	\hline
	\endfoot
	
	\hline
	\endlastfoot
	
		$A$ & Technologieparemeter im Konsumgutsektor\\%S.18
		$B$ & Technologieparemeter im Bildungssektor\\%S.16
		$\bar{B}$ & Diffusionparameter durch Offenheit\\%S.31
		$c(t)$ & Konsumg�termenge\\%S.17
		$c(t)_{ex}$ & Exportg�termenge\\%S.31
		$c(t)_{im}$ & Importg�termenge\\%S.31
		$F[K,L]$ & Produktionsfunktion des Konsumgutsektors\\%S.16
		$G[K,L]$ & Produktionsfunktion des Bildungssektors\\%S.16
		$\mathbb{H}$ & Hamiltionfunktion\\%S.19
		$h(t)$ & durchschnittlicher Humankapitalbestand\\%S.16
		$h_{0}$ & Startwert des durchschnittlichen Humankapitalbestands\\%S.18
		%$\dot{h}(t)$ & Bewegungsgesetz der Humankapitalakkumulation\\%S.18
		$K(t)$ & physisches Kapital/Sachkapital\\ %S.15
		$k(t)$ & durchschnittlicher physischer Kapitalbestand\\%S.16
		$k_{0}$ & Startwert des durchschnittlichen physischen Kapitalbestands\\%S.18
		%$\dot{k}(t)$ & Bewegungsgesetz des physischen Kapitalakkumulation\\%S.18
		$L(t)$ & Arbeit\\%S.15
		$M$ & Grenzprodukt des Kapitals einer geschlossenen Volkswirtschaft\\
	  $\bar{M}$ & Grenzprodukt des Kapitals einer offenen Volkswirtschaft\\%S.43
		$N(t)$ & Bev�lkerungsgr�{\ss}e\\%S.15
		$N_0$ & Startwert der Bev�lkerungsgr�{\ss}e\\
		$n$ & Bev�lkerungswachstumsrate\\%S.17
		$p*$ & Preis des Importgutes\\%S.33
		$t$ & Zeit\\%S.17
		$u(t)$ & Anteil des Humankapitals im Konsumgutsektor\\%S.15
		$(1-u(t))$ & Anteil des Humankapitals im Bildungssektor\\%S.15
		$V$ & Lebenszeitnutzen\\%S.17
		$v(t)$ & Anteil des physischen Kapitals im Konsumgutsektor\\%S.15
		$(1-v(t))$ & Anteil des physischen Kapitals im Bildungssektor\\%S.15
		$\alpha$ & Produktionselastizit�t im Konsumgutsektor des physischen Kapitals\\%S.16
		$\gamma_{1}$ & Schattenpreis des physischen Kapitals\\%S.19
	  %$\gamma_{1im}$ & Schattenpreis des pysischen Kapitals Importg�termenge???\\%S.33
		$\gamma_{2}$ & Schattenpreis des Humankapitals\\%S.19
		$\dot{\gamma}$ & Abschreibungsrate des Schattenpreises\\%S.21
		$\eta$ & Produktionselastizit�t im Bildungssektor des physischen Kapitals\\%S.16
		$\rho$ & Diskontrate\\%S.17
		$\sigma$ & intertemporale Substitutionselastizit�t\\%S.17
		$\phi$ & Absorbtionsrate\\%S.31
		$\chi$ & Kapital-Konsumquote\\%S.26
		$\chi_{ex}$ & Kapital-Exportg�terquote\\%S.39
		$\chi_{im}$ & Kapital-Importg�terquote\\%S.39
		$\psi$ & Nutzen \\%S.17

\hline
\end{longtable}
	


	%\end{longtable}
\end{document}
