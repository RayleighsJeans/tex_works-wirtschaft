%\documentclass[12pt]{article}
%\usepackage[latin1]{inputenc}% erm\"oglich die direkte Eingabe der Umlaute 
%\usepackage[T1]{fontenc} % das Trennen der Umlaute
%\usepackage{ngerman}
%\usepackage{a4wide}
%\usepackage{color}
%\usepackage{amsmath} % braucht man um die gleichungen zu labeln und zu zitieren
%\usepackage{lmodern,amssymb,xcolor}
\chapter{Auswertung}\label{Auswertung}

In Bezug auf die in Kapitel \ref{Einleitung} aufgestellten Thesen zeigt die vorangegangene Analyse, dass Au{\ss}enhandel die Bedeutung des Humankapitals f{\"u}r den Wachstumsprozess betont. In einem weiteren Schritt wurde gezeigt, dass nicht nur Humankapital per se wichtig ist, sondern auch seine Zusammensetzung,  bezogen auf heterogene F{\"a}higkeit, der imitierenden und innovierenden Art. Au{\ss}enhandel stimuliert die Innovationsf{\"a}higkeit von Volkswirtschaften und best{\"a}tigt zudem, dass abh{\"a}ngig vom Entwicklungsstand auch die Imitationsstrategie den Entwicklungsprozess voranbringt. Es wurde gezeigt, dass Freihandel durch eine entsprechende Strategiewahl die Entwicklung aller L{\"a}nder beg{\"u}nstigt, unabh{\"a}ngig vom Entwicklungsstand und der Beschaffenheit der Welttechnologiegr{\"o}{\ss}e. Technologisch kleine sowie technologisch gro{\ss}e L{\"a}nder profitieren vom Wissenstransfer und ebnen damit die M{\"o}glichkeit f{\"u}r dauerhaftes Wachstum.\\
Es konnte gezeigt werden, dass weniger weit entwickelte L{\"a}nder vom Handel mit relativ humankapitalreich produzierten G{\"u}tern profitieren und sich dadurch ihr eigenes Bildungswesen verbessert. Diese Anhebung des Bildungsniveaus eines Landes f{\"u}hrt zu einem h{\"o}heren Wachstumspfad, durch die Weiterentwicklung des technologischen Entwicklungsstands. So bewirkt Au{\ss}enhandel mit humankapitalreichen G{\"u}tern, dass relativ weiter entwickelte L{\"a}nder die Innovationsstrategie verfolgen und relativ weniger weit entwickelte die Imitationsstrategie.\footnote{Je nach N{\"a}he zur WTG k{\"o}nnte jedoch ein Wechsel angeraten werden.} Unabh{\"a}ngig von der technologischen Gr{\"o}{\ss}e des Landes resultiert das gleiche Ergebnis. Demzufolge wird auch bei technologisch gro{\ss}en L{\"a}ndern, die eine endogene WTG bedingen, mit abnehmender Distanz zur WTG die Innovationsstrategie pr{\"a}feriert. Der Abstand zur WTG wird sich zwar ausweiten, jedoch f{\"u}hrt Au{\ss}enhandel zu einem absolut gesehen zuk{\"u}nftig h{\"o}heren Entwicklungsstand.\footnote{In der folgenden Auswertung wird nicht mehr zwischen den Ergebnissen bei einer endogenen und exogenen WTG unterschieden, da dies keinen wesentlichen Einfluss auf die strategische Entscheidung eines Landes hat.}\newline
Differenziert man zus{\"a}tzlich zwischen den Sektoren, was aus der exportunterst{\"u}tzenden Politik folgte\footnote{Diese wurde in Kapitel \ref{Papier1} modelliert.}, dann zeigt dies ebenfalls, dass bei einem sehr hohen Abstand zur WTG durchaus die Imitationsstrategie im Importsektor zu pr{\"a}ferieren ist. Dies sollte einer Volkswirtschaft dazu verhelfen eine Basis an technologischem Wissen und Humankapital aufzubauen, indem zun{\"a}chst Wissen importiert wird, das nun nachgeahmt werden kann, bevor ein Wechsel zur Innovationsstrategie sinnvoll ist. Dieser Zusammenhang wurde bereits von \citet{Glass.1999} gezeigt und kann hier auf andere Art und Weise best{\"a}tigt werden. Der Schwerpunkt des Importsektors liegt sowohl bei den weniger weit entwickelten, als auch bei den relativ weiter entwickelten Volkswirtschaften auf der Imitationsstrategie. Der Exportsektor hingegen stellt sich besser bzw. es ist profitabler der Innovationsstrategie zu folgen.\\ 
Weiterhin konnte gezeigt werden, dass der Schwerpunkt eines weniger weit entwickelten Landes im Importsektor auf der humankapitalintensiven Produktion liegt und es folgt damit die Imitationsstrategie. Durch die {\"O}ffnung des Landes verschlechtern sich die Entwicklungschancen des Sektors. 
Au{\ss}erdem l{\"a}sst sich daraus schlussfolgern, dass Volkswirtschaften, die der Imitationsstrategie folgen und einen relativ geringen Entwicklungsstand vorweisen tats{\"a}chlich ein h{\"o}heres Wachstum erzielen, indem sie, unabh{\"a}ngig von einem existierenden Bildungssektor, mehr physisches Kapital in die Konsumg{\"u}terproduktion investieren. Wird nun das fehlende Kapital noch aus der Humankapitalakkumulation bezogen, wie es in der hier angef{\"u}hrten Modellwelt aus Kapitel \ref{Papier2} vorgesehen ist, ist zu erwarten, dass dies den Effekt verst{\"a}rkt und eine noch geringere Wachstumsrate folgen k{\"o}nnte. Entwicklungsstrategisch wird damit die Annahme best{\"a}tigt, dass weniger weit entwickelte L{\"a}nder in den Bildungssektor nur Humankapital investieren sollten.\newline
Die gewonnenen Erkenntnisse best{\"a}tigen eine Arbeit von \citet{Mies.2013}, die den Ansatz verfolgte den Humankapitaleinsatz bei der Produktion von Imitationen hinsichtlich ihrer Intensit{\"a}t zu unterscheiden und danach eine Strategie zu w{\"a}hlen. Sie kommt zu dem Ergebnis, dass in relativ weniger weit entwickelten L{\"a}ndern bei einem hohen Einsatz von Humankapital im Herstellungsprozess adaptierter G{\"u}ter ein Wachstumspfad erreicht wird, der in einem geringen gleichgewichtigen Einkommen m{\"u}ndet. Wohingegen ein geringer Einsatz von Humankapital bei der Produktion zu einem h{\"o}heren Einkommen f{\"u}hren kann. Die Wahl der Produktionsstrategie im adaptierenden Sektor h{\"a}ngt demzufolge vom Entwicklungsstand des Landes ab. Je weiter entwickelt ein Land ist, desto mehr Humankapital sollte in den Produktionsprozess eingehen und desto weiter entwickelte Technologien k{\"o}nnen angewendet werden \citep{Mies.2013}.\\
Ferner zeigt eine zusammenfassende Analyse der Handelseffekte, dass auch diese in den hier angef{\"u}hrten und sehr verschiedenen Wachstumsmodellen best{\"a}tigt werden konnten.\\
Der \textbf{Wettbewerbseffekt} beschreibt die gestiegene Rivalit{\"a}t der Anbieter durch den Zusammenschluss des heimischen mit dem ausl{\"a}ndischen Markt. Der Wettbewerbsdruck veranlasst die Produzenten zu geringeren Grenzkosten zu produzieren, indem Ineffizienzen behoben werden, oder aber neue G{\"u}ter zu entwickeln und sich somit von den Mitstreitern abzusetzen. Beides geht mit Innovationen einher. Demzufolge bedingt Freihandel einen h{\"o}heren Innovationsanreiz und es folgt eine h{\"o}here Innovationsrate. \newline Bei erfolglos innovierenden Unternehmen geht das Risiko einher den bisherigen Absatz an ausl{\"a}ndische erfolgreichere Anbieter zu verlieren. Es folgt der sogenannte Flucht-Eintritts-Effekt, der ein Bestreben der Unternehmen das Risiko eines Marktaustritts zu mindern bewirkt. Das Risiko des Marktaustritts erh{\"o}ht also die Innovationsrate. Hinzu kommt au{\ss}erdem, dass es den Unternehmen nicht nur um eine bestehende Position am Markt geht, sondern nun auch die M{\"o}glichkeit existiert die ausl{\"a}ndischen Anbieter zu verdr{\"a}ngen und zus{\"a}tzliche Gewinne zu erwirtschaften. \newline Der Wettbewerbseffekt f{\"u}hrt einerseits im Inland zum Ausscheiden unproduktiver Unternehmen und damit zu einem Anstieg der gesamten Wirtschaftsleistung eines Landes. Andererseits induziert der zus{\"a}tzliche Wettbewerbsdruck eine steigende Innovationst{\"a}tigkeit der Unternehmer. \newline 
Die vorangegangenen Untersuchungen haben gezeigt, dass es zu einem Anstieg der Innovationsrate per Au{\ss}enhandel kommt. Das er{\"o}rterte Modell in Kapitel \ref{Papier1} verdeutlicht dies durch einen grunds{\"a}tzlich fr{\"u}heren Wechsel zur Innovationsstrategie, unabh{\"a}ngig vom technologischen Entwicklungsstand eines Landes. F{\"u}r die Umsetzung dieser Strategie ist qualifizierte Arbeit notwendig. Der Anstieg der Nachfrage an ausgebildeter Arbeit wird, wie auch in Kapitel \ref{Papier2} gezeigt, durch den Au{\ss}enhandel induzierten Anreiz befriedigt, der die Haushalte veranlasst tendenziell eher in Weiterbildung zu investieren.\footnote{Es wurde gezeigt, dass durch Au{\ss}enhandel die Entscheidungsvariable im Gleichgewicht $u^*$ ansteigt.} Demzufolge konnte der Wettbewerbseffekt in dieser Arbeit best{\"a}tigt werden. \newline
Der \textbf{Marktgr{\"o}{\ss}eneffekt} spielt bei der {\"O}ffnung eines Landes ebenfalls eine Rolle. In einem {\"o}konomisch kleinen Land besteht die M{\"o}glichkeit, dass sich die Durchf{\"u}hrung einiger Innovationen nicht lohnen w{\"u}rde, da die Forschungs- und Entwicklungskosten den erwarteten Gewinn {\"u}bersteigen. Der erwirtschaftete Gewinn einer Innovation ist aus beiden M{\"a}rkten  deutlich h{\"o}her, als wenn die Innovation nur in einem Markt eingef{\"u}hrt worden w{\"a}re. Demnach kann eine zuvor noch unrentable Innovation nun lohnend sein. Alle weiteren Innovationen die bei geringen Gewinnaussichten durchgef{\"u}hrt worden w{\"a}ren, f{\"u}hren bei steigender Marktgr{\"o}{\ss}e zu deutlich h{\"o}heren Ertr{\"a}gen.
Auch hier steigt die Innovationst{\"a}tigkeit an und hat abh{\"a}ngig von der {\"o}konomischen Gr{\"o}{\ss}e eines Landes unterschiedliche Wachstumswirkungen. Denn in {\"o}konomisch gro{\ss}en L{\"a}ndern {\"a}ndert sich die Marktgr{\"o}{\ss}e nicht so stark wie {\"o}konomisch kleine L{\"a}nder, die sich dem Handel {\"o}ffnen. Demzufolge ist auch der Wachstumseffekt in {\"o}konomisch kleinen L{\"a}ndern h{\"o}her als in {\"o}konomisch gro{\ss}en Volkswirtschaften.\footnote{In diesem Zusammenhang ist nicht der negative Wachstumseffekt bei einer Ausweitung der endogenen Welttechnologiegrenze die Rede, sondern von der zus{\"a}tzlichen Entwicklung eines Landes, die sich in Wachstum {\"a}u{\ss}ert.}\\
Grunds{\"a}tzlich bedingt auch dies wieder einen h{\"o}heren Entwicklungsstand durch die Innovationsstrategie und ist nun auch f{\"u}r L{\"a}nder mit einem relativ gesehen gr{\"o}{\ss}eren Abstand zu WTG ratsam, als in der Autarkiesituation. Wie schon zuvor beschrieben wird mehr ausgebildete Arbeit nachgefragt, die auch tats{\"a}chlich vorhanden ist.\footnote{Zumindest in einem gr{\"o}{\ss}eren Umfang als in geschlossenen Volkswirtschaften.} Jedoch wird im Hauptteil dieser Arbeit nicht zwischen {\"o}konomischen L{\"a}ndergr{\"o}{\ss}en unterschieden. Demzufolge kann auch ein st{\"a}rkerer Wachstumseffekt bei {\"o}konomisch kleinen L{\"a}ndern nicht nachgewiesen werden. Da hier nur {\"o}konomisch kleine L{\"a}nder betrachtet werden wird lediglich angenommen, dass der Marktgr{\"o}{\ss}eneffekt deutlich sp{\"u}rbar sein m{\"u}sste. Neben der Innovationst{\"a}tigkeit bewirkt der Marktgr{\"o}{\ss}eneffekt allgemein, dass grunds{\"a}tzlich {\"o}konomisch kleine L{\"a}nder st{\"a}rker  vom Handel profitieren als dies bei gro{\ss}en L{\"a}ndern der Fall ist. Dies ist ebenfalls durch das Ausma{\ss} der Ver{\"a}nderung der Marktgr{\"o}{\ss}e zu erkl{\"a}ren, woraus sich auch andere Gewinnm{\"o}glichkeiten ergeben. Somit ist der Zugewinn eines kleinen Landes relativ h{\"o}her, als der eines gro{\ss}en Landes, welches nur im geringen Ma{\ss}e von der Markt\-erweiterung profitiert. \newline Diesen Zusammenhang best{\"a}tigen auch \citet{Alesina.2005} in ihrer Regression von Wachstum auf die Handelsoffenheit. Die Autoren haben in ihren Untersuchungen einen negativen Koeffizienten zwischen der Offenheit eines Landes und der Landesgr{\"o}{\ss}e festgestellt. Dabei endogenisieren sie die Gr{\"o}{\ss}e eines Landes und k{\"o}nnen den Einfluss vom Au{\ss}enhandel auf die L{\"a}ndergr{\"o}{\ss}e hinsichtlich des {\"o}konomischen Wachstums beobachten.
\newline

Durch die Einf{\"u}hrung einer Exportf{\"o}rderung k{\"o}nnen au{\ss}erdem hinsichtlich der Sektorgr{\"o}{\ss}e Aussagen getroffen werden. Wie in Kapitel \ref{Papier1} anhand des Modells gezeigt wurde, f{\"u}hrt dies zu einer Fokussierung auf den Exportsektor, dem damit tendenziell gr{\"o}{\ss}ere Projekte zugeteilt werden. Daraus resultieren unterschiedlich gro{\ss}e Ex- und Importsektoren. Obwohl der Exportsektor aktiv unterst{\"u}tzt wird, profitiert der nun relativ kleinere Importsektor st{\"a}rker von Au{\ss}enhandel.\footnote{Dieser Zusammenhang konnte hier f{\"u}r ein technologisch kleines Land jedoch nicht best{\"a}tigt werden. Denn im Importsektor verschlechtern sich durch Handel die Entwicklungsm{\"o}glichkeiten, wohingegen sie sich im Exportsektor verbessern. In einem technologisch gro{\ss}en Land hingegen trifft diese Aussage zu und Au{\ss}enhandel f{\"o}rdert den vorwiegend imitierenden kleineren Importsektor st{\"a}rker als den Exportsektor.} Diesen Zusammenhang zeigten ebenfalls \citet{Aghion.2013} anhand der Daten S{\"u}dafrikas. 
Dabei weisen sie eine F{\"o}rderung des Produktivit{\"a}tswachstums durch die stetige {\"O}ffnung des Landes in kleineren Sektoren nach.  Weil S{\"u}dafrika von einer heterogenen Struktur der Sektoren gepr{\"a}gt ist, k{\"o}nnen sie sogar spezifisch zeigen, dass Handel in relativ kleinen Sektoren einen st{\"a}rkeren positiven Effekt hat als in relativ gro{\ss}en Sektoren.\\
Bezieht man sich nun auf die verschiedenen Entwicklungsst{\"a}nde einer Volkswirtschaft 
wurde grunds{\"a}tzlich gezeigt, dass weniger weit entwickelte Volkswirtschaften eher der Imitationsstrategie folgen sollten. Die Bedeutung von Innovationen nimmt also erst mit steigendem Abstand zur WTG ab. Denn es ist m{\"o}glich, dass sich ein Land von seiner "`schlechten"' Position entmutigen l{\"a}sst und somit Handel negative Innovationsanreize setzt. Dieser Entmutigungseffekt f{\"u}hrt bei relativ r{\"u}ckst{\"a}ndigen Volkswirtschaften zu dem Impuls sich von jeglichen Innovationst{\"a}tigkeiten abzuwenden. Zwar regen erfolgreiche Innovationen den Aufholprozess an, dies erscheint jedoch in Anbetracht m{\"o}glicher Imitationen als sehr ressourcenaufwendig und nicht wirtschaftlich.  
Dieser Effekt konnte durch die vorgenommene Analyse nachgewiesen werden. Es wird vielmehr verdeutlicht, dass die Unternehmen nicht nur entmutigt werden, sondern, dass es grunds{\"a}tzlich auch rentabler ist, mit einem relativ geringen technischen Entwicklungsstand zu imitieren als zu innovieren.\footnote{Da in der vorgelagerten Untersuchung aus Kapitel \ref{Papier2} nicht zwischen innovierenden und imitierenden T{\"a}tigkeiten unterschieden wird, k{\"o}nnen auch zu diesem Punkt keine Aussagen getroffen werden.} \\
Die Wirkung vom Au{\ss}enhandel ist auch vom Entwicklungsstand eines Landes abh{\"a}ngig. Die {\"O}ffnung eines Landes stimuliert das Wachstum, jedoch profitieren die weniger weit entwickelten L{\"a}nder st{\"a}rker von den sogenannten \textbf{Wissens-Spillover-Effekten} als die weiter entwickelten L{\"a}nder, die das Wissen "`abgeben"' \citep{Sachs.1995,Grossman.1990b}. Hier ist das Ausma{\ss} der Aufholm{\"o}glichkeit entscheidend. Je weiter ein Land entwickelt ist, desto geringer sind die zus{\"a}tzlichen Gewinne, die durch die Einf{\"u}hrung neuer Technologien generiert werden k{\"o}nnen. Ein relativ weniger weit entwickeltes Land hingegen kann hinsichtlich des technologischen Fortschritts deutlich st{\"a}rker aufholen und profitiert somit mehr von handelsliberalisierenden Ma{\ss}nahmen, als  ein Land, das weit entwickelt ist und somit relativ wenig M{\"o}glichkeiten hat neue Technologien einzuf{\"u}hren durch die es bereichert wird \citep{Keller.2004}. Handel verst{\"a}rkt eindeutig diesen Effekt, weil beispielsweise ausl{\"a}ndische Forschungsinvestitionen mit zunehmendem Offenheitsgrad zu inl{\"a}ndischen Produktivit{\"a}tseffekten f{\"u}hren \citep{Coe.1995}.\newline 
Die Ber{\"u}cksichtigung des Entwicklungsstandes in dieser Arbeit erlaubt es Aussagen {\"u}ber den Spillover-Effekt vom Handel treffen zu k{\"o}nnen. Er ist sogar Kern der {\"U}berlegung, dass ein weniger weit entwickeltes Land von dem Handel mit einem weiter entwickelten Land profitiert. So beeinflusst zwar einerseits der Wissenstransfer die Innovationsrate positiv, andererseits f{\"u}hrt dies ebenfalls zu einer h{\"o}heren Imitationst{\"a}tigkeit, die ebenso den technologischen Wissensstand eines Landes erh{\"o}ht. Dieser Effekt des Entscheidungsproblem basiert auf der Humankapitalakkumulation durch den Wissenstransfer und f{\"u}hrt in weniger weit entwickelten L{\"a}ndern zu einem Aufholprozess.\\



Es bleibt jedoch noch die Frage nach der hier entwickelten Entwicklungsstrategie zu kl{\"a}ren. Bis in die 1970er Jahre war es in vielen Entwicklungsl{\"a}ndern {\"u}blich die importierten Industrieg{\"u}ter durch heimische Produkte zu ersetzen und somit die Importe einzuschr{\"a}nken. Diese Importsubstitution und auch andere protektionistische Ma{\ss}nahmen f{\"u}hrten beispielsweise in L{\"a}ndern Lateinamerikas wie Brasilien oder Mexiko zu einem zu starken wirtschaftlichen  Wachstum. {\"U}berholt wurden diese mittlerweile stagnierenden L{\"a}nder durch noch st{\"a}rker wachsende Volkswirtschaften wie HongKong oder Singapur, deren Wachstum durch noch st{\"a}rker wettbewerbseinschr{\"a}nkende politische Ma{\ss}nahmen stimuliert wurde. Auf andere Art und Weise, jedoch genauso erfolgreich, gelang es L{\"a}ndern wie Japan und Korea ein hohes Wirtschaftswachstum zu generieren. Sie haben auf starke Wettbewerbseinschr{\"a}nkungen verzichtet und der Schwerpunkt wurde auf hohe Investitionst{\"a}tigkeiten, staatliche Subventionen und Konglomerate gelegt. Dieser strategische Ansatz wurde auch in Kapitel \ref{Papier1} implementiert und verdeutlichte die Wirkung vom Au{\ss}enhandel auf den technologischen Fortschritt eines Landes.\newline
Die vorgelegte Arbeit zeigt, dass es in dem hier angef{\"u}hrten Zusammenhang nicht notwendig ist weniger weit entwickelte L{\"a}nder durch protektionistische Ma{\ss}nahmen zu sch{\"u}tzen. Denn 
L{\"a}nder die noch weit von der WTG entfernt sind, stellen sich mit hohen Markteintrittsbarrieren, wie beispielsweise Z{\"o}lle oder Kontingente nicht zwingend besser. Staatliche Eingriffe, die den Freihandel unterst{\"u}tzen f{\"u}hren zu einer geeigneten strategischen Ausrichtung mit einem anhaltenden Wachstum. So wurde bereits gezeigt, dass durch gezielte Investitionen die Strategie gelenkt werden kann. Weil die Innovations- und Imitationst{\"a}tigkeiten von verschieden ausgebildeten Arbeitern durchgef{\"u}hrt werden, kann man aus diesem Umstand eine gezielte Entwicklungsstrategie ableiten. Wird der Bildungsstand wie von \citet{Benhabib.1994} anhand der Bildungsausgaben charakterisiert, dann f{\"u}hren Bildungsausgaben in den Bildungsbereich, der eine solide Grundausbildung der Bev{\"o}lkerung sichert, zu erfolgreichen Imitationen. Die Innovationst{\"a}tigkeit eines Landes wird durch die Unterst{\"u}tzung des h{\"o}heren Bildungsbereiches intensiviert. Wie in dieser Arbeit gezeigt wurde steigt mit der N{\"a}he zur WTG die Bedeutung von Innovationen. Dann folgt daraus, dass auch Investitionen im h{\"o}heren Bildungsbereich mit der N{\"a}he zur WTG an Bedeutung zunehmen.\footnote{Auf politischer Ebene l{\"a}sst sich laut \citet{Vandenbussche.2006} daraus herleiten, dass technologisch weniger weit entwickelte L{\"a}nder besser durch Bildungsinvestitionen in die Grundausbildung unterst{\"u}tzt werden, wohingegen das Produktivit{\"a}tswachstum relativ weit entwickelter L{\"a}nder durch Investitionen in den h{\"o}heren Bildungsbereich gef{\"o}rdert werden.}
Wohingegen in L{\"a}ndern, die relativ weit von der WTG entfernt sind, eher von Bildungsausgaben profitieren, die Grundkenntnisse und einfache Fertigkeiten f{\"o}rdern.\\

Zusammenfassend und in Bezug auf die aufgestellten Thesen l{\"a}sst sich festhalten, dass politische Handlungsempfehlungen von der Lage zur Welttechnologiegrenze abh{\"a}ngen. Wird zwischen Innovationen und Imitationen anhand des Abstandes zur Welttechnologiegrenze unterschieden, dann lassen sich diesen verschiedene Segmente des Bildungssystems zuordnen. Die Bedeutung von Investitionen in die Grundausbildung, welche vor allem die Imitationst{\"a}tigkeit unterst{\"u}tzen, nimmt mit der N{\"a}he zur WTG ab. Wohingegen die Rolle h{\"o}herer Bildungsinvestitionen mit der Lage zur WTG zunimmt.\\
In der vorliegenden Arbeit wurde eine Entwicklungsstrategie vorgestellt, die zun{\"a}chst ein Angebot an qualifizierter Arbeit bereitstellt, damit diese anschlie{\ss}end durch gezielte Investitionen den technologischen Entwicklungsstand eines Landes und somit letztendlich auch das Wachstum beg{\"u}nstigt. Begr{\"u}ndet wird das Wachstum durch den technischen Fortschritt und die Humankapitalakkumulation, ausgel{\"o}st und beg{\"u}nstigt durch den Au{\ss}enhandel und den damit einhergehenden Effekten.