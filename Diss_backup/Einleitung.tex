\chapter{Einleitung}\label{Einleitung}
Blickt man auf das vergangene Jahrtausend zur{\"u}ck, so haben drei einschneidende interaktive Ereignisse den Entwicklungsprozess der Welt bestimmt \citep{Maddison.2001}.\footnote{Dabei sind hier vor allem Ereignisse mit �konomischer Wirkung von Bedeutung somit werden politische Begebenheiten und die damit zusammenh�ngenden wirtschaftlichen Konsequenzen vernachl�ssigt.} \begin{itemize}
	\item Umsiedlung und Landerschlie{\ss}ung
	\item Handel
	\item technischer Fortschritt
\end{itemize}
Die Besiedelung und Bewirtschaftung unerschlossener Regionen f�hrte zu einer Erweiterung des Faktors Boden. Als kurze Beispiele dienen China und die Erschlie{\ss}ung des amerikanischen Kontinents.  Neue Verfahren im Reisanbau erm{\"o}glichten eine Anpassung an die geologischen Rahmenbedingungen und er�ffneten neue geographische Anbaum�glichkeiten. Nun konnte auch die Region s{\"u}dlich des Flusses Yangtse bewirtschaftet werden. Daraus folgte, dass sich vom achten~bis zum dreizehnten Jahrhundert Chinas Bev{\"o}lkerung ma{\ss}geblich umsiedelte und sich damit an die neuen Bedingungen anpasste. Der prozentuale Bev{\"o}lkerungsanteil hat sich s{\"u}dlich des Yangtse mehr als verdoppelt. {\"A}hnlich verhielt es sich mit der Erschlie{\ss}ung Amerikas durch die europ{\"a}ische Bev{\"o}lkerung. Unbekanntes fruchtbares Land sowie neue Ressourcen wurden entdeckt und eingesetzt, so dass die Produktivit{\"a}t anstieg und letztlich ein Einkommenszuwachs verursacht wurde \citep{Maddison.2001}.\\
Das zweite einschl{\"a}gige Ereignis des letzten Jahrtausends war die Aufnahme von Handel zu anderen Staaten. Dies hat vor allem nach \citet{Maddison.2001} die europ{\"a}ischen L{\"a}nder und weniger die afrikanischen und asiatischen L{\"a}nder in ihrer Entwicklung  beeinflusst. Vom Jahr 1000 bis 1500 waren die strategische Lage und das Wissen um den Schiffbau Grund f{\"u}r die Bedeutung Venedigs bez{\"u}glich des internationalen maritimen Handels. Es wurden {\"u}berwiegend Seide und Gew{\"u}rze mit fern{\"o}stlichen L{\"a}ndern wie China und Syrien gehandelt. Auch schon damals bedingte die Offenheit eines Landes nicht nur die Einfuhr unbekannter G{\"u}ter, sondern auch den Transfer von Produktionstechnologien und Wissen. Das westliche europ{\"a}ische Handelszentrum war Portugal. Ein weiterer Mitstreiter auf dem Gew{\"u}rzhandelsmarkt waren die Niederlande, die jedoch zeitlich etwas sp{\"a}ter erst ab 1500 eine {\"a}hnliche Flotte einsetzten. 1700 waren in den Niederlanden nur 40{\%}  der Erwerbsbev{\"o}lkerung im landwirtschaftlichen Sektor besch{\"a}ftigt. Der gr{\"o}{\ss}te Teil des Volkseinkommens wurde durch die Seefahrt und den Dienstleistungssektor erwirtschaftet. �hnlich verhielt es sich in Spanien, einer weiteren wichtigen maritimen Handelsmacht. Diese Zeit wurde gepr{\"a}gt durch starkes Konkurrenzdenken zu Lasten der Mitstreiter, denn Kooperationen wurden gr�{\ss}tenteils vernachl�ssigt. Ebenfalls der Schifffahrt schlossen sich im 18. Jahrhundert Frankreich und England an. Englands Vorteil gegen{\"u}ber seinen Mitstreitern lag in einem ausgebauten Netz an Institutionen im Banken- und Finanzsektor sowie staatlicher Einrichtungen. Das Wachstum Gro{\ss}britanniens war zu dieser Zeit h{\"o}her, als bei allen anderen europ{\"a}ischen L{\"a}nder. Unterst{\"u}tzend f{\"u}r die weltweiten Handelsrouten waren die Kolonien, die  die Erschlie{\ss}ung von Ressourcen und Rohstoffen erlaubten und Grund f{\"u}r die {\"U}berwindung bisher ferner Distanzen lieferten \citep{Maddison.2001}.
Mit der Industrialisierung begann hinsichtlich des Wirtschaftswachstums ein neues Zeitalter. Bedingt durch den technischen Fortschritt wuchs das Pro-Kopf-Einkommen Gro{\ss}britanniens schneller als jemals zuvor. Es gelang den Engl{\"a}ndern ihren physischen Kapitalstock erheblich aufzubauen, sowie die steigende Nachfrage nach qualifizierten Arbeitskr{\"a}ften durch den Ausbau des Bildungssystems zu befriedigen. Au{\ss}erdem begann das britische Empire Handelsbeschr{\"a}nkungen zu reduzieren, was einen positiven Effekt auf die {\"u}brige Welt hatte, da dies auch den Diffusionsprozess von technischem Wissen beg{\"u}nstigte und somit die Industrialisierung in andere L{\"a}nder trug. Auch die Einf{\"u}hrung eines Eigentumsrechte-Systems des Staates steigerte die Attraktivit{\"a}t f{\"u}r Investoren.\footnote{Dieses Eigentumsrechte-System ist mit dem heutigen Patentrecht zu vergleichen.} England war ein wohlhabender Staat, der mit jeder Entwicklung die Welttechnologiegrenze ausweitete. \newline Die beiden Weltkriege zerst{\"o}rten die Ordnung des freien Handels und das weltweite Wirtschaftswachstum war bis 1950 mehrheitlich deutlich geringer als bis zum Beginn des ersten Weltkrieges 1913. Die Nachkriegszeit, nach dem zweiten Weltkrieg, brachte vor allem in den europ{\"a}ischen L{\"a}ndern eine Zeit des Aufschwungs mit sich. Das weltweite BIP stieg j{\"a}hrlich um etwa 5{\%} an, der weltweite Handel wuchs um 8{\%} und das Pro-Kopf Einkommen um 3{\%} j{\"a}hrlich.\footnote{Diese Informationen basieren auf den Daten der OECD laut \citet{Maddison.2001}.} Die beiden Weltkriege brachten zudem eine neue politische Ordnung hervor. Der kalte Krieg brach die Verbindung zwischen der westlichen Welt mit dem russisch wohl gesonnenen Osten ab. Internationaler Handel war trotz ausgebauter Transportm{\"o}glichkeiten eingeschr{\"a}nkter als Anfang des 20ten Jahrhunderts \citep{Maddison.2001}.\\
Als dritten interaktiven Prozess nach \citet{Maddison.2001} wird erneut auf das letzte Jahrtausend zur{\"u}ckgeblickt, jedoch diesmal unter dem Aspekt der technologischen Entwicklung und der Einbettung von Institutionen. \newline Der technische Fortschritt war zwar von 1000-1820 verglichen zu heutigen Verh{\"a}ltnissen relativ gering, war aber schon damals ein entscheidender Faktor f{\"u}r das Wirtschaftswachstum. Nur durch technische Errungenschaften der Seefahrt wie beispielsweise der Kompass, die Sanduhr und  weitere Entwicklungen der Schifffahrt gelang es den Handel in deutlich weiter entfernte L{\"a}nder aufzunehmen. Au{\ss}erdem konnten Neuerungen im landwirtschaftlichen Sektor das immer weiter ansteigende Bev{\"o}lkerungswachstum kompensieren und ernsthafte Hungersn{\"o}te verhindern. Bis zum 15. Jahrhundert wurden viele technologische Neuerungen aus dem asiatischen und arabischen Raum nach Europa transferiert. Trotzdem profitierten letztendlich die europ{\"a}ischen L{\"a}nder st{\"a}rker als die Herkunftsl{\"a}nder selbst. Als einer der entscheidenden Unterschiede sieht \citet{Maddison.2001} die angesprochenen Institutionen wie das intakte Finanz-, Versicherungs- und Bankensytem, dessen Vorreiter England war. Auch der Devisenmarkt erleichterte den H{\"a}ndlern der damaligen Zeit ihre Arbeit und minderte ihre Transaktionskosten erheblich.  \newline Der Transfer dieses Systems oder neuer Technologie von Europa aus in die �brige Welt war jedoch relativ gering. Ein funktionierender Wirkungskanal des 18.Jahrhunderts waren die Kolonien Gro{\ss}britanniens in Nordamerika \citep{Maddison.2001}.

Die Argumentation Maddisons verdeutlicht m�gliche Einflussfaktoren auf den Entwicklungsprozess. \citet{Gandolfo.1998} f�hrt �hnliche Gr�nde f�r Wachstum an, vernachl�ssigt jedoch den Einfluss des Handels. Sein Fokus liegt zun�chst auf der Faktorakkumulation, bei \citet{Maddison.2001} am Beispiel des Produktionsfaktors Boden, aber auch Migration und somit der Produktionsfaktor Arbeit w�re m�glich. Nachdem die Faktorakkumulation lange als Ursprung {\"o}konomischen Wachstums angesehen wurde, hat sich die Wissenschaft einer neuen Richtung gewidmet, die den technologischen Wandel als Kern des Wachstums ansieht. Der Motor des Wachstums der "`Neuen Wachstums{\"o}konomie"' oder auch "`Endogenen Wachstums�konomie"'wird im technischen Fortschritt gesehen \citep{Gandolfo.1998,Maddison.2001}.\newline Diese Arbeit wird sich vornehmlich mit den zwei Str{\"o}mungen dieser Richtung besch{\"a}ftigen und jeweils eine Modellvariation einer offenen Volkswirtschaft vorstellen. \newline Bei dem ersten Modell, das in Kapitel \ref{Papier2} folgt, stehen Wissensexternalit{\"a}ten bei der Humankapitalakkumulation im Vordergrund, die den technischen Fortschritt begr{\"u}nden. Das Modell basiert auf dem Ansatz von \citet{Lucas.1988}, der neben \citet{Romer.1990} einer der Hauptvertreter dieser Ausrichtung ist. \newline Das zweite Modell in Kapitel \ref{Papier1} fokussierte sich auf private Investitionen im Forschungs- und Entwicklungssektor als Ursache f{\"u}r {\"o}konomisches Wachstum. Angeh{\"o}rige dieser Forschngsrichtung sind beispielsweise \citet{Romer.1990,Grossman.1991c} sowie \citet{Aghion.1992}. Dabei f{\"u}hren Investitionen der Unternehmen zu Innovationen\footnote{Dies ist unabh�ngig davon ob die Anzahl der verf�gbaren G�ter gleich bleibt \citep{Aghion.1992} oder ansteigt \citep{Romer.1990}.}, die letztlich den technischen Fortschritt beschreiben. Die hier vorgestellte Modellvariation basiert auf dem Papier von \citet{Acemoglu.2006}, die den Grundgedanken der zuvor genannten Abhandlungen aufgreifen und Aussagen {\"u}ber makro{\"o}konomische strategische Einscheidungen zulassen. \newline Der Schwerpunkt beider Modellvariationen liegt in der Einbettung von internationalem Handel in diese Wachstumsmodelle. Au{\ss}enhandel verbindet L{\"a}nder und f{\"u}hrt deren Reaktionen und Situationen auf dem Weltmarkt zusammen. Diese wechselseitigen Beziehungen gehen sowohl mit Wissensdiffusion und anderweitigen Interaktionen einher. Der Kern dieser Arbeit ist die �berpr�fung der folgenden These: Handel f�hrt zu einer Entwicklungsstrategie, die eine innovative bzw. imitative Ausrichtung der Unternehmen anstrebt und ein anhaltendes positives Wachstum bedingt. Dabei spielen politische Entscheidungen und Spillover-Effekte eine Rolle.\\
Werden die Modellvariationen aus \ref{Papier2} und \ref{Papier1} getrennt voneinander betrachtet, dann f�hrt Handel zum einen zu einem besseren Bildungssystem, zum anderen zu einem h�heren technischen Entwicklungsstand durch politische Ma{\ss}nahmen. Kombiniert man beide Modelle (\ref{Papier2} und \ref{Papier1}) miteinander, dann resultiert zun�chst ein besseres Bildungssystem, dass dann wiederum die technologische Entwicklung eines Landes beg�nstigt. \\
Um die Hauptthese zu untersuchen ist die Aufstellung folgender Nebenthese notwendig: Ein relativ weniger weit entwickeltes Lande folgt der Imitationsstrategie, wohingegen ein weiter entwickeltest Land die Innovationsstrategie pr�feriert. Neben der Tatsache, dass Humankapitalakkumulation zu einem h�heren Entwicklungsstand f�hrt kommt au{\ss}erdem der Zusammensetzung des Humankapitals eine besondere Bedeutung zu. \\
Der Einflu{\ss} des Handels soll hier unterstrichen werden und zeigen, dass unabh�ngig von der Modellvariation ein besseres Bildungssystem resultiert und der Au{\ss}enhandel die technologische Entwicklung eines Landes beg�nstigt.
Denn die Erweiterung eines endogenen Wachstumsmodells um Handel zeigt, dass nicht nur der G{\"u}terhandel die Entwicklung eines Landes beeinflusst, sondern, dass es auch zu Wissensstr{\"o}men kommt, die die Wohlfahrt eines Landes erh{\"o}hen.
Die Entwicklungspolitik orientiert sich weg von physischen Investitionsprojekten und hin zur F{\"o}rderung von Bildung. Auch hier wird dieser Ansatz aufgegriffen, indem Au{\ss}enhandel ein h�heres Angebot an Humankapital bedingt, welches anschlie{\ss}end durch exportf�rdernde Investitionen gezielt eingesetzt wird. \\
Die vorliegende Arbeit pr�ft vornehmlich in Kapitel \ref{Papier2}, \ref{Papier1} und \ref{Kombi} die aufgestellten Thesen indem in endogene Wachstumsmodelle Handel integriert wird. Kapitel \ref{Papier2} und \ref{Papier1} behandelt die beiden Modellvariationen endogener Wachstumsmodelle, deren Ergebnisse anschlie{\ss}end in Kapitel \ref{Kombi} kombiniert werden. Daf�r werden in Kapitel \ref{Wachstum} und \ref{sec:Globalisierung} die theoretischen Grundlagen dargelegt. Die Analyse der in Kapitel \ref{sec:Globalisierung} vorgestellten Handelseffekte werden in den weiterf�hrenden Kapiteln besonders ber�cksichtigt. Kapitel \ref{Auswertung} wertet die Ergebnisse aus und widmet sich der Belegung bzw. Wiederlegung der hier aufgestellten Thesen.