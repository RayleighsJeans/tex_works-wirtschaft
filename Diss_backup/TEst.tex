\documentclass[12pt, xcolor=dvipsnames]{report}
%hatte vorher sehr lange report als documentclass
%\documentclass[a4paper, 12pt, DIVcalc, oneside, headsepline, ngerman, smallheadings, openany, liststotoc, bibtotoc]{scrbook}
%\usepackage{scrpage2} % zur Manipulation der Kopf und Fusszeilen
\usepackage{makeidx} % zum Erstellen und Indexen (Abkürzungsverzeichnis)
\makeindex % Befehl dazu?
\usepackage[intoc]{nomencl} % Abkuerzungsverzeichnis (und ins Inhaltsverzeichnis aufnehmen)
\let\abk\nomenclature % Befehl umbenennen in abk
\renewcommand{\nomname}{Abkürzungsverzeichnis} % Deutsche Überschrift
\makenomenclature
%\usepackage{endnotes} % End- und Fu{\ss}noten
\usepackage[breaklinks=true]{hyperref} % klickbare Kapitel und Link-Fraben
%\usepackage{hypdvips} % ermöglicht trotz hyperref den Zeilenumbruch im Inhaltsverzeichnis
\hypersetup{colorlinks=true, linkcolor=blue, urlcolor=blue, citecolor=blue}
\usepackage{caption} % für z.B. Tabellenüberschriften
\usepackage{tocbibind} %gibt im Inhaltsverzeichnis bspw. das Abbildungsverzeichnis an
%\usepackage{subcaption}
\usepackage{subfigure}
\usepackage{floatrow}
\usepackage{float}
\usepackage[percent]{overpic} %damit kann man in einem Bild zusätzlich beschriften
\usepackage{color}
%\usepackage{svg}
\usepackage{fancyhdr}
%\usepackage[utf8]{inputenc}%direkte Angabe von Umlauten; brauch ich für Literaturverzeichnis; aber zerschie{\ss}t mir alles
\usepackage[T1]{fontenc}
%\usepackage{lmodern}
%\usepackage{tocstyle}
%\usetocstyle{KOMAScript}
\usepackage[authoryear]{natbib}%hardvard style zitieren
\usepackage[ngerman]{babel} %deutscher Satzbau
\usepackage{ngerman} %wenn ich das reinnehmen: viele Fehler
\usepackage{paralist} 
\usepackage{enumitem} 
%\usepackage[ansinew]{inputenc} %damit in Windows die umlaute funktionieren
\usepackage{a4wide}%einstellung der Seitenränder
\usepackage{lmodern,amssymb,xcolor,amsmath}
%%\usepackage[style=authortitle-icomp]{biblatex}
\usepackage{natbib}
\usepackage{tikz}
\usepackage{fancybox} % für die verschiedenen Boxen
\usepackage{ulem} % für unterstreichen und durchstreichen im Text
\usetikzlibrary{arrows,positioning}
\usepackage{float}
\usepackage{acronym}% notwendig für die Definition von Abkürzungen und deren Anzeige im Abkürzungsverzeichnis
\usepackage{ulem}% um gestrichelte oder gepunktete underlines zu machen
\usepackage{psfrag,epsfig,graphicx,subfigure,enumerate,pifont,bibentry,pstricks,marvosym,rotating}
%\usepackage{geometry} %passt die Seitenränder an

\usepackage{lmodern}
\usepackage{longtable}
\usepackage{verbatim} % kann damit Kommentare angeben, die im Texcode sind
\usepackage[section]{placeins}% damit Abbildungen nicht ins nächste Kapitel rutschen
\pagestyle{fancy} 
%\geometry{top=2cm}


\begin{document}
%\pagenumbering{Roman}  %römische Nummerierung aktivieren; rausgenommen um Inhaltsverzeichnis auf Seite I zu haben
%%ab hier Titelseite%%%%%%%%%%%%%%%%%%%%%%%%%%%%%%%%%%%%%%%%%%%%%%%%%%%%%%%%%%%
%\maketitle
\pagenumbering{gobble}
\thispagestyle{empty}

\begin{center}
%\begin{verbatim}
%Universität Greifswald

%\end{verbatim}
\begin{figure}[h!]
 \vspace{-2.8cm}\centering 
  \begin{tabular}{@{}r@{}} 

\includegraphics[width=0.45\textwidth]{C:/Users/BibiKiBa/Diss/Abbildungen/unilogo.eps}\\
%\epsfig{figure=\figpath/200JahreBIP.eps,width=0.95\textwidth}\\
  %\footnotesize\sffamily\textbf{Quelle:} Galor (2011) %\cite{pps} 
  \end{tabular}  
	\label{unilogo}
\end{figure}
\hline{}\\[0.5cm]
 	\begin{Huge} 
	\centering \textbf{Endogenes Wachstum und Internationaler Handel}
	
 	\end{Huge} {}\\[1cm]
 	\begin{large} 
 - Die Wirkung von Au{\ss}enhandelseffekten auf den technischen Fortschritt-\\[0.5cm]
	\end{large}
 \hline
{}\\[1.5cm]
\centering
%\includegraphics[width=0.3\textwidth]{unilogo.png}\\[6cm] %% Unilogo kann eingefügt werden



Inaugural-Dissertation \\
\vspace{4mm} zur \\
\vspace{4mm} Erlangung des Grades eines Doktor rerum politicarum\\
\vspace{4mm} der \\
\vspace{4mm} Universit{\"a}t Greifswald \\
\vspace{4mm} im \\
\vspace{4mm} Bereich der Wirtschaftswissenschaften \\
\vspace{10mm}
\begin{flushleft}
		\begin{tabular}{l} 
			vorgelegt von Birgit Kirschbaum\\
			Greifswald den \today\\
			{}\\
			{}\\
			Erstgutachter: Prof. Dr. rer. pol. Susanne Soretz\\
			Zweitgutachter: Prof. Dr. rer. pol. Armin Rohde
			
		\end{tabular}
\end{flushleft}
\end{center}
\newpage
%%bis hier Titelseite%%%%%%%%%%%%%%%%%%%%%%%%%%%%%%%%%%%%%%%%%%%%%%%%%%%%%%%%%%%

\end{document}