
\chapter{Konvergenz}
Ein weiterer, sehr bedeutender Bereich der Wachstumstheorie, die Konvergenztheorie, wurde bisher vernachl{\"a}ssigt und wird im folgenden kurz erl{\"a}utert. Bei dieser steht nicht die Ergr{\"u}ndung von Wachstum im Vordergrund, sondern die Erkl{\"a}rung der Entwicklung verschiedener Wachstumspfade.\\ Konvergenz beschreibt die Ann{\"a}herung des Pro-Kopf-Einkommens bzw. der Wachstumsrate an einen Referenzwert, meist den der {\"u}brigen Welt oder den {\"a}hnlicher Volkswirtschaften. So zeigte die Entwicklung des Pro-Kopf Einkommens von 13 L{\"a}ndern zwischen den Jahren 1870-1989, dass viele L{\"a}nder zu parallelen Wachstumspfaden konvergieren \citep{Evans.1996}. F{\"u}r die Konvergenz sind zwei Erkl{\"a}rungsans{\"a}tze denkbar. Konvergenz kann entweder ein Ergebnis abnehmender Ertr{\"a}ge der Kapitalakkumulation sein oder aufgrund internationaler Wissens-Spillover-Effekte entstehen. \newline
Wird von abnehmenden Ertr{\"a}gen der Kapitalakkumulation ausgegangen, f{\"u}hrt dies zur \textbf{absoluten Konvergenz} der Wachstumsraten. Empirisch belegt wurde diese These von \citet{SalaiMartin.2002}. Er zeigt, dass w{\"a}hrend des Beobachtungszeitraums von 1970-2000 die Einkommensungleichheit abnahm und die betrachteten L{\"a}nder zueinander konvergierten. Die Arbeiten von \citet{Mankiw.1992} oder \citet{Barro.1997} best{\"a}tigen ebenfalls, dass relativ weniger weit entwickelte L{\"a}nder schneller wachsen und gegen die f{\"u}hrende Technologie, also die Welttechnologiegrenze, konvergieren.
Aber vergleicht man die {\"o}konomischen Daten der {\"a}rmsten und reichsten L{\"a}nder der Welt miteinander, dann f{\"a}llt auf, dass eine hohe Ungleichheit zwischen beiden Extremen besteht und diese noch weiter voneinander divergieren \citep{Maddison.2001}. Das Ph{\"a}nomen der \textbf{Gro{\ss}en Divergenz} beschreibt die Ausweitung des Abstandes des Lebensstandards um ein f{\"u}nffaches zwischen den {\"a}rmsten und reichsten L{\"a}ndern von 1870  bis zum Jahr 1990 \citep{Pritchett.1997}. \citet{Helpman.2004} f{\"u}hrt einen bedeutenden Teil der Einkommensunterschiede zwischen den L{\"a}ndern auf die verschiedenen totalen Faktorproduktivit{\"a}ten zur{\"u}ck.\\
Auf die Beobachtungen einer divergierenden Welt st{\"u}tzte sich \citet{MayerFoulkes.2006} und kategorisierte zun{\"a}chst seine Daten, indem er f{\"u}nf L{\"a}ndergruppen bildete. Dabei stellt er fest, dass die Ungleichheit innerhalb einer Gruppe zwar {\"u}ber den Zeitraum hinweg abgenommen hat, dass aber das Einkommen zwischen den Gruppen divergiert. Die Wachstumsraten vieler armer weniger weit entwickelter L{\"a}nder divergieren und die relative Divergenzl{\"u}cke zwischen den Pro-Kopf-Einkommen der {\"a}rmsten und reichsten Konvergenzgruppen nahm vom Jahr 1960 bis 1995 um den Faktor 2,6 zu \citep{MayerFoulkes.2006}. Dabei handelt es sich hier um die sogenannte \textbf{bedingte Konvergenz}. Die Wachstumsraten bzw. die Pro-Kopf-Einkommen innerhalb einer L{\"a}ndergruppe n{\"a}her sich an, die Konvergenzclubs\footnote{Ein Konvergenzclub besteht aus Volkswirtschaften mit zueinander konvergierenden Wachstumsraten. Dies ist in der Regel dann der Fall, wenn Innovationen entwickelt werden und somit ein Technologietransfer stattfindet. Der damit einhergehende Wissenstransfer f{\"u}hrt f{\"u}r weniger weit entwickelte Volkswirtschaften zu einer Eingliederung in den Konvergenzclub, sofern diesem entsprechende Ressourcen f{\"u}r innovierende T{\"a}tigkeiten bereitgestellt werden. So besteht bspw. eine Konvergenz zwischen L{\"a}ndern, die innovativ ausgerichtet sind und dadurch mit der gleichen Rate wachsen. Dies bedeutet dann gleichzeitig, dass mit der Einstellung innovativer T{\"a}tigkeiten das Wachstum der Volkswirtschaft langfristig stagniert \citep{Aghion.2015}.} an sich entfernen sich aber von einander \citep{Quah.1993,Howitt.2000,Howitt.2005}. Abh{\"a}ngig von dem Entwicklungsstand eines Landes besteht die M{\"o}glichkeit, dass einige L{\"a}nder fr{\"u}hzeitig stagnieren und das hohe Niveau an der WTG nicht erreichen \citep{Aghion.1992,Barro.1997,Howitt.2005}.\\

Der Schwerpunkt dieser Arbeit liegt auf den internationalen Spillover-Effekten durch Au{\ss}enhandel.
Die Vertreter dieses zweiten Erkl{\"a}rungsansatzes folgen dem schumpetrianischem Ansatz der Wachstumstheorien. Freihandel beg{\"u}nstigt die M{\"o}glichkeit der weniger weit entwickelten L{\"a}nder sich den Industrienationen anzuschlie{\ss}en. Technologisches Wissen passiert die Grenzen und alle beteiligten L{\"a}nder k{\"o}nnen davon profitieren. 
Ein Technologietransfer f{\"u}hrt zu einer Anpassung der Produktivit{\"a}ten bei einander {\"a}hnlichen Volkswirtschaften, den beschriebenen Konvergenzclubs \citep{Durlauf.1995, Quah.1993,Quah.1997}.\\
Die M{\"o}glichkeit zu den f{\"u}hrenden L{\"a}ndern aufzuschlie{\ss}en ist als "`catching up"' Prozess bekannt und durch den R{\"u}ckstand eines Landes hinsichtlich des Entwicklungsstandes bedingt. Zu dieser Erkenntnis des Vorteils des R{\"u}ckstands kommen auch \citet{Barro.1990,Barro.1991,Barro.1992}. Sie implizieren ebenfalls, dass dadurch die meisten L{\"a}nder zu parallelen Wachstumspfaden konvergieren. Um ihre These zu belegen untersuchen sie zun{\"a}chst anhand der Daten von 48 US-Bundesstaaten den Einkommenszuwachs seit 1840 und stellen fest, dass relativ {\"a}rmere Bundesstaaten schneller wachsen als reichere und es zu einer Konvergenz aller kommt. In einer folgenden Arbeit, gemeinsam mit \citet{Blanchard.1989}, erweitern sie ihre Analyse auf selbstst{\"a}ndige Staaten und vergleichen einerseits das Wachstum relativ weniger weit entwickelter L{\"a}nder mit relativ weiter entwickelten L{\"a}ndern und andererseits das Wachstum unterschiedlicher Regionen, wie beispielsweise die Ann{\"a}herung S{\"u}ditaliens an Norditalien \citep{Barro.1992}.\newline
Der Aufholprozess weniger weit entwickelter L{\"a}nder zur WTG kann auch auf wettbewerbseinschr{\"a}nkende Staatseingriffe zur{\"u}ckgef{\"u}hrt werden. Im 19ten Jahrhundert gelang es den relativ wenig entwickelten L{\"a}ndern wie Deutschland, Frankreich und Russland durch die Adaption bestehender Produktionsprozesse und einem damit verbundenen hohen Investitionsaufwand die L{\"u}cke zu den weiter entwickelten L{\"a}ndern zu schlie{\ss}en \citep{Gerschenkron.1962}. Dabei h{\"a}ngt der Einfluss von Staatsausgaben auf den Konvergenzprozess entscheidend von dem Entwicklungsstand eines Landes ab. Je weniger weit entwickelt eine Volkswirtschaft ist, desto h{\"o}her ist die Konvergenzgeschwindigkeit durch die Staatsausgaben. In relativ weit entwickelten Volkswirtschaften ist die L{\"u}cke zur Welttechnologiegrenze per se nicht so gro{\ss} und dementsprechend der Aufholprozess relativ langsamer \citep{Ott.2011}. 
Beziehen sich die Staatsausgaben auf die Finanzierung eines {\"o}ffentlichen Bildungssystems, f{\"o}rdert dies den Anpassungsprozess weniger weit entwickelter L{\"a}nder \citep{Glomm.1992}. Volkswirtschaften entscheiden sich f{\"u}r ein {\"o}ffentliches Bildungssystem, deren Bev{\"o}lkerung gr{\"o}{\ss}tenteils unterhalb des durchschnittlichen Einkommens liegt und somit tendenziell weniger weit entwickelt ist.
Untersucht man Humankapital und die unterschiedliche Wirkung {\"o}ffentlich und privat finanzierter Bildungssysteme, zeigt sich, dass zwar in L{\"a}ndern mit einem {\"o}ffentlichen Bildungssystem die Einkommensungleichheit schneller zur{\"u}ck geht, jedoch bei privater Bildung ein h{\"o}heres Pro-Kopf-Einkommen erzielt wird, sofern die anf{\"a}nglichen Einkommensunterschiede nicht erheblich waren \citep{Glomm.1992}.\\
Neben Staatsausgaben wird der Konvergenzprozess zus{\"a}tzlich beschleunigt durch die globale Integration handelsliberalisierter L{\"a}nder. Denn im Vergleich zu geschlossenen Volkswirtschaften  besteht f{\"u}r ge{\"o}ffnete L{\"a}nder dieser Vorteil des R{\"u}ckstands, der zu einem catching up Prozess f{\"u}hren kann. Denn Handel bedingt ein schnelleres Aufholen weniger weit entwickelter L{\"a}nder \citep{Sachs.1995}. Dieser R{\"u}ckstand erkl{\"a}rt auch das starke Wachstum exportorientierter osteurop{\"a}ischer Staaten \citep{Ventura.1997}.\\
Der Technologietransfer zwischen L{\"a}ndern f{\"u}hrt jedoch nur zu einer Konvergenz der Wachstumspfade, sofern sich ein Land im geschlossenen Zustand ebenfalls gem{\"a}{\ss} einer positiven Wachstumsrate entwickelt hat. Ist dies nicht der Fall wird die Volkswirtschaft stagnieren. Dadurch wird verdeutlicht, dass die Offenheit eines Landes keinen wesentlichen Einfluss auf die Konvergenz einer Volkswirtschaft hat \citep{Howitt.2000}. 
Es sei denn, internationaler Handel ist durch Produktivit{\"a}tsunterschiede der Volkswirtschaften bedingt, dann kann dies zu einer einheitlichen weltweiten Einkommensverteilung f{\"u}hren \citep{Howitt.2000,Acemoglu.2002,Eaton.2001}.\\
Anderer Meinung sind \citet{Galor.2006,Galor.2008}. Sie verdeutlichen, dass die Auswirkungen der Au{\ss}enhandels{\"o}ffnung stark den Entwicklungsstand eines Landes beeinflussen kann. Sie f{\"u}hren die Divergenz zwischen den industrialisierten und den nicht-industrialisierten L{\"a}ndern darauf zur{\"u}ck, dass die entsprechenden L{\"a}ndergruppen unterschiedlich mit ihren Handelsgewinnen umgegangen sind und jeweils eine andere Strategie verfolgt haben. So lag der Schwerpunkt der heute weniger weit entwickelten, nicht-industrialisierten, L{\"a}nder in der Drosselung des Bev{\"o}lkerungswachstums. Wohingegen die heutigen industrialisierten L{\"a}nder bestrebt waren den Pro-Kopf-Output zu erh{\"o}hen, indem sie beispielsweise den Bildungssektor f{\"o}rderten \cite{Galor.2006}.\\
Zusammenfassen l{\"a}sst sich festhalten, dass Konvergenz durch den internationalen Wissenstransfer beg{\"u}nstigt wird und eine Handelsoffenheit zu einem catching up Prozess f{\"u}hren kann, jedoch ist dies von der Entwicklungsstrategie einer Volkswirtschaft abh{\"a}ngig. 

