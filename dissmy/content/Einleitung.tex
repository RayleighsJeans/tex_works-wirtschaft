\chapter{Einleitung}\label{Einleitung}
Blickt man auf das vergangene Jahrtausend zurück, so haben drei einschneidende interaktive Ereignisse den Entwicklungsprozess der Welt bestimmt \citep{Maddison.2001}.\footnote{Dabei sind hier vor allem Ereignisse mit ökonomischer Wirkung von Bedeutung somit werden politische Begebenheiten und die damit zusammenhängenden wirtschaftlichen Konsequenzen vernachlässigt.}

\begin{itemize}
	\item Umsiedlung und Landerschließung
	\item Handel
	\item technischer Fortschritt
\end{itemize}

Die Besiedelung und Bewirtschaftung unerschlossener Regionen führte zu einer Erweiterung des Faktors Boden. Als kurze Beispiele dienen China und die Erschließung des amerikanischen Kontinents.  Neue Verfahren im Reisanbau ermöglichten eine Anpassung an die geologischen Rahmenbedingungen und eröffneten neue geographische Anbaumöglichkeiten. Nun konnte auch die Region südlich des Flusses Yangtse bewirtschaftet werden. Daraus folgte, dass sich vom achten~bis zum dreizehnten Jahrhundert Chinas Bevölkerung maßgeblich umsiedelte und sich damit an die neuen Bedingungen anpasste. Der prozentuale Bevölkerungsanteil hat sich südlich des Yangtse mehr als verdoppelt. Ähnlich verhielt es sich mit der Erschließung Amerikas durch die europäische Bevölkerung. Unbekanntes fruchtbares Land sowie neue Ressourcen wurden entdeckt und eingesetzt, so dass die Produktivität anstieg und letztlich ein Einkommenszuwachs verursacht wurde \citep{Maddison.2001}.\\


Das zweite einschlägige Ereignis des letzten Jahrtausends war die Aufnahme von Handel zu anderen Staaten. Dies hat vor allem nach \citet{Maddison.2001} die europäischen Länder und weniger die afrikanischen und asiatischen Länder in ihrer Entwicklung  beeinflusst. Vom Jahr 1000 bis 1500 waren die strategische Lage und das Wissen um den Schiffbau Grund für die Bedeutung Venedigs bezüglich des internationalen maritimen Handels. Es wurden überwiegend Seide und Gewürze mit fernöstlichen Ländern wie China und Syrien gehandelt. Auch schon damals bedingte die Offenheit eines Landes nicht nur die Einfuhr unbekannter Güter, sondern auch den Transfer von Produktionstechnologien und Wissen. Das westliche europäische Handelszentrum war Portugal. Ein weiterer Mitstreiter auf dem Gewürzhandelsmarkt waren die Niederlande, die jedoch zeitlich etwas später erst ab 1500 eine ähnliche Flotte einsetzten. 1700 waren in den Niederlanden nur 40{\%}  der Erwerbsbevölkerung im landwirtschaftlichen Sektor beschäftigt. Der größte Teil des Volkseinkommens wurde durch die Seefahrt und den Dienstleistungssektor erwirtschaftet. Ähnlich verhielt es sich in Spanien, einer weiteren wichtigen maritimen Handelsmacht. Diese Zeit wurde geprägt durch starkes Konkurrenzdenken zu Lasten der Mitstreiter, denn Kooperationen wurden größtenteils vernachlässigt. Ebenfalls der Schifffahrt schlossen sich im 18. Jahrhundert Frankreich und England an. Englands Vorteil gegenüber seinen Mitstreitern lag in einem ausgebauten Netz an Institutionen im Banken- und Finanzsektor sowie staatlicher Einrichtungen. Das Wachstum Großbritanniens war zu dieser Zeit höher, als bei allen anderen europäischen Länder. Unterstützend für die weltweiten Handelsrouten waren die Kolonien, die  die Erschließung von Ressourcen und Rohstoffen erlaubten und Grund für die Überwindung bisher ferner Distanzen lieferten \citep{Maddison.2001}.
Mit der Industrialisierung begann hinsichtlich des Wirtschaftswachstums ein neues Zeitalter. Bedingt durch den technischen Fortschritt wuchs das Pro-Kopf-Einkommen Großbritanniens schneller als jemals zuvor. Es gelang den Engländern ihren physischen Kapitalstock erheblich aufzubauen, sowie die steigende Nachfrage nach qualifizierten Arbeitskräften durch den Ausbau des Bildungssystems zu befriedigen. Außerdem begann das britische Empire Handelsbeschränkungen zu reduzieren, was einen positiven Effekt auf die übrige Welt hatte, da dies auch den Diffusionsprozess von technischem Wissen begünstigte und somit die Industrialisierung in andere Länder trug. Auch die Einführung eines Eigentumsrechte-Systems des Staates steigerte die Attraktivität für Investoren.\footnote{Dieses Eigentumsrechte-System ist mit dem heutigen Patentrecht zu vergleichen.} England war ein wohlhabender Staat, der mit jeder Entwicklung die Welttechnologiegrenze ausweitete. \newline Die beiden Weltkriege zerstörten die Ordnung des freien Handels und das weltweite Wirtschaftswachstum war bis 1950 mehrheitlich deutlich geringer als bis zum Beginn des ersten Weltkrieges 1913. Die Nachkriegszeit, nach dem zweiten Weltkrieg, brachte vor allem in den europäischen Ländern eine Zeit des Aufschwungs mit sich. Das weltweite BIP stieg jährlich um etwa 5{\%} an, der weltweite Handel wuchs um 8{\%} und das Pro-Kopf Einkommen um 3{\%} jährlich.\footnote{Diese Informationen basieren auf den Daten der OECD laut \citet{Maddison.2001}.} Die beiden Weltkriege brachten zudem eine neue politische Ordnung hervor. Der kalte Krieg brach die Verbindung zwischen der westlichen Welt mit dem russisch wohl gesonnenen Osten ab. Internationaler Handel war trotz ausgebauter Transportmöglichkeiten eingeschränkter als Anfang des 20ten Jahrhunderts \citep{Maddison.2001}.\\


Als dritten interaktiven Prozess nach \citet{Maddison.2001} wird erneut auf das letzte Jahrtausend zurückgeblickt, jedoch diesmal unter dem Aspekt der technologischen Entwicklung und der Einbettung von Institutionen. \newline Der technische Fortschritt war zwar von 1000-1820 verglichen zu heutigen Verhältnissen relativ gering, war aber schon damals ein entscheidender Faktor für das Wirtschaftswachstum. Nur durch technische Errungenschaften der Seefahrt wie beispielsweise der Kompass, die Sanduhr und  weitere Entwicklungen der Schifffahrt gelang es den Handel in deutlich weiter entfernte Länder aufzunehmen. Außerdem konnten Neuerungen im landwirtschaftlichen Sektor das immer weiter ansteigende Bevölkerungswachstum kompensieren und ernsthafte Hungersnöte verhindern. Bis zum 15. Jahrhundert wurden viele technologische Neuerungen aus dem asiatischen und arabischen Raum nach Europa transferiert. Trotzdem profitierten letztendlich die europäischen Länder stärker als die Herkunftsländer selbst. Als einer der entscheidenden Unterschiede sieht \citet{Maddison.2001} die angesprochenen Institutionen wie das intakte Finanz-, Versicherungs- und Bankensytem, dessen Vorreiter England war. Auch der Devisenmarkt erleichterte den Händlern der damaligen Zeit ihre Arbeit und minderte ihre Transaktionskosten erheblich.  \newline Der Transfer dieses Systems oder neuer Technologie von Europa aus in die übrige Welt war jedoch relativ gering. Ein funktionierender Wirkungskanal des 18.Jahrhunderts waren die Kolonien Großbritanniens in Nordamerika \citep{Maddison.2001}.


Die Argumentation Maddisons verdeutlicht mögliche Einflussfaktoren auf den Entwicklungsprozess. \citet{Gandolfo.1998} führt ähnliche Gründe für Wachstum an, vernachlässigt jedoch den Einfluss des Handels. Sein Fokus liegt zunächst auf der Faktorakkumulation, bei \citet{Maddison.2001} am Beispiel des Produktionsfaktors Boden, aber auch Migration und somit der Produktionsfaktor Arbeit wäre möglich. Nachdem die Faktorakkumulation lange als Ursprung ökonomischen Wachstums angesehen wurde, hat sich die Wissenschaft einer neuen Richtung gewidmet, die den technologischen Wandel als Kern des Wachstums ansieht. Der Motor des Wachstums der "`Neuen Wachstumsökonomie"' oder auch "`Endogenen Wachstumsökonomie"'wird im technischen Fortschritt gesehen \citep{Gandolfo.1998,Maddison.2001}.\newline Diese Arbeit wird sich vornehmlich mit den zwei Strömungen dieser Richtung beschäftigen und jeweils eine Modellvariation einer offenen Volkswirtschaft vorstellen. \newline Bei dem ersten Modell, das in Kapitel \ref{Papier2} folgt, stehen Wissensexternalitäten bei der Humankapitalakkumulation im Vordergrund, die den technischen Fortschritt begründen. Das Modell basiert auf dem Ansatz von \citet{Lucas.1988}, der neben \citet{Romer.1990} einer der Hauptvertreter dieser Ausrichtung ist. \newline Das zweite Modell in Kapitel \ref{Papier1} fokussierte sich auf private Investitionen im Forschungs- und Entwicklungssektor als Ursache für ökonomisches Wachstum. Angehörige dieser Forschngsrichtung sind beispielsweise \citet{Romer.1990,Grossman.1991c} sowie \citet{Aghion.1992}. Dabei führen Investitionen der Unternehmen zu Innovationen\footnote{Dies ist unabhängig davon ob die Anzahl der verfügbaren Güter gleich bleibt \citep{Aghion.1992} oder ansteigt \citep{Romer.1990}.}, die letztlich den technischen Fortschritt beschreiben. Die hier vorgestellte Modellvariation basiert auf dem Papier von \citet{Acemoglu.2006}, die den Grundgedanken der zuvor genannten Abhandlungen aufgreifen und Aussagen über makroökonomische strategische Einscheidungen zulassen. \newline Der Schwerpunkt beider Modellvariationen liegt in der Einbettung von internationalem Handel in diese Wachstumsmodelle. Außenhandel verbindet Länder und führt deren Reaktionen und Situationen auf dem Weltmarkt zusammen. Diese wechselseitigen Beziehungen gehen sowohl mit Wissensdiffusion und anderweitigen Interaktionen einher. Der Kern dieser Arbeit ist die Überprüfung der folgenden These: Handel führt zu einer Entwicklungsstrategie, die eine innovative bzw. imitative Ausrichtung der Unternehmen anstrebt und ein anhaltendes positives Wachstum bedingt. Dabei spielen politische Entscheidungen und Spillover-Effekte eine Rolle.\\


Werden die Modellvariationen aus \ref{Papier2} und \ref{Papier1} getrennt voneinander betrachtet, dann führt Handel zum einen zu einem besseren Bildungssystem, zum anderen zu einem höheren technischen Entwicklungsstand durch politische Maßnahmen. Kombiniert man beide Modelle (\ref{Papier2} und \ref{Papier1}) miteinander, dann resultiert zunächst ein besseres Bildungssystem, dass dann wiederum die technologische Entwicklung eines Landes begünstigt. \\


Um die Hauptthese zu untersuchen ist die Aufstellung folgender Nebenthese notwendig: Ein relativ weniger weit entwickeltes Lande folgt der Imitationsstrategie, wohingegen ein weiter entwickeltest Land die Innovationsstrategie präferiert. Neben der Tatsache, dass Humankapitalakkumulation zu einem höheren Entwicklungsstand führt kommt außerdem der Zusammensetzung des Humankapitals eine besondere Bedeutung zu. \\


Der Einfluß des Handels soll hier unterstrichen werden und zeigen, dass unabhängig von der Modellvariation ein besseres Bildungssystem resultiert und der Außenhandel die technologische Entwicklung eines Landes begünstigt.
Denn die Erweiterung eines endogenen Wachstumsmodells um Handel zeigt, dass nicht nur der Güterhandel die Entwicklung eines Landes beeinflusst, sondern, dass es auch zu Wissensströmen kommt, die die Wohlfahrt eines Landes erhöhen.
Die Entwicklungspolitik orientiert sich weg von physischen Investitionsprojekten und hin zur Förderung von Bildung. Auch hier wird dieser Ansatz aufgegriffen, indem Außenhandel ein höheres Angebot an Humankapital bedingt, welches anschließend durch exportfördernde Investitionen gezielt eingesetzt wird. \\


Die vorliegende Arbeit prüft vornehmlich in Kapitel \ref{Papier2}, \ref{Papier1} und \ref{Kombi} die aufgestellten Thesen indem in endogene Wachstumsmodelle Handel integriert wird. Kapitel \ref{Papier2} und \ref{Papier1} behandelt die beiden Modellvariationen endogener Wachstumsmodelle, deren Ergebnisse anschließend in Kapitel \ref{Kombi} kombiniert werden. Dafür werden in Kapitel \ref{Wachstum} und \ref{sec:Globalisierung} die theoretischen Grundlagen dargelegt. Die Analyse der in Kapitel \ref{sec:Globalisierung} vorgestellten Handelseffekte werden in den weiterführenden Kapiteln besonders berücksichtigt. Kapitel \ref{Auswertung} wertet die Ergebnisse aus und widmet sich der Belegung bzw. Wiederlegung der hier aufgestellten Thesen.