
\chapter{Wirtschaftliche Globalisierung durch Au{\ss}enhandel}\label{sec:Globalisierung}
 \chaptermark{Au{\ss}enhandel}
Das Regelwerk des freien Marktes, wie wir es heute verstehen, wird vornehmlich auf zwei Ökonomen zurückgeführt: David Ricardo und Robert Malthus. Beide spielten eine entscheidende Rolle in der Hausbildung der Britischen Gesellschaft des 19. Jahrhunderts, die als Vorreiter der Entwicklungsgeschichte der heutigen industrialisierten Volkswirtschaften gilt.\\
Dabei hat vor allem die Theorie des komparativen Vorteils in den letzten 40 Jahren als Marktlogik zu einer immer arbeitsteiligeren globalen Wirtschaft geführt, die einerseits tiefgreifende gesellschaftliche und politische Veränderungen mit dem steigenden Wohlstand vieler Volkswirtschaften mit sich gebracht hat, aber auch ebenso vielen Ländern schadete.\newline


Unter der Annahme, dass der heutige Wettbewerb sich in einem ständigen Spannungsfeld zwischen wirtschaftlichen und politischen Interessen bewegt, soll in diesem Kapitel der Prozess der Globalisierung betrachtet und der Frage nachgegangen werden, ob sich die Entwicklung der letzten 40 Jahre auch tatsächlich auf die Theorie des frühen 19. Jahrhunderts zurückführen lässt oder aber eher durch politische und wirtschaftliche Interessen zu erklären ist? \newline
Legt man hierbei die geschichtliche Entwicklung der unterschiedlichsten Volkswirtschaften zugrunde, wurde vor allem eine Entwicklungsstrategie favorisiert, die vielerlei Anwendung fand, sich dann aber recht spät als suboptimal herausstellten.\\ 
Die vorliegende Arbeit hat den Anspruch, nicht nur einen Eindruck von der Vielschichtigkeit möglicher Strategien zu vermitteln, sondern auch deren Motive und Konsequenzen zu analysieren. Dabei wird von der These ausgegangen, dass keine perfekte und allgemeingültige Entwicklungsstrategie existiert, sondern wirtschaftliche Entwicklungen einerseits zwar stark von den internen Rahmenbedingungen einer Volkswirtschaft abhängen, aber in gewissem Ma{\ss}e auch von externen Einflussgrö{\ss}en, die nicht unmittelbar gesteuert werden können.\\


Für die Entwicklung zu den heute industrialisierten Volkswirtschaften und die weltweite Handelsstruktur spielte die Theorie David Ricardos mit ihrem Ansatz vom komparativen Vorteil eine entscheidende Rolle. Im letzten Jahrhundert fand seine Theorie mehrfach von weniger weit entwickelten Volkswirtschaften als Entwicklungsstrategie Anwendung. So zählt  der 1772 in England geborene Ricardo heute zu den bedeutendsten Vertretern der klassischen englischen National{\"o}konomen. Er war Sohn einer holl{\"a}ndisch-j{\"u}dischen Einwandererfamilie, die zu den wohlhabendsten Familien seiner Zeit z{\"a}hlten. \citet{Lin.2007} schildert weiter, dass er bereits ab seinem 14. Lebensjahr an der B{\"o}rse zusammen mit seinem Vater arbeitete. Die B{\"o}rse entsprach zu der damaligen Zeit eher einem losen Zusammenschluss von Menschen, die sich in Caf{\'e}h{\"a}usern trafen. Der Aktienh{\"a}ndler Ricardo arbeitet schon in jungen Jahren auf eigenes Risiko und war bereits mit 20 Jahren ein erfolgreicher, gestandener und reicher Mann. \newline Im Jahre 1796 reiste er nach Bath, dort las er das erste Mal im Hauptwerk von Adam Smith dem "`Wohlstand der Nationen"' und begann anschlie{\ss}end mit seinen Studien {\"u}ber die Wirtschaftspolitik. Die Ansichten Adam Smiths p\textsl{}r{\"a}gten nicht nur David Ricardo, sondern ver{\"a}nderten die Weltanschauung der folgenden Generationen. Erst Jahre nach seinem Tod wurde der Einfluss und das Ausma{\ss} der G{\"u}te Smiths' Hauptwerkes "`Wohlstand der Nationen"' deutlich \citep{Lin.2007}. Vor diesem Hintergrund ist zunächst eine Auseinandersetzung mit Adam Smith sinnvoll, um dabei den komparativen Vorteil fundiert darstellen zu können.\newline


Adam Smith war einer der ersten gro{\ss}en Wirtschaftsdenker und Begr{\"u}nder der klassischen Schule der National{\"o}konomie. Er beschreibt als erster die Gesetze des Marktes, die zur Stabilit{\"a}t der Gesellschaft beitragen sollten.  Sein liberales Weltbild zeigt sich in diesen Gesetzen des Marktes, die eine selbstregulierende Gesellschaft hervorbringen sollen. Er setzte sich f{\"u}r die Abschaffung der zentralen Instanz ein, die die Wirtschaftsabl{\"a}ufe steuert, demnach ist er einer der ersten Kritiker der Institution "`Staat"'. Der absolutistische Staat soll sich in eine Welt mit eigenverantwortlichen selbstbestimmten Individuen verwandeln \citep{Lin.2007,Huther.2006}.\\


Statt den Handel politisch zu lenken, propagiert Smith den freien Austausch von Waren und Dienstleistungen. Auch innenpolitisch war er der Meinung, dass die Kr{\"a}fte des Wettbewerbs ausreichen, um die Wirtschaft zu steuern und dadurch staatliche Eingriffe nicht mehr notwendig sind. Diese eigenst{\"a}ndigen Mechanismen seien nur funktionsf{\"a}hig, wenn der Staat durch die unsichtbare Hand ersetzt werden w{\"u}rde. \newline Smith ist der Ansicht, dass dem Staat die Aufgaben zufallen, die f{\"u}r die ganze Gesellschaft n{\"u}tzlich sind. Dazu z{\"a}hlen die Bereiche der sozialen Sicherung wie das Rechtswesen und Aufbau bzw. Instandhaltung einer intakten Infrastruktur. Er verlangt beispielsweise auch produktive Staatsausgaben wie in Bildung \citep{Huther.2006}. \newline In seinem Hauptwerk"'An Inquiry into the Nature and Causes of the Wealth of Nations"' von 1776 analysiert er unter anderem die wohlfahrtsmehrenden Effekte von Arbeitsteilung und freien M{\"a}rkten. Arbeitsteilung f{\"u}hrt seiner Ansicht nach zur Erreichung des wichtig\-sten {\"o}konomischen Ziels: Effizientes Arbeiten. Arbeitsteilung bedeutet die Untergliederung der zu verrichtenden Arbeit in kleinere Aufgaben und f{\"u}hrt zu kleinen spezifischen Arbeitsschritten, auf die sich die Arbeiter spezialisieren. Es f{\"o}rdert die Geschicklichkeit der Menschen und ihre F{\"a}higkeiten k{\"o}nnen leichter weiterentwickelt werden, wodurch ihre Arbeitsproduktivit{\"a}t ansteigt. Als Konsequenz kann in der gleichen Zeit mehr produziert werden. Ohne eine starke Arbeitsteilung w{\"a}re die Industrialisierung nicht denkbar gewesen - und sie l{\"o}ste ein vorher nie gekanntes Wirtschaftswachstum aus. Dass Arbeitsteilung und Spezialisierung die Menschen aus einer jahrhundertelangen {\"o}konomischen Stagnation auf einen dauerhaften Wachstumspfad gebracht haben, wurde bereits in \ref{Unified} geschildert.\newline Au{\ss}erdem ist es Smiths Verdienst, dass die {\"O}konomie sich zu einer eigenst{\"a}ndigen wissenschaftlichen Disziplin entwickelt hat \citep{Lin.2007,Huther.2006}. \newline Die Welt zu Smiths Zeiten war wirtschaftlich und politisch im Wandel, denn die einsetzende industrielle Revolution machte die zuvor agrar-dominierte Wirtschaft immer komplexer. Es stellten sich Lohn-, Preis- und Verteilungsfragen und neue Ph{\"a}nomene wie Arbeitsteilung, Massenproduktion und ein wachsender Finanzsektor traten auf \citep{Lin.2007}.
\newline Adam Smith stellte daher eine Verbindung der Wirtschaft mit dem Staat und dem Recht her. Ricardo hingegen erkl{\"a}rt die Wirtschaft durch die Wirtschaft und ermöglichte dadurch eine rein {\"o}konomische Reflektion, indem er die Wirtschaftswissenschaften von den anderen Sozialwissenschaften isolierte. Hierdurch kam es zu einem wichtigen Wendepunkt in der Geschichte der Wirtschaftswissenschaften und David Ricardo wurde zu einem der ersten Globalisierungstheoretikern und einem f{\"u}hrenden Vertreter der klassischen National{\"o}konomie. \newline Anders als Adam Smith hatte Ricardo erkannt, dass die gesellschaftlichen Schichten unterschiedlich vom wirtschaftlichen Wachstum profitieren und dadurch ein Ungleichgewicht entstehen wird. Dies zeigte sich auch kurz darauf in der Umsetzung eines Getreidegesetzes, welches Schutzz{\"o}lle f{\"u}r die Getreideeinfuhr vorsah. Die Durchsetzung dieses Gesetzes wurde vor allem durch die m{\"a}chtigen Gro{\ss}grundbesitzer erm{\"o}glicht, deren Wohlstand erheblich von den Getreidepreisen abhingen und somit dieses Gesetz bef{\"u}rworteten \citep{Kurz.2008}.
\newline Anfang des 19 Jahrhunderts beobachtete Ricardo gewisse gesellschaftliche und politische Vorkommnisse mit Sorge. Dazu zählten, dass die Landflucht zunahm, ebenso wie das Bev{\"o}lkerungswachstum und der immer weiter voranschreitende Prozess der Industrialisierung. Er fragte sich, wie sich die gesellschaftlichen Reicht{\"u}mer langfristig verteilen werden. Nach seinen Schlussfolgerungen k{\"o}nnte der Preis f{\"u}r das Agrarland allen Wohlstand absorbieren und somit w{\"a}ren die Grundbesitzer, ohne staatliche Intervention von au{\ss}en, irgendwann unermesslich reich. Den Grund sah Ricardo darin, dass fruchtbarer Boden durch das anhaltende Bev{\"o}lkerungswachstum zu einem extrem seltenen und kostbaren Gut wird. Die L{\"a}ndereien waren {\"u}berbewertet und konnten die Bev{\"o}lkerung nicht mehr ern{\"a}hren. Die Grundbesitzer k{\"o}nnten ihren Lebensunterhalt allein durch die Verpachtung ihres Grund und Bodens bestreiten. \newline


Mitverantwortlich f{\"u}r den aufkommenden Pessimismus ist ein Freund Ricardos, der oben bereits vorgestellte Thomas Malthus, der im Jahre 1766 in der englischen Grafschaft Surrey s{\"u}dlich von London geboren wurde. Er befasste sich intensiv mit der Problematik des Bev{\"o}lkerungswachstums und teilte gr{\"o}{\ss}tenteils die Ansichten Ricardos .\newline
\citet{Lin.2007} bezeichnet Malthus als den ersten professionellen National{\"o}konomen, der den weltweit ersten Lehrstuhl f{\"u}r Geschichte und politische {\"O}konomie in England inne hatte. Bedingt durch seine pessimistische Einstellung wurde er als der am meisten gehasste Mann seiner Epoche beschrieben. \newline Sein erstes Werk "`An Essay on the Principle of Population as It Affects the Future Improvement of Society"' ver{\"o}ffentlichte er 1798 anonym. Es handelt von dem Bev{\"o}lkerungswachstum und der damit einhergehenden Prognose drohender Hungersn{\"o}te und Verelendung. Er beschreibt den Zusammenhang zwischen dem Bev{\"o}lkerungswachstum und der Nahrungsmittelproduktion.\\

Seiner These nach w{\"a}chst die Bev{\"o}lkerung Englands schneller, als die F{\"a}higkeit gen{\"u}gend Lebensmittel zu produzieren. Dem exponentiellen Bev{\"o}lkerungswachstum zur Folge verdoppelt sich die Menschheit etwa alle 25 Jahre, wohingegen die Lebensmittelproduktion im selbigen Zeitraum nur linear w{\"a}chst. Demnach wird ein Zeitpunkt eintreten, zu dem die Ressourcen der Erde nicht mehr ausreichen w{\"u}rden, um die Bev{\"o}lkerung ausreichend zu ern{\"a}hren. Die damit einhergehenden Probleme wie Krankheit, Elend und Tod erh{\"o}hen die Sterblichkeitsrate und korrigieren damit das Bev{\"o}lkerungswachstum nach unten. Als erster beschreibt er dabei die Bev{\"o}lkerungsfalle und dessen Folgen \citep{Malthus.1798}. \newline Vor seiner Arbeit ging man davon aus, dass in einem Land mit der wachsenden Bev{\"o}lkerung auch eine wachsende wirtschaftliche Leistungsf{\"a}higkeit einhergeht. Laut seiner Bev{\"o}lkerungstheorie kommt es aber zu Verarmung und Verelendung des betrachteten Landes. Au{\ss}erdem hinterfragt er in seiner Arbeit wodurch die Zahl der Menschen begrenzt wird und was letztlich zu dem beobachteten R{\"u}ckgang der Sterblichkeitsrate gef{\"u}hrt hat. Seine politische Besorgnis l{\"a}sst die Angst vor {\"U}berbev{\"o}lkerung und pessimistische Grundausrichtung nachvollziehen, denn Zeit seines Lebens herrschte die franz{\"o}sische Revolution mit der ein Gro{\ss}teil der Probleme einhergingen.\newline Erst Jahre sp{\"a}ter zeigte sich, dass er vor allem den Menschen selbst und seinen Erfindergeist untersch{\"a}tzte. Er war skeptisch hinsichtlich der Geschwindigkeit des technischen Fortschritts, die vor allem in der Landwirtschaft die Produktivit{\"a}t erheblich erh{\"o}hte und damit die Ernten vergr{\"o}{\ss}erte. Diese Aspekte wurden seinerseits vernachl{\"a}ssigt. \newline Hinterfragt man noch einmal die Vorhersagen von Malthus, dann lag er hinsichtlich England mit seiner Analyse sicherlich falsch. Allerdings besteht weiterhin das Problem wachsender Hungersn{\"o}te vor allem in Entwicklungsl{\"a}ndern. Die Nahrungsmittelproduktion {\"u}berholte das Bev{\"o}lkerungswachstum um ein Vielfaches und der Hungertod ist heute seltener als Ende des 18.Jahrhunderts. Die Ursache ist in weniger weit entwickelten L{\"a}ndern jedoch eine andere, als die von Malthus beschriebene, denn hier beruhen Hungersn{\"o}te vornehmlich auf sozialer Ungerechtigkeit und nicht auf dem Unverm{\"o}gen ausreichend Nahrungsmittel zu produzieren \citep{Hesselbein.2000}.


Ricardo hinterfragte zu seinerzeit ebenfalls die Thesen von Thomas Malthus und besch{\"a}ftigte sich mit seinen pessimistischen Auffassungen. Das ansteigende Bev{\"o}lkerungswachstum w{\"u}rde langfristig zu einer Bestellung qualitativ schlechterer B{\"o}den f{\"u}hren und zu einem Anstieg der Nahrungsmittelpreise f{\"u}hren. Er begann, wie Adam Smith, die Gesetzm{\"a}{\ss}igkeiten des Wirtschaftswachstums zu erforschen und erarbeitete eine Reihe konkreterer Vorschl{\"a}ge zur Liberalisierung des Marktes und zur F{\"o}rderung des privaten Unternehmertums \citep{Lin.2007}. Dabei entstand eines der ersten Wirtschaftsmodelle: Die \textbf{Theorie des komparativen Kostenvorteils}.\newline

 Kern dieser Theorie ist, dass jeder das macht, was er am besten kann und jeder von dem Wissen und der Erfahrung des anderen profitiert. In Anbetracht der damaligen politischen Lage und einer unsicheren Zukunft produzierten alle L{\"a}nder aus Angst alles was sie brauchten selbst und erhoben hohe Z{\"o}lle auf ausl{\"a}ndische Waren. In seiner Arbeit suggerierte Ricardo erstmals ein Interesse die M{\"a}rkte f{\"u}r freien Warenaustausch zu {\"o}ffnen \citep{Ricardo.1817}.\newline So bem{\"u}hte er sich auch um die Aufhebung des Getreidegesetzes, welches die lobbyistisch m{\"a}chtige Stellung der Grundbesitzer veranlasst hatte den Getreidepreis k{\"u}nstlich zu regeln. Ricardo ging es im wesentlichen um die Verringerung bzw. Abschaffung der Z{\"o}lle auf Getreide. Dies w{\"u}rde die Dynamik des Wettbewerbs steigern und zu einem geringeren Brotpreise f{\"u}hren. \newline David Ricardo veranschaulichte seine Theorie anhand des Beispiels von Tuch und Wein, die in den beiden L{\"a}ndern England und Portugal hergestellt wurden. Er hinterfragte, warum ein Land beide G{\"u}ter herstellen sollte, wenn sich ein Land auch spezialisieren kann und dann mehr von einem Gut herstellt, welches er gegen das andere Gut eintauschen kann. Er zeigte in seinem Modell, dass Handel in allen beteiligten L{\"a}ndern den Wohlstand erh{\"o}ht, auch wenn ein Land absolute Kostenvorteile aufweist. Der Kern seiner Theorie liegt im komparativen Kostenvorteil. Dieser geht aus den technologischen Gegebenheiten und den damit verbundenen Produktivit{\"a}tsunterschieden beider L{\"a}nder hervor und f{\"u}hrt zur Vorteilhaftigkeit von Freihandel. Die daraus resultierende theoretische Schlussfolgerung war nun, dass die Theorie universal g{\"u}ltig sei f{\"u}r alle L{\"a}nder \citep{Ricardo.1817}.\newline So wurde diese im 21. Jahrhundert von der Welthandelsorganisation (WTO) in die Praxis umgesetzt. Dabei wurde das Prinzip des komparativen Vorteils genutzt, um den Freihandel popul{\"a}r zu machen. Eindeutig war es für die meisten Wirtschaftswissenschaftler aber noch nicht, ob es sich tats{\"a}chlich um einen Motor f{\"u}r Wachstum und Wohlstand handelt. Seine Arbeit {\"u}ber den komparativer Vorteil setzte nämlich die Annahme der Vollbesch{\"a}ftigung voraus und dass alle L{\"a}nder Zugang zu allen Technologien haben, somit kein Technologietransfer stattfindet. \newline Im Nachhinein  l{\"a}sst sich sagen, dass die Theorie sich heute als effizienter erwiesen hat, als sie es in der Vergangenheit jemals war. Damals waren Entfernungen in der Welt wichtig und stellten eine Hemmschwelle f{\"u}r den derzeitigen internationalen Handelsaustausch da. Die fremden L{\"a}nder und m{\"o}gliche Handelspartner waren weit entfernt und der Transport war somit kostspielig und zeitaufwendig. Zweihundert Jahre sp{\"a}ter sind die Transport\-kosten deutlich geringer. Der technologische Fortschritt und eine ausgepr{\"a}gte Infrastruktur erleichtern die {\"U}berwindung von Distanzen. Einschl{\"a}gige Beispiele hierf{\"u}r sind technologische Errungenschaften, wie das Internet und eine verbesserte Verkehrsanbindung und Transportm{\"o}glichkeit durch Containerschiffe \citep{Flassbeck.2010,Rosner.2012}. \newline Problematisch bei der realen Anwendung ist jedoch, dass sein Modell die M{\"o}glichkeit der Arbeitslosigkeit ausschlie{\ss}t. Jeder der einen Arbeitsplatz verliert bekommt einen Arbeitsplatz in der anderen Branche. Im Modell von Tuch und Wein steckt die Annahme dahinter, dass alle Arbeitspl{\"a}tze in der jeweils anderen Branche finden, unabhängig von Qualifikationsvoraussetzungen \citep{Ricardo.1817}.\newline Au{\ss}erdem lehnt er Faktormobilit{\"a}t ab und schlie{\ss}t aus, dass ein Kapitalist mit seiner Technologie nicht in einem anderen Land zu billigeren L{\"o}hnen produzieren kann und in dem urspr{\"u}nglichen Herkunftsland sein Gut nur noch verkauft.\newline Diese Auswirkungen der Modellrestriktionen werden durch das Beispiel von General Motors (GM) verdeutlicht. Der Grund f{\"u}r die Schlie{\ss}ung des Standorts in Lynn von GM in den USA war nicht der Freihandel oder die Theorie Ricardos, sondern nur die Suche nach billigerer Arbeitskraft. \newline Wie aber vereinbarte Ricardo die faktische Suche nach billigen Arbeitskr{\"a}ften mit seinem Anspruch nach Wohlstandsgewinn f{\"u}r alle? \newline Steht das wirtschaftliche Interesse {\"u}ber dem sozialen Interesse, dem Wohlfahrtsgewinn, dann kann bei dem Gedanken vom freien Handel ein wesentlicher Bestandteil sein die Interessen bestimmter Gruppen zu f{\"o}rdern. Dies ist meist der Fall bei gro{\ss}en Unternehmen, die ihre Standorte schnell verlegen k{\"o}nnen, wie es bei GM der Fall war. Der {\"O}ffentlichkeit wird ein bestimmtes Gesellschaftsmodell dargeboten, das mit der Argumentation wissenschaftlicher Erkenntnisse untermauert wird.
 
 
So verlagerte General Motors in Flint, Michigan USA, ab 1978 ihre Produktionsst{\"a}tten nach Mexiko und sp{\"a}ter nach China. Es ist durchaus denkbar, dass der Verlust von ca. 40 000 Arbeitspl{\"a}tzen die Folge vom Freihandel ist und den damit einhergehenden politischen Entscheidungen.\newline


Befasst man sich nur mit der Geschichte des freien Welthandels ungeachtet der Wissenschaft und Ricardos Grundthese des komparativen Vorteils, so gr{\"u}nden sehr fr{\"u}he Handelsbeziehungen auf Zwang durch Waffengewalt. \newline England war im 18. Jahrhundert das Handelszentrum der Welt und in mehrere weltweite Handelskriege verwickelt. Der weltweite freie Markt wurde vom Imperialismus geschaffen, denn ohne Kolonisation h{\"a}tte das britische Empire kaum M{\"a}rkte f{\"u}r seine Produkte schaffen k{\"o}nnen. So kam es letztlich in China Ende des 18. Jahrhunderts zum Opiumkrieg. \newline

 \citet{Straubhaar.2011} sieht das Motiv des Krieges zun{\"a}chst in einer bis ca. 1820 unausgeglichenen bilateralen Handelsbilanz, zugunsten der Chinesen. Die Europ{\"a}er hatten den begehrten chinesischen Exportartikeln, wie Tee und Seide, meist wenig entgegenzusetzen. Die Briten wollten Textilien aus Baumwolle und Wolle auf dem chinesischem Markt verkaufen, scheiterten jedoch mit beidem, weil das Material f{\"u}r die dortigen Verh{\"a}ltnisse zu warm war und es schon eine fortschrittlichere Textilindustrie gab. Um Tee, Rohseide und andere Produkte in China zu kaufen, musste Gro{\ss}britannien gro{\ss}e Mengen Silber ausgeben. Die damit verbundenen Devisenabfl{\"u}sse nach China f{\"u}hrten in Europa zu einer sp{\"u}rbaren Silberverknappung, die wiederum fatale Auswirkungen auf die dortigen Volkswirtschaften hatte. Um dem zu begegnen, gingen die Engl{\"a}nder dazu {\"u}ber, im von ihnen beherrschten Indien mehr Opium produzieren zu lassen. Es sollte eine Nachfrage nach Opium erzeugt werden, um somit Silber als Zahlungsmittel zu umgehen.  Dieses Opium, f{\"u}r das es in China einen sehr aufnahmef{\"a}higen Markt gab, wurde dann mit Unterst{\"u}tzung bestochener Hafen- und Verwaltungsbeh{\"o}rden auf dem chinesischen Markt verkauft. Den andauernden Handel mit China konnte England nur gewaltsam und durch den Verkauf von Opium ausbalancieren. Jetzt kehrten sich der {\dq}Silberfluss{\dq} und die Handelsbilanz zugunsten der Europ{\"a}er um.\newline Hierin begr{\"u}ndete sich eine der ersten dokumentierten {\dq}freien{\dq} Handelsbeziehungen der Geschichte \citep{Straubhaar.2011}.\newline Chinas Kaiser lie{\ss} das Opium verbrennen und vernichten. Er verbot die Einfuhr, sowie den Verkauf und Konsum. Fraglich ist jedoch, ob diese Ma{\ss}nahmen wirtschaftlich motiviert waren oder die massenhafte Opiumsucht, die inzwischen auch die oberen Gesellschaftsschichten ergriffen hatte, den Grund f{\"u}r das Verbot darstellten. \newline Dem Verbot des Kaisers begegnete Gro{\ss}britannien mit dem ersten Opiumkrieg. Queen Victoria setzte mit dem Zwang Chinas zu freiem Handel ein Exempel und verdeutlichte den anderen L{\"a}ndern die Konsequenzen bei Zuwiderhandlungen. Sie wollte verhindern, dass auch die anderen Koloniall{\"a}nder sich ebenfalls weigern Freihandel zu betreiben.  Im Jahre 1840 griffen britische Truppen den Freihandelshafen Kanton an. Es entwickelte sich der fast drei Jahre dauernde Opiumkrieg. In dessen Verlauf besiegten die {\"u}berlegen ausger{\"u}steten britischen Landungstruppen, unter dem Schutz der modernen englischen Kriegsschiffe, die chinesischen Truppen \citep{Straubhaar.2011}.\newline Nach der Niederlage musste China den Opiumhandel wieder zulassen, Hongkong an England abtreten und weitere Handelspunkte {\"o}ffnen. Mit dem Nanjing-Vertrag und anderen ungleichen Vertr{\"a}gen verlor China seine politische Unabh{\"a}ngigkeit. Als Folge eines weiteren Opiumkriegs erzwangen 1844 die USA und Frankreich weitere Vertr{\"a}ge. Mit denen verlor China seine Zollautonomie, also das Recht, Z{\"o}lle zu erheben. China war gezwungen, seinen wirtschaftlichen Protektionismus aufzugeben \citep{Straubhaar.2011}.
Neben China liefert Haiti ein weiteres Beispiel f{\"u}r die Anwendung Ricardos Theorie. Die Entwicklung, die Haiti durchlief, ist in vielerlei Hinsicht bemerkenswert. Alexander \citet{King.2005} beschreibt am Beispiel Haiti den Einfluss der Globalisierung auf die Entwicklung eines Landes. Der Entdeckung im 15. Jahrhundert durch Christoph Columbus folgte die Ausl{\"o}schung der indigenen Bev{\"o}lkerung und die Wiederbev{\"o}lkerung durch die Kolonialm{\"a}chte mit aus Afrika stammenden Sklaven im 17. Jahrhundert. Zu Zeiten der franz{\"o}sischen Kolonialisierung galt Haiti als eines der reichsten L{\"a}nder Lateinamerikas und z{\"a}hlt heute zu den am wenigsten weit entwickelten L{\"a}ndern der Welt \citep{Beck.2008,IBP.2013,Stauber.2014b}.\newline Die Industrialisierung wurde in dem Mutterland Frankreich erheblich unterst{\"u}tzt, jedoch wurde dies gleichzeitig in der Kolonie unterbunden. Dies geschah beispielsweise durch ein Verbot von verarbeitendem Gewerbe in der Kolonie selbst, wodurch die Wirtschaft zus{\"a}tzlich abh{\"a}ngig  von dem~Mutterland wurde und die Instabilit{\"a}t gef{\"o}rdert wurde. Die {\"o}konomischen Potenzen einer Kolonie wurden nur hinsichtlich des Nutzens f{\"u}r die Kolonialm{\"a}chte gef{\"o}rdert, jedoch nicht, um langfristig die Entwicklung Haitis zu unterst{\"u}tzen \citep{King.2005}.\newline


Unmittelbar nach der Liberalisierung 1980 lie{\ss} Haiti Handel mit der {\"u}brigen Welt zu. Zur{\"u}ckzuf{\"u}hren ist dies auf eine Bedingung der WTO, um internationale Anleihen zu erhalten. Dies brachte jedoch schwere Folgen f{\"u}r den landwirtschaftlichen Sektor Haitis mit sich. Ein Land, dass sich zuvor noch selbst versorgen konnte, verzeichnete nun Hungersn{\"o}te in der Bev{\"o}lkerung. Der vorhandene fruchtbare Boden wurde unter der Bev{\"o}lkerung aufgeteilt und die Agrarstruktur bestand nun aus kleinen Parzellen, deren Produktivit{\"a}t deutlich geringer war, als die der Gro{\ss}plantagen. Das Problem der Bodenerosion verst{\"a}rkte diesen Effekt und die Abholzung des beinah gesamten Regenwalds f{\"u}hrte zus{\"a}tzliche zur Desertifikation. Die {\"U}bernutzung des {\"u}brigen fruchtbaren Bodens war die Folge. Dennoch galt das Land als Exporteur von Kaffee, Kakao, H{\"a}uten und Bauholz \citep{King.2005}.\newline Vor Haitis Unabh{\"a}ngigkeit war die~Bev{\"o}lkerung noch f{\"a}hig die eigene Ern{\"a}hrung durch Reisanbau zu sichern. Landesweit f{\"u}hrte der Verlust an landwirtschaftlichen Fl{\"a}chen f{\"u}r den eigenen Verbrauch zu sozialer Destabilisierung des Landes.\newline Auch die Vergabe von Krediten durch den Internationalen W{\"a}hrungsfond (IWF) war an die Bedingung des Freihandels gekoppelt. Diese begr{\"u}ndeten ihr Vorgehen darin, dass offene M{\"a}rkte als Wachstumsfaktor gef{\"o}rdert werden sollten und versuchten damit Ricardos Theorie in die Realit{\"a}t zu {\"u}bertragen \citep{InternationalMonetaryFund.2007,Weiss.2008}.\newline Der komparative Vorteil Haitis lag in den g{\"u}nstigen Arbeitskr{\"a}ften und den nat{\"u}rlichen Umweltbedingungen in der Landwirtschaft. Daraus leitete sich eine Entwicklungsstrategie ab, die den Schwerpunkt Haitis auf exportorientierte Landwirtschaft und Montageindustrie legte. Doch liefert gerade Haiti ein~Negativbeispiel f{\"u}r Ricardos Ansichten. Die Ern{\"a}hrungssicherung in Haiti wurde durch die Verdr{\"a}ngung der Kleinproduktion in den 1980er und 1990er Jahren gef{\"a}hrdet, weil Importe vom subventionierten US-amerikanischen Reis und Zucker den heimischen Markt dominierten. Der Reisanbau lohnte f{\"u}r viele Bauern nicht mehr und sie waren gezwungen ihr Land aufzugeben. Zeitgleich wurden Kaffee- und Mangoplantagen durch Gelder der US-amerikanischen Entwicklungszusammenarbeit gef{\"o}rdert.\footnote{M{\"o}glicherweise liegt in diesem Spezialanbau tats{\"a}chlich ein komparativer Kostenvorteil Haitis.} Doch konnte der steigende Nahrungsmittelbedarf, der durch das Bev{\"o}lkerungswachstum bedingt ist, nicht durch die kleiner werdende Lebensmittelproduktion gedeckt werden. Das Einkommen aus dem Anbau von Kaffee und Mangos ist zu gering, um eine importbasierte Sicherung der Ern{\"a}hrung zu gew{\"a}hrleisten. Die Entwicklungsstrategie sah vor, den Zugang zu lebensnotwendigen G{\"u}tern {\"u}ber den Importmarkt sicher zu stellen, was jedoch unvereinbar mit der Selbstversorgung des Landes war. Da der haitianische Binnenmarkt zu klein erschien, wurde f{\"u}r den internationalen Markt produziert. Der zweite Schwerpunkt, die Montageindustrie, sollte dabei die Kaufkraft f{\"u}r die importierten G{\"u}ter sicherstellen. Trotz erheblicher Steuernachl{\"a}sse, die der Unterst{\"u}tzung der Montageindustrie dienten, waren deren Entwicklungspotenziale beschr{\"a}nkt. Im Jahr 1984 befanden sich 96 Montagebetriebe auf Haiti und erreichten damit ihren H{\"o}hepunkt \citep{King.2005}. \newline Die H{\"a}lfte der Bev{\"o}lkerung ist arm und unterern{\"a}hrt. Belegt ist diese Aussage durch FAO-Angaben von 2010 und durch Daten des ausw{\"a}rtigen Amts aus dem Jahre 2007, die besagen, dass die H{\"a}lfte der Bev{\"o}lkerung mit weniger als 1 US-Dollar pro Tag auskommen muss. Dieser Wert liegt laut WTO unter der Armutsgrenze von 1 US Dollar am Tag. Bei einer Gesamtbevölkerung von 9,4 Millionen Einwohnern entspricht dies 5,5 Millionen Haitanern.  \newline Die UNO sieht die Schuld  am Scheitern der Agrarproduktion bei den Liberalisierungsprogrammen.\footnote{~King (2005) nennt hier bei Haiti ein 1995 beschlossenes Strukturanpassungsprogramm, das neben wettbewerbspolitischen Ma{\ss}nahmen, wie der Privatisierung der neun grö{\ss}ten Staatsbetriebe, unter anderem auch die Verringerung von Importzöllen regelte.} Die Entwicklungsländer sind nicht industrialisierter als zuvor. Wie dieses Beispiel zeigt ist das Gegenteil der Fall: Viele Länder können heute ihr eigenes Volk nicht mehr ernähren.\newline Die Liberalisierung wurde stark durch die USA befürwortet. Seit 1981 verfolgte die amerikanische Politik den Standpunkt den weniger entwickelten Ländern niedrigere Güter wie Nahrungsmittel zu liefern, um sich auf die Industrialisierung zu konzentrieren und den Sprung ins industrielle Zeitalter zu schaffen.  Wie Präsident Clinton in dieser Zeit öffentlich zugab, habe diese Strategie nicht funktioniert und bekannte diese Vorgehensweise als folgenschweren Fehler.\newline


Das Beispiel Haitis zeigt die Probleme auf, die nach Ansicht der Globalisierungskritiker durch die Inanspruchnahme von Krediten des IWF entstehen k{\"o}nnen. \newline Weltweit steigende Grundnahrungsmittelpreise f{\"u}hrten dazu, dass sich die Regierung Haitis 1986 an den IWF wandte, um Kredite aufzunehmen. Die Grundidee des IWF basiert auf der St{\"a}rkung des politischen Friedens und dem weltweiten Wohlergehen. Die Ziele des 1944 gegr{\"u}ndeten Weltw{\"a}hrungsfonds sind unter anderem die F{\"o}rderung der internationalen Zusammenarbeit in der W{\"a}hrungspolitik, die Stabilisierung der internationalen Finanzm{\"a}rkte und die {\"U}berwachung der Geldpolitik.  Ein weiteres und f{\"u}r die vorliegende Arbeit zentrales Ziel, ist eine Analyse der Konsequenzen einer Ausweitung des Welthandels.


 Mit der Unabh{\"a}ngigkeit vieler L{\"a}nder in den  1950er und insbesondere in den 1960iger Jahren, wurde die Notwendigkeit dieser Sonderorganisation deutlich. Das Wachstumspotential der meist weniger weit entwickelten L{\"a}nder konnte nur durch weitere Investitionen ausgesch{\"o}pft werden. Die finanzielle Hilfe gew{\"a}hrte der IWF, jedoch unter strengen Auflagen und Bedingungen, die der Ideologie des freien Marktes folgen. Das Beispiel Haiti zeigt, dass es zur Auspl{\"u}nderung von Rohstoffen durch transnationale Konzerne kommen kann und die sozialen Auswirkungen von Krisen und Hilfsma{\ss}nahmen nicht bedacht wurden \citep{IBP.2013}.\newline
 
 
Ebenfalls unter dem Einflu{\ss} der Kolonialm{\"a}chte stand Ghana. Trotz der bedeutenden wirtschaftlichen Stellung des Landes, aufgrund der Goldvorkommen, z{\"a}hlt auch Ghana zu den {\"a}rmsten L{\"a}ndern der Welt. Im~Jahr 2003 belief sich der Anteil der Bev{\"o}lkerung mit einem Einkommen von weniger als einem US-Dollar pro Tag auf 45 {\%} \citep{Regeher.2013}. In den 80er Jahren wurden Ghana Darlehen zur Schuldenreduzierung der gro{\ss}en Organisationen WTO und IWF gew{\"a}hrt, unter der Auflage eines Strukturanpassungsprogramms. Dieses beinhaltete wieder die {\"O}ffnung des Marktes f{\"u}r ausl{\"a}ndische Investoren und hatte Massenarbeitslosigkeit, eine wachsende Schattenwirtschaft und einen R{\"u}ckgang lokaler landwirtschaftlicher Erzeugnisse zur Folge. Ebenso wie bei Haiti f{\"u}hrte die Wirtschaftsliberalisierung zu Monokulturen und Reisimporten. \newline Auch dieses Beispiel verdeutlicht kritische Anmerkungen am Globalisierungsgedanken. Wird ein Wettbewerb zwischen armen und reichen bzw. zwischen strukturschwachen und -starken L{\"a}ndern zugelassen, dann wird voraussichtlich das weniger weit entwickelte Land den K{\"u}rzeren ziehen. Investiert ein relativ reiches Land Kapital in ein weniger weit entwickeltes Land, dann garantiert dieses Vorgehen noch nicht die gesellschaftliche und politische Entwicklung des weniger weit entwickelten Landes. \newline Diese beiden Beispiele zeigen die negativen Aspekte des Freihandels. Blickt man jedoch auf die vergangenen 50 Jahre Wirtschaftsgeschichte zur{\"u}ck, so gibt es auch zahlreiche positive Beispiele. Die positive Wendung trat im Fall Ghana relativ sp{\"a}t ein, wie der politische Sonderbericht Ghanas zeigt. Im Jahr 2014 sank der Anteil der Bev{\"o}lkerung mit einem Einkommen von weniger als einem US-Dollar pro Tag von 45 {\%} (2003) auf 28,5{\%} und konnte ein Wirtschaftswachstum von 7,43{\%} pro Jahr verzeichnen.\footnote{Die im Jahr 2007 entdeckten Ölvorkommen stellen eine weitere Entwicklungschance für Ghana dar. Jedoch zeigt das bisher tendenziell schleppende Wachstum, dass die Gefahr des "`Ressourcenfluchs"' besteht \citep{Regeher.2013}.}\newline  Im Schnitt liefern L{\"a}nder, die Handel zulassen bessere Wirtschaftsdaten als L{\"a}nder die nicht oder dies nur im beschr{\"a}nkten Masse getan haben.\newline


Ein Musterbeispiel f{\"u}r den Erfolg von Freihandel liefert die Koreanische Halbinsel. Anhand der Entwicklungsprozesse der letzten 60 Jahre lassen sich durch einen Vergleich von Nord- und S{\"u}dkorea R{\"u}ckschl{\"u}sse {\"u}ber die Wirkungsweise politischer Entscheidungen ziehen. \newline Die Grundvoraussetzungen auf der Koreanischen Halbinsel waren die gleichen, wie Rohstoffvorkommen, Kultur, Milit{\"a}r und die wirtschaftlichen Institutionen. Vor dem zweiten Weltkrieg stand Korea unter japanischer Herrschaft und wurde bedingt durch den japanischen Einfluss gegen Ende des 19. Jahrhunderts zur {\"O}ffnung von drei Handelsh{\"a}fen gezwungen \citep{Engelhard.2004,Lee.1999}.\newline 
S{\"u}dkorea ist heute durch seine stete Handelsoffenheit, die nach dem zweiten Weltkrieg ausgedehnt wurde, gut entwickelt, w{\"a}hrend Nordkorea in einem wirtschaftlich desolaten Zustand ist, weil es weitestgehend verschlossen agierte und sich damit weiter isolierte.\newline Die Erfolgsfaktoren und Ereignisse der s{\"u}dlichen Halbinsel werden im folgenden ausf{\"u}hrlicher dargelegt.
Der Entwicklungsprozess S{\"u}dkoreas wurde zun{\"a}chst bis Ende der 80er Jahre strengen Grunds{\"a}tzen folgend von der Regierung gesteuert. Die Wirtschaftsplanung erfolgte flexibel und ideologisch ungebunden, strebte jedoch weiterhin einen exportorientierten Ausbau des Industriesektors an. Dabei war der Staat vor allem kontrollierend t{\"a}tig. Die staatlichen Investitionen wurden wachstums- und exportf{\"o}rdernd eingesetzt und es war dem Staat gestattet in die F{\"u}hrung privater Unternehmen einzugreifen, wie beispielsweise gr{\"o}{\ss}ere Investitionsentscheidungen mitzutragen. Dieser staatlich bestimmte Entwicklungsprozess l{\"a}sst sich nach  \citet{Engelhard.2004} in drei  Phasen gliedern. \newline Die erste Phase umrei{\ss}t den Zeitraum von 1962-1973. Der Schwerpunkt lag in der arbeitsintensiven Exportf{\"o}rderung der Leichtindustrie sowie dem Aufbau einer modernen physischen Infrastruktur. Es gelang S{\"u}dkorea in kurzer Zeit, dass die Textilindustrie 38{\%} des Gesamtexportwerts ausmachte. Der aus dem Wohlstandsgewinn darauf folgende rasche Bev{\"o}lkerungszuwachs wurde durch eine Auflage f{\"u}r die Familienplanung reguliert und der Ausbau der Verkehrsinfrastruktur ebnete die Basis f{\"u}r die folgende wirtschaftliche Entwicklung. \newline So folgte in der Zeit von 1973 bis 1982 der Übergang von der Leichtindustrie zur Schwerindustrie. Trotz starker Handelsorientierung wurden Importz{\"o}lle erlegt, um gro{\ss}e Branchen, wie beispielsweise die Stahlindustrie zu sch{\"u}tzen. Der Staat investierte in dieser Zeit 70{\%} der verf{\"u}gbaren finanziellen Mittel in die schwer- und petrochemische Industrie. Jedoch zeigten sich auch gro{\ss}e Probleme, beispielsweise stiegen die Einkommensunterschiede an. Dies war vor allem darauf zur{\"u}ckzuf{\"u}hren, dass nun eine deutlich gr{\"o}{\ss}ere Nachfrage nach qualifizierten Arbeitskr{\"a}ften herrschte, die einen Lohnanstieg entsprechender Branchen mit sich f{\"u}hrte. Au{\ss}erdem wurde durch den andauernden Import neuer Technologien die Entwicklung eigener Technologien vernachl{\"a}ssigt, was die Wettbewerbsf{\"a}higkeit minderte. \newline Die L{\"o}sung dieser Schwachpunkte bildeten den Beginn der dritten Phase. Man wendete sich von der arbeitsintensiven Produktion ab und konzentrierte sich von 1980-1987 auf die Industrialisierung kapitalintensiver Investitionsg{\"u}ter. Schwerpunkte stellten dabei der Ausbau der Maschinen- und  Automobilindustrie dar. Deren Exporte summierten sich auf 50{\%} der gesamten Exportmenge. Au{\ss}erdem stellte diese Phase auch eine politische Wende dar, da nun ein Gro{\ss}teil der Wirtschaftsprozesse liberalisiert und Handelsbeschr{\"a}nkungen reduziert wurden. Die Kontrollfunktion des Staates wurde zudem herabgesetzt. \newline Auf diese drei Phasen folgte Ende der 80er Jahre der Umschwung hin zur Demokratisierung und der F{\"o}rderung von Hochtechnologiebranchen. Um diese zu erweitern konzentrierten sich staatliche und private Investitionen auf den Forschungs- und Enwicklungssektor. Durch den technologischen Fortschritt musste das Bildungswesen reformiert werden, um eine fortf{\"u}hrende qualitative Ausbildung der Bev{\"o}lkerung zu gew{\"a}hrleisten. Der Beitritt zur WTO und der OECD f{\"u}hrten zu einem andauernden Abbau protektionistischer Handelsbarrieren und der Wettbewerb wurde immer st{\"a}rker den Marktkr{\"a}ften {\"u}berlassen \citep{Engelhard.2004}.\newline Der enorme Aufschwung brachte jedoch nicht nur Nutznie{\ss}er zu Tage, der Verlierer der Entwicklungsstrategie war in erster Linie die Bev{\"o}lkerung. Eine so stark wachstumsorientierte Strategie ging nicht mit sozialer Gerechtigkeit einher. Politische Gegenstr{\"o}mungen wurden unterdr{\"u}ckt \citep{Engelhard.2004}.\newline Auch andere offene asiatische L{\"a}nder wie Taiwan, Japan und China erreichen westliches Produktionsniveau, vor allem weil sie ihre Produktivit{\"a}t verbessert und technologisch aufgeholt haben. Dies ist nicht nur durch massive Investitionen des Westens zu begr{\"u}nden. Korea hat binnen 40 Jahren eine der schnellsten sozio{\"o}konomischen Transformationen in der Geschichte der Menschheit gehabt. Die wirtschaftliche Ver{\"a}nderung des Landes entspricht der Entwicklung Englands von der Kolonialisierung\footnote{Als zeitlicher Rahmen dient hier die Regentschaft von George des Dritten, als die vereinigten Staaten noch britische Kolonie war.} bis heute. Erreicht wurde dieses enorme Wachstum durch den Schutz junger Wirtschaftszweige, wie beispielsweise \citep{Lee.1999}.\newline


Mit Wachstumsraten zwischen 8 und 9 {\%} bis 1995 ist die Entwicklung S{\"u}dkoreas ein beispielhafter Aufholprozess. Es {\"u}bersprang den langwierigen Prozess der technologischen Entwicklung, indem es jegliche Technologien importierte. Die relativ reichlich vorhandene qualitativ hochwertige Arbeit wurde genutzt und beschleunigte die Entwicklung.\newline Der Humankapitalreichtum bef{\"a}higte S{\"u}dkorea sich nur auf rohstoffsparende Technologien zu beschr{\"a}nken und somit ihre eigenen Rohstoffe gezielt einsetzten zu k{\"o}nnen, ohne diese zwingend aus der {\"u}brigen Welt importieren zu m{\"u}ssen. \newline Ein weiterer Erfolgsfaktor war das Intervenieren und Lenken des Staates.  Mit General Park Chung Hee {\"u}bernahm 1961 das Milit{\"a}r die staatliche F{\"u}hrung S{\"u}dkoreas mit einer klar formulierten Entwicklungsstrategie: "`growth first /export first"' \citep{Engelhard.2004}. Der Staat hatte erheblichen Einfluss auf die wirtschaftlichen Prozesse und agierte eher wie ein Unternehmen. Dazu z{\"a}hlten die gezielte Lenkung von Investitionen, die Aufteilung der Branchenstrukturen, Anreizregulierung oder auch die betriebliche Standortwahl der Unternehmen, um nur einige der Ma{\ss}nahmen zu nennen. Diese stark wachstumsorientierte Strategie ging einher mit einer Exportorientierung. Sich dem Au{\ss}enhandel zu {\"o}ffnen, sollte nicht nur das eigene Wirtschaftswachstum beg{\"u}nstigen, sondern war au{\ss}erdem notwendig, um die Vorhaben im eigenen Land zu erm{\"o}glichen. Neben Technologien mussten auch erg{\"a}nzende Rohstoffe f{\"u}r die heimischen Industriezweige importiert werden, au{\ss}erdem war das Potential des inl{\"a}ndischen Binnenmarkt, d.h. nicht genug Käufer bzw. Nachfrager, zu gering um die Kapazitäten vollst{\"a}ndig ausnutzen zu k{\"o}nnen. Man erhoffte sich aus dem durch Handel resultierenden Marktgrö{\ss}eneffekt eine Ausnutzung der vorhandenen Kapazitäten. \newline Die zentrale Rolle des Staates {\"a}u{\ss}erte sich in dem Instrument der Kontrolle. Der Kreditmarkt unterlag strengen Vergabekriterien, sowie auch der Einsatz der genehmigten Gelder streng kontrolliert wurde, damit diese nicht f{\"u}r nicht produktive Absichten eingesetzt wurden. Eine weitere Ma{\ss}nahme war trotz Handelsoffenheit der Schutz bestimmter heimischer Industrien. So wurde beispielsweise die  Automobilindustrie durch Importz{\"o}lle gesch{\"u}tzt \citep{Engelhard.2004}. \newline


Die Beispiele verdeutlichen, das David Ricardos Theorie in der realen Welt vielfach angewandt wurde. Das prinzipielle Konzept, das dahinter stand, funktionierte zwar, jedoch waren die weitreichenden negativen Folgen nicht absehbar. Ricardos Argumente waren durch seine Arbeit zu stark an das theoretische Model gebunden. Er stellte die Welt so dar, als basierte die gesamte Wirtschaft nur auf Handel. Er ber{\"u}cksichtigt weder Schulden, Arbeitslosigkeit noch Geld. Er gilt als Begr{\"u}nder unserer heutigen Mathematisierung der Wirtschaftswissenschaft. Er lieferte Konzepte, die sich geschickt mathematisch umsetzen lassen und zeigen, dass es zu einem Gleichgewicht kommt, auch wenn es in der Realit{\"a}t nicht der Fall ist. Er erkannte, dass die schlechte Anwendbarkeit vor allem auf der Annahme der Vollbesch{\"a}ftigung beruhte. Um sich diesem Aspekt anzun{\"a}hern trat er sehr f{\"u}r den vermeintlichen Segen der Arbeitsfreiz{\"u}gigkeit ein \citep{Huther.2006}. \newline Die Situation zu Lebzeiten Ricardos verdeutlichten ihm den Handlungsbedarf. Die englischen St{\"a}dte des 18. Jahrhunderts waren {\"u}berf{\"u}llt mit notleidenden Bauern. Dabei sollte jeder Mensch vor {\"a}u{\ss}erster Not gesch{\"u}tzt sein, denn das Armengesetz garantierte jedem ein Recht auf Unterst{\"u}tzung durch die Gemeinde. Dazu lieferte David Ricardo die Grundlage f{\"u}r ein national einheitliches System der Unterst{\"u}tzung bed{\"u}rftiger Menschen und wird heute als einer der ersten sozialpolitischen Eingriffe des Staates gesehen. Der dortige fr{\"u}here Zustand m{\"u}sste mit dem heutigen Port-au-Prince der Hauptstadt Haitis vergleichbar sein. Nur, dass es dort kein Wohlfahrtssystem gibt wie in England. Vor allem Thomas Malthus hielt nicht viel von Wohlfahrtssystemen, da es den Menschen die Motivation zum arbeiten nimmt. Die Armengesetze produzierten Armut statt diese zu lindern. Sie erm{\"o}glichten dem Einzelnen, trotz finanzieller Schwierigkeiten und Grundversorgungsproblemen, zu heiraten und Kinder zu bekommen. Finanziell schlechter gestellte erhielten finanzielle Unterst{\"u}tzung gem{\"a}{\ss} der Anzahl ihrer Kinder. Dies war laut Malthus ein~Anreiz mehr Kinder in die Welt zu setzen, als von den Eltern ern{\"a}hrt werden konnten. \citet{Lin.2007} schildert weiter, dass mit Beginn des 18. Jahrhunderts Arbeitsh{\"a}user eingerichtete wurden,  in die die Armen eingewiesen wurden. Dort sollten sie auf ihre Arbeitswilligkeit hin getestet werden und ihre finanziellen Zuwendungen mit Arbeitsleistung ausgleichen. \newline Bis zum Ende des 19. Jahrhunderts konnten die Arbeiter auf der Suche nach einer Besch{\"a}ftigung nicht ohne Weiteres in eine andere Stadt ziehen. Es galt das Herkunftsprinzip, bei dem einem B{\"u}rger nur dann staatliche Unterst{\"u}tzung zustand, wenn die Personen in der Gemeinde geboren, verheiratet oder ausgebildet wurden. Das f{\"u}hrte zu einem sehr unflexiblen Arbeitsmarkt \citep{Wende.2001}.\newline Die Industrialisierung und das Wohlfahrtssystem f{\"u}hrten zu ansteigendem Bev{\"o}lkerungswachstum und der Zunahme der Verst{\"a}dterung. Dadurch entstanden erhebliche Kosten f{\"u}r die Armenunterst{\"u}tzung, die das System ineffektiv machten. Malthus setzte sich gemeinsam mit David Ricardo f{\"u}r den freien Wettbewerb ein. Die Setzung der L{\"o}hne wurde der Kontrolle des Gesetzgebers entzogen. Sie waren der Ansicht, dass die {\"o}ffentliche F{\"u}rsorge den Gesetzen des Marktes schadet \citep{Fischer.1972,Baek.2010}.\newline Durch ihren Einsatz wurde 1834 ein neues Gesetz zum Armenrecht erlassen, darin wurde unter Ber{\"u}cksichtigung der Argumente von Malthus und Ricardo {\"u}ber die verpflichtende Einweisung in Arbeitsh{\"a}user verf{\"u}gt. Der starke Andrang f{\"u}hrte zu deutlich verschlechterten~Lebensbedingungen in den Arbeitsh{\"a}usern. Ziel der Gesetzes{\"a}nderung war die Kostensenkung durch die K{\"u}rzung sozialer Zuwendungen. Jedoch waren die Zust{\"a}nde in den {\"u}berf{\"u}llten Arbeitsh{\"a}usern so schlecht, dass in den Bed{\"u}rftigen die Motivation geweckt wurde, ihren Lebensunterhalt eigenst{\"a}ndig durch Arbeit zu verdienen und sie nicht l{\"a}nger auf das Wohlfahrtssystem angewiesen sein m{\"u}ssen. Die Arbeiter mussten eine Besch{\"a}ftigung finden und das Lohnniveau wurde durch die Kr{\"a}fte des Marktes bestimmt. Nach der gesetzlichen {\"A}nderung konnten sie sich auch wieder frei bewegen, da eine interessante Unterst{\"u}tzung nicht mehr existierte. Beide Wissenschaftler verhalfen der britischen Gesellschaft dazu eine reine kapitalistische Marktwirtschaft zu werden \citep{Wende.2001}.\newline


Die Befreiung der Arbeitskraft f{\"u}hrte jedoch zu weitreichenden Folgen. In Gro{\ss}britannien fand eine Entwicklung weg vom landwirtschaftlichen, hin zum Industriesektor statt. Diesen~Strukturwandel unterzog sich auch China in den vergangenen 30- 40 Jahren und zeigt noch deutlicher welche zus{\"a}tzlichen Konsequenzen dies f{\"u}r den Arbeitsmarkt hatte.  Vor ca. 30 Jahren lebte ein Gro{\ss}teil der Bev{\"o}lkerung auf dem Land und China war weitgehend eine b{\"a}uerliche Gesellschaft. Wenn bei einer {\"u}berwiegend l{\"a}ndlichen, landwirtschaftlich gepr{\"a}gten Bev{\"o}lkerung, Land das Gemeinbesitz war zum Privatbesitz gemacht wird, f{\"u}hrt es langfristig zu einer Struktur von wenigen Gro{\ss}grundbesitzern und wenigen kleinen Landbesitzern. Viele der ehemaligen Bauern besitzen gar kein Land mehr und  sind somit potenzielle Arbeiter f{\"u}r den Industriesektor. Die hinzugewonnenen Arbeiter machten in einem Land wie China mit seiner sehr hohen Bev{\"o}lkerungszahl einen betr{\"a}chtlichen Anteil aus. Dank David Ricardo und Thomas Malthus konnte sich diese Arbeitskraft auf der Suche nach einer Besch{\"a}ftigung frei bewegen. Den Gro{\ss}teil der ehemaligen Bauern f{\"u}hrte ihr Weg vom Land in die St{\"a}dte und konnten ihre Arbeitskraft auf einem globalen kapitalistischem Markt anbieten \citep{Franke.2013,Menzel.2013,Reisach.1997}.\newline


Die zunehmende Verst{\"a}dterung und das gewachsene Potential an Arbeitskr{\"a}ften bot den westlichen Industriestaaten die M{\"o}glichkeit die Produktionsst{\"a}tten in weniger entwickelte L{\"a}nder auszulagern, in denen das Arbeitsangebot hoch und der Lohn somit gering war. Dies geschah auch bei General Motors. Die amerikanischen Arbeiter in Flint wurden arbeitslos, da sie im Wettbewerb mit den chinesischen und mexikanischen Arbeitern nicht mithalten konnten. In den vereinigten Staaten wiederholten sich gewisse Z{\"u}ge der britischen Geschichte. Der Staat Michigan entwarf eine Art Neuauflage des Armutsgesetze ganz im Stil von Malthus.\newline Der Einfluss Ricardos und Malthus ist auch in der heutigen Zeit noch sp{\"u}rbar. Je h{\"o}her die Mindestl{\"o}hne sind, desto besser k{\"o}nnen die Grundbed{\"u}rfnisse befriedigt werden und desto mehr Macht bekommen die Arbeiter. Im globalen Kontext wird dies als gro{\ss}es Problem gesehen. \newline David Ricardo starb am 11.09.1823 im Alter von 51 Jahren. In der {\"O}ffentlichkeit ist der Theoretiker kaum bekannt, dabei hat seine Lehre die globale Wirtschaftsgeschichte nachhaltig beeinflusst. Ricardo und Malthus hatten gro{\ss}en Anteil an einer Umstrukturierung der Gesellschaft entsprechend der Logik des Marktes. Ihre Theorien und Ansichten schufen Reichtum und Armut gleicherma{\ss}en \citep{Heilbroner.2011}.\newline


Die eingangs gestellte Frage nach den Motiven für Handelsbeziehungen lässt sich zusammenfassend als ein Problemlösungsansatz der damaligen Zeit sehen bzw. beantworten. Die angeführten Beispiele zeigen, dass in vielen Fällen die Anwendung der Theorie Ricardos und Malthus auf wirtschaftliche und politische Interessen zurückzuführen sind. Der Kerngedanke zielte jedoch auf die Erhöhung der Wohlfahrt aller beteiligter Länder ab. Ihnen schwebte eine ausgeglichene Gesellschaft mit geringen Standesunterschieden vor, ein noch immer zeitgemä{\ss}es Ideal im andauernden Prozess der Globalisierung.
 

\section{Grundlagen und Handelstheorien}\label{Handelstheorien}
Die Diskussion {\"u}ber den aktuellen Nutzen und die zukünftig m{\"o}glichen Entwicklungspotenziale durch Freihandel wurde im vorherigen Kapitel \ref{sec:Globalisierung} sowohl anhand historischer als auch aktueller Beispiele bereits ausführlich vorgestellt. \newline Dabei konnte festgestellt werden, dass f{\"u}r die {\"O}ffnung eines Landes  verschiedenste Argumente sprechen, die sich zwar unterschiedlicher Analysen bedienen und dabei aber die Motive, Blickwinkel und Intensionen der jeweiligen Betrachter ber{\"u}cksichtigen. In diesem Zusammenhang stellte sich aber die Frage nach einem richtigen Ma{\ss} f{\"u}r die jeweilige Ausprägung vonFreihandel bzw. Protektionismus. Ab wann {\"u}berwiegen die Nachteile bzw.  bis wann kann der Nutzen diese aufw{\"a}gen? Globalisierungsbef{\"u}rworter gewichten eine Handelsliberalisierung st{\"a}rker als beispielsweise Politiker, die einerseits innenpolitische Probleme l{\"o}sen m{\"u}ssen, andererseits die Interessen derer Vertretern, die ihnen zu einer Wiederwahl verhelfen.\newline Zun{\"a}chst wird auf der Ebene der Wohlfahrtsanalyse das Effizienzargument f{\"u}r Freihandel angef{\"u}hrt, weil der durch den Au{\ss}enhandel entstandene Wohlfahrtsanstieg durch Protektionismus gemindert werden w{\"u}rde. Demnach wäre es effizient auf Eingriffe zu verzichten und den Marktkr{\"a}ften zu vertrauen.
Handelt es sich jedoch um ein {\"o}konomisch gro{\ss}es Land, dann kann theoretisch die Wohlfahrt dar{\"u}ber hinaus durch protektionistische Ma{\ss}nahmen gesteigert werden. Dies besagt z.B. das Terms of Trade Argument und zeigt, dass dies bei einem Optimalzoll zwar zutrifft, in der Realit{\"a}t aber selten Anwendung findet \citep{Ventura.1997,  Acemoglu.2002}.\footnote{Dies ist zum Einen dadurch bedingt, dass er nur dann wohlfahrtssteigernd wirkt, wenn sich die übrige Welt nicht widersetzt und ebenfalls den Handel beschränkt. Zum anderen mangelt es häufig an der politischen Durchsetzbarkeit.}\newline Die Intention des Staates den Handel einzuschr{\"a}nken kann auch dadurch bedingt sein ein bestehendes inl{\"a}ndische Marktversagen ausgleichen zu wollen. Dies ist meist dann der Fall, wenn ein zus{\"a}tzlicher nicht erfasster Nutzen, der aus der heimischen Produktion hervorgeht, den gesellschaftlichen Gesamtnutzen steigert \citep[Kapitel 10]{Krugman.2015}.\newline


Neben zus{\"a}tzlicher Wohlfahrt kann es durch Freihandel noch zu weiteren Gewinnen durch die Sondierung produktiver und weniger produktiver Unternehmen durch den bereits angesprochenen Wettbewerbseffekt kommen. Der erh{\"o}hte Wettbewerb setzt Anreize innovativ t{\"a}tig zu sein und verdr{\"a}ngt weniger produktive Unternehmen vom Markt, so dass lang\-fri\-stig die volkswirtschaftliche Produktivit{\"a}t steigt. \newline


Der Wettbewerbseffekt st{\"a}rkt zwar die davon profitierenden gr{\"o}{\ss}eren Unternehmen, jedoch wird dieser Effekt auch h{\"a}ufig als Argument gegen die {\"O}ffnung eines Marktes verwendet. Da durch Au{\ss}enhandel die weniger effiziente Unternehmen vom Markt verdr{\"a}ngt werden, bef{\"u}rchten Unternehmen aus technologisch weniger weit entwickelten L{\"a}ndern, nicht zu unrecht dem erh{\"o}hten Wettbewerbsdruck nicht standhalten zu k{\"o}nnen. Die Vielfalt an klein- und mittelst{\"a}ndischen~Unternehmen sinkt. Jedoch w{\"u}rden nicht nur einzelne Unternehmen unten den Konsequenzen leiden, sondern ganze Branchen eines Landes k{\"o}nnten betroffen sein.\newline


Das dritte Argument f{\"u}r Freihandel betrifft die politische Durchsetzbarkeit. So scheitern  Handelshemmnisse selbst dann schon, wenn  es politisch durchaus sinnvoll ist den Handel einzuschr{\"a}nken. Letztlich sichert Freihandel dem Politiker die Wiederwahl und ist h{\"a}ufig der Weg des geringsten Wiederstandes. \newline 


Somit bedingt die Gunst des Freihandels bei den potentiellen W{\"a}hlern die politische Durchsetzbarkeit. Dabei tritt das Problem im Rahmen der politischen {\"O}konomie auf, denn häufig werden die Anliegen m{\"a}chtiger Interessensgruppen eher vertreten, als die dem Gemeinwohl dienlichen. Auch werden tendenziell in technologisch relativ weiter entwickelten L{\"a}ndern eher Bedenken bez{\"u}glich einer {\"O}ffnung angef{\"u}hrt, hinsichtlich m{\"o}glicher Einkommensdefizite. Wird beispielsweise Handel mit arbeitskr{\"a}fteintensiven G{\"u}tern aus weniger weit entwickelten Volkswirtschaften betrieben, kann dies zu einer Anpassung des Lohnniveaus und letztlich zu einem geringeren Lebensstandard f{\"u}hren. Diese Option weckt das Begehren nach protektionistischen Ma{\ss}nahmen um diese Einkommensanpassung zu mindern, bzw. zu verhindern \citep[Kapitel 1]{Krugman.2015}.\newline


Auch wenn durch Au{\ss}enhandel auf gesamtwirtschaftlicher Ebene die Wohlfahrt ansteigt, f{\"u}hrt er innenpolitische Probleme herbei. Dazu z{\"a}hlt auch das Verteilungsproblem des Einkommens, weil nicht jede Gruppe gleicherma{\ss}en beg{\"u}nstigt bzw. einige sogar benachteiligt werden. Die Einkommensschwerpunkte verlagern sich beispielsweise von den Arbeitnehmern zu den Kapitaleignern. In diesem Fall können durch staatliche Regulierung Wohlfahrtsgewinne zugunsten schlechter gestellter Bev{\"o}lkerungsgruppen umverteilt werden \citep{Dixit.1980}. Au{\ss}erdem k{\"o}nnen importkonkurrierende Branchen, in denen spezifische Faktoren eingesetzt werden, unter Au{\ss}enhandel leiden, da es nur sehr schlecht bis gar nicht m{\"o}glich ist diese Faktoren in anderen Bereichen einzusetzen.


Die Argumente basieren auf theoretischen Modellen und empirischen {\"U}berpr{\"u}fungen \citep[Kapitel 1]{Krugman.2015}.

\subsection{Au{\ss}enwirtschaftstheorien}
Grundsätzlich lässt sich in Bezug auf die derzeit bekannten Au{\ss}enwirtschaftstheorien feststellen, dass sie sich überwiegend mit wirtschaftlichen Interaktionen zwischen den Volkswirtschaften befassen.
Entsprechend den verschiedenen Erklärungsansätzen nach denen die Gründe, warum Länder miteinander Handel betreiben, recht unterschiedlich sind, werden im folgenden diese möglichen Gründe vorgestellt.


Dabei liegt der Schwerpunkt weniger auf intertemporalen Entscheidungen, da davon ausgegangen wird, dass alle Wirtschaftsteilnehmer zu jedem Zeitpunkt alles haben k{\"o}nnen. Gerade hinsichtlich der Koordination von Produktionsprozessen ist eine intertemporale Optimierung nicht notwendig, da die G{\"u}ter und auch die Produktionsfaktoren jederzeit aus der {\"u}brigen Welt bezogen werden k{\"o}nnen.


\subsubsection{Ricardo - Technologieunterschiede}
Die Darlegung der Beweggründe für ökonomischen Handel liefert einen kurzen {\"U}berblick {\"u}ber die Hauptstr{\"o}mungen der Handelstheorien, denen die Leitfrage aller traditionellen Handelstheorien zugrunde liegt: Welches Land exportiert welches Gut? \newline 


Die klassische Theorie des Au{\ss}enhandels wurde vor allem durch David Ricardos Arbeit von 1817 gepr{\"a}gt. Seine Idee basiert auf dem gleichen Konzept, dass Robert \citet{Torrens.1815} in seinem Aufsatz {\"u}ber den Getreidehandel verfasste. Dabei liegt der hier angef{\"u}hrte Grund f{\"u}r Au{\ss}enhandel in der Verschiedenheit der Technologien und den damit verbundenen Produktivit{\"a}tsunterschieden. Das Ursprungsmodell beschreibt den Handel zwischen den beiden L{\"a}ndern Portugal und England mit den G{\"u}tern Wein und Tuch. Produziert werden beide G{\"u}ter nur mit dem Einsatzfaktor Arbeit.  Allerdings unterscheiden sich die jeweils notwendigen Einsatzmengen f{\"u}r die Produktion eines Gutes, bedingt durch den Einsatz unterschiedlicher Produktionstechnologien. Somit f{\"u}hren die Produktivit{\"a}tsunterschiede zwischen den L{\"a}ndern zu unterschiedlichen Produktionskosten. Dieser komparative Kostenvorteil beschreibt den relativen Vorteil eines Landes, der durch den Einsatz verschiedener Technologien zu Stande kommt und stellt hier den Grund f{\"u}r Au{\ss}enhandel dar. Dabei stellen sich die teilnehmenden Wirtschaftssubjekte durch die Aufnahme von Au{\ss}enhandel besser, weil jedes Land immer einen komparativen Vorteil in irgendeinem Sektor hat \citep{Ricardo.1817}. \newline Ricardo widersprach damit den Annahmen Adam Smiths, dass absolute Vorteile einer {\"O}konomie zwingend notwendig sind, damit absolute Arbeitsteilung, also Handel im weiteren Sinne, f{\"u}r beide Seiten sinnvoll ist.\newline


Kann ein Land in allen Branchen effizienter produzieren, dann geht dies nicht zwangsl{\"a}ufig mit einer kosteng{\"u}nstigeren Produktion einher, denn vergleicht man die Opportunit{\"a}tskosten der beteiligten L{\"a}nder in den entsprechenden Branchen, dann zeichnet sich allein schon dadurch der absolute vom komparativen Vorteil ab. So können weniger effiziente L{\"a}nder schon durch niedrigere L{\"o}hne ihre Konkurrenzf{\"a}higkeit erhalten und zu geringeren Opportunit{\"a}tskosten produzieren. Somit ist ihr komparativer Vorteil dann durch den produktiveren Einsatz des Faktors Arbeit bedingt, also durch g{\"u}nstige Arbeitskraft. Dieser Zusammenhang beschreibt den Unterschied zwischen dem absoluten und dem komparativen Vorteil \citep{Ricardo.1817}.\newline


Auf Ricardos grundlegende Arbeit "`On the Principles of Political Economy and Taxation{\dq} von  (1817) st{\"u}tzen sich eine Vielzahl von empirischen Untersuchungen und Modellvariationen, von denen hier nur einige wenige vorgestellt werden.\\
Bei der Variation des Modells des komparativen Vorteils von \citet{Dornbusch.1977} handelt sich um eine vereinfachte Version des Ricardo Modells. Jedoch werden nicht nur zwei G{\"u}ter produziert und gehandelt, sondern sehr viele G{\"u}ter, sodass sich ein Kontinuum an handelbaren G{\"u}tern ergibt. Dies f{\"u}hrt Ricardos These mit der realen Welt ein wenig n{\"a}her zusammen.\\


So ist ein Vergleich der Produktivit{\"a}ten der USA mit denen von Gro{\ss}britannien Gegenstand vieler empirischer Untersuchungen, in denen die Theorie Ricardos dahingehend best{\"a}tigt wurde, dass die theoretischen komparativen Vorteile mit den tats{\"a}chlichen {\"u}bereinstimmen \citep{MacDougall.1951,MacDougall.1952,Stern.1962,Balassa.1963}.\\
Ebenfalls empirisch ist die Arbeit von \citet{Golub.2000}. Sie untersuchen den Zusammenhang zwischen den Verh{\"a}ltnissen der relativen Produktivit{\"a}ten und bilateralen Handelsstrukturen der USA. Dabei stellen sie fest, dass die Struktur nicht komplett durch den komparativen Vorteil erkl{\"a}rt werden kann, aber diese dennoch in Teilen erklärt.\\
Beschr{\"a}nkt man die Betrachtung des Handels ausschlie{\ss}lich auf Industrieprodukte, dann liegt der Grund f{\"u}r Handel mit diesen in der technologischen Ausstattung der L{\"a}nder bzw. dem technischen Entwicklungsstand eines Landes. Empirische Beobachtungen, die Aufschluss {\"u}ber die Handelsstruktur geben, bestätigen ebenfalls Ricardos Aussagen \citet{Dosi.1988}.\newline


Die Hauptaussage der Theorie, dass jedes Land bei der Produktion eines Gutes einen komparativen Vorteil hat, klingt gerade f{\"u}r weniger weit entwickelte L{\"a}nder vielversprechend. Auch ein Vergleich, mit beispielsweise den USA, betont die relativ schlechte Situation dieser L{\"a}nder, aufgrund fehlender absoluter Vorteile. Jedoch ändert sich dieses Bild sobald die komparativen Vorteile hinzugezogenen werden. Diese können auf unterschiedliche Argumente zur{\"u}ckgef{\"u}hrt werden. Dazu z{\"a}hlen Faktoren wie das Klima, nat{\"u}rliche Ressourcen, besondere akkumulierte F{\"a}higkeiten, {\"U}berschussangebote an g{\"u}nstigen Arbeitskr{\"a}ften oder auch gezielt hervorgerufene komparative Vorteile durch staatliche F{\"o}rderungsma{\ss}nahmen eines bestimmten Sektors. Den komparativen Vorteil k{\"o}nnen entweder Faktoren bedingen, die relativ fest und {\"u}ber die Zeit unver{\"a}nderlich sind, oder auch andere Faktoren, die sich erst noch {\"u}ber die Zeit entwickeln werden.\newline 


Diesen Aspekt greift auch \citet{Helpman.2011} auf. Demnach ist es einzelnen Unternehmen m{\"o}glich einen komparativen Vorteil f{\"u}r ein Land zu generieren. Dies zeigt, dass mikro{\"o}konomische Entscheidungen betr{\"a}chtlichen Einfluss auf das makro{\"o}konomische Gleichgewicht haben k{\"o}nnen. In dem ausf{\"u}hrlich dargelegten Modell in Kapitel \ref{Papier1} wird ein {\"a}hnlicher Ansatz verfolgt. Die technologischen Entwicklungen einzelner Unternehmen erh{\"o}hen nicht nur die Produktivit{\"a}t eines Landes, sondern im offenen Modell sogar die der {\"u}brigen Welt.\\
Kritiker Ricardos bezeichnen seine Theorie zwar als {\"u}berholt, da er die Produktionsfaktormobilit{\"a}t und den~Technologietransfer nicht ber{\"u}cksichtigt \citep{Irwin.2009}. Allerdings wird in der herrschenden Meinung die Ansicht vertreten, dass seine Hauptaussagen auch heute immer noch aktuell sind.


\subsubsection{Heckscher-Ohlin - Ausstattungsunterschiede}
Ein weiteres Modell geht davon aus, dass Handel auch dann vorteilhaft ist, wenn verschiedene Länder zwar die gleiche Technologie verwenden, sich aber in ihrer Ausstattung mit Produktionsfaktoren unterscheiden. Die Vertreter dieser neoklassischen Theorie des Au{\ss}enhandels sind Eli Filip Heckscher und Bertil Ohlin, die Begr{\"u}nder des nach ihnen benannten Heckscher-Ohlin-Modells. Ausgehend von technologisch {\"a}hnlichen oder gleichen L{\"a}ndern, stellten sie einen komparativen Preisvorteil bei Volkswirtschaften fest. Dabei führt die Aufnahme von Freihandel zu einer Spezialisierung des gesamtwirtschaftlichen Produktionsvolumens, hin zu einem Gut. Genau zu dem Gut, bei dem der bei der Produktion relativ reichlicher vorhandene Produktionsfaktor intensiver genutzt wird. Dieses Faktorproportionentheorem ist der Kern des Heckscher-Ohlin-Modells und veranschaulicht welche Handelsstruktur sich bilden wird.\\
Das Heckscher-Ohlin-Modell wurde erstmals von \citet{Jones.1965} algebraisch formuliert und liefert damit den Ausgangspunkt zahlreicher Modellvarianten \citep{Davis.2001,Trefler.1993,Deardorff.1984,Jones.1984}\footnote{Liefert einen allgemein Überblick über die Au{\ss}enhandelstheorien, sowie das Faktorproportionentheorem im "`Handbook of International Economics"'.}.\\


\citet{Leontief.1953} besch{\"a}ftigte sich als einer der Ersten mit der empirischen {\"U}berpr{\"u}fung des Heckscher-Ohlin-Modells. Er zeigte am Beispiel der USA, das relativ reichlicher mit Kapital ausgestattet ist, dass dort nicht die Handelsstruktur besteht, die das Faktorproportionentheorem vorhersagt.  Die Handelsstr{\"o}me der USA sind {\"u}berwiegend durch relativ arbeitsintensive Exporte und kapitalintensive Importe gepr{\"a}gt \citep{Leontief.1953}. Diese Ergebnissen widerlegten schlie{\ss}lich die Theorie von Heckscher und Ohlin und wurde als das Leontief Paradoxon bekannt. Für weitere industrialisierte L{\"a}nder konnten {\"a}hnliche Ergebnisse belegt werden \citep{Gruber.1970,Maskus.1985}.\\


\citet{Trefler.1993} widerspricht dem Leontief Paradoxon und zeigt anhand einer modifizierten Variante des Heckscher-Ohlin-Modells, dass dieses best{\"a}tigt werden kann, sofern Produktivit{\"a}tsunterschiede zwischen den beteiligten L{\"a}ndern zugelassen werden. Ebenso widerlegt auch \citet{Leamer.1980} Leontiefs Untersuchungen, indem er einen Test anwendete, der auf dem Vergleich der Faktorintensit{\"a}ten der produzierten und konsumierten G{\"u}ter gr{\"u}ndet. Die Allgemeingültigkeit wurde jedoch nicht belegt, da das Leontief-Paradoxon nur in bestimmten Jahren Anwendung fand \citep{Stern.1981}.\\


Auch \citet{Davis.1995} besch{\"a}ftigen sich mit der Anwendbarkeit der Theorie. Sie vertreten die Meinung, dass trotz fehlender empirischer Best{\"a}tigung der Theorie von Heckscher und Ohlin der Kerngedanke und das Ergebnis des Modells anwendbar ist. {\"A}hnlich wie \citet{Trefler.1993} modifizieren sie es, indem sie die Grundannahmen anpassen und erhalten f{\"u}r die Daten Japans die Theorie st{\"u}tzende Ergebnisse. Dabei sehen sie zum einen von der Annahme ab, dass die Technologien f{\"u}r die betrachteten L{\"a}nder gleich sein sollten und sich somit nicht ausschlie{\ss}lich durch ihre Ausstattung unterscheiden. Zum anderen analysieren sie die Produktions- und Konsumstruktur separat, ohne die direkten Handelsdaten zu nutzen.  Beides zusammen f{\"u}hrt dazu, dass sie das Heckscher-Ohlin-Modell empirisch f{\"u}r Japan best{\"a}tigen k{\"o}nnen.\\


Einen anderen Ansatz w{\"a}hlen \citet{Bond.}, die eine dynamische Version des Heckscher-Ohlin Modells graphisch l{\"o}sen und stellen dabei neben der Existenz, die Dynamik und Stabilit{\"a}t m{\"o}glicher Gleichgewichte dar.\\
Es ist auch durchaus üblich verschiedene Ansätze miteinander zu kombinieren, wodurch der Handel zwischen L{\"a}ndern mit {\"a}hnlicher Ressourcenausstattung erklärt werden kann. Daf{\"u}r wurde die Idee des komparativen Vorteils Ricardos in das Heckscher-Ohlin Modell implementiert. Bei {\"a}hnlichen Faktoreinsatzverh{\"a}ltnissen in {\"a}hnlichen L{\"a}ndern ist der technische Unterschied der L{\"a}nder von Bedeutung und bestimmt die Handelsstruktur \citep{Davis.1995b}.\\
Eine Kombination mit dem Ansatz der Neuen Handelstheorien best{\"a}tigt die Faktorproportionentheorie, sowie das Rybczynski Theorem weitestgehend, vor allem jedoch f{\"u}r humankapitalreiche L{\"a}nder. Diese Erweiterung des Heckscher-Ohlin-Modells nahm \citet{Romalis.2004} vor, indem er es um Transportkosten und den Ansatz der monopolistischen Konkurrenz nach \citet{Krugman.1980} erweiterte.\\


\textit{Faktorpreisausgleichstheorem}\\


In einem engen Zusammenhang mit dem Faktorpropotionentheorem bzw. Heckscher-Ohlin-Theorem steht das Faktorpreisausgleichstheorem bzw. Stolper-Samuelson-Theorem. Nachdem zun{\"a}chst die Reaktionen auf den G{\"u}term{\"a}rkten betrachtet wurden, werden hier die sich ergebenden Konsequenzen auf den Faktorm{\"a}rkten dargelegt. Das Faktorpreisausgleichstheorem geht auf die Arbeit von \citet{Samuelson.1941} zur{\"u}ck, in der sie die Wirkung durch die Aufnahme von Handel auf die Faktorpreise zeigen. Dabei greifen sie die Idee ihres Kollegen \citet{Ohlin.1933} auf, der ebenso wie \citet{Heckscher.1919}, den Zusammenhang zwischen der Handelsstruktur und der Resourccenausstattung eines Landes thematisiert. Das daraus resultierende Heckscher-Ohlin-Theorem besagt, dass ein Land stets das Gut exportieren wird, das den relativ reichlicher vorhandenen Produktionsfaktor intensiver bei der Herstellung verwendet. \newline Samuelsons weiterführenden {\"U}berlegungen basieren auf den beiden Regionen USA und Europa, die sich seinerzeit hinsichtlich ihrer Bev{\"o}lkerungsdichte und dem verf{\"u}gbaren fruchtbaren Boden deutlich unterschieden. Demzufolge werden durch Handel die relativ hohen L{\"o}hne im eher d{\"u}nn besiedelten Amerika sinken und der Bodenpreis in Europa  wird ansteigen. Somit werden sich langfristig die Faktorpreise auf dem Weltmarkt angleichen. Es ist dann in der theoretischen Welt nicht mehr kosteng{\"u}nstiger Produktionsfaktoren zu im- oder exportieren, um diese dann weiter zu verarbeiten, wenn durch den Preisausgleich ein direkter G{\"u}teraustausch zum gleichen Ergebnis f{\"u}hrt \citep{Samuelson.1948}.\\
Ein weiteres Papier von \citet{Samuelson.1949} kn{\"u}pft an seine vorherige Arbeit an und besch{\"a}ftigt sich wieder mit dem Faktorpreisausgleichstheorem. Auch hier formuliert er die Gedanken Ohlins formal und best{\"a}tigt erneut das Stolper-Samuelson-Theorem.\\


Dem Zusammenhang zwischen der Faktormobilit{\"a}t und Handel widmet sich \citet{Mundell.1957} in seiner theoretischen Arbeit. Dabei geht er zun{\"a}chst von immobilen Produktionsfaktoren aus und zeigt, dass mit der Zunahme protektionistischer handelseinschr{\"a}nkender Ma{\ss}nahmen die Motivation zur Mobilit{\"a}t der Faktoren ansteigt. Weiterhin kommt er zu dem umgekehrten Ergebnis, dass mit der Einschr{\"a}nkung der Faktormobilit{\"a}t der Handel mit G{\"u}tern zunimmt. Somit best{\"a}tigt auch er, dass Faktormobilit{\"a}t und G{\"u}termobilit{\"a}t substituierbar sind \citep{Mundell.1957}.\\


Sobald jedoch ein Modell von der Grundannahme, die der gleichen bzw. ähnlichen Technologien, abweicht, werden sich die Faktorpreise nicht mehr vollst{\"a}ndig angleichen \citep{Jones.1970,Davis.2001}.\footnote{Interessant ist hier vor allem der Aspekt dass in empirischen {\"U}berpr{\"u}fung verschiedener Handelstheorien festgestellt wurde, dass vollkommene Spezialisierung, tendenziell realistischer ist, bzw. h{\"a}ufiger vorkommt, als Autarkie oder der hier thematisierte Faktorpreisausgleich \citep{Cunat.2001}.}\\


Nur indirekt mit dem technischen Entwicklungsstand beschäftigt sich \citet{Trefler.1993}. Er widerspricht zunächst dem Leontief-Paradoxon und zeigt dann anhand einer modifizierten Variante des Heckscher-Ohlin-Modells, dass dieses best{\"a}tigt werden kann, sofern Produktivit{\"a}tsunterschiede zwischen den beteiligten L{\"a}ndern zugelassen werden. Dabei handelt es sich um einen bedingten Faktorpreisausgleich. In dem ursprünglichen Stolper-Samuelson-Theorem gleichen sich die Faktorpreise, wie der Lohn $w$ an. Bei Treflers bedingter Variante steht der Lohn jedoch im Verh{\"a}ltnis zum technologischen Wissen\footnote{Das technische Wissen ist hier durch den Parameter $A$ gekennzeichnet.}, somit gleicht sich nur das Verh{\"a}ltnis $w/A$ beider an.\bigskip\\


F{\"u}r beide aufeinander aufbauenden Theorien gilt: Spezialisierung und Handel lohnen sich umso mehr, je verschiedener die Handelspartner sind. Der interindustrielle Handel, erkl{\"a}rt durch das Heckscher-Ohlin oder Ricardo Modell, nimmt zu, je unterschiedlicher sich die L{\"a}nder hinsichtlich ihrer Ausstattung sind. Wohingegen intraindustrieller Handel auf Skaleneffekte bei monopolistischer Konkurrenz zur{\"u}ckzuf{\"u}hren ist. Dabei sind die Handelsbeziehungen umso intensiver je {\"a}hnlicher die L{\"a}nder sich einander sind \citep{Dosi.1993}. Dieser Erklärungsansatz wird im Rahmen der Neuen Handelstheorien behandelt.


\subsubsection{Krugman - interne Skalenerträge}
Eine weitere Handelstheorie basiert auf dem Ansatz der internen Skalenerträge von Paul \citet{Krugman.79}. Die Arbeit von Robert \citet{Solow.1956} hatte indirekten Einfluss auf seine Au{\ss}enhandelstheorien. In Solows Theorie {\"u}ber unvollst{\"a}ndigen Wettbewerb wurde eine sehr realistische Welt dargestellt, in der Unternehmen durch steigende Skalenertr{\"a}ge Gewinne erwirtschaften k{\"o}nnen. Denn auf eine gro{\ss}e Produktionsmenge können die fixen Kosten stärker umgelegt werden. 
Der Grundgedanke der Gr{\"o}{\ss}envorteile, die internen Skalenertr{\"a}ge, geht auf die Ideen Ricardo und Smith zur{\"u}ck. Danach f{\"u}hrt das Konzept der Arbeitsteilung zu fallenden St{\"u}ckkosten, aufgrund der Gr{\"o}{\ss}envorteile. Demzufolge ist es für die Unternehmen und die gesamte Volkswirtschaft lohnend sich zu spezialisieren und Handel zu betreiben, und zwar unabhängig von Ausstattungs- oder Technologieunterschieden. Interne Skalen\-ertr{\"a}ge f{\"u}hren jedoch zu einer Marktmacht, die nicht mit vollkommenem Wettbewerb vereinbar ist. Diese Bedingung setzt Krugman mit Hilfe des Modells von \citet{Dixit.1977} um, die ein formales Modell zur monopolistischen Konkurrenz entwickelt hatten.\newline


Hinzu kommt ein weiterer Punkt, der von Krugman berücksichtigt wurde. Die Produktvielfalt ist den Unternehmungen eher unwichtig, denn bei ihnen steht die Massenproduktion im Vordergrund. Aus Sicht der Konsumenten gilt allerding das Umgekehrte: Sie bevorzugen eine m{\"o}glichst gro{\ss}e Auswahl und legen Wert darauf, möglichst viele verschiedene Produkte zu haben und ihnen ist dies wichtiger als von einem einzigen Produkt eine gro{\ss}e Menge zu erhalten.\\
\citet{Krugman.79} zeigt, dass der durch Handel induzierte Marktgrö{\ss}eneffekt die Bed{\"u}rfnisse beider befriedigen kann. Bei den Unternehmen entsteht durch den Zugewinn des ausländischen Marktes eine gr{\"o}{\ss}ere Nachfrage, f{\"u}r den nun ebenfalls produziert werden kann und die Konsumenten k{\"o}nnen durch ausl{\"a}ndische Anbieter ein vielf{\"a}ltigeres Angebot nutzen. \newline


Lohnender Au{\ss}enhandel basiert aber in diesem Fall nicht auf dem klassischen Argument des Produktivit{\"a}tsvorteils, sondern zeigt hier auf warum einander {\"a}hnliche Industriel{\"a}nder miteinander handeln und machen zudem auch deutlich warum sie dies  gerade innerhalb derselben Branchen tun. Krugman lieferte damit die wirtschaftstheoretische Erklärung f{\"u}r die Handelsströme des Europ{\"a}ischen Binnenmarktes.\newline


Mit Hilfe der bisherigen theoretischen Modelle konnten allerdings einige der bis hier angef{\"u}hrten empirischen Beobachtungen noch nicht zutreffend vorhergesagt werden, denn die Handelsmodelle von Ricardo und Heckscher-Ohlin reichten nicht aus, um die derzeitige weltweite Handelsstruktur vollst{\"a}ndig erkl{\"a}ren zu k{\"o}nnen. So wurde weder der Au{\ss}enhandel zwischen den sich {\"a}hnelnden Industriel{\"a}ndern begr{\"u}ndet, noch die M{\"o}glichkeit der G{\"u}tervielfalt als Wohlfahrtsgewinn wahrgenommen. Diese beiden Erkl{\"a}rungsdefizite, Gr{\"o}{\ss}envorteile und Produktdifferenzierung sowie der damit einhergehende unvollkommene Wettbewerb wurden bereits von \citet{Balassa.1967,Kravis.1978} sowie \citet{Grubel.1967,Grubel.1970} als Kernbestandteile der sogenannten Neuen Handelstheorien angedeutet. Krugman gab dem Erkl{\"a}rungsansatz der aufkeimenden Neuen Wachstumstheorie in seiner Arbeit von 1980 einen formalen Rahmen.
Sein Ansatz begr{\"u}ndet damit den Handel zwischen L{\"a}ndern, die sich nicht drastisch unterscheiden. In den bisherigen Theorien wurde der Austausch von unterschiedlichen G{\"u}tern zwischen verschiedenen L{\"a}ndern erkl{\"a}rt. Es handelte sich dabei um interindustriellen Handel. In diesem Ansatz geht es um die Erkl{\"a}rung von Handel mit {\"a}hnlichen G{\"u}tern zwischen {\"a}hnlichen L{\"a}ndern, dem intraindustriellem Handel. \newline Eine weitere Neuerung ist die Annahme bez{\"u}glich der Pr{\"a}ferenzen der Konsumenten. Nicht mehr die absolute Gesamtmenge von~G{\"u}tern steht im Vordergrund, sondern deren Vielfalt. Unter der Voraussetzung, dass alle Güter den selben Preis haben, möchten die  Nachfrager eher so viele unterschiedliche G{\"u}ter wie m{\"o}glich beziehen, statt ausschlie{\ss}lich ein Gut zu konsumieren. \newline In den Neuen Handelstheorien lässt sich keine eindeutige Handelsstruktur zuordnen.\footnote{Als Ausnahme gelten hier die sogenannten Nord-Süd Modelle, die Wachstum und Handel miteinander kombinieren. In dieser Modellart wird Handel zwischen der Region des relativ weniger weit entwickelten Süden mit dem relativ weit entwickelten Norden beschrieben. Diese Einteilung geht auf die Beobachtung zurück, dass auf der Nordhalbkugel ein Gro{\ss}teil der entwickelten bzw. industrialisierten Länder zu finden ist, wohingegen auf der Südhalbkugel viele der weniger weit entwickelten Länder liegen. Dabei muss unter anderem von den Pazifikstaaten Australien und Neuseeland abstrahiert werden. Aus dieser regionalen Aufteilung bestimmt sich die Handelsstruktur.  Der weniger weit entwickelte Süden importiert die neu entwickelten Güter \citep{Grossman.1991a,Krugman.1990}.
Dieser Modellaufbau zeigt wie der technische Fortschritt in die Neuen Handelstheorien integriert werden kann. 
 } Aufgrund der Ähnlichkeit der Länder wird sich diese durch Zufall ergeben. Dabei erh{\"o}ht sich die Produktvielfalt aller beteiligter L{\"a}nder durch Au{\ss}enhandel. Die weltweite Nachfrage nach einem Gut ist dann so gro{\ss}, dass die sich bei der Produktion ergebenden Gr{\"o}{\ss}envorteile die Produktionskosten pro Stück reduzieren und das Gut g{\"u}nstiger angeboten werden kann. Die internen Skalenertr{\"a}ge k{\"o}nnen ausgenutzt werden und erm{\"o}glichen eine Spezialisierung auf einige wenige G{\"u}ter. Die absolute Anzahl der Produkte auf dem Weltmarkt ist zwar geringer, als die Summe aller im Autarkiefall, jedoch besteht eine h{\"o}here Produktvielfalt in allen beteiligten L{\"a}ndern. Dadurch steigt die Wohlfahrt, weil die Konsumenten die Vielfalt der G{\"u}ter sch{\"a}tzen. Krugmans Theorie hebt die Rolle gro{\ss}er heimischer M{\"a}rkte als k{\"u}nftige aufstrebende Exportzweige hervor. Dabei profitieren alle beteiligten L{\"a}ndern von internen Skalenerträgen und es ist wirtschaftlich und wohlfahrtstheoretisch sinnvoll sich zu spezialisieren und miteinander Handel zu betreiben \citep{Krugman.79,Krugman.1983,Melvin.1969}.\newline 
Die Neuen Handelstheorien unterscheiden sich von den bisherigen der Neoklassik dahingehend, dass die grundlegenden Bedingungen, wie die Voraussetzung des vollkommenen Wettbewerbs und die Annahme {\"u}ber die Homogenit{\"a}t der G{\"u}ter nicht zwingend G{\"u}ltigkeit finden und nur h{\"o}chstens eine von beiden Voraussetzungen noch zutrifft. Weitere Charakteristika sind zum einen der Erkl{\"a}rungsansatz des intra-industriellen Handels und zum anderen die M{\"o}glichkeit der Einbeziehung von steigenden Skalenertr{\"a}gen. \\


In einem nachfolgendem Papier \citep{Krugman.1979b} formuliert Krugman ein weiteres Handelsmodell, dass eine Kombination aus dem Ansatz von Hecker-Ohlin und einem intrasektoralen Ansatz ist, der mit steigenden Skalenertr{\"a}gen einhergeht. Dabei hinterfragt er, welches Handelsmuster sich ergibt, wenn sich L{\"a}nder zwar {\"a}hneln, sich aber dennoch in ihrer Ausstattung unterscheiden. Je {\"a}hnlicher sich L{\"a}nder  auch hinsichtlich ihrer Ausstattung sind, desto eher ergibt sich die Handelsstruktur gem{\"a}{\ss} dem Ansatz der Skaleneffekte \citep{Krugman.79}.\newline
Dies verdeutlicht im allgemeinen, dass interindustrieller Handel und intraindustrieller Handel  nicht komplett voneinander getrennt werden sollten, denn es besteht ein Zusammenhang dergestalt, dass je {\"a}hnlicher sich L{\"a}nder werden, desto eher entwickelt sich intraindustrieller Handel \citep{Krugman.1981}. Mit der Entwicklung eines Landes ändert sich der Grund für Handel.\newline 


Eine zusätzliche Modellerweiterung berücksichtigt nun auch die Transport\-ko\-sten und zeigt dadurch welche Wirkung Z{\"o}lle und politische Eingriffe haben k{\"o}nnen \citep{Krugman.1980}.\footnote{Weitere Modellvariationen und theoretische Arbeiten, die zu den Neuen Handelstheorien z{\"a}hlen liefern zum Beispiel \citet{Grossman.1991b}.
Best{\"a}tigt wird der Ansatz Krugmans durch zahlreiche empirische Untersuchungen \citep{Antweiler.2002}.}\\
\citet{Lancaster.1980} analysiert das Ausma{\ss} von Handelsvolumen, die durch monopolistische Konkurrenz bedingt sind. Auch wenn L{\"a}nder hinsichtlich Technologie und Ausstattung identisch sind, jedoch die Marktform der monopolistischen Konkurrenz vorliegt, handeln sie intraindustriell miteinander. Er vergleicht jetzt das hypothetische Handelsvolumen durch einen komparativen Vorteil mit dem des m{\"o}glichen  intraindustriellen Handels und kommt zu dem Ergebnis, dass das Volumen deutlich h{\"o}her ist, wenn sich die L{\"a}nder nicht zwingend ihrer komparativen Vorteile spezialisieren.\newline


Im sp{\"a}teren Verlauf der Arbeit wird das in Kapitel \ref{Papier2} er{\"o}rterte Modell dem Ansatz Ricardos folgen. Anschlie{\ss}end in Kapitel \ref{Papier1} wird Handel durch Ausstattungsunterschiede nach Heckscher-Ohlin begründet.


\section{Wirkung von Handel auf Wachstum}
Die Ansätze der Neuen Handelstheorien haben gezeigt, dass die Forschungszweige Handel und Wachstum eng miteinander verbunden sind. In diesem Rahmen wurden immer mehr Faktoren in die Modelle implementiert, die erst durch Au{\ss}enhandel in ein Land kommen und dann langfristig Einfluss auf das Wachstum der Volkswirtschaft haben. Die Handelsgewinne beeinflussen das ökonomische Wachstum und verdeutlichen die Bedeutung des Freihandels für den Entwicklungsprozess eines Landes. Zu den Hauptvertretern dieser zusammenführenden Ansätze zählen Gene Grossman, Elhanan Helpman und Alwyn Young.\newline  
Sie beschreiben die dynamischen Effekte des internationalen Handels auf das Wirtschaftswachstum \citep{Young.1991,Grossman.1995}. Dabei kann grundsätzlich zwischen exogenen Wachstumsmodellen unterschieden zwerden, die den Handel implementiert haben \citep{Dixit.1980,Ethier.1982, Krugman.1979ab,Krugman.1981,Lancaster.1980} und endogenen Wachstumsmodellen offener Volkswirtschaften \citep{Dinopoulos.,Feenstra.,Grossman.1989a,Grossman1989b.,Grossman.1990d,Grossman.1991c, Krugman.1990,Segerstrom.1990,Young.1991,Backus.} unterschieden werden. Die Hauptergebnisse der bisherigen endogenen Wachstumsmodelle konnten auch in Verbindung mit Handel bestätigt werden \citep{vanLong.1997}.

\citet{Atkeson.2000} sowie \citet{Cunat.2001} kombinieren den Handel nach dem Heckscher-Ohlin-Model mit einem Wachstumsmodell. 


Bei der Kombination der Wachstums{\"o}konomie mit den Handelstheorien, gibt es zwei m{\"o}gliche Betrachtungsweisen. Zum einen wird die Wirkung von Au{\ss}enhandel, also der Offenheit eines Landes, auf das Wirtschaftswachstum untersucht werden. Zum anderen wird der Einfluss wachstumsstimullierender Faktoren, wie beispielsweise der technische Fortschritt, auf die Handelsstruktur, das Handelsvolumen oder die Terms of Trade\footnote{Die Wirkung des technischen Fortschritts auf die Terms of Trade h{\"a}ngt in erster Linie von der Art des technischen Fortschritts ab und in welchem Sektor dieser angewendet wird. So w{\"u}rde beispielsweise ein arbeitsvermehrender technischer Fortschritt in dem relativ arbeitsintensiven Importsektor zu einem Anstieg der Term of Trade f{\"u}hren und das innovierende Land besser stellen \citep{Gandolfo.1998}.} analysiert. Dabei werden in erster Linie die Abweichungen und Ver{\"a}nderung der genannten Gr{\"o}{\ss}en in bereits offenen Volkswirtschaften ermittelt, wohingegen bei der zuerst angeführten Betrachtungsweise erstmalig eine Handelsstruktur mit einem dazugeh{\"o}rigen Handelsvolumen entsteht und diese neuen Wechselwirkungen das Wachstum beeinflussen. Der Schwerpunkt liegt auf der Analyse von offenen Wachstumsmodellen, bei denen die Entwicklung und das Wachstum eines Landes untersucht werden.

\subsection{Effekte des Au{\ss}enhandels}\label{Effekte Handel}
Diese in der Forschung vertretenen Herangehensweisen der Wirkungsmechanismen von Au{\ss}enhandel gehen auf die Unterscheidung der Handelsgewinne zurück. Es wird unterschieden zwischen direkten und indirekten Handelsgewinnen. Alle Wirtschaftsteilnehmer profitieren durch Au{\ss}enhandel, also durch Arbeitsteilung, die zur Spezialisierung f{\"u}hrt, und Tausch. Dies wurde in den vorangegangen Kapiteln erl{\"a}utert und geht zur{\"u}ck auf die {\"U}berlegungen von Adam Smith und David Ricardo. Aus Arbeitsteilung und Tausch resultiert ein Handelsgewinn, der als direkt bezeichnet wird \citep{Mill.1909}. Die indirekten Handelsgewinnen entstehen durch die folgenden drei Effekte und begr{\"u}nden dass Handel das Einkommen in der Welt steigert, da das Produktivit{\"a}tswachstum gef{\"o}rdert wird \citep{Aghion.1998}.\footnote{Ein Land profitiert von Handelsliberalisierungen auf zwei verschiedene Arten: statisch und dynamisch \citep{Grossman.1989a,Grossman.1991b,Grossman.1991c,RiveraBatiz.,RiveraBatiz.1991a}. Dies ist eine andere M{\"o}glichkeit die Handelsgewinne zu untergliedern. Der statische Gewinn fasst h{\"o}here Produktqualit{\"a}ten oder auch ein gr{\"o}{\ss}eres Variantenreichtum zusammen. Dynamischer Gewinn beschreibt hingegen eine h{\"o}here Innovationsrate, die den stetigen Prozess neuer Produktentwicklungen eines Landes meint. Grossman und Helpman beschreiben dabei sowohl den Prozess der Innovation als auch den der Imitation. Beide Prozesse ben{\"o}tigen finanzielle Ressourcen, physisches Kapital und Arbeitskr{\"a}fte. Ferner muss bei beiden mit der Unsicherheit des Erfolgs gerechnet werden. In ihren Beitr{\"a}gen beschreiben sie eine Modellwelt, in der die L{\"a}nder mit einem relativ hohen Lohnniveau einen komparativen Vorteil im Forschungssektor haben und somit g{\"u}nstiger Innovationen entwickeln k{\"o}nnen. Niedriglohnl{\"a}nder hingegen sind bef{\"a}higt diese nachzuahmen und sich somit ebenfalls weiter zu entwickeln. Ausgehend von einer Nord-S{\"u}d Handelswelt werden sich die Produktionsst{\"a}tten der G{\"u}ter langfristig vom Norden in den S{\"u}den verlagern \citep{Grossman.1991c}.}

	\begin{itemize}
		\item [1.] Marktgr{\"o}{\ss}eneffekt
		\item [2.] Wissens-Spillover-Effekt
		\item [3.] Wettbewerbseffekt
	\end{itemize}

Bei dem \textit{Marktgrö{\ss}eneffekt} f{\"u}hrt die {\"O}ffnung eines Landes zu neuen M{\"a}rkten, also zu einem insgesamt gr{\"o}{\ss}eren Absatzmarkt, dem Weltmarkt. Je Gr{\"o}{\ss}er ein Markt ist, desto h{\"o}here Gewinne k{\"o}nnen erwartet werden. Durch die {\"O}ffnung der Grenzen steigt der Absatzmarkt eines Landes um die {\"u}brige Welt an. Es k{\"o}nnen insgesamt h{\"o}here St{\"u}ckzahlen produziert und abgesetzt werden, wovon alle Produzenten gleicherma{\ss}en profitieren.\footnote{Diese Grö{\ss}eneffekte beschreibt \citet{Jones.1995a}, indem er allgemein endogene Wachstumsmodelle empirisch testet.} Dadurch nimmt die Bedeutung steigender Skaleneffekte und Learning-by-Doing Externalit{\"a}ten deutlich zu \citep[Kapitel 15]{Aghion.1998}.\\


Der zweite Effekt, der \textit{Wissens-Spillover-Effekt}, bezieht sich nicht mehr auf die G{\"u}term{\"a}rkte, sondern beschreibt Wissensstr{\"o}me zwischen Regionen bzw. L{\"a}ndern. Er beschreibt die Wissens- und Technologiediffusion, die unmittelbar aus internationalem Handel resultiert. In der Regel kommt es zu einem Austausch von technischem Wissen zu weniger weit entwickelten Regionen der relativ weiter entwickelten Regionen \citep{Sachs.1995}. 
Den Diffusionsprozess, bedingt durch die industrielle Revolution beschreibt \citet{Lucas.2007}, indem er untersucht, ob sich Unterschiede hinsichtlich der Offenheit von L{\"a}ndern feststellen lassen. Dabei legt er die Kriterien\footnote{Bei diesen Kriterien handelt es sich um die Regelung der maximalen Höhe von Handelsbeschränkungen, sowie institutioneller und wettbewerbspolitischer Art.} f{\"u}r Offenheit von \citet{Sachs.1995} zugrunde, die erf{\"u}llt sein m{\"u}ssen, damit eine Volkswirtschaft als offen kategorisiert werden kann. Er stellt fest, dass mit der Offenheit eines Landes auch die Diffusionsdurchl{\"a}ssigkeit zunimmt.


Den Einfluss des Au{\ss}enhandels auf eine Branche beschreibt der dritte Effekt, der \textit{Wettbewerbseffekt}. Die {\"O}ffnung eines Landes ist mit einer Vergr{\"o}{\ss}erung des Marktes verbunden, wodurch der Wettbewerb zwischen den Produzenten steigt. In der theoretischen Modellwelt wird meist angenommen, dass es ein repr{\"a}sentatives Unternehmen gibt und sich somit die Gesamtheit aller Unternehmer gem{\"a}{\ss} der Symmetrie der Unternehmen nach diesem richtet. In der Realit{\"a}t ist die Gesamtheit der Unternehmen aber nicht homogen. Somit ist auch der Einfluss von Handel auf die Unternehmen verschieden. Diese verhalten sich gerade nicht komplett gleich und weisen unterschiedliche Produktivit{\"a}ten auf. Zwar er{\"o}ffnen die hinzugewonnenen Absatzm{\"o}glichkeiten allen Marktteilnehmern neue M{\"o}glichkeiten, jedoch f{\"u}hrt der gestiegene Wettbewerbsdruck dazu, dass die am wenigsten leistungsf{\"a}higen Unternehmen aus dem Markt gedr{\"a}ngt werden. Was wiederum dazu f{\"u}hrt, dass die akkumulierte Produktivit{\"a}t einer Volkswirtschaft ansteigt.\footnote{Dieses Argument untermauert auch \citet{Trefler.2004} in seinem Aufsatz über die Produktivitätssteigerung Kanadas durch Handelsliberalisierung.}\\


Der Wettbewerbseffekt äu{\ss}ert sich demnach in einem Selektionseffekt. \citet{Melitz.2003} betont in seiner Arbeit diesen Selektionseffekt. Au{\ss}enhandel erm{\"o}glicht den Zugang zu neuen M{\"a}rkten und vergr{\"o}{\ss}ert somit das Absatzgebiet eines jeden Unternehmens. Neben der Nachfrage weitet sich jedoch auch das Feld der Anbieter aus, durch die der Wettbewerb des Marktes ansteigt. Die Marktkr{\"a}fte f{\"u}hren dazu, dass die weniger effizienten Produzenten aus dem Markt ausscheiden, da sie nun durch ausl{\"a}ndische Mitstreiter verdr{\"a}ngt wurden. Dieser Selektionseffekt beschränkt sich nicht nur auf die lokalen Unternehmen, sondern setzt sich auch im internationalen Wettbewerb zwischen den Unternehmen fort.\\Insgesamt werden jetzt nur noch die heimischen Unternehmen am Markt bleiben, die ein bestimmtes Effizienzniveau erf{\"u}llen. Das gestiegenen Effizienzniveau eines Landes wirkt sich direkt positiv auf das gesamtwirtschaftliche Einkommen aus \citep[Kapitel 15]{Aghion.2015}.


\subsection{Auswirkung der Effekte}
Die Kernfragen, die sich daraus ergeben lauten:  Welche Wirkung hat Handel auf das {\"o}konomische Wachstum? Welche Folgen ergeben sich aus den genannten Effekten? Dies h{\"a}ngt im wesentlichen von der Modellierung des Handelsmodells ab und letztlich auch von den Gr{\"u}nden f{\"u}r {\"o}konomisches Wachstum.\\
Die wissenschaftlichen Meinungen {\"u}ber den Einfluss von Handel auf das {\"o}konomische Wachstum gehen auseinander. Vorherrschend ist, dass Au{\ss}enhandel Wachstum f{\"o}rdert und somit ein positiver Zusammenhang zwischen Handel und Wachstum besteht \citep{Dollar.1992,Sachs.1995}. Auch empirisch wurde nachgewiesen, dass mit zunehmenden Handelsbeziehungen das Pro-Kopf-Einkommen ansteigt und somit auch das Wirtschaftswachstum \citep{Frankel.1999}.\footnote{Um dies zeigen zu können wurde ein me{\ss}barer und berechenbarer Grad der Offenheit eines Landes entwickelt, mit dem sich die L{\"a}nder einzeln katalogisieren lassen.}\newline


Mikro{\"o}konomisch basierte Ans{\"a}tze, wie von \citet{Bernard.2003} und \citet{Bernard.2004} zeigen, dass Unternehmen, die f{\"u}r den Exportsektor produzieren, produktiver sind. \\


Mit Au{\ss}enhandel und der Heterogenit{\"a}t von Unternehmen beschäftigt sich die Arbeit von \citet{Melitz.2003}. Bei ihm f{\"u}hrt Handelsliberalisierung zu einer dem Produktivit{\"a}tsgrad entsprechenden Unternehmensstruktur. Nur die produktivsten Unternehmen produzieren f{\"u}r den Export, weniger produktive Unternehmen befriedigen die heimische Nachfrage und die schw{\"a}chsten Unternehmen scheiden aus dem Markt aus.\footnote{Dabei handelt es sich um den angeführten Selektionseffekt.} Dies führt zu unternehmensinternen Umstrukturierungsprozessen, die der zusätzliche Wettbewerb fordert. Jedoch ber{\"u}cksichtigt er in seinem Modell nicht den Einfluss von Handel auf die Innovationst{\"a}tigkeit.\newline
Berücksichtigt man den Entwicklungsstand eines Landes, wird der Einfluss von Handel in weniger weit entwickelten L{\"a}nder hervorgehoben \citep{Pavcnik.2002}. Hier zeigt sich die Wirkung des Wissens-Spillover-Effekts, denn der Import von Technologien aus relativ weiter entwickelten Volkswirtschaften erh{\"o}ht die Produktivit{\"a}t der weniger weit entwickelten Ländern durch den Technologietransfer. \newline 


Die Nord-Süd-Modelle berücksichtigen ebenfalls den technologischen Fortschritt durch Innovationsentwicklung. Der weniger weit entwickelte Süden profitiert dabei vom Technologietransfer durch den Import von Innovationen. Neben diesem Spillover-Effekt verstärkt der Au{\ss}enhandel weiterhin den technische Fortschritt durch die nun vorhandenen Imitationsmöglichkeiten. Der Import von Gütern erlaubt es dem Süden mit einer zeitlichen Verzögerung diese Güter nachzuahmen, währenddessen wieder neu entwickelte importiert werden \citep{Grossman.1991a,Krugman.1990}.\\
Vertreter der Mindermeinung  hinterfragen die positive Wirkung durchaus kritisch und zeigen teilweise, dass Au{\ss}enhandel sogar die Wachstumsraten einiger L{\"a}nder mindern kann \citep{RodriguezCaballero.2000,Matsuyama.,Young.1991,Galor.2008}.\bigskip\\


Grundsätzlich wirkt Au{\ss}enhandel auf das Wachstum {\"u}ber die beiden von \citet{Gandolfo.1998} genannten Kan{\"a}le, der Faktorvermehrung und dem technischen Fortschritt.\\


Bei der \textbf{Faktorvermehrung}, dem ersten Wirkungskanal, stehen den Volkswirtschaften durch die Zunahme der Marktgrö{\ss}e insgesamt mehr Produktionsfaktoren zur Verf{\"u}gung und die Technologiediffusion offener Volkswirtschaften erh{\"o}ht die Effizienz des Faktoreinsatzes \citep{Gandolfo.1998}. Die Gr{\"o}{\ss}envorteile k{\"o}nnen unternehmensintern ausgenutzt werden. Aus mikro{\"o}konomischer Sicht k{\"o}nnen die Produktionsfaktoren effizienter genutzt werden und mit der gleichen Einsatzmenge kann nun eine h{\"o}here Ausbringungsmenge produziert werden. Das Grenzprodukt steigt an und somit steigt auch die Wachstumsrate. Der Wettbewerbseffekt bedingt ebenfalls die volkswirtschaftliche Produktivit{\"a}t, da er zur Selektion nur der konkurrenzf{\"a}higsten Unternehmen f{\"u}hrt. Dies zeigt makro{\"o}konomisch, dass Unternehmen aus dem Markt austreten und nur die produktivsten Unternehmen eines Landes verbleiben. Somit liegt jetzt eine produktivere Gesamtheit aller Unternehmen vor, als in der geschlossenen Volkswirtschaft. Au{\ss}erdem wirkt der Marktgr{\"o}{\ss}eneffekt und die dadurch implizierte Unternehmensselektion durch den Wettbewerbseffekt auf den Forschungs- und Entwicklungssektor. Unternehmen streben st{\"a}rker nach monopolistischer Marktmacht und das erh{\"o}ht den Anreiz zur Innovationsentwicklung.  Dies bildet den Übergang zu dem zweiten Wirkungskanal des Wachstums, dem technischen Fortschritt. Er resultiert nicht nur aus der erhöhten Innovationstätigkeit in einer Volkswirtschaft, sondern auch durch den Wissens-Spillover-Effekt, der die internationale Technologiediffusion ermöglicht.\newline


Der \textbf{technische Fortschritt} wird im folgenden durch die Intensität der \textit{Innovationstätigkeit} eines Landes untersucht. Dabei wird die Vorteilhaftigkeit von Au{\ss}enhandel f{\"u}r die Wohlfahrt eines Landes durch den Einfluss der Offenheit auf die Innovationst{\"a}tigkeit eines Landes gezeigt.\footnote{Gleichwohl ist auch eine hemmende Wirkung von Au{\ss}enhandel auf die Innovationst{\"a}tigkeit und letztlich das Wirtschaftswachstum m{\"o}glich. Denn unterscheiden sich beide L{\"a}nder durch ihre ursprünglichen Produktivit{\"a}tsniveaus bei Autarkie, dann kann Au{\ss}enhandel die Innovationst{\"a}tigkeit hemmen, sofern es sich um anf{\"a}nglich relativ weniger weit entwickelte L{\"a}nder handelt \citep{Devereux.1994,RiveraBatiz.1991a}.} In diesem Zusammenhang werden die drei Effekte des Au{\ss}enhandels nochmals verdeutlicht. \newline Erfolgreiche Innovatoren profitieren vom Marktgr{\"o}{\ss}eneffekt, da sich nun eher Innovationen finanzieren lassen, so dass diese auch tats{\"a}chlich produziert werden k{\"o}nnen. \newline 


Der Wissens-Spillover-Effekt, wirkt sich durch den nun m{\"o}glichen internationalen Wissens- und Technologietransfer aus. Dieser Diffussionsprozess ist jedoch nicht zwingend notwendig, um die Vorteilhaftigkeit des Handels zu zeigen. Werden m{\"o}gliche zus{\"a}tzliche Wissens-Spillover-Effekte ausgeschlossen, da die ge{\"o}ffneten Volkswirtschaften {\"u}ber die gleichen Technologien verf{\"u}gen, dann f{\"u}hren trotzdem dynamische Effekte zu h{\"o}heren Wachstumsraten in ge{\"o}ffneten L{\"a}ndern \citep{Grossman.1991b}. Der Ursprung des Wachstums liegt ebenfalls in der Entwicklung von Innovationen.\newline Wird Wissenstransfer jedoch zugelassen, dann ist ein Zugewinn von technologischem Wissen {\"u}ber die Grenze hinweg m{\"o}glich und  Pioneerunternehemen k{\"o}nnen das Wissen verwenden, um damit neue Innovationen zu entwickeln.\footnote{Nicht nur Innovationen, sondern auch die Imitationsrate wird durch Freihandel gefördert. Dies kann ebenfalls durch den Wissenstransfer im Zuge des Wissens-Spillover-Effektes begründet werden.} \newline 
Auch der aus Au{\ss}enhandel resultierende Wettbewerbseffekt steigert die Innovationsrate, da insgesamt die Produktivit{\"a}t eines Unternehmens, einer Branche und letztlich eines Landes zunimmt. Demzufolge wird ein innovierendes Unternehmen in einer offenen Volkswirtschaft theoretisch mehr Innovationen entwickeln als in einer geschlossenen.
Denn der zusätzliche Wettbewerb stellt einen Anreiz zur Innovationsentwicklung dar, damit Unternehmen sich von den konkurrierenden Anbietern abheben können, um Marktmacht zu erlangen. Dabei f{\"o}rdert der durch Handelsliberalisierung verst{\"a}rkte Wettbewerb den Innovationsprozess, welcher sich in dauerhaften Produktverbesserungen {\"a}u{\ss}ert \citep{Segerstrom.1990}.\\


Es bleibt festzuhalten, dass über alle drei Wirkungskan{\"a}le: Marktgr{\"o}{\ss}e, Wissens-Spillover-Effekt und Wettbewerbseffekt sich die Offenheit eines Landes positiv auf die Innovationsrate auswirkt.\\
Doch kann die Reaktion eines Landes nicht immer eindeutig vorhergesagt werden. Denn erörtert man diese Situation für technologisch kleine Länder mit einem gro{\ss}en Abstand zur WTG, dann kann ein Entmutigungseffekt bezüglich der Innovationstätigkeit auftreten. Die hohe Rückständigkeit lässt die aufzuholende Lücke hinreichend gro{\ss} erscheinen, sodass es wenige Bestrebungen gibt an den technologischen Entwicklungsstand anzuschlie{\ss}en. Die Folge wäre ein Rückgang der Wachstumsrate und Handel würde in diesem Fall das Wachstum sogar reduzieren \citep{Aghion.2015}. \citet{Hicks.1968} behandelt diesen Zusammenhang zwischen Handel und Wachstum, basierend auf der Thematik nach dem zweiten Weltkrieg bez{\"u}glich Deutschland und den USA. Der Entwicklungsunterschied war so gro{\ss}, dass es Bedenken gab, die L{\"u}cke nicht mehr schlie{\ss}en zu k{\"o}nnen. Auch wenn die Thematik nicht mehr zeitgem{\"a}{\ss} ist, lassen sich die Bedenken und Ans{\"a}tze auf heutige Konstellationen anwenden.\\


Ebenfalls in technologisch kleinen Länder, deren Strategie nicht darin liegt Innovationen zu entwickeln ist es denkbar, dass die Wachstumsrate geschmälert wird, da von einem Flucht-Eintritts-Effekt\footnote{Dieser Effekt beschreibt einen zusätzlichen Impuls innovativ tätig zu sein, da in offenen Volkswirtschaften die Konkurrenzsituation zwischen den innovierenden Unternehmen deutlich stärker ist. Es besteht die Möglichkeit, dass ausländische Unternehmen schneller sind und somit eher eine Neuerung am Markt ansiedeln.} abstrahiert werden kann. Auch der Gr{\"o}{\ss}eneffekt bezüglich der Innovationsrate kann vernachl{\"a}ssigt werden, sofern es sich um ein {\"o}konomisch kleines Land handelt. Diese beiden Faktoren k{\"o}nnen dazu führen, dass sich ein Land  durch die au{\ss}enwirtschaftliche {\"O}ffnung verschlechtert und die Wachstumsrate sinkt. \newline  Die beiden zuletzt genannten Argumente lassen eine politische Empfehlung für ökonomisch und technologisch kleine Länder ableiten. Denn wenn zun{\"a}chst die Innovationst{\"a}tigkeit gef{\"o}rdert wird, so dass es sektoral eine technologische F{\"u}hrerschaft gibt, und anschlie{\ss}end Handelsliberalisierug zugelassen wird, dann werden sich die Wachstumsaussichten verbessern \citep{Aghion.2015}.


Der Import von Innovationen kann die heimischen Innovationsbestrebungen mindern bzw. sogar ersetzen. Selbst L{\"a}nder die nicht innovieren, k{\"o}nnen ein h{\"o}heres Produktivit{\"a}tswachstum erreichen, indem sie Handel betreiben \citep{Aghion.2015}. Die Anpassung und die damit einhergehende Konvergenz zum Weltmarkt stellen dann die Vorteilhaftigkeit von Au{\ss}enhandel dar und nicht die Steigerung der Innovationsintensität.\\
Die Wirkung des Handels {\"u}ber den technischen Fortschritt durch die Innovationst{\"a}tigkeit auf das Wachstum wurde ausf{\"u}hrlich erl{\"a}utert. Weitere Effekte des Handels werden  durch \textit{learning-by-doing Externalit{\"a}ten}, die in einigen Sektoren auftreten, bedingt \citep{Young.1991,Matsuyama.}.\\
Learning-by-doing steht im engen Zusammenhang mit Grö{\ss}eneffekten, da mit steigender Ausbringungsmenge die Effizienz der Produktionsverfahren zunimmt. Bei kleineren Stückzahlen wirkt sich die Erfahrung durch leaning-by-doing noch nicht hinreichend positiv auf die Produktivität aus. Wohingegen durch Au{\ss}enhandel die Bedeutung dieses Effekts durch den erweiterten Markt ansteigt \citep{Arrow.1962}.\\


Wird von Skaleneffekten abstrahiert, steigt zwar zun{\"a}chst die Innovationsrate an, steigert jedoch nicht langfristig die Wachstumsrate. Dies ist nur ein Argument, dass aufzeigt, warum junge Industriezweige anf{\"a}nglich vor der internationalen Konkurrenz gesch{\"u}tzt werden sollen und wird in Kapitel \ref{Entwicklungsstrategien} nochmals aufgegriffen, um mögliche Entwicklungsstrategien aufzuzeigen.\newline Au{\ss}enhandel kann sich auch bei der Herausbildung der Handelsstruktur negativ auswirken. So kann internationaler Handel dazu f{\"u}hren, sich entgegen der tats{\"a}chlichen komparativen Vorteile zu spezialisieren und somit Branchen zu f{\"o}rdern, die ein vergleichsweise geringes Wachstumspotential aufweisen \citep{Acemoglu.2009}.\newline 


Hinzu kommt die Möglichkeit des Ausbleibens von learning-by-doing-Effekten. Sind in weniger weit entwickelten Volkswirtschaften die Produktionsverfahren sehr traditionell gepr{\"a}gt und haben sich bereits {\"u}ber einen langen Zeitraum hinweg optimiert, dann werden learning-by-doing-Effekte die Produktivit{\"a}t nicht ma{\ss}geblich verbessern, da diese bereits weitestgehend ausgesch{\"o}pft wurden \citep{Young.1991}. Eine {\"O}ffnung f{\"u}r Handel, die wiederum zu grenz{\"u}bergreifendem learning-by-doing f{\"u}hrt, h{\"a}tte keinen oder sogar einen hemmenden Einfluss auf das Wachstum der Volkswirtschaft. \newline Diesen Zusammenhang zeigt \citet{Young.1991}, indem er ein Land jeweils vor und nach der Einf{\"u}hrung von Au{\ss}enhandel analysiert. Handelt es sich um ein technologisch weniger weit entwickeltes Land, dann ist die Wachstumsrate des Einkommens in der geschlossenen Volkswirtschaft gr{\"o}{\ss}er oder gleich der einer ge{\"o}ffneten Volkswirtschaft. Umgekehrt verh{\"a}lt sich ein relativ weit entwickeltes Land, das seine Situation durch Handel verbessert. So steigt nicht nur der technologische Entwicklungsstand an, sondern auch das Wirtschaftswachstum. Somit ist in dieser Konstellation die Wirkung des Au{\ss}enhandels von dem Entwicklungsstand des Landes abhängig.\newline
Die Ver{\"a}nderung des langfristigen Wachstums einer nun offenen Volkswirtschaft, verglichen mit derselbigen im geschlossenen Zustand untersuchen ebenfalls \citet{Grossman.1995} in ihrem Modell. Sie arbeiten dabei zwei Einflussfaktoren heraus, die zu einer relativen Ver{\"a}nderung der Wachstumsgeschwindigkeit beitragen. Von Bedeutung ist zum einen die Reichweite der learning-by-doing-Effekte, denn es ist fraglich ob diese nur national oder gar international wirken. Zum anderen beeinflusst die Spezialisierung der Produktion auf einzelne Sektoren induziert durch Handel die Wachstumsgeschwindigkeit. Je nach Gestaltung der Produktionsschwerpunkte und der daraus resultierenden Handelsstruktur kann das Wachstum eines Landes langfristig beschleunigt oder verlangsamt werden.\newline


{\"A}hnliche Ansichten wie \citet{Grossman.1995} teilt auch \citet{Krugman.1987}, der ebenfalls die Reichweite und damit das Wirkungsgebiet von learning-by-doing analysiert. Sein Handelsmodell beschreibt die dynamische Entwicklung des komparativen Vorteils, der durch learning-by-doing hervorgeht.  Wirkt learning-by-doing nur national, f{\"u}hrt Handel nicht automatisch zur Konvergenz von Wachstumsraten der am Handel beteiligter L{\"a}nder. Jedoch l{\"o}sen durch Handel hervorgerufene internationale learning- by-doing-Effekte gegenseitige Reaktionen aus, die die Wachstumsraten konvergieren lassen. \newline 


Es wurde dargelegt, dass die {\"O}ffnung eines Landes die beiden Wirkungskan{\"a}le des Wachstums beeinflusst. Es ver{\"a}ndert sich die inl{\"a}ndische Produktivit{\"a}t einer Volkswirtschaft, durch die drei angef{\"u}hrten Effekte. Dabei wurde die Rolle des technologischen Fortschritts besonders betont und wird im folgenden nochmals am Beispiel Japans hervorgehoben.\\


\citet{Grossman.1990} entwickelte ein dynamisches Modell des komparativen Vorteils, bei dem Innovationen endogen sind. Es werden zwei L{\"a}nder betrachtet und zwei G{\"u}ter produziert, ein Hightech Konsumgut und ein gew{\"o}hnliches Konsumgut. In diesem Modell nimmt Japan die Rolle des Landes ein, das relativ reichlich mit sehr gut ausgebildeten Arbeitskr{\"a}ften, aber weniger gut mit nat{\"u}rlichen Ressourcen ausgestattet ist. In beiden L{\"a}ndern verwenden Unternehmen ihre Ressourcen f{\"u}r Forschung und Entwicklung, um die Qualit{\"a}t der G{\"u}ter zu verbessern. Motiviert sind sie durch bestehende Profitm{\"o}glichkeiten auf dem Weltmarkt durch Innovationen bzw. der daraus resultierenden Monopolstellung. Demzufolge kann Japan neue Technologien besser entwickeln und Hightech Produkte g{\"u}nstiger produzieren.  Der komparative Vorteil Japans liegt in der Herstellung von Hightech G{\"u}tern, was auch durch die Daten best{\"a}tigt werden konnte. \citet{Grossman.1990} ergr{\"u}ndet neben der Handelsstruktur auch die Wirkung handelspolitischer Ma{\ss}nahmen wie Importz{\"o}lle und Exportsubventionen. Diese politischen Eingriffe erh{\"o}hen zwar die Wettberwerbsf{\"a}higkeit der Hightech Branche in dem jeweiligen Land, mindern jedoch die Innovationsquote, da weniger Anreize bestehen zu innovieren. F{\"u}r die Entwicklung des technische Fortschritts bedeutet dies laut \citet{Grossman.1990}, dass die Wachstumsrate sinkt, sofern das Land die protektionistischen Ma{\ss}nahmen einf{\"u}hrt, bei dem der komparative Vorteil in der Hightech Produktion liegt, also hier Japan. Hat das sich sch{\"u}tzende Land einen komparativen Nachteil in der Hightech Produktion, dann steigt die Rate des technischen Fortschritts an \citep{Grossman.1990}.
Die Richtung des technischen Fortschritts l{\"a}sst sich insofern bestimmen, dass bei unvollst{\"a}ndigen geistigen Eigentumsrechten mit Handel deutlich mehr Fachkr{\"a}fte notwendig sind und ein h{\"o}herer technologischer Entwicklungsstand realisiert werden kann als in einer geschlossenen Volkswirtschaft \citep{Acemoglu.2003,Thoenig.2003,Epifani.2006}.