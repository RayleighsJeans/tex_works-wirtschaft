\chapter{mathematischer Anhang Kapitel: einfaches Handelsmodel}

Das Nutzenmaximierungsproblem der Konsumenten für alle Situationen wird in dem folgenden Abschnitt etwas ausführlicher dargelegt. Zunächst ergibt sich die optimale Aufteilung $m$ des Faktors Arbeit auf die beiden Sektoren.

\begin{itemize}
\item \underline{geschlossene Volkswirtschaft}
\begin{equation} U_j=ln(c_1)+ln(c_2)\end{equation}
\begin{displaymath}\frac{\partial U}{\partial m}=\frac{\partial U}{\partial c_1}\frac{\partial c_1}{\partial m}+\frac{\partial U}{\partial c_2}\frac{\partial c_2}{\partial m}=0\end{displaymath} 
\begin{displaymath}\frac{\Theta_1}{c_1}=\frac{\Theta_2}{c_2}\end{displaymath}
\begin{equation} c_1=\frac{\Theta_1}{\Theta_2}c_2\end{equation}
\begin {equation} m=0,5\end{equation}

\item \underline{offene Vokswirtschaft}
\begin{equation} U^i_j=ln(c^i_1)+ln(c^i_2)\end{equation}
\begin{equation}\frac{\partial U^i}{\partial m^i}=\frac{\partial U^i}{\partial c_1^i}\frac{\partial c_1^i}{\partial m^i}+\frac{\partial U^i}{\partial c_2^i}\frac{\partial c_2^i}{\partial m^i}\overset{!}{=}0\end{equation}
\begin{displaymath} \frac{\Theta_1^i}{c_1^i}=\frac{\Theta_2^i}{c_2^i}\end{displaymath}\begin{equation} c_1^i=\frac{\Theta_1^i}{\Theta_2^i}c_2^i\end{equation}
\begin{equation} m^i=0,5\end{equation}
\end{itemize}


Die Nutzenmaximierung unter Berücksichtigung des zur Verfügung stehenden Einkommens. 
\begin{itemize}
\item \underline{geschlossene Volkswirtschaft}
\begin{equation} maxU=\ln(c_1)+\ln(c_2)-\lambda (p_1c_1+p_2c_2-y)\end{equation}
\begin{displaymath}\frac{\partial U}{\partial c_1}=\frac{1}{c_1}-\lambda p_1=0\end{displaymath}
\begin{displaymath} \frac{\partial U}{\partial c_2}=\frac{1}{c_2}-\lambda p_2=0\end{displaymath}
\begin{displaymath} \frac{\partial U}{\partial \lambda}=-p_1c_1-p-_2c_2+y=0\end{displaymath}
\begin{equation} c_1=\frac{y}{2p_1}=d_1\end{equation}
\begin{equation} c_2=\frac{y}{2p_2}=d_2\end{equation}

\item \underline{offene Vokswirtschaft}
\begin{equation} maxU^i=\ln(c^i_1)+\ln(c^i_2)-\lambda^i (p_1c^i_1+p_2c^i_2-y^i)\end{equation}
\begin{equation}\frac{\partial U^i}{\partial c^i_1}=\frac{1}{c^i_1}-\lambda^i p_1\overset{!}{=}0\end{equation}
\begin{equation}\frac{\partial U^i}{\partial c^i_2}=\frac{1}{c^i_2}-\lambda^i p_2\overset{!}{=}0\end{equation}
\begin{equation}\frac{\partial U^i}{\partial \lambda^i}=-p_1c^i_1-p-_2c^i_2+y^i\overset{!}{=}0\end{equation}
\begin{equation} c^i_1=\frac{y^i}{2p_1}=d^i_1\end{equation}
\begin{equation} c^i_2=\frac{y^i}{2p_2}=d^i_2\end{equation}

\item \underline{offene Vokswirtschaft mit Transportkosten}
\begin{equation} maxU^i=\ln(c^i_1)+\ln(c^i_2)-\lambda^i (p_1c^i_1+p_2(1+\tau)c^i_2-y^i)\end{equation}
\begin{equation}\frac{\partial U^i}{\partial c^i_1}=\frac{1}{c^i_1}-\lambda^i p_1\overset{!}{=}0\end{equation}
\begin{equation} \frac{\partial U^i}{\partial c^i_2}=\frac{1}{c^i_2}-\lambda^i p_2(1+\tau)\overset{!}{=}0\end{equation}
\begin{equation} \frac{\partial U^i}{\partial \lambda^i}=-p_1c^i_1-p-_2(1+\tau)c^i_2+y^i\overset{!}{=}0\end{equation}
\begin{equation} c^i_1=\frac{y^i}{2p_1}=d^i_1\end{equation}
\begin{equation} c^i_2=\frac{y^i}{2p_2(1+\tau)}=d^i_2\end{equation}
\end{itemize}



Die Gewinnmaximierung der Unternehmen verhält sich je nach Situation folgendermaßen: 
\begin{itemize}
\item \underline{geschlossene Volkswirtschaft}
\begin{equation} max\Pi_j=p_jc_j(m)-w_jc_j(m)\end{equation}
\begin{equation}\frac{\partial \Pi_j}{\partial c_j}=p_j-w_j\overset{!}{=}0\end{equation}
\begin{equation}p_j=w_j\end{equation}
\begin{equation}\frac{\partial \Pi_j}{\partial m}=p_j\frac{\partial c_j}{\partial m}-w_j\frac{\partial c_j}{\partial m}\overset{!}{=}0\end{equation}
\begin{equation}p_j\frac{\partial c_j}{\partial m}=w_j\frac{\partial c_j}{\partial m}\end{equation}
\begin{equation}p_j=w_j\end{equation}
\begin{center}
\begin{flushright}
\fbox{\parbox{5cm}{\begin{displaymath}\frac{p_1}{p_2}=\frac{w_1}{w_2}=\frac{\Theta_1}{\Theta_2}\end{displaymath}}}\newline
\end{flushright}
\end{center}

\item \underline{offene Vokswirtschaft}
\begin{equation} max\Pi_j=p_jc^i_j-w^i_jc^i_j\end{equation}
\begin{equation}\frac{\partial \Pi^i_j}{\partial c^i_j}=p_j-w^i_j\overset{!}{=}0\end{equation} 
\begin{equation}p_j=w^i_j\end{equation}
\begin{equation}\frac{\partial \Pi^i_j}{\partial m^i}=p_j\frac{\partial c^i_j}{\partial m^i}-w^i_j\frac{\partial c^i_j}{\partial m^i}\overset{!}{=}0\end{equation}
\begin{displaymath}p_j\frac{\partial c^i_j}{\partial m^i}=w^i_j\frac{\partial c^i_j}{\partial m^i}\end{displaymath}
\begin{equation}p_j=w^i_j\end{equation}
\begin{flushright}
\begin{center}
\fbox{\parbox{7cm}{\begin{displaymath}\frac{p^h_1}{p^h_2}=\frac{w^h_1}{w^h_2}=\frac{\Theta^h_1}{\Theta^h_2}<\frac{p^*_1}{p^*_2}=\frac{w^*_1}{w^*_2}=\frac{\Theta^*_1}{\Theta^*_2}\end{displaymath}}}
\end{center}
\end{flushright}

\item \underline{offene Vokswirtschaft mit Transportkosten}
\begin{equation} max\Pi_j=p_jc^i_j-w^i_j (1+\tau) c^i_j\end{equation}
\begin{equation}\frac{\partial \Pi^i_j}{\partial c^i_j}=p_j-w^i_j(1+\tau)\overset{!}{=}0\end{equation}
\begin{equation}p_j=w^i_j(1+\tau)\end{equation}
\begin{equation}\frac{\partial \Pi^i_j}{\partial m^i}=p_j\frac{\partial c^i_j}{\partial m^i}-w^i_j (1+\tau)\frac{\partial c^i_j}{\partial m^i}\overset{!}{=}0\end{equation}
\begin{displaymath}p_j\frac{\partial c^i_j}{\partial m^i}=w^i_j (1+\tau)\frac{\partial c^i_j}{\partial m^i}\end{displaymath}
\begin{equation}p_j=w^i_j (1+\tau)\end{equation}
\begin{flushright}
\begin{center}
\fbox{\parbox{9cm}{\begin{displaymath}\frac{p^h_1}{p^h_2}=\frac{w^h_1}{w^h_2}=\frac{\Theta^h_1}{\Theta^h_2}<\frac{p^*_1}{p^*_2}=\frac{w^*_1}{w^*_2(1+\tau)}=\frac{\Theta^*_1}{\Theta^*_2}\end{displaymath}}}
\end{center}
\end{flushright}
\end{itemize}
