%\documentclass[12pt]{report}

%%\documentclass[a4paper, 12pt, DIVcalc, oneside, headsepline, ngerman, smallheadings, openany, liststotoc, bibtotoc]{scrbook}
%%\usepackage{scrpage2} % zur Manipulation der Kopf und Fusszeilen
%\usepackage{makeidx} % zum Erstellen und Indexen (Abkürzungsverzeichnis)
%\makeindex % Befehl dazu?
%\usepackage[intoc]{nomencl} % Abkuerzungsverzeichnis (und ins Inhaltsverzeichnis aufnehmen)
%\let\abk\nomenclature % Befehl umbenennen in abk
%\renewcommand{\nomname}{Abkürzungsverzeichnis} % Deutsche Überschrift
%\makenomenclature
%%\usepackage{endnotes} % End- und Fußnoten
%\usepackage{hyperref} % klickbare Kapitel und Link-Fraben
%\hypersetup{colorlinks=true, linkcolor=blue, urlcolor=blue, citecolor=blue}
%\usepackage{caption} % für z.B. Tabellenüberschriften
%\usepackage{tocbibind} %gibt im Inhaltsverzeichnis bspw. das Abbildungsverzeichnis an
%\usepackage{subcaption}
%\usepackage{subfigure}
%\usepackage[percent]{overpic} %damit kann man in einem Bild zusätzlich beschriften
%\usepackage{color}
%%\usepackage{svg}


%\usepackage[ngerman]{babel} %deutscher Satzbau
%\usepackage{ngerman}
%\usepackage{paralist} 
%\usepackage{enumitem} 
%\usepackage[ansinew]{inputenc} %damit in Windows die umlaute funktionieren
%\usepackage{a4wide}
%\usepackage{lmodern,amssymb,xcolor,amsmath}
%%\usepackage[style=authortitle-icomp]{biblatex}
%\usepackage{natbib}
%\usepackage{tikz}
%\usepackage{fancybox} % für die verschiedenen Boxen
%\usepackage{ulem} % für unterstreichen und durchstreichen im Text
%\usetikzlibrary{arrows,positioning}
%\usepackage{float}
%\usepackage{psfrag,epsfig,graphicx,subfigure,enumerate,pifont,bibentry,pstricks,marvosym,rotating}

%\renewcommand{\baselinestretch}{1.35}\normalsize

%\begin{document}

\chapter{einfaches Handelsmodell}
Das folgenden Kapitel stellt ein einfaches Handelsmodell dar, an welches schrittweise herangef{\"u}hrt wird. Zun{\"a}chst wird die Situation ohne Handel, die Autarkiesituation, beschrieben. Anschlie{\ss}end findet Handel zwischen den Regionen statt. In einem letzten Schritt werden Transportkosten eingef{\"u}hrt, die vom Prinzip eine Handelsbarriere darstellen und wie ein Zoll fungieren.  Das modell folgt der Idee   nach Ricardo mit Technologieunterschieden.

\section{geschlossene Volkswirtschaft}
Zun{\"a}chst wird eine Volkswirtschaft betrachtet, in der mit nur einem Produktionsfaktor Arbeit ($L$)\footnote{$L= L_{1}+L_{2}$} zwei Güter ($c_j$ mit $j=1; 2$) hergestellt werden.\newline Die Produktionstechnologie des Landes beschreibt einen konstanten Arbeitskräftebedarf.
\begin{equation} c_j=\Theta_jL_j \end{equation}
Dabei werden die Arbeitskräfte zwischen den Sektoren $1$ und $2$ aufgeteilt gemäß 
$L_1=m ~\text{und}~L_2=(1-m)$. Gem{\"a}{\ss} der Grenzproduktivit{\"a}tsentlohnung entspricht das Grenzprodukt der Arbeit dem Lohnsatz, was wiederum gleich dem Produktivit{\"a}tsparameter $\Theta_j$ ist. 
\begin{displaymath} \frac{\partial c_j}{\partial L_j}=w_j \Rightarrow \Theta_j=w_j\end{displaymath}

\subsection{Konsum}
Alle Konsumenten haben identische Pr{\"a}ferenzen. Mit dem Konsum der Güter 1 und 2 steigt der Nutzen der Haushalte an.Die Nutzenfunktion eines repräsentativen Haushalts lautet: 
\begin{equation} U_j=ln(c_1)+ln(c_2)\label{NutzenAutarkie}\end{equation}
Er maximiert seinen Nutzen wie folgt ohne Ber{\"u}cksichtigung einer Budgetrestriktion: 
\begin{equation}\frac{\partial U}{\partial m}\overset{!}{=}0\end{equation} 
\begin{equation} c_1=\frac{\Theta_1}{\Theta_2}c_2\end{equation}
\begin {equation} m=0,5\end{equation}
Der Produktionsfaktor Arbeit sollte gemäß der optimale Aufteilung $m$ hälftig auf beide Sektoren verteilt werden.
Wird das verf{\"u}gbare Einkommen $y$ des Haushalts ber{\"u}cksichtigt, ergibt sich folgendes Maximierungsproblem: 
\begin{equation} maxU=\ln(c_1)+\ln(c_2)-\lambda (p_1c_1+p_2c_2-y)\end{equation}
\begin{equation}\frac{\partial U}{\partial c_1}\overset{!}{=}0\label{fo1}\end{equation}
\begin{equation} \frac{\partial U}{\partial c_2}\overset{!}{=}0\label{fo2}\end{equation}
\begin{equation} \frac{\partial U}{\partial \lambda}\overset{!}{=}0\end{equation}
Aus Gleichung \eqref{fo1} und \eqref{fo2} ergibt sich: 
\begin{equation} c_1=\frac{y}{2p_1}=d_1\end{equation}
\begin{equation} c_2=\frac{y}{2p_2}=d_2\end{equation}
Die Konsummengen $c_j$ entsprechen hierbei exakt der nachgefragten Mengen $d_j$.

\subsection{Produktion} 
Nach der Nutzenmaximierung der Konsumenten folgt nun die Gewinnmaximierung der Produzenten. 
\begin{equation} max\Pi_j=p_jc_j(m)-w_jc_j(m)\end{equation}
\begin{equation}\frac{\partial \Pi_j}{\partial c_j}\overset{!}{=}0\end{equation}
\begin{equation}p_j=w_j\end{equation}
\begin{equation}\frac{\partial \Pi_j}{\partial m}\overset{!}{=}0\end{equation}
\begin{equation}p_j=w_j\end{equation}
\begin{center}
\begin{flushright}
\fbox{\parbox{5cm}{\begin{displaymath}\frac{p_1}{p_2}=\frac{w_1}{w_2}=\frac{\Theta_1}{\Theta_2}\end{displaymath}}}\newline
\end{flushright}
\end{center}

\subsection{Marktgleichgewicht}
Die Konsumenten fragen die gesamte produzierte Menge nach.\footnote {$s_j$ steht dabei für die produzierte und somit angebotene Menge, wohingegen $d_j$ die nachgefragte Menge darstellt.}
\begin{equation} s_1=d_1\end{equation}
\begin{equation} s_2=d_2\end{equation}
Das Verh{\"a}ltnis von Produktion und Konsum ist somit ausgeglichen.

\section{offene Volkswirtschaft}
Es wird angenommen, dass der Weltmarkt nur aus zwei L{\"a}ndern/Regionen $i$ besteht, dem Inland {\dq}h{\dq} und dem Ausland {\dq}*{\dq}. Beide sind gleich gro{\ss} und mit den gleichen Rohstoffen ausgestattet. Die Konsumenten beider Regionen pr{\"a}ferieren die gleichen G{\"u}ter. Au{\ss}erdem werden Markteintrittsbarrieren vernachl{\"a}ssigt. In diesem Zwischenschritt wird noch davon ausgegangen, dass es keine Transportkosten f{\"u}r die Ein- und Ausfuhr von G{\"u}tern gibt.

\subsection{Konsum}
Da nun offene Volkswirtschaften betrachtet werden, die miteinander Handel betreiben k{\"o}nnen, wird neben der zweiten Region auch ein weiteres Konsumgut eingef{\"u}hrt. Es werden in jedem Land beide G{\"u}ter produziert $c^i_j: c^i_1; c^i_2$. Die Produktion der Konsumg{\"u}ter erfolgt lediglich durch den Einsatz des Faktors Arbeit, welche wiederum zwischen den beiden Sektoren aufgeteilt werden muss. Es gilt:
\begin{displaymath}L^i=L^i_1+L^i_2\end{displaymath}
Aufgrund der gleichen Pr{\"a}ferenzen entsprechen sich auch die Nutzenfunktionen beider L{\"a}nder und sind somit nicht verschiedenen zu der Situation einer geschlossenen Volkswirtschaft \eqref{NutzenAutarkie}.\newline 
Beide L{\"a}nder unterscheiden sich durch ihre Produktionstechnologien. Das Inland ist produktiver als das Ausland, kann also die gleiche Menge des Einsatzfaktors Arbeit ausgiebiger nutzen und einen h{\"o}heren Output generieren. Die allgemeine Form der Produktionsfunktion  lautet
\begin{displaymath}c^i_j=\Theta^i_jL^i_j \end{displaymath} 
wobei hier jetzt gilt, dass 
\begin{equation}\Theta_1^h<\Theta_1^*~\text{sowie}~\Theta_2^h>\Theta_2^*\end{equation} mit $L_1^i=m ^i  ~\text{und} L_2^i=(1-m^i)$.\newline 
Auch in einer offenen Volkswirtschaft entspricht das Grenzprodukt dem Faktorpreis und es gilt auch hier: 
\begin{displaymath}\frac{\partial c^i_j}{\partial L^i_j}=w^i_j \Rightarrow \Theta^i_j=w^i_j\end{displaymath}
Nach der {\"O}ffnung des Landes stehen alle Haushalte den gleichen G{\"u}terpreisen gegen{\"u}ber. Sie haben nicht nur die gleichen Pr{\"a}ferenzen, sondern treffen ihrer Konsumentscheidung unabh{\"a}ngig vom Produktionsort des Gutes. Die Konsumenten differenzieren demnach nicht wo das Gut produziert wurde. \newline 
Im folgenden wird die optimale Aufteilung der Arbeit auf die beiden Sektoren berechnet. \begin{equation} U^i_j=ln(c^i_1)+ln(c^i_2)\label{NutzenTrade}\end{equation}
\begin{equation}\frac{\partial U^i}{\partial m^i}\overset{!}{=}0\end{equation}
\begin{equation} c_1^i=\frac{\Theta_1^i}{\Theta_2^i}c_2^i\end{equation}
\begin{equation} m^i=0,5\end{equation}
%h{\"a}tte eher gedacht, dass $m^h=1$ und $m^*=0 $ist\newline \newline Auch in einer offenen Volkswirtschaft sollte die Arbeit h{\"a}lftig in beiden Sektoren eingesetzt werden.  ---> macht aber irgendwie keinen Sinn, wenn man am ende nur in einem Sektor Produziert....
Daraus ergibt sich in beiden Regionen das gleiche Maximierungsproblem. Der Nutzen der Haushalte wird unter Ber{\"u}cksichtigung des verf{\"u}gbaren Einkommens optimiert und man erh{\"a}lt die sektorale Nachfrage $d_j^i$. 
\begin{equation} maxU^i=\ln(c^i_1)+\ln(c^i_2)-\lambda^i (p_1c^i_1+p_2c^i_2-y^i)\end{equation}
\begin{equation}\frac{\partial U^i}{\partial c^i_1}\overset{!}{=}0\end{equation}
\begin{equation}\frac{\partial U^i}{\partial c^i_2}\overset{!}{=}0\end{equation}
\begin{equation}\frac{\partial U^i}{\partial \lambda^i}\overset{!}{=}0\end{equation}
\begin{equation} c^i_1=\frac{y^i}{2p_1}=d^i_1\end{equation}
\begin{equation} c^i_2=\frac{y^i}{2p_2}=d^i_2\end{equation}

\subsection{Produktion}
Wird die Anbieterseite betrachtet, dann führt der komparative Vorteil in jedem Land und f{\"u}r jeden Sektor zu einem anderen Maximierungsproblem der Unternehmen. 
\begin{equation} max\Pi_j=p_jc^i_j-w^i_jc^i_j\end{equation}
\begin{equation}\frac{\partial \Pi^i_j}{\partial c^i_j}\overset{!}{=}0\end{equation} 
\begin{equation}p_j=w^i_j\end{equation}
\begin{equation}\frac{\partial \Pi^i_j}{\partial m^i}\overset{!}{=}0\end{equation}
\begin{equation}p_j=w^i_j\end{equation}
\begin{flushright}
\begin{center}
\fbox{\parbox{7cm}{\begin{displaymath}\frac{p^h_1}{p^h_2}=\frac{w^h_1}{w^h_2}=\frac{\Theta^h_1}{\Theta^h_2}<\frac{p^*_1}{p^*_2}=\frac{w^*_1}{w^*_2}=\frac{\Theta^*_1}{\Theta^*_2}\end{displaymath}}}
\end{center}
\end{flushright}
Das Inland hat sich vollkommen auf die Produktion von Gut 1 spezialisiert, bedingt durch den relativ hohen Lohn in Sektor eins, aufgrund der h{\"o}heren Produktivit{\"a}t und somit dem komparativen Vorteil. Die Herstellung von Gut 2 wird im Inland eingestellt und komplett aus dem Ausland bezogen. Im Ausland verh{\"a}lt es sich entgegengesetzt. Dieses spezialisiert sich auf die Produktion von Gut 2 und importiert die gesamte nachgefragte Menge von Gut 1 aus dem Inland. 

\subsection{Marktgleichgewicht}
Auch im Handelsmodell m{\"u}ssen die angebotene und nachgefragte Menge des Marktes, hier der Weltmarkt, im Gleichgewicht {\"u}bereinstimmten. Es gilt: 
\begin{displaymath}w_1^i=p_1; w_2^i=p_2\end{displaymath}
\begin{equation} p_1^ic_1^i+p_2^ic_2^i\leq w_1^i \Theta_1 ^im^i+w_2^i\Theta_2^i(1-m^i)\end{equation}
%Ausgaben sind kleiner gleich Einnahmen?
Bei der Gegen{\"u}berstellung von Angebots und Nachfrage l{\"a}sst sich die Handelsstruktur des Modells nachweisen. In Sektor eins ist das inl{\"a}ndische Angebot gr{\"o}{\ss}er als die Nachfrage nach diesem Gut. 
\begin{displaymath}s_1^h>d_1^h\end{displaymath}
Wohingegen im Ausland mehr von Gut 1 nachgefragt, als hergestellt wird. 
\begin{displaymath} s_1^*<d_1^*\end{displaymath}
Wie im Optimierungsproblem bereits er{\"o}rtert wurde, wird Gut 1 ausschlie{\ss}lich vom Inland produziert, demnach muss gelten, dass:
\begin{displaymath}s_1^h=d_1^h+d_1^* ~\text{und}~ s_1^*=0\end{displaymath}
Es ergibt sich das Gut 1 vom Inland exportiert wird und somit die Importe des Auslands darstellen. 
\begin{displaymath}Ex = d_1^*=Im^*\end{displaymath}
Die Analyse f{\"u}r Sektor zwei ist sehr {\"a}hnlich. Hier ist im Inland die Nachfrage nach diesem Gut kleiner als das Angebot und im Ausland herrscht ein {\"U}berschussangebot nach Gut 2.
\begin{displaymath}s_2^h<d_2^h\end{displaymath}
\begin{displaymath} s_2^*>d_2^*\end{displaymath}
Im Inland wird Gut 2 nicht produziert, sondern nur im Ausland
\begin{displaymath}s_2^*=d_2^h+d_2^*~ \text{und} ~s_2^h=0\end{displaymath}
Das Ausland exportiert Gut 2 in das Inland. Das Inland importiert Gut 2. 
\begin{displaymath}Ex*=d_2^h=Im\end{displaymath}
Auf Weltmarkt bedeutet dies f{\"u}r die Anbieter und Nachfrager, das gilt: 
\begin{displaymath}s_1^h>s_1^*\end{displaymath}
\begin{displaymath}s_2^*>s_2^h\end{displaymath}
\begin{displaymath}d_1^h=d_1^*\end{displaymath}
\begin{displaymath}d_2^*=d_2^h\end{displaymath}
Au{\ss}erdem gilt auch hier wieder, dass alle produzierten G{\"u}ter auch konsumiert werden. 
\begin{equation}s_1^h+s_1^*+s_2^h+s_2^*=d_1^h+d_1^*+d_2^h+d_2^*\end{equation}
Damit sich das Modell im Gleichgewicht befindet muss der Wert aller exportierten G{\"u}ter dem der importierten entsprechen.

\section{offene Volkswirtschaft mit Transportkosten}
Dem im vorherigen Kapitel beschriebenem Handelsmodell werden jetzt Transportkosten hinzugef{\"u}gt. Diese Modellvariation zeigt die allgemeine Wirkung handelserschwerender Ma{\ss}nahmen und l{\"a}sst Aussagen {\"u}ber politischen Eingriffe zu.\newline 
Es werden Iceberg-Transportkosten zu Grunde gelegt.\newline 
\textcolor[rgb]{1,0,0}{Definition iceberg Transportkosten.}\\
Sie werden nicht in Geld- sondern in G{\"u}tereinheiten bemessen. Die Transportkosten $\tau_j^i$ bewirken, dass ein geringerer Anteil der exportierten G{\"u}ter in dem Zielland eintrifft und Kosten in H{\"o}he {\dq}verlorener{\dq} G{\"u}tereinheiten entstehen. Um eine bestimmte Menge zu exportieren muss demnach mehr produziert werden, damit nach Abzug der Kosten diese Menge auch tats{\"a}chlich ankommt. In dem hier formulierten Fall werden nur Transportkosten in Sektor zwei des Auslandes angenommen. $\tau_1^* = \tau_1^h = \tau_2^h = 0 ~\text{und} ~\tau=\tau_2^*>0$.

\subsection{Konsum}
Dies f{\"u}hrt zu einem angepassten Optimierungsproblem der Konsumenten: 
\begin{equation} maxU^i=\ln(c^i_1)+\ln(c^i_2)-\lambda^i (p_1c^i_1+p_2(1+\tau)c^i_2-y^i)\end{equation}
\begin{equation}\frac{\partial U^i}{\partial c^i_1}\overset{!}{=}0\end{equation}
\begin{equation} \frac{\partial U^i}{\partial c^i_2}=\overset{!}{=}0\end{equation}
\begin{equation} \frac{\partial U^i}{\partial \lambda^i}\overset{!}{=}0\end{equation}
\begin{equation} c^i_1=\frac{y^i}{2p_1}=d^i_1\end{equation}
\begin{equation} c^i_2=\frac{y^i}{2p_2(1+\tau)}=d^i_2\label{Nachfrage mit TK}\end{equation}
Gleichung \eqref{Nachfrage mit TK} zeigt, das mit steigenden Transportkosten die Nachfrage nach Gut 2 sinkt. Dies kann im Extremfall dazu f{\"u}hren, dass ein Transport per se nicht mehr rentabel ist und Gut 2 zu einem nicht-handelbarem Gut wird.

\subsection{Produktion}
Das Maximierungsproblem der Unternehmen muss ebenfalls angepasst werden, um die Transportkosten zu ber{\"u}cksichtigen.
\begin{equation} max\Pi_j=p_jc^i_j-w^i_j (1+\tau) c^i_j\end{equation}
\begin{equation}\frac{\partial \Pi^i_j}{\partial c^i_j}\overset{!}{=}0\end{equation}
\begin{equation}p_j=w^i_j(1+\tau)\end{equation}
\begin{equation}\frac{\partial \Pi^i_j}{\partial m^i}\overset{!}{=}0\end{equation}
\begin{equation}p_j=w^i_j (1+\tau)\end{equation}
\begin{flushright}
\begin{center}
\fbox{\parbox{9cm}{\begin{displaymath}\frac{p^h_1}{p^h_2}=\frac{w^h_1}{w^h_2}=\frac{\Theta^h_1}{\Theta^h_2}<\frac{p^*_1}{p^*_2}=\frac{w^*_1}{w^*_2(1+\tau)}=\frac{\Theta^*_1}{\Theta^*_2}\end{displaymath}}}
\end{center}
\end{flushright}
\textcolor[rgb]{1,0,0}{eventuell doch $\tau_j^i$ verwenden und in der Fu{\ss}note nochmal die Einschr{\"a}nkung bringen?}
Transportkosten f{\"u}hren im Allgemeinen nicht zu der Verlagerung des komparativen Vorteils. Dieser kann jedoch gr{\"o}{\ss}er oder kleiner ausfallen. In dem hier dargestellten Fall f{\"u}hren Transportkosten zu einem kleineren komparativen Vorteil des Auslandes in Sektor zwei. Die L{\"o}hne unterscheiden sich noch und in Sektor zwei ist dieser noch immer geringer als in Sektor eins. Jedoch hat sich die relative Differenz beider Ver{\"a}ndert. Die Handelsstruktur ver{\"a}ndert sich erst dann, wenn $w_1^h/w_2^h>w_1^*/w_2^*(1+\tau)$. In diesem Fall würde das Inland nur Gut 2 produzieren und das Ausland w{\"u}rde sich vollst{\"a}ndig auf die Produktion von Gut 1 spezialisieren.

\subsection{Marktgleichgewicht} 
Das Marktgleichgewicht ist genau dann ausgeglichen, wenn gilt, dass 
\begin{equation} p_1^ic_1^i+p_2^ic_2^i\leq w_1^i\Theta_1 ^im^i+w_2^i(1+\tau)\Theta_2^i(1-m^i)\end{equation} 
\begin{displaymath}\text{mit}\quad w_1^i=p_1; w_2^i=p_2,\end{displaymath} so dass letztlich ein Gleichgewicht zwischen Produktion und Konsum besteht: 
\begin{equation}s_1^h+s_1^*+s_2^h+s_2^*=d_1^h+d_1^*+d_2^h+d_2^*\end{equation}







%\end{document}