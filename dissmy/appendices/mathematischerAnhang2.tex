\chapter[Mathematischer Anhang zu Kapitel \ref{Papier1}]{Mathematischer Anhang \\ zu Kapitel \ref{Papier1}} 
\chaptermark{Appendix}
\section[Gewinnmaximale Preis-Mengen-Kombination ]{Gewinnmaximale Preis-Mengen-Kombination \sectionmark{Preis-Mengen-Kombination}}\label{sec:Appendix-Gewinn}



 Die Produktionsfunktion \eqref{Produktionsfunktion} wird nach der einzusetzenden Menge an Zwischengütern $\nu$ abgeleitet
 
 
\begin{equation}
\frac{\partial y_j}{\partial x_{tj}(\nu)}\overset{!}{=}0
\end{equation} 


Gemä{\ss} der Grenzproduktivitätsentlohnung  und  unter Berücksichtigung von \eqref{Grenzproduktentlohnung} erhält man dann folgenden Limit-Preis:


\begin{equation}
\boxed{\chi_j=\left(\frac{A(\nu)N_{tj}}{x_{tj}(\nu)}\right)^{1-\alpha_j}}\label{LimitPreis}
\end{equation}


Wird dieser nach $x_{tj}(\nu)$ umgestellt, erhält man die gleichgewichtige Nachfrage $x(\nu)$ nach Zwischengütern.


\begin{equation*}
\chi_j^{\frac{1}{1-\alpha_j}}=\frac{A(\nu)N_{tj}}{x_{tj}(\nu)}
\end{equation*}


\begin{equation}
x_{tj}(\nu)=\chi_j^{-\frac{1}{1-\alpha_j}}A_t(\nu)N_{tj}\label{gleichgewichtige Nachfrage}
\end{equation}


Daraus lässt sich wiederum der gleichgewichtige Gewinn in den Zwischengütersektoren berechnen. Ausgehend von


\begin{equation*}
\pi_{tj}(\nu)=p_t(\nu)x_{tj}(\nu)-x_{tj}(\nu)
\end{equation*}


Wird Gleichung \eqref{Grenzproduktentlohnung} und \eqref{gleichgewichtige Nachfrage} eingesetzt, dann erhält man


\begin{equation}
\pi_{tj}(\nu)=[\chi_j-1]A(\nu)N_{tj}\chi_j^{-\frac{1}{1-\alpha_j}}
\end{equation}


Die Intensität der Konkurrenz lässt sich an dem Gewinn pro Unternehmen ablesen. Mit steigender Anbieterzahl sinkt dieser und kann dargestellt werden als:


\begin{equation}
\frac{\pi_{tj}(\nu)}{A(\nu)N_{tj}}=[\chi_j-1]\chi_j^{-\frac{1}{1-\alpha_j}}
\end{equation}


\begin{equation*}
\delta_j= \chi_j^{-\frac{1}{1-\alpha_j}}[\chi_j-1] \qquad  \text{mit } \qquad \chi_j\leq\frac{1}{\alpha_j}
\end{equation*}


\begin{equation}
\pi_{tj}(\nu)=\delta_jA(\nu)N_{tj}
\end{equation}


Die aggregierte produzierte Menge des Endprodukts berechnet sich aus \eqref{Produktionsfunktion}, \eqref{gleichgewichtige Nachfrage} und $A_t\equiv \int \limits_{o}^\infty{A_t(\nu)}d\nu$ (noch die Grenzen des Integrals nachtragen)


\begin{equation*}
y_{tj}=\frac{1}{\alpha_j}N_{tj}^{1-\alpha_j}A(\nu)^{1-\alpha_j}\left(A(\nu)N_{tj}\chi_j^{-\frac{1}{1-\alpha_j}}\right)^{\alpha_j}
\end{equation*}


\begin{equation*}
y_{tj}=\frac{1}{\alpha_j}N_{tj}^{1-\alpha_j}A(\nu)^{1-\alpha_j} A(\nu)^{\alpha_j}N_{tj}^{\alpha_j}\chi_j^{-\frac{\alpha_j}{1-\alpha_j}}
\end{equation*}


\begin{equation}
\boxed{y_{tj}=\frac{1}{\alpha_j}N_{tj}A(\nu)\chi_j^{-\frac{\alpha_j}{1-\alpha_j}}}
\end{equation}

\section{Gleichgewichtiger Lohnsatz}\label{sec:Appendix-Lohnsatz}


Der gleichgewichtige Lohnsatz bestimmt sich, indem die Produktionsfunktion aus Gleichung \eqref{Produktionsfunktion} unter der Restriktion der Kosten nach den Produktionsfaktoren abgeleitet wird.


\begin{equation*}
\max{\mathcal{L}}=\frac{1}{\alpha_j}N_{tj}^{1-\alpha_j}\left(\int{A(\nu)^{1-\alpha_j}x_{tj}(\nu)^{\alpha_j}d\nu}\right)+\lambda[C_j-w_jN_{tj}-\chi_{j}x_{tj}(\nu)]
\end{equation*} 


\begin{equation*}
\frac{\partial \mathcal{L}}{\partial N_{tj}} \overset{!}{=} 0
\end{equation*}


\begin{equation*}
\frac{1}{\alpha_j}(1-\alpha_j)N_{tj}^{-\alpha_j}A(\nu)^{1-\alpha_j}x_{tj}(\nu)^{\alpha_j}-\lambda w_j = 0
\end{equation*}


\begin{equation}
\frac{1-\alpha_j}{\alpha_j}N_{tj}^{-\alpha_j}A(\nu)^{1-\alpha_j}x_{tj}(\nu)^{\alpha_j}=\lambda w_j
\end{equation}


\begin{equation*}
\frac{\partial \mathcal{L}}{\partial x_{tj}(\nu)} \overset{!}{=} 0
\end{equation*}


\begin{equation*}
\frac{1}{\alpha_j}\alpha_jN_{tj}^{1-\alpha_j}A(\nu)^{1-\alpha_j}x_{tj}(\nu)^{\alpha_j-1}-\lambda \chi_j = 0
\end{equation*}


\begin{equation}
N_{tj}^{1-\alpha_j}A(\nu)^{1-\alpha_j}x_{tj}(\nu)^{\alpha_j-1}=\lambda \chi_j
\end{equation}


\begin{equation}
\frac{w_j}{\chi_j}=\frac{\frac{1-\alpha_j}{\alpha_j}A(\nu)^{1-\alpha_j}x_{tj}(\nu)^{\alpha_j} N_{tj}^{-\alpha_j}}{A(\nu)^{1-\alpha_j}x_{tj}(\nu)^{\alpha_j-1}N_{tj}^{1-\alpha_j}}
\end{equation}


Werden beide Grenzprodukte zusammengefasst, erhält man das Faktorpreisverhältnis. 


\begin{equation}
\frac{w_j}{\chi_j}=\frac{\frac{1-\alpha_j}{\alpha_j}x_{tj}(\nu)}{N_{tj}}
\end{equation}


Dieser Ausdruck kann nach $x_{tj}(\nu)$ oder $N_{tj}$ umgestellt werden, um es an späterer Stelle einsetzen zu können

 
\begin{equation}
x_{tj}(\nu)=\frac{w_jN_{tj}\alpha_j}{\chi_j(1-\alpha_j)}\label{nach x}
\end{equation}


\begin{equation}
N_{tj}=\frac{\frac{1-\alpha_j}{\alpha_j}x_{tj}(\nu)\chi_j}{w_j} \label{nach N}
\end{equation}


\begin{equation*}
\frac{\partial \mathcal{L}}{\partial \lambda}\overset{!}{=} 0
\end{equation*}


\begin{equation*}
C_j-w_jN_{tj}-\chi_jx_{tj}(\nu)=0
\end{equation*}


\begin{equation}
C_j=w_jN_{tj}+\chi_jx_{tj}(\nu) \label{nach C}
\end{equation}


In diesem Ausdruck \eqref{nach C} wird Gleichung \eqref{nach x} einsetzen.


\begin{equation*}
C_j=w_jN_{tj}+\chi_j\frac{w_jN_{tj}\alpha_j}{\chi_j(1-\alpha_j)}
\end{equation*}


\begin{equation*}
C_j=w_jN_{tj}\left(1+\frac{\alpha_j}{(1-\alpha_j)}\right)
\end{equation*}


\begin{equation*}
C_j=w_jN_{tj}\frac{1}{(1-\alpha_j)}
\end{equation*}


Nach $N_{tj}$ umgestellt erhalten wir die gleichgewichtige Arbeiterzahl


\begin{equation}
\boxed{N_{tj}=\frac{C_j(1-\alpha_j)}{w_j}}\label{GGArbeiter}
\end{equation}


Beim Zwischengütermarkt wird anschlie{\ss}end Gleichung \eqref{nach N} in Gleichung \eqref{nach C} eingesetzt.


\begin{equation*}
C_j=w_j\frac{\frac{1-\alpha_j}{\alpha_j}x_{tj}(\nu)\chi_j}{w_j}+\chi_jx_{tj}(\nu)
\end{equation*}


\begin{equation*}
C_j=x_{tj}(\nu)\chi_j\left(1+\frac{1-\alpha_j}{\alpha_j}\right)
\end{equation*}


\begin{equation*}
C_j=x_{tj}(\nu)\chi_j\frac{1}{\alpha_j} 
\end{equation*}


Dies wiederum nach $x_{tj}(\nu)$ umgestellt ergibt die gleichgewichtige Anzahl an Zwischengütern. 


\begin{equation}
\boxed{x_{tj}(\nu)=\frac{C_j\alpha_j}{\chi_j}}\label{ZwischenGG}
\end{equation}


Werden jetzt beide gleichgewichtigen Mengen \eqref{GGArbeiter} und \eqref{ZwischenGG} in die Produktionsfunktion \eqref{Produktionsfunktion} eingesetzt ergibt sich die Produktionsmenge: 


\begin{equation*}
y_j=\frac{1}{\alpha_j}\left(\frac{C_j(1-\alpha_j)}{w_j}\right)^{1-\alpha_j}A(\nu)^{1-\alpha_j}\left(\frac{C_j\alpha_j}{\chi_j}\right)^{\alpha_j}
\end{equation*}


\begin{equation*}
y_j=\frac{1}{\alpha_j}\frac{C_j^{1-\alpha_j}(1-\alpha_j)^{1-\alpha_j}}{w_j^{1-\alpha_j}}A(\nu)^{1-\alpha_j}\frac{C_j^{\alpha_j}\alpha_j^{\alpha_j}}{\chi_j^{\alpha_j}}
\end{equation*}


\begin{equation}
y_j=\frac{1}{\alpha_j}C_j(1-\alpha_j)^{1-\alpha_j}\alpha_j^{\alpha_j}w_j^{\alpha_j-1}A(\nu)^{1-\alpha_j}\chi_j^{-\alpha_j}\label{alles in y eingesetzt}
\end{equation}


Diese wird wiederum in die Kostenfunktion $C_j=c_jy_j$ eingesetzt, um die Grenzkosten zu berechnen. 


\begin{equation*}
y_j=\frac{1}{\alpha_j}c_jy_j(1-\alpha_j)^{1-\alpha_j}\alpha_j^{\alpha_j}w_j^{\alpha_j-1}A(\nu)^{1-\alpha_j}\chi_j^{-\alpha_j}\
\end{equation*}


\begin{equation*}
1=\alpha_j^{\alpha_j-1}(1-\alpha_j)^{1-\alpha_j}w_j^{\alpha_j-1}A(\nu)^{1-\alpha_j}\chi_j^{-\alpha_j}c_j
\end{equation*}


\begin{equation}
c_j=\alpha_j^{1-\alpha_j}(1-\alpha_j)^{\alpha_j-1}w_j^{1-\alpha_j}A(\nu)^{\alpha_j-1}\chi_j^{\alpha_j}\label{GK}
\end{equation}


Nachdem das Grenzprodukt für Zwischengüter bereits berechnet wurde wird jetzt für die Lohnsatzbestimmung das Grenzprodukt der Arbeit gebildet. Die Produktionsfunktion \eqref{Produktionsfunktion} wird nach $N_j$ abgeleitet. 
$w_j=WGP_j \Longleftrightarrow w_j=p_jGP_j$


\begin{equation}
\frac{\partial y_j}{\partial N_j}\overset{!}{=}0
\end{equation}


\begin{equation}
\frac{1}{\alpha_j}(1-\alpha_j)N_j^{1-\alpha_j-1}A(\nu)^{1-\alpha_j}x_{tj}^{\alpha_j}=0\label{GPArbeit}
\end{equation}


Gemä{\ss} der Gewinnmaximierungsbedingung $p_j=c_j$ ergibt sich durch das Einsetzen von Gleichung \eqref{GK} 


\begin{equation}
p_j=\alpha_j^{1-\alpha_j}(1-\alpha_j)^{\alpha_j-1}w_j^{1-\alpha_j}A(\nu)^{\alpha_j-1}\chi_j^{\alpha_j}\label{Guterpreis}
\end{equation}


Im Gewinnmaximum entspricht au{\ss}erdem der Faktorpreis dem Wertgrenzprodukt und somit entspricht der Lohnsatz $w-j$ dem Grenzprodukt der Arbeit laut Gleichung \eqref{GPArbeit} multipliziert mit dem Güterpreis $p_j$ gemä{\ss} Gleichung \eqref{Guterpreis} gleichgesetzt. 


\begin{equation}
w_j=\frac{1}{\alpha_j}(1-\alpha_j)N_j^{-\alpha_j}A(\nu)^{1-\alpha_j}x_{tj}(\nu)^{\alpha_j}\alpha_j^{1-\alpha_j}(1-\alpha_j)^{\alpha_j-1}w_j^{1-\alpha_j}A(\nu)^{\alpha_j-1}\chi_j^{\alpha_j}
\end{equation}


\begin{equation*}
w_j=N_j^{-\alpha_j}x_{tj}(\nu)^{\alpha_j}\alpha_j^{-\alpha_j}(1-\alpha_j)^{\alpha_j}w_j^{1-\alpha_j}\chi_j^{\alpha_j}
\end{equation*}


\begin{equation*}
w_j^{1-1+\alpha_j}=\left(\frac{1-\alpha_j}{\alpha_j}\right)^{\alpha_j}\chi_j^{\alpha_j}N_j^{-\alpha_j}x_{tj}(\nu)^{\alpha_j}
\end{equation*}


\begin{equation*}
w_j=\frac{1-\alpha_j}{\alpha_j}\chi_jN_j^{-1}x_{tj}(\nu)
\end{equation*}


\begin{equation}
w_j=\frac{1-\alpha_j}{\alpha_j}\frac{\chi_jx_{tj}(\nu)}{N_j}
\end{equation}


Mit der gleichgewichtigen Nachfrage nach Zwischengütern \eqref{gleichgewichtige Nachfrage}ergibt sich zunächst


\begin{equation}
w_j=\frac{1-\alpha_j}{\alpha_j}\frac{\chi_j\chi_j^{-\frac{1}{1-\alpha_j}}A(\nu)N_j}{N_j}
\end{equation}


\begin{equation}
w_j=\frac{1-\alpha_j}{\alpha_j} \chi_j^{-\frac{1}{1-\alpha_j}}A(\nu)
\end{equation}


und mit dem gewinnmaximalen Preis $\chi_j=p_j(\nu)$ entnommen aus Gleichung \eqref{LimitPreis}, lässt sich der sektorspezifische Lohnsatz herleiten. 


\begin{equation*}
w_j=\frac{1-\alpha_j}{\alpha_j} \left(\left(\frac{A(\nu)N_j}{x_{tj}(\nu)}\right)^{1-\alpha_j}\right)^{-\frac{1}{1-\alpha_j}}A(\nu)
\end{equation*}


\begin{equation*}
w_j=\frac{1-\alpha_j}{\alpha_j}\frac{A(\nu)^{-\alpha_j}N_j^{-\alpha_j}}{x_{tj}(\nu)^{-\alpha_j}}A(\nu)
\end{equation*}


\begin{equation}
\boxed{w_j=\frac{1-\alpha_j}{\alpha_j}A(\nu)^{1-\alpha_j}N_j^{-\alpha_j}x_{tj}(\nu)^{\alpha_j}}
\end{equation}


Berechnung des gleichgewichtigen Lohnsatzes zwischen beiden Sektoren ergibt sich aus der Summe der einzelnen Sektoren. 


\begin{equation}
N_1+N_{2}=N=1
\end{equation}


\begin{equation}
N_1=1-N_{2}
\end{equation}


Beide Lohnsätze werden miteinander gleichgesetzt.


\begin{equation}
w_1=w_{2}
\end{equation}


\begin{equation*}
\frac{1-\alpha_I}{\alpha_I}A(\nu)^{1-\alpha_I}(1-N_{II})^{-\alpha_I}x_{tI}(\nu)^{\alpha_I}=\frac{1-\alpha_{II}}{\alpha_{II}}A(\nu)^{1-\alpha_{II}}N_{II}^{-\alpha_{II}}x_{tII}(\nu)^{\alpha_{II}}
\end{equation*}


\begin{equation}
\boxed{\frac{\frac{1-\alpha_I}{\alpha_I}A(\nu)^{1-\alpha_I}x_{tI}(\nu)^{\alpha_I}}{\frac{1-\alpha_{II}}{\alpha_{II}}A(\nu)^{1-\alpha_{II}}x_{tII}(\nu)^{1-\alpha_{II}}}=\frac{N_{II}^{-\alpha_{II}}}{(1-N_{II})^{-\alpha_I}}}
\end{equation}


\section{Abstand zur WTG eines Landes}\label{sec:Abstand WTG}


Der Abstand zur Welttechnologiegrenze wird definiert als relative Lage der LTG zur WTG


\begin{equation}
a_{tj}=\frac{A_{tj}}{\overline{A}_{tj}}\label{DistanzGrob}
\end{equation}


Dabei gilt, dass die Produktivität eines Landes, die LTG, zu gleichen Teilen aus jungen und alten Unternehmen besteht. 


\begin{equation*}
A_{tj}=\frac{A_{tj}^y+A_{tj}^o}{2} \qquad \text{und für die WTG gilt}\qquad \overline{A}_{tj}= \overline{A}_{t-1j}(1-g)
\end{equation*}


Werden beide Ausdrücke in \eqref{DistanzGrob} eingesetzt erhält man eine detailierte Aufschlüsselung über die Produktivitäten der Unternehmensarten.  


\begin{equation}
a_{tj}=\frac{A_{tj}^y+A_{tj}^o}{2\overline{A}_{t-1j}(1-g)}\label{WTGgrob}
\end{equation}


Die Produktivität junger Ingenieure beträgt:

 
\begin{equation}
A_{tj}^{y}=\sigma_j(\eta\overline{A}_{t-1}+\lambda\gamma A_{t-1})\label{jung}
\end{equation}


Die Produktivität eines Unternehmens, wenn alle Ingenieure in dem Unternehmen verbleiben und die älteren weniger qualifizierten nicht ausgetauscht werden, also $(R_{tj}=1)$, beträgt: 


\begin{equation}
A_{tj}^{o}(R_{tj}=1)=\eta\overline{A}_{t-1}+\lambda\gamma A_{t-1}\label{behalten}
\end{equation}


Setzt man \eqref{jung} und \eqref{behalten} in \eqref{WTGgrob} ein, dann erhält man:


\begin{equation}
a_{tj}(R_{t}=1)=\frac{\sigma_j(\eta\overline{A}_{t-1}+\lambda\gamma A_{t-1})+(\eta\overline{A}_{t-1}+\lambda\gamma A_{t-1})}{2\overline{A}_{t-1j}(1-g)}
\end{equation}


Die gegenwärtige WTG wird auf $1$ normiert, $\overline{A}_{tj}=1$, und es gilt, dass $A_{t-1j}=a_{t-1j}$.


\begin{equation}
a_{tj}(R_{t}=1)=\frac{\sigma_j(\eta+\lambda\gamma a_{t-1})+(\eta + \lambda\gamma a_{t-1})}{2(1-g)}
\end{equation}  


Dies kann auch umformuliert werden als:


\begin{equation}
\boxed{a_{tj}(R_{t}=1)=\frac{1+\sigma_j}{2(1+g)}[\eta+\lambda \gamma a_{t-1}]}
\end{equation}


Werden die älteren weniger qualifizierten Ingenieure ausgetauscht und ersetzt, $(R_{tj}=0)$, dann liegt der Abstand zur WTG bei: 


\begin{equation}
A_{tj}^o(R_{tj}=0)=\lambda(\eta\overline{A}_{t-1}+\gamma A_{t-1})+(1-\lambda)\sigma_j(\eta\overline{A}_{t-1}+\lambda\gamma A_{t-1})\label{austausch}
\end{equation}


Dies wiederum zusammen eingesetzt mit \eqref{jung} in \eqref{WTGgrob} ergibt:


\begin{equation}
a_{tj}(R_{t}=0)=\frac{\sigma_j(\eta\overline{A}_{t-1}+\lambda\gamma A_{t-1})+\lambda(\eta\overline{A}_{t-1}+\gamma A_{t-1})+(1-\lambda)\sigma_j(\eta\overline{A}_{t-1}+\lambda\gamma A_{t-1})}{2\overline{A}_{t-1j}(1-g)}
\end{equation}


Nach der Normierung $\overline{A}_{tj}=1$ und Substitution von $A_{t-1j}=a_{t-1j}$ lautet dies:


\begin{equation}
a_{tj}(R_{t}=0)=\frac{\sigma_j(\eta+\lambda\gamma a_{t-1})+\lambda(\eta+\gamma a_{t-1})+(1-\lambda)\sigma_j(\eta+\lambda\gamma a_{t-1})}{2(1-g)}
\end{equation}


Erneut umformuliert ergibt sich: 


\begin{equation}
\boxed{a_{tj}(R_{t}=0)=\frac{1}{2(1-g)}\left[\lambda+\sigma_j+(1-\lambda)\sigma_j)\eta+(1+\sigma_j+(1-\lambda)\sigma_j)\lambda\gamma a_{t-1}\right]}
\end{equation}


Beide möglichen Abstände zur WTG zusammengefasst:


\begin{equation}
a_{tj}=\begin{cases}\frac{1+\sigma_j}{2(1+g)}[\eta+\lambda \gamma a_{t-1}] \hfill\text{if  } R_{tj}=1 \\
\frac{1}{2(1+g)}[(\lambda+\sigma_j+(1-\lambda)\sigma_j)\eta+(1+\sigma_j+(1-\lambda)\sigma_j)\lambda\gamma a_{t-1 j}] \quad\hfill\text{if   }R_{tj}=0
\end{cases}
\end{equation}


\section{Berechnung des Schwellenwerts $a_{rj}$}


\sectionmark{Schwellenwert $a_{rj}$}
Für die Herleitung von $a_{rj}$ wird der Nutzen mit dem erwarteten Nutzen gleichgesetzt.
\begin{equation}
V_{tj}(v| s=1, e=O, z=L) = E_{t}V_{tj}(v| s= \sigma_j, e=y)
\end{equation}


Der gegenwärtige Nutzen $V_{tj}$ gro{\ss}er Projekte mit älteren Ingenieuren, die über relativ wenig Fähigkeiten verfügen beläuft sich auf: 


\begin{equation}
V_{tj}(v|s=1, e=O, z=L) = \left [(1- \mu)\delta_{j} N_{j} \eta \bar{A}_{t-1j}- max (\kappa_{j} \bar{A}_{t-1j} -RE_{t}, 0) \right]
\end{equation}


Ersetzt man die Gewinnrücklage mit $RE_{t} =\frac{1+r}{1+g_{j}} \sigma_{j} \mu \delta_{j} N_{j} \mu \bar{A}_{t-1j}$, dann erhält man für den gegenwärtigen Nutzen:


\begin{equation}
V_{tj}(v| s=1, e=O, z=L) = \left [ (1- \mu) \delta_{j} N_{j} \eta \bar{A}_{t-1j}-max\left (\kappa_{j} \bar{A}_{t-1j}- \frac{1+r}{1+g_{j}}\sigma_{j} \mu \delta_{j}N_{j} \eta \bar{A}_{t-1j}, 0\right) \right ]
\end{equation}


Der erwartete Nutzen $E_{t}V_{tj}$, eines jungen Ingenieurs, dessen Fähigkeiten noch unbekannt sind und somit nur kleine Projekte bearbeitet entspricht: 


\begin{equation}
E_{t}V_{tj}= (1- \mu) \delta_{j} N_{j} \sigma_{j}(\eta + \lambda \gamma a_{t-1j}) \bar{A}_{t-1j} - \phi \kappa_{j} \bar{A}_{t-1j}
\end{equation}


Es wird die Welttechnologiegrenze der vergangenen Periode auf 1 normiert, $\bar{A}_{t-1j} = 1$, um den Grenzwert $ a_{t-1} = a_r $ zu erhalten.


\begin{equation}
V_{tj}(v| s=1, e=O, z=L) =\left [ (1- \mu) \delta_{j} N_{j} \eta - \kappa_{j} + \frac{1+r}{1+g_{j}} \sigma_{j} \mu \delta_{j} N_{j} \eta \right]
\end{equation}


\begin{equation}
E_{t}V_{tj}(v| s=\sigma, e=Y) = (1- \mu) \delta_{j} N_{j} \sigma_{j} (\eta + \lambda \gamma a_{r_{j}}) - \phi \kappa_{j}
\end{equation}


Werden nun beide gleich gesetzt erhält man: 


\begin{equation}
\left [ (1-\mu) \delta_{j} N_{j} \eta - \kappa_{j} + \frac{1+r}{1+g_{j}} \sigma_{j} \mu \delta_{j}N_{j} \eta  \right ]= (1- \mu) \delta_{j} N_{j} \sigma_{j} (\eta + \lambda \gamma a_{r_{j}}) - \phi \kappa_{j}
\end{equation}


Die folgenden Schritte dokumentieren die Auflösung nach dem Wert $a_{rj}$.


\begin{equation}
\left((1-\mu) +\frac{1+r}{1+g_{j}} \sigma_{j} \mu \right) \delta_{j} N_{j} \eta - \kappa_{j} + \phi \kappa_{j} = (1- \mu) \delta_{j} N_{j} \sigma_{j} \eta + (1- \mu) \delta_{j} N_{j} \sigma_{j} \lambda \gamma a_{r_{j}}
\end{equation}


\begin{equation}
\left((1-\mu) +\frac{1+r}{1+g_{j}} \sigma_{j} \mu \right) \delta_{j} N_{j} \eta - (1- \mu) \delta_{j} N_{j} \sigma_{j} \eta + \kappa_{j}( \phi -1)= (1- \mu) \delta_{j} N_{j} \sigma_{j} \lambda \gamma a_{r_{j}}
\end{equation}


\begin{equation}
\left [ (1- \mu) (1- \sigma _{j}) + \frac{1+r}{1+g_{j}} \sigma_{j} \mu \right ] \delta_{j} N_{j} \eta - \kappa_{j}(1- \phi) =(1- \mu) \delta_{j} N_{j} \sigma_{j} \lambda \gamma a_{r_{j}}
\end{equation}


\begin{equation} 
a_r{_j}(\mu,\delta)=\frac{[(1-\mu)(1-\sigma_j)+\frac{1+r}{1+g}\mu\sigma_j]\eta-\frac{\kappa(1-\phi)}{\delta N_j}}{(1-\nu)\sigma_j\lambda\gamma}
\end{equation}


\section[Abhängigkeit des Schwellenwerts $a_{rj}$ von der Projektgrö{\ss}e $\sigma$]{Abhängigkeit des Schwellenwerts $a_{rj}$\\ von der Projektgrö{\ss}e $\sigma$ \sectionmark{Projektgrö{\ss}e $\sigma$ und Schwellenwert $a_{rj}$}}\label{SchwellenwertAr}
\sectionmark{Einfluss Projektgrö{\ss}e $\sigma$ auf den Schwellenwert $a_{rj}$}


Dieser Abschnitt zeigt wie der Schwellenwert $a_r$ von der Projektgrö{\ss}e $\sigma$ abhängt. Dafür wird $a_{rj}$ nach $\sigma$ abgeleitet, um die Veränderung zu zeigen.


\begin{equation}
\frac{da_{rj}}{d \sigma}= \frac{\partial a_{rj}}{\partial \sigma}+ \frac{\partial a_{rj}}{\partial g}*\frac{\partial g}{\partial \sigma} + \frac{\partial a_{rj}}{\partial \delta}\frac{\partial \delta}{\partial \sigma}
\end{equation}


\begin{equation}
\begin{split}
\frac{(\eta + \lambda \gamma) (1-\mu)+ \eta\left(-1+ \mu - \frac{2(1+ i)[\eta(2- \lambda)+(2- \lambda) \lambda \gamma] \mu  \sigma}{(\lambda \gamma[1+ \sigma + ( 1 - \lambda) \sigma] +\eta [\lambda + \sigma + (1- \lambda)\sigma])^{2}} + \frac{2(1+i)\mu}{\lambda \gamma [1+ \sigma + (1- \lambda)\sigma]+ \eta[\lambda+ \sigma +(1- \lambda) \sigma]}\right)} {\gamma \lambda(1- \mu) \sigma}\\
- \frac{-(\eta + \lambda \gamma)(1 -  \mu)(1- \sigma)+ \eta\left((1- \mu)(1- \sigma)+ \frac{2(1+i) \mu \sigma}{\lambda \gamma [1+ \sigma + (1- \lambda) \sigma ]+\eta[ \lambda + \sigma + (1- \lambda) \sigma]}\right) } {\gamma \lambda (1- \mu) \sigma^{2}}
\end{split}
\end{equation}


Durch Vereinfachung und Umformulierung erhält man:


\begin{equation}
- \frac{\frac{\eta^{2}[-4 +2\lambda + i(-4 + 2\lambda)]\mu\sigma^{2}}{\left(\eta[\lambda(-1 + \sigma)- 2\sigma]+\lambda\gamma[-1+(-2+\lambda)\sigma]\right)^{2})} + \lambda \gamma \left[1 + \eta [-1 + \frac{\eta (-4+2\lambda + i (-4+2 \lambda)] \sigma^{2}} {(\eta [ \lambda (-1+ \sigma)- 2\sigma]+ \lambda \gamma [-1+ (-2+ \lambda) \sigma])^{2}})\right] }{\gamma \lambda (-1 + \mu) \sigma^{2}}
\end{equation}


Eine erneute Ableitung zeigt, um welche Art von Extremwert es sich handelt.


\begin{equation}
\begin{split}
\frac{\eta\left(\frac{4 (1+ i)[\eta(2- \lambda)+ (2- \lambda) \lambda \gamma]^{2} \mu \sigma}{\left(\lambda \gamma[1 + \sigma + (1 - \lambda) \sigma] + \eta [\lambda + \sigma + (1- \lambda)\sigma]\right)^{3}}- \frac{4(1+i)[\eta ( 2- \lambda)+ (2- \lambda) \lambda \gamma] \mu}{\left(\lambda \gamma [1+ \sigma + (1- \lambda) \sigma] + \eta [\lambda + \sigma + (1- \lambda) \sigma]\right)^{2}}\right)}{\gamma \lambda (1- \mu) \sigma}\\
-\frac{2\left((\eta +\lambda\gamma)(1-\mu)+\eta\left[-1+\mu-\frac{2(1+i) [\eta (2- \lambda)+(2-\lambda)\lambda \gamma]\mu\sigma}{\left(\lambda \gamma [1+\sigma+(1-\lambda)\sigma] +\eta[\lambda+\sigma +(1-\lambda)\sigma]\right)^{2}}+\frac{2(1+i)\mu}{\lambda\gamma[1+\sigma+(1-\lambda)\sigma]+\eta[\lambda+\sigma+(1-\lambda)\sigma]}\right]\right)}{\gamma\lambda(1-\mu)\sigma^{2}}\\
+\frac{2\left(-(\eta+\lambda \gamma)(1- \mu)(1- \sigma) +\eta\left[(1-\mu)(1-\sigma)+\frac{2 (1+i)\mu\sigma}{\lambda \gamma[1+\sigma + (1-\lambda)\sigma] +\eta[\lambda+\sigma+(1- \lambda)\sigma]}\right]\right)}{\gamma\lambda(1-\mu)\sigma^3}
\end{split}
\end{equation}


Wird dieser Term wiederum vereinfacht, lautet er: 


\begin{equation}
\begin{split}
\left.-\frac{1}{\gamma \lambda (-1 + \mu) \sigma ^{3}}\Bigg[\frac{(1+i) \eta (-2 + \lambda)(-4\eta \lambda - 4\lambda \gamma)(\eta+\lambda \gamma)\mu\sigma^{2}}{\left(\eta [\lambda (-1+ \sigma) -2\sigma]+\lambda\gamma[-1 +(-2+\lambda)\sigma]\right)^{3}}
+2\Big[\lambda \gamma(-1 + \mu +\sigma-\mu\sigma)+\right.\\ 
\left.\frac{(-2-2i)\eta\mu\sigma}{\eta[\lambda(-1+\sigma)-2 \sigma] + \lambda\gamma[-1+(-2+\lambda)\sigma]}\Big]-2\sigma\bigg[-( \eta + \lambda \gamma)(-1 + \mu) + \eta\Big(-1 +\mu+\right.\\
\left. \frac{2(1+i)(-2+ \lambda)(\eta + \lambda \gamma) \mu \sigma}{\left(\eta[\lambda (-1+ \sigma) -2\sigma] + \lambda \gamma[-1+(-2+\lambda)\sigma]\right)^2}-\frac{2 (1+ i)\mu}{\eta [\lambda (-1+\sigma)-2\sigma] + \lambda \gamma [-1 + (-2 + \lambda) \sigma]} \Big)\bigg]\Bigg]\right.
\end{split}
\end{equation}


\section[Nicht-Konvergenz-Falle für die Imitationsstrategie, ${[R=1]}$]{Nicht-Konvergenz-Falle \\ für die Imitations\-strategie, [$R=1$] \sectionmark{Nicht-Konvergenz-Falle}}\label{NichtKonvergenzFalleImitation}
\sectionmark{Nicht-Konvergenz-Falle}


Es wird der Schnittpunkt mit der $45^\circ$ Linie berechnet. Der gegenwärtige Entwicklungsstand entspricht dem zukünftigen Entwicklungsstand, bei der jeweiligen Strategie.


\begin{equation}
\tilde{a_j}_{R=1}=\frac{1+\sigma_j}{2(1+g)}[\eta+\lambda\gamma\tilde{a_j}_{R=1}]
\end{equation}


\begin{equation}
-\frac{1+\sigma_j}{2(1+g)}\eta=\lambda\gamma\tilde{a_j}_{R=1}\left(\frac{1+\sigma_j}{2(1+g)}\right)-\tilde{a_j}_{R=1}
\end{equation}


\begin{equation}
\frac{1+\sigma_j}{2(1+g)}\eta=\tilde{a_j}_{R=1}\left(\frac{1+\sigma_j}{2(1+g)}\lambda\gamma-1\right)
\end{equation}


\begin{equation}
\tilde{a_j}_{R=1}= \frac{\frac{1+\sigma_j}{2(1+g)}\eta}{-\frac{1+\sigma_j}{2(1+g)}\lambda\gamma+\frac{\frac{1+\sigma_j}{2(1+g)}}{\frac{1+\sigma_j}{2(1+g)}}}
\end{equation}


\begin{equation}
\boxed{\tilde{a_j}_{R=1}=\frac{(1-\sigma_j)\eta}{2(1+g)-\lambda\gamma(1+\sigma_j)}}
\end{equation}


\section[Nicht-Konvergenz-Falle für die Innovationsstrategie, ${[R=0]}$]{Nicht-Konvergenz-Falle \\ für die Innovations\-strategie, [$R=0$]}\label{NichtKonvergenzFalleInnovation}
\sectionmark{Nicht-Konvergenz-Falle}


Hier wird ebenfalls der Schnittpunkt mit der $45^\circ$ Linie berechnet. Dann gilt, dass der gegenwärtige Entwicklungsstand gleich dem zukünftigem Entwicklungsstand ist, bei der entsprechenden Strategie.


\begin{equation}
\tilde{a_j}_{R=0} =\frac{1}{2(1+g)}[(\lambda+\sigma_j+(1-\lambda)\sigma_j)\eta+(1+\sigma_j+(1-\lambda)\sigma_j)\lambda\gamma\tilde{a_j}_{R=0}]
\end{equation}


\begin{equation}
\frac{1}{2(1+g)}(\lambda+\sigma_j+(1-\lambda)\sigma_j)\eta = \tilde{a_j}_{R=0}-\tilde{a_j}_{R=0}\frac{(1+\sigma_j+(1-\lambda)\sigma_j)\lambda\gamma}{2(1+g)}
\end{equation}


\begin{equation}
\frac{1}{2(1+g)}(\lambda+\sigma_j+(1-\lambda)\sigma_j)\eta = \tilde{a_j}_{R=0}\left[1-\frac{(1+\sigma_j+(1-\lambda)\sigma_j)\lambda\gamma}{2(1+g)}\right]
\end{equation}

\begin{equation}
\tilde{a_j}_{R=0} = \frac{(\lambda+\sigma_j+(1-\lambda)\sigma_j)\eta}{2(1+g)}*\frac{2(1+g)}{2(1+g)-(1+\sigma_j+(1-\lambda)\sigma_j)\lambda\gamma}
\end{equation}


\begin{equation}
\boxed{\tilde{a_j}_{R=0} = \frac{(\lambda+\sigma_j+(1-\lambda)\sigma_j)\eta}{2(1+g)-(1+\sigma_j+(1-\lambda)\sigma_j)\lambda\gamma}}
\end{equation}


\section[Effekte der Exportförderung auf die Strategien]{Effekte der Exportförderung auf die Strategien \sectionmark{Effekte der Exportförderung}}\label{math:Effekte}
\sectionmark{Effekte der Exportförderung}


Wie verändert sich $[R=1]$, wenn sich $\sigma$ ändert, bei einer endogenen WTG


\begin{equation}
a_t [R=1]=\frac{(1+\sigma)}{2(1+g(\sigma))}(\eta+\lambda\gamma a_{t-1})
\end{equation}


\begin{equation}
\frac{\partial a_t[R=1]}{\partial\sigma}=0
\end{equation}


\begin{equation}
\begin{split}-\frac{(\eta+a_{t-1}\gamma\lambda)(\eta(2-\lambda)+\gamma(2-\lambda)\lambda)(1+\sigma)}{(\gamma\lambda(1+\sigma+(1-\lambda)\sigma)+\eta(\lambda+\sigma+(1-\lambda)\sigma))^2}\\
+\frac{\eta+a_{t-1}\gamma\lambda}{\gamma\lambda(1+\sigma+(1-\lambda)\sigma)+\eta(\lambda+\sigma+(1-\lambda)\sigma)}=0
\end{split}
\end{equation}


vereinfacht man diesen Term, erhält man


\begin{equation}
\frac{(\eta+a_{t-1}\gamma\lambda)(\gamma\lambda(\lambda-1)+\eta(2\lambda-2))}{(\eta(\lambda(\sigma-1)-2\sigma)+\gamma\lambda((\lambda-2)\sigma-1))^2}
\end{equation}


Betrachten wir nur den Ordinatenabschnitt lässt sich bestimmen ob dieser ansteigt oder sinkt bei einer Veränderung der Projektgrö{\ss}e. 


\begin{equation}
\begin{split}
\frac{\partial{\frac{(1+\sigma)}{2(1+g(\sigma))}\eta}}{\partial\sigma}=-\frac{\eta(\eta(2-\lambda)+\gamma(2-\lambda)\lambda)(1+\sigma)}{\gamma\lambda(1+\sigma+(1-\lambda)\sigma)+\eta(\lambda+\sigma+(1-\lambda)\sigma))^2}\\
+\frac{\eta}{\gamma\lambda(1+\sigma(1-\lambda)\sigma)+\eta(\lambda+\sigma+(1-\lambda)\sigma)}
\end{split}
\end{equation}


vereinfacht erhält man


\begin{equation}
\frac{\partial{\frac{(1+\sigma)}{2(1+g(\sigma))}\eta}}{\partial{\sigma}}=\frac{\eta(\gamma\lambda(\lambda-1)+\eta(2\lambda-2))}{(\eta(\lambda(\sigma-1)-2\sigma)+\gamma\lambda(\sigma(\lambda-2)-1))^2}
\end{equation}


Im Folgenden wird die Reaktion der Steigung genauer betrachtet 


\begin{equation}
\begin{split}
\frac{\partial{\frac{(1+\sigma)}{2(1+g(\sigma))}\lambda\gamma a_{t-1}}}{\partial\sigma}=-\frac{a_{t-1}\gamma\lambda(\eta(2-\lambda)+\gamma(2-\lambda)\lambda)(1+\sigma)}{(\gamma\lambda(1+\sigma+(1-\lambda)\sigma)+\eta(\lambda+\sigma+(1-\lambda)\sigma))^2}\\
+\frac{a_{t-1}\gamma\lambda}{\gamma\lambda(1+\sigma+(1-\lambda)\sigma)+\eta(\lambda+\sigma+(1-\lambda)\sigma)}
\end{split}
\end{equation}


dies wiederum vereinfacht ergibt


\begin{equation}
\frac{\partial{\frac{(1+\sigma)}{2(1+g(\sigma))}\lambda\gamma a_{t-1}}}{\partial\sigma}=\frac{a_{t-1}\gamma\lambda(\gamma\lambda(\lambda-1)+\eta(2\lambda-2))}{(\eta(\lambda(\sigma-1)-2\sigma)+\gamma\lambda((\lambda-2)\sigma-1))^2}
\end{equation}


Wie verändert sich [R=0], wenn sich $\sigma$ ändert, bei einer endogenen WTG


\begin{equation}
\frac{1}{2(1+g)}[(\lambda+\sigma+(1-\lambda)\sigma)\eta+(1+\sigma+(1-\lambda)\sigma)\lambda\gamma a_{t-1}]
\end{equation}


\begin{equation}
\frac{\partial a_t[R=0]}{\partial\sigma}=0
\end{equation}


\begin{equation}
\begin{split}
\frac{\eta (2 - \lambda) +a_{t-1} \gamma(2 - \lambda) \lambda}{\gamma \lambda (1 + \sigma +(1- \lambda) \sigma)+ \eta (\lambda + \sigma + (1- \lambda) \sigma)}\\
- \frac{ (\eta (2- \lambda)+ \gamma (2 - \lambda) \lambda)(a_{t-1} \gamma \lambda (1+ \sigma +(1- \lambda)\sigma) + \eta (\lambda + \sigma + (1 -\lambda) \sigma))}{(\gamma \lambda (1+ \sigma + (1 - \lambda) \sigma)+ \eta (\lambda + \sigma + (1 - \lambda) \sigma))^{2}}=0
\end{split}
\end{equation}


vereinfacht man diesen Term, erhält man


\begin{equation}
\frac{\gamma \eta \lambda (2 -3 \lambda +\lambda^{2} + a_{t-1} (-2 +3 \lambda -\lambda^{2}))}{(\eta ( \lambda(-1 + \sigma) -2 \sigma) + \gamma \lambda(-1 +(-2 + \lambda) \sigma))^{2}}
\end{equation}


Betrachten wird hier nur den Ordinatenabschnitt, dabei lässt sich bestimmen ob dieser ansteigt oder sinkt bei einer Veränderung der Projektgrö{\ss}e. 


\begin{equation}
\begin{split}
\frac{\partial\frac{(\lambda + \sigma + (1- \lambda) \sigma) \eta}{2(1+ g(\sigma))}}{\partial\sigma}=- \frac{\eta( \eta (2- \lambda) + \gamma (2- \lambda) \lambda)(\lambda + \sigma + (1- \lambda) \sigma)}{(\gamma \lambda (1+ \sigma +(1- \lambda) \sigma) + \eta( \lambda + \sigma + (1- \lambda) \sigma))^{2}}\\
+ \frac{\eta (2- \lambda)}{\gamma \lambda (1+ \sigma +(1- \lambda) \sigma) + \eta( \lambda + \sigma + (1- \lambda) \sigma)}
\end{split}
\end{equation}


vereinfacht erhält man


\begin{equation}
\frac{\partial\frac{(\lambda + \sigma + (1- \lambda) \sigma) \eta}{2(1+ g(\sigma))}}{\partial\sigma}=\frac{\gamma \eta \lambda(2 -3\lambda +\lambda^{2})}{(\eta(\lambda(-1 + \sigma)-2 \sigma) + \gamma \lambda (-1 + (-2 + \lambda) \sigma))^{2}}
\end{equation}


Im Folgenden wird die Reaktion der Steigung genauer betrachtet 


\begin{equation}
\begin{split} 
\frac{\partial\frac{(1+\sigma+(1-\lambda)\sigma)\lambda\gamma a_{t-1}}{2(1+g)}}{\partial\sigma} =- \frac{a_{t-1} \gamma \lambda ( \eta (2- \lambda)+\gamma(2-\lambda) \lambda)(1+ \sigma + (1- \lambda) \sigma)}{(\gamma \lambda (1+ \sigma + (1- \lambda) \sigma) + \eta (\lambda + \sigma +(1- \lambda) \sigma))^{2}}\\
+ \frac{a_{t-1} \gamma (2- \lambda) \lambda}{ \gamma \lambda (1+ \sigma + (1- \lambda) \sigma) + \eta (\lambda + \sigma +(1- \lambda) \sigma)}
\end{split}
\end{equation}


dies wiederum vereinfacht ergibt


\begin{equation}
\frac{\partial\frac{(1+\sigma+(1-\lambda)\sigma)\lambda\gamma a_{t-1}}{2(1+g)}}{\partial\sigma} =\frac{a_{t-1}\gamma \eta \lambda (-2 +3 \lambda -\lambda^{2})}{(\eta (\lambda (-1 + \sigma) -2 \sigma) + \gamma \lambda (-1+(-2 + \lambda) \sigma))^{2}}
\end{equation}