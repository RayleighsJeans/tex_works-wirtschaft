
\section[Wachstumstheorien beruhend auf technischem Fortschritt]{Wachstumstheorien beruhend auf\\ technischem Fortschritt \sectionmark{Wachstumstheorien}}\label{sec:Wachstumstheorien}
\sectionmark{Wachstumstheorien}
Das folgende Unterkapitel befasst sich mit der Entwicklung der Wachstumstheorien, die sich vornehmlich mit den Ursachen des Wirtschaftswachstums besch{\"a}ftigen. Beginnend mit der relativ jungen Wirtschaftstheorie der "`unified growth theory"', zu deutsch die Theorie des einheitlichen Wachstums, wird anschlie{\ss}end wieder die Struktur \citeauthor{Gandolfo.1998}s aufgegriffen, die die Wirtschaftstheorien gem{\"a}{\ss} ihrer Gr{\"u}nde f{\"u}r Wachstum untergliedert. Gandolfo sah als direkte Ursachen von Wachstum zum einen die Akkumulation von Produktionsfaktoren und zum anderen den technischen Fortschritt. Die Akkumulation von physischem Kapital wird unter anderem im neoklassischen Solow-Modell thematisiert. Darauf folgt die Abgrenzung zu den endogenen Wachstumstheorien, wie beispielsweise dem Romer-Modell. Anschlie{\ss}end wird der technische Fortschritt in schumpeterianischen Modellen genauer analysiert, bevor abschlie{\ss}end anhand des Uzawa-Lucas-Modells die Akkumulation von Humankapital als Voraussetzung f{\"u}r den technischen Fortschritt behandelt wird. \newline In diesem Rahmen werden die verschiedenen Ans{\"a}tze und Modelle kurz vorgestellt, um die im Hauptteil folgenden Modellvariationen darin einordnen zu k{\"o}nnen. 

\paragraph{unified growth thoery}\label{Unified}
Die "`unified growth theory"' wurde von Oded Galor begründet und versucht einen zeitlich allumfassenden Erklärungsansatz für das Wirtschaftswachstum zu finden. Dabei wird das langfristige Wachstums vor der Zeit der Industrialisierung mit einbezogen, wodurch eine stärkere Bedeutung des demographischen Wandels bedingt wird \citep{Galor.2011}.

	\begin{figure}[h!]
 \centering 
 \begin{tabular}{@{}r@{}} 
		\includegraphics[width=0.95\textwidth]{images/Abbildungen/Karte.eps}\\
 \footnotesize\sffamily\textbf{Quelle:} Galor (2011) %\cite{pps} 
 \end{tabular}  
		\caption{Pro-Kopf-Einkommen der Welt im Jahre 2010} 
		\label{KarteEinkommen}
	\end{figure}

Abbildung \ref{KarteEinkommen} zeigt das Pro-Kopf-Einkommen der Weltbev{\"o}lkerung aus dem Jahr 2010. Das BIP ist das Ma{\ss} für das wirtschaftliche Wachstum, wobei eine Pro-Kopf-Betrachtung eine internationale Vergleichbarkeit ermöglicht. Dabei wird ersichtlich, dass auf der Nordhalbkugel und in den Pazifikstaaten Australien und Neuseeland das durchschnittliche Einkommen pro Kopf bei mindestens 15.000 US-Dollar pro Jahr liegt. F{\"u}hrend sind Nordamerika, Europa, sowie Australien und Neuseeland. Das durchschnittliche Einkommen dieser L{\"a}nder ist gr{\"o}{\ss}er als 30.000 US-Dollar. Mit weniger als 3.000 US-Dollar im Jahr m{\"u}ssen die Einwohner im Norden Sub-Sahara-Afrikas auskommen \citep[Kapitel 1]{Galor.2014}.\newline
		
		\begin{figure}[h!]
			\centering 
				\begin{tabular}{@{}r@{}}  
				\includegraphics[width=0.95\textwidth]{images/Abbildungen/BIP.eps}\\
				\footnotesize\sffamily\textbf{Quelle:} Galor (2011) %\cite{pps} 
				\end{tabular}  
			\caption{Pro-Kopf-Einkommen von 1820-2010} 
			\floatfoot{-~zu den Western Offshoots zählen Australien, Kanada, Neuseeland und USA~-}
			\label{BIP200Jahre}
		\end{figure}
		
Abbildung \ref{BIP200Jahre} zeigt das BIP pro Kopf im Zeitverlauf der letzten 200 Jahre. Es sind immer noch deutliche regionale Unterschiede zu verzeichnen, doch viel auff{\"a}lliger ist, dass gegen Ende des 19. Jahrhunderts, in den heute relativ weit entwickelten L{\"a}ndern, eine Phase der Stagnation endete. Au{\ss}erdem gab es weltweit nach dem zweiten Weltkrieg einen erneuten Wachstumsschub.

		\begin{figure}[h!]
			\centering 
			\begin{tabular}{@{}r@{}} 
				\psfrag{A}{Asien} 
				\includegraphics[width=0.95\textwidth]{images/Abbildungen/200JahreBIP.eps}\\
				\footnotesize\sffamily\textbf{Quelle:} Galor (2011) %\cite{pps} 
			\end{tabular}  
			\caption{Pro-Kopf-Einkommen von 1-2010} 
			\label{BIP2000Jahre}
		\end{figure}
		
Ein erweiterter Blick auf Schätzungen\footnote{Diese wurden von \citet{Galor.2011} vorgenommen und gehen zurück auf die Daten von \citet{Maddison.2001}.} der letzten 2000 Jahre in Abbildung \ref{BIP2000Jahre} zeigt, dass die Phase der Stagnation seit Beginn unserer Zeitrechnung andauert. Der erste Wachstumsschub gegen Ende des 19. Jahrhunderts  gr{\"u}ndet auf der Erfindung der Dampfmaschine und der damit einhergehenden industriellen Revolution. Zun{\"a}chst in England, dann in ganz Westeuropa, Japan und in den USA kam es zu dem {\"U}bergang von der Agrar- zur Industriegesellschaft. Die Industrialisierung bedingte eine stark beschleunigte Entwicklung von Technologie, Produktivit{\"a}t und Wissenschaft.\newline


In der vorliegenden Arbeit soll aufgezeigt werden, dass es sich hierbei um wesentliche Einflussfaktoren wirtschaftlichen Wachstums handelt. Jedoch ist der Grenzertrag dieser Neuerungen abnehmend und somit f{\"u}r die Industriel{\"a}nder von geringerer Bedeutung. Auf das wirtschaftliche Wachstum noch relativ wenig entwickelter Länder üben diese Faktoren heute aber einen deutlichen Einfluss aus. Die regionale Ausbreitung der industriellen Entwicklung, der Technologietransfer, erfolgt entweder durch Migration oder durch den G{\"u}terhandel, dem zweiten Schwerpunkt dieser Arbeit.\newline 


In dem Bereich der "`unified growth theory"' besch{\"a}ftigt sich Oded Galor vornehmlich mit  Forschungsfragen {\"u}ber den Ursprung der sozialen Ungleichheit zwischen den L{\"a}ndern:\footnote{Die soziale Ungleichheit hat sich in den vergangenen 2000 Jahren enorm verändert. Wird nur Westeuropa betrachtet, so ist der Faktor 40 Mal so gro{\ss}, als zu Beginn unserer Zeitrechnung. In Ländern Afrikas, hat sich die Ungleichheit hingegen "`nur"' vervierfacht. Die Folge des hohen Wirtschaftswachstums ist eine gr{\"o}{\ss}ere Kluft zwischen den armen und reichen Bev{\"o}lkerungsschichten \citep{Galor.2011}.} Welche Faktoren hemmten die Konvergenz armer L{\"a}nder an reichere in den letzten Jahrzehnten? Welche Rolle spielen die originären Faktoren, wie kulturelle, geologische und geographische  Eigenschaften eines Landes bei der Erkl{\"a}rung der beobachteten komparativen Vorteile?\newline


Die Bev{\"o}lkerungsfalle oder auch Malthusianische Katastrophe genannt, bildet die Grundlage der einheitlichen Wachstumstheorie und stellt ein Hemmnis f{\"u}r Entwicklung und Wachstum dar. Der Grundgedanke geht auf Thomas Malthus (1798) zur{\"u}ck. Er behauptete, dass langfristiges Wachstum des Lebensstandards nicht m{\"o}glich sei. \citet{Galor.2011} greift seine Theorie auf und unterteilt dabei die letzten 2000 Jahre in drei verschiedene Epochen. Die Malthusianische Epoche, die Post-Malthusianische Epoche und die Zeit des Modernen Wachstums.\footnote{Neben \citet{Galor.2006} befassen sich ebenso die Aufsätze von \citet{Hansen.2002}, sowie \citet{Ashraf.2008} mit dieser zeitlich allumfassenden Wachstumstheorie.}\\

		\begin{figure}[htbp]
			\centering 
			\begin{tabular}{@{}r@{}}  
				\includegraphics[width=0.90\textwidth]{images/Abbildungen/EpochenMalthus.eps}\\
				\footnotesize\sffamily\textbf{Quelle:} Galor (2011) %\cite{pps} 
			\end{tabular}  
			\caption{Entwicklungsphasen des Wachstums (der "`unified growth theory"')} 
			\label{Epochen}
		\end{figure}
		
Die Malthusianische Epoche nimmt 99,8{\%} der letzten 2000 Jahre ein und endet in den 50er Jahren des 18. Jahrhunderts. Die verbleibenden 0,2{\%} bilden die Post-Malthusianische Epoche, welche ca. 120 Jahre andauerte und durch die Industrielle Revolution eingeleitet wurde, sowie die anschlie{\ss}ende Zeit des Modernen Wachstums. Diese begann in den 1870ern und dauert bis heute an \cite{Galor.2014}.\newline


\textit{Malthusianische Epoche}\newline
\citet{Ashraf.2011} charakterisieren die Malthusianische Epoche vor allem durch den sehr langsamen Prozess des technischen Fortschritts. Dieser war nicht das Ergebnis organisierter Wissensakkumulation, wie es seit der Industrialisierung und in den Industrie\-l{\"a}ndern {\"u}blich war, sondern basierte auf Erkenntnissen, Erfahrungen und Experimenten des Alltags sowie der Notwendigkeit Probleme zu l{\"o}sen, um das {\"U}berleben zu sichern. Jedoch wurde in Anbetracht des sehr langen Zeitraums von knapp 2000 Jahren relativ wenig Neuerungen eingef{\"u}hrt und es resultierte laut der Sch{\"a}tzungen von \citet{Maddison.2001} nur eine j{\"a}hrliche Wachstumsrate von $\frac{1}{19}\%$ des Pro-Kopf-Einkommens. In diesem Zeitabschnitt entspricht das Pro-Kopf-Einkommen ungef{\"a}hr dem Existenzminimum. Der geografisch begrenzte Produktionsfaktor Land stellt die Haupteinnahmequelle der Bev{\"o}lkerung dar. Der fruchtbare Boden konnte nur bedingt bewirtschaftet werden und f{\"u}hrte langfristig zu abnehmenden Grenzertr{\"a}gen des Bodens und der Arbeit. Geht man von einem Land aus, das nur landwirtschaftliche G{\"u}ter herstellt, dann ben{\"o}tigt die Volkswirtschaft fruchtbares Land $X$, Arbeit $L$ und den Produktivit{\"a}tsparameter $A$ um das Gut $Y$ herzustellen.

	\begin{equation}
		Y=AX^\beta L^{1-\beta},  \qquad \text{mit}\quad 0< \beta < 1
	\end{equation}
	
Wenn davon ausgegangen wird, dass jedes Mitglied der Bev{\"o}lkerung arbeitet und das fruchtbare Land  auf $X=1$ normiert wird, dann ergibt sich f{\"u}r das Pro-Kopf-Einkommen $y$ folgende Gleichung. 

	\begin{equation}
		y=\frac{Y}{L}=AL^{-\beta}
	\end{equation}
	
Hier l{\"a}sst sich formal darstellen, dass ein positiver Zusammenhang zwischen der Produktivit{\"a}t $A$ und dem Pro-Kopf-Einkommen $y$ besteht und ein negativer mit der Bev{\"o}lkerungsgr{\"o}{\ss}e $L$. Damals wie heute bestimmt das Einkommen die Familienplanung. Ein hohes Pro-Kopf-Einkommen geht mit einem hohen Lebensstandard einher. Je st{\"a}rker das Pro-Kopf-Einkommen w{\"a}chst, desto schneller w{\"a}chst die Bev{\"o}lkerung.\\


Drei externe Effekte beeinflussen in diesem Zeitabschnitt das Pro-Kopf-Einkommen positiv: der technologische Fortschritt, die Ausweitung des bestellbaren Bodenbestands und ein R{\"u}ckgang der Bev{\"o}lkerung durch exogene Schocks, wie Krankheiten oder Hungersn{\"o}te. Diese f{\"u}hren kurzfristig zu einem positiven Pro-Kopf-Einkommenseffekt. Der Wohlstandsanstieg der Bev{\"o}lkerung bedingt dann wiederum ein h{\"o}heres Bev{\"o}lkerungswachstum. Langfristig bedeutet das jedoch, dass das Niveau des Pro-Kopf Einkommens wieder sinkt. Beispielhaft f{\"u}r das Verhalten des Einkommens auf einen exogenen Schock ist die Pest, die in England von 1250 bis 1270 w{\"u}tete. Die Bev{\"o}lkerungszahl sank sehr stark, wodurch der Produktionsfaktor Arbeit knapper und dadurch teurer wurde. Ein stark ansteigendes reales Lohnniveau war die Folge. Erst mit dem Anstieg der Bev{\"o}lkerung sank auch das Lohnniveau wieder. \newline Ein weiterer Zusammenhang besteht zwischen der Bodenproduktivit{\"a}t und der Bev{\"o}lkerungsdichte. Je produktiver das Land ist und je mehr Lebensmittel angebaut und geerntet werden k{\"o}nnen, desto st{\"a}rker w{\"a}chst die Bev{\"o}lkerung. In diesem Fall vornehmlich in Volkswirtschaften, denen relativ viel bestellbares Land zur Verf{\"u}gung steht. Jedoch hat die Zunahme der Produktivit{\"a}t des Bodens keinen direkten Einfluss auf das Pro-Kopf-Einkommen, weil der anf{\"a}ngliche Einkommenszuwachs durch den Produktivit{\"a}tsgewinn, durch die Bev{\"o}lkerungszunahme ausgeglichen wird. \newline Bei dem dritten positiven Effekt dieser Zeit, dem Technologischen Fortschritt verh{\"a}lt es sich {\"a}hnlich. Anf{\"a}nglich steigert dieser die Produktivit{\"a}t und somit das Einkommen, aber ein h{\"o}heres Einkommen f{\"u}hrt zu einer h{\"o}heren Geburtenrate und gleicht somit den kurzfristigen Effekt wieder aus. Ansonsten lässt sich zwischen technologischem Fortschritt und Pro-Kopf-Einkommen nur ein geringer positiver Zusammenhang feststellen \citep{Galor.2014}.\\


\textit{Post-Malthusische Epoche}\newline


Der {\"U}bergang der Malthusischen zu der Post-Malthusischen Epoche ist durch den Startpunkt, den "`take-off-point"', des wirtschaftlichen Wachstums charakterisiert. Dabei wird die Phase der Stagnation durch Wachstum abgel{\"o}st. 
Laut der Theorie nach \citet{Hansen.2002} sowie \citet{Ashraf.2008}\footnote{Das Papier von \citet{Ashraf.2008} bestätigt den Wandel des Bevölkerungswachstums in der malthusischen Epoche empirisch.} wurde der technische Fortschritt  durch die Industrielle Revolution im 18. Jahrhundert stark beschleunigt.\footnote{Eine andere Theorie besagt, dass die Humankapitalakkumulation im Vordergrund steht und letztlich zur Industrialisierung, dem Übergang von der Stagnation zum Wachstum, geführt hat \citep{Galor.2000}. Die Ansammlung von Humankapital führt zu technischem Fortschritt, der somit durch einen Skaleneffekt der Bevölkerungsgrö{\ss}e entsteht. Andererseits führt erst der Produktivitätsfortschritt zu einer Nachfrage nach Humankapital und es kommt zu ständigen positiven Wechselwirkungen zwischen der Humankapitalakkumulation und dem technischen Fortschritt.} Dadurch kam es zu einem sehr starken Anstieg des totalen Outputs und auch des Pro-Kopf-Einkommens. Das Pro-Kopf-Einkommen hatte noch immer einen positiven Effekt auf das Bev{\"o}lkerungswachstum, jedoch ist dieser nun im Vergleich zur Malthusischen Epoche  abnehmend. Es herrschte also ein vergleichsweise schnelles Wachstum des Pro-Kopf-Einkommens und der Bev{\"o}lkerung. \newline Dieser Wachstumsstartpunkt ist jedoch regional verschieden, vor allem, weil es schon regionale Entwicklungsunterschiede gab. So vor allem in technologisch weiter entwickelten Volkswirtschaften und auch in L{\"a}ndern, die sehr reichlich mit dem Faktor Boden ausgestattet waren. Hier gab es grunds{\"a}tzlich eine h{\"o}here Bev{\"o}lkerungsdichte und gr{\"o}{\ss}tenteils {\"a}hnliche Einkommenslevel in den Bev{\"o}lkerungsschichten. Somit waren diese schon in der Malthusianischen Epoche relativ weiter entwickelt, was wiederum einen fr{\"u}heren "`take-off point"' mit einem relativ st{\"a}rker andauernden Wachstum bedingte.\newline Werden die Regionen anhand der Industrialisierung pro Kopf \footnote{Dies kann als die Arbeitsleistung pro Kopf gesehen werden, die durch den Einsatz fortschrittlicherer Verfahren ansteigt und wird gemessen an der industriellen Produktion pro Kopf.} miteinander verglichen, verdeutlicht dies, dass die Industrialisierung in Gro{\ss}britannien ihren Ursprung hat \citep{Galor.2014}.\\


Durch Migration und Handel bedingt, kam es erst {\"u}ber 50 Jahre sp{\"a}ter in den {\"u}brigen europ{\"a}ischen L{\"a}ndern, wie Frankreich und Deutschland, sowie Nordamerika zu einem Anstieg der Pro-Kopf-Industrialisierung.  In den heutigen Entwicklungsl{\"a}ndern sank sogar in der Zeit der Malthusischen Epoche die Industrialisierung pro Kopf aufgrund des starken Bev{\"o}lkerungswachstums. Erst in der Zeit des modernen Wachstums, ab dem Jahre 1920, gelangte ein Wachstumsimpuls in die L{\"a}nder der dritten Welt. Ein deutlich st{\"a}rkerer Wachstumsimpuls auf deren Industrialisierung folgte mit etwas zeitlicher Verz{\"o}gerung nach dem zweiten Weltkrieg im Jahre 1960. Jedoch handelt es sich hierbei um einen deutlich geringeren Wachstumsschub, als er durch die Industrialisierung in den heutigen Industrieländern hervorgerufen wurde \citep{Galor.2014}.\\


\textit{Epoche des modernen Wachstums}\newline


Die Epoche des modernen Wachstums beschreibt den Zeitabschnitt in dem das anhaltende {\"o}konomische Wachstum beginnt. Der technische Fortschritt war in dieser Zeit so intensiv, dass es eine starke Nachfrage nach Humankapital gab. Die Bev{\"o}lkerung begann daher in ihre Ausbildung zu investieren und baute Humankapital auf. Die Menschen mussten aber Priorit{\"a}ten setzen und ihre Zeit zwischen Erwerbst{\"a}tigkeit, Kindererziehung und ihrer eigenen Weiterbildung aufteilen. Dies geschah zu Lasten der Geburtenrate, welche mit steigenden Humankapital schlie{\ss}lich sank. Qualifizierte Mitarbeiter f{\"o}rderten von nun an den andauernden Industrialisierungsprozess und der technische Fortschritt nahm weiterhin zu. Die gesunkene Geburtenrate f{\"u}hrte letztlich zu einem geringeren Bev{\"o}lkerungswachstum. Von diesem Zeitpunkt an war das {\"o}konomische Wachstum unabh{\"a}ngig von den Bev{\"o}lkerungsbewegungen und es kam zu keiner Kompensation positiver wachstumsf{\"o}rdernder Effekte durch Bev{\"o}lkerungszunahme. Die drei angef{\"u}hrten Punkte, technologischer Fortschritt, gemindertes Bev{\"o}lkerungswachstum und Humankapitalakkumulation generierten von da an langfristiges gleichm{\"a}{\ss}iges {\"o}konomisches Wachstum. \newline Werden die Wachstumsraten der verschiedenen Volkswirtschaften betrachtet, so handelt es sich seit 1950 bis zum heutigen Zeitpunkt um gr{\"o}{\ss}tenteils gleichm{\"a}{\ss}iges positives Wachstum. Die unterschiedlichen Entwicklungsst{\"a}nde werden durch die verschiedenen Niveaus des BIPs pro Kopf deutlich. Diese resultieren aus den unterschiedlichen Startsituationen in der Malthusianischen Epoche und den daraus folgenden "`take off points"' induziert durch die Industrialisierung \citep{Galor.2014}.\\


Die Entwicklung der Geburtenrate greift \citet{Galor.2014} erneut auf und analysiert in seiner Wachstumstheorie deren R{\"u}ckgang. Die Daten zeigen, dass nicht nur die Entwicklung der L{\"a}nder zeitlich versetzt ist, sondern auch die Geburtenraten {\"a}hnlich reagieren. L{\"a}nder mit relativ schlechteren Anfangsbedingungen und somit einem sp{\"a}teren "`take off"' verzeichnen auch einen verz{\"o}gerten Anstieg und sp{\"a}terem Absinken der Geburtenrate. Die Geburtenrate w{\"a}chst zun{\"a}chst durch das zus{\"a}tzliche Einkommen aus der industrialisierten Wirtschaft und sinkt mit zunehmenden Bildungsstand der Bev{\"o}lkerung. Werden die asiatischen oder afrikanischen Volkswirtschaften betrachtet, so stieg dort die Geburtenrate erst im Jahr 1870 an. F{\"u}nfzig Jahre sp{\"a}ter begann in den L{\"a}ndern der westlichen Welt zu diesem Zeitpunkt die Geburtenrate bereits wieder zu sinken \citep{Galor.2014}.
Oded Galors "`unified growth theory"' fand viele Anh{\"a}nger, die ihre Aufgabe darin sahen die Entwicklung r{\"u}ckblickend zu er{\"o}rtern.\newline Das Malthusische Modell zeigt, dass die Produktion mit einem fixen Faktor, dem Land bzw. dem fruchtbaren Boden, und zunehmenden Bev{\"o}lkerungswachstum von der Pro-Kopf-Output-Rate abh{\"a}ngt. Dabei führt ein hohes Pro-Kopf-Einkommen zu einem Anstieg der Bev{\"o}lkerung, was wiederum die Pro-Kopf-Rate mindert und die Bev{\"o}lkerungszahl sinkt.  Langfristig ergibt sich eine Stagnation der Wachstumsrate. Wird der Ansatz von \citet{Malthus.1798} um eine AK-Produktionstechnologie erweitert, dann simuliert dies die Zeit des 1900 Jahrhundert, in der die industrielle Revolution zu grundlegenden Ver{\"a}nderungen f{\"u}hrte. Diese Modellerweiterung nach \citet{Hansen.2002}, sowie \citet{Ashraf.2008} veranschaulicht, dass sofern der Wissensparameter gro{\ss} genug ist, es zu einem Strukturwandel vom prim{\"a}ren Landwirtschaftssektor zum sekund{\"a}ren Industriesektor kommt. Somit wird die Kompetenz und Qualifiziertheit der Unternehmer in Zeiten struktureller Ver{\"a}nderungen, wie beispielsweise dem Wandel im Zuge der Industrialisierung betont \cite{Galor.1997}.
Die Volkswirtschaft bewegt sich damit aus der Stagnation heraus und die Wirtschaft w{\"a}chst langfristig. Sie sehen den Grund für den Entwicklungsprozess stagnierender zu wachsenden Volkswirtschaften in dem Wandel von landintensiver Produktion hin zu technologieintensiver Produktion, auch als Folge der Industrialisierung. Dieser Zusammenhang ebnet den {\"U}bergang zur neoklassischen Wachstumstheorie, dessen f{\"u}hrender Vertreter Robert Solow ist \citep{Hansen.2002}.


\subsection{Exogene Wachstumsmodelle}
Die folgenden traditionellen Wirtschaftstheorien beschäftigen sich vornehmlich mit der Erklärung des Wachstums seit dem Industrialisierungsprozess. Ein Wachstumsmodell wird immer dann als exogen bezeichnet, wenn die Ursachen des technischen Fortschritts nicht hinterfragt werden und per Annahme in das Modell eingehen. Dies belegt das Solow-Modell, indem Kapitalakkumulation zu einem Anpassungswachstum hin zum Gleichgewicht führt und technischer Fortschritt als exogene Annahme einer langfristigen Stagnation entgegenwirkt.


\paragraph{Solow-Modell}
Robert Merton Solow wurde 1924 in New York City geboren und fand, nach dem zweiten Weltkrieg w{\"a}hrend eines volkswirtschaftlichen Studiums in Harvard, in Wassily Leontief seinen Lehrer \citep{Lin.2007}. Aus seinem bedeutendsten Papier "` A Contribution to the Theory of Economic Growth"` von 1956 entwickelte er ein Wachstumsmodell basierend auf einer gesamtwirtschaftlichen Produktionsfunktion. Die beiden Produktionsfaktoren Arbeit und Kapital werden in einem flexiblen Verh{\"a}ltnis eingesetzt und f{\"u}hren zu einer gleichgewichtig wachsenden Wirtschaft. Dabei zeigt das sogenannte Solow-Modell in seiner Einfachheit die Bedeutung des technischen Fortschritts f{\"u}r die {\"o}konomische Entwicklung eines Landes und beschreibt den gleichgewichtigen Zustand einer Volkswirtschaft, bei dem die Abschreibung und das Bevölkerungswachstum genau durch die Ersparnis kompensiert wird. In diesem Gleichgewicht verändert sich die Kapitalintensit{\"a}t nicht mehr.\footnote{Nach einem Anpassungswachstum verändert sich die Kapitalintensität pro Kopf $k(t)$ nicht mehr über die Zeit, deshalb gilt $\dot{k}=0$.} Das Modell setzt sich zun{\"a}chst aus einer Produktionsfunktion und einem Bewegungsgesetz zusammen.

	\begin{equation}
		Y=A K^\alpha L^{1-\alpha}
	\end{equation}

Das Gut bzw. Volkseinkommen $Y$ wird mit den Produktionsfaktoren Kapital und Arbeit hergestellt. Die Produktionselastizit{\"a}t $\alpha <1$ beschreibt abnehmende Grenzertr{\"a}ge des Kapitals und $A$ ist ein Produktionsparameter. \newline Das Bewegungsgesetz beschreibt die Abh{\"a}ngigkeit der Kapitalakkumulation von den Investitionen, die sich aus der Ersparnis $sY$ ergibt, und der Abschreibung auf das Kapital.

	\begin{equation}
		\dot{K}=sY-\delta K
	\end{equation}\label{Bewegungsgesetz Solow}

Dabei ist $\dot{K}$ das aggregierte Sparen und entspricht der aggregierten Investition, $\delta K$ beschreibt die aggregierte Abschreibung \citep{Solow.1956}.\newline Die Kernaussage des Solow-Modells ist, dass langfristiges Wirtschaftswachstum nicht durch {\"o}konomische Bedingungen herbeigef{\"u}hrt wird. Das Pro-Kopf-Einkommen ${Y}/{L}$ kann nur dann wachsen, wenn auch der Produktivitätsparameter $A$ w{\"a}chst. Dieser wird auch als technischer Fortschritt bezeichnet, der jedoch weder erkl{\"a}rt noch begr{\"u}ndet wird. Langfristig ist Wirtschaftswachstum nur dann m{\"o}glich, wenn es zu technischem Fortschritt kommt. Neben diesem Ergebnis zeigt Robert Solow erstmals, dass eine Volkswirtschaft intrinsisch bestrebt ist Stabilit{\"a}t zu erreichen.\newline Bis zur Entwicklung seines Modells galt der Faktor Kapital als limitierend f{\"u}r das Wirtschaftswachstum.\footnote{Als Beispiel dient hier das Harrod-Domar Wachstumsmodell \citep{Harrod.1939,Domar.1946}.} Basierend auf den Gedanken Ricardos zeigt Solow, dass ohne technischen Fortschritt eine Kapitals{\"a}ttigung und somit eine Stagnation eintreten wird \citep{Solow.1956} \newline Die Ergebnisse seiner Arbeit belegte Solow selbst im Jahr 1957 empirisch am Beispiel der USA. Er argumentiert, dass nicht der erh{\"o}hte Einsatz von Kapital und Arbeit die Entwicklung f{\"o}rderten, sondern rund 90 Prozent des Wachstums durch technischen Fortschritt verursacht wurden. Dies gelang ihm mit Hilfe des Solow-Residuums. Dieser Term, auch als Totale Faktorproduktivit{\"a}t bezeichnet, beschreibt den Zuwachs der Produktivit{\"a}t, der weder durch eine erh{\"o}hte Kapitalzufuhr, noch durch zus{\"a}tzliche Arbeit hervorgerufen wird und sich demnach nur auf den technischen Fortschritt zur{\"u}ckf{\"u}hren l{\"a}sst.\newline 


Das Solow-Model ist der Ausgangspunkt vieler weiterer Wachstumstheorien und Str{\"o}mungen, die auf den folgenden Seiten skizziert werden \citep{Aghion.2015}.


\paragraph{Ramsey-Modell}
In seinem dynamischen Model maximiert \citet{Ramsey.1928} die Wohlfahrt {\"u}ber einen unendlichen Zeithorizont intertemporal. Dabei unterscheidet sich seine Arbeit von der Solows durch die Annahme hinsichtlich der Beschaffenheit der Sparquote. Im Solow-Modell ist diese konstant und somit exogen, wohingegen Ramsey sie endogenisiert. Darin liegt auch der Kern seines Modells: die Konsum- bzw. Sparentscheidung der Haushalte. Sein endogenes Wachstumsmodell bestimmt den optimalen Konsumpfad in Form der Keynes-Ramsey-Regel, indem der Nutzen intertemporal maximiert wird, ergibt sich die optimale Wachstumsrate des Konsums \cite{Ramsey.1928}.
\bigskip


\citet{Solow.1956} und \citet{Ramsey.1928} stehen stellvertretend für die exogenen Wachstumsmodelle, die die Ursachen des technischen Fortschirtts vernachlässigen. Diese vorhandenen Erklärungsdefizite der exogenen Modelle versuchen die endogenen Modelle zu beheben. 


\subsection{Endogene Wachstumsmodelle}
Bis zu den neueren Wachstumstheorien oder auch endogenen Wachstumstheorien wurden weder die Möglichkeit unvollständiger Konkurrenz noch Externalitäten als Einflussfaktoren auf das Wirtschaftswachstum berücksichtigt. Externe Effekte durch Investitionen in Human- oder Sachkapital können zu einem gesamtwirtschaftlich langfristigen Wachstum führen, unabhängig davon, ob der Effekt intraindustriell eine Branche betrifft, oder aber interindustriell branchenübergreifend wirkt. Das hier vorherrschende Beispiel für einen positiven externen Effekt entsteht durch zunehmende Bildung, denn ein höherer Bildungsstand verbessert nicht nur die eigene Produktivität im Berufsleben, sondern trägt auch zur Verbreitung von Wissen bei, wie durch die Weitergabe an die nächste Generation.\\
Wird in der Theorie von unvollständigem Wettbewerb ausgegangen, birgt dies für Unternehmen Anreize den technischen Fortschritt zu beschleunigen, um von Monopolmacht profitieren zu können. \\
Eine weiteres Charakteristika endogener Wachstumsmodelle ist, dass sie nicht von abnehmenden Grenzerträgen des Kapitals ausgehen.\\


\citeauthor{Gandolfo.1998}s \citeyear{Gandolfo.1998} Struktur, Wachstumsmodelle hinsichtlich ihrer Wachstumsursachen, Faktorakkumulation und technischem Fortschritt, zu untergliedern kann auch bei den endogenen Modellen angewandt werden. Die folgenden Abbildung \ref{endoWachstumsmodelle} spezifiziert die Ursache und ordnet entsprechend charakterisierende Modelle zu \citep{Frenkel.1999}.\\


		\begin{figure}[h!]
			\centering 
			\begin{tabular}{@{}r@{}} 
				\psfrag{e}{X} 
				\includegraphics[width=0.78\textwidth]{images/Abbildungen/uebersichtEndogene.eps}\\
				\hfill\footnotesize\sffamily\textbf{Quelle:} ENTWURF in Anlehnung an \citep{Frenkel.1999} %\cite{pps} 
			\end{tabular}  
			\caption{Übersicht endogener Wachstumsmodelle} 
			\label{endoWachstumsmodelle}
		\end{figure}


Endogenen Wachstumsmodelle werden von \citet{Frenkel.1999} in zwei Strömungen untergliedert. Wird der Technologieparameter als konstant angenommen, ist Wachstum auf die Kapitalakkumulation zurückzuführen. Diese Modelle zeigen, dass auch ohne technischen Fortschritt das Grenzprodukt des Kapitals nicht abnimmt. Die Zweite Strömung endogenisiert den technischen Fortschritt, indem Innovationen aktiv angestrebt werden \citep{Frenkel.1999}. Beiden Strömungen ist jedoch gemein, dass in diesen Modellen  die Wissenschaftler die M{\"o}glichkeit haben auf das Wissen vorangegangener Generationen zur{\"u}ckzugreifen, aus diesen zu lernen und das Wissen weiter aufzubauen. Der endogene Faktor besteht in der Weitergabe des Wissens, also dem daraus resultierenden augenblicklichen Wissensstand und nicht in einer erh{\"o}hten Investitionst{\"a}tigkeit in den Forschungssektor \citep{Romer.1990,Rebelo.1991}.
 
\subsubsection{Endogene Wachstumsmodelle mit konstantem Technologieparameter}


Wird von einer Linearität zwischen dem Kapital und dem Volkseinkommen ausgegangen, dann handelt es sich um eine konstante Kapitalproduktivität, die ein abnehmendes Grenzprodukt des Kapitals ausschlie{\ss}t, so wie im AK-Modell. 


\paragraph{AK-Modell}


Das AK-Modell ist ein weiteres richtungsweisendes Modell, eines der er\-sten endogenen Wachstumsmodelle in Hinblick auf den technischen Fortschritt und basiert auf dem Papier von \citet{Rebelo.1991}. Es unterscheidet sich dahingehend vom Solow-Modell, dass der technische Fortschritt den abnehmenden Grenzertr{\"a}gen entgegenwirkt und diesen "`Wachstumshemmer"' unterbindet.  Der technische Fortschritt wird nicht einzeln aufgef{\"u}hrt, sondern bedingt die Akkumulation von Humankapital, die ein Bestandteil der allgemeinen Kapitalakkumulation ist. Die Produktionsfunktion besteht, wie der Name des Modells bereits sagt, aus Kapital $K$ und der Konstanten $A$, jedoch ohne abnehmende Ertr{\"a}ge.

	\begin{equation}
		Y=AK
	\end{equation}

Er modelliert ein endogenes Wachstumsmodell, obwohl er von konstanten Skalenertr{\"a}gen ausgeht. Denn \citet{Rebelo.1991} erachtet, anders als \citet{Romer.1990}, steigende Skalenertr{\"a}ge als nicht notwendig, um Wachstum zu generieren, sofern f{\"u}r die Investitionsg{\"u}terproduktion nur akkumulierbare Einsatzfaktoren eingebracht werden \citep{Rebelo.1991}. \newline


Die Kapitalakkumulation entspricht der des Solow-Modells und ist demnach der Gleichung \eqref{Bewegungsgesetz Solow} zu entnehmen. Die Wachstumsrate $g$ der {\"O}konomie beschreibt das langfristige Wachstum und wird durch eine hohe Ersparnis des BIPs hervorgerufen.

	\begin{equation}
		g=\frac{\dot{K}}{K}=s\frac{Y}{K}-\delta=sA-\delta
	\end{equation}

Das Modell kann sowohl auf industrialisierte L{\"a}nder als auch auf Entwicklungsl{\"a}nder angewendet werden. Der beschriebene  Wachstumsprozess ist unabh{\"a}ngig von der Entwicklung der {\"u}brigen Welt und schlie{\ss}t zun{\"a}chst den Handel mit anderen Volkswirtschaften aus. Wird dieser ber{\"u}cksichtigt, dann ver{\"a}ndern sich die Bedingungen der Kapitalakkumulation und das Modell m{\"u}sste modifiziert werden.\newline Das AK-Modell ist immer dann hilfreich, wenn die Unterscheidung von Innovation und Akkumulation irrelevant ist. Da in diesem Rahmen unter anderem der Einfluss von Innovationen untersucht werden soll, werden im folgenden die innovationsbasierten Wachstumsmodelle genauer betrachtet \citep{Aghion.2015}.


\paragraph{Uzawa-Lucas-Modell} In diesem Modell verhindert die Akkumulation von Sach- und Humankapital ein abnehmendes Grenzprodukt, jedoch nicht durch eine Ausweitung des Kapitals, wie dies zuvor bei der Faktormehrung exogener Modelle der Fall war, sondern durch eine Erhöhung der Produktivität des Kapitals. Bildung stellt in dem Modell von Uzawa-Lucas den Hauptgrund für die Akkumulation von Humankapital dar und erklärt damit langfristiges Wachstum.\footnote{Eine ausführliche Darstellung folgt in Kapitel \ref{Papier2}.}


\paragraph{Learning-by-doing}
Die dritte Strömung endogener Modelle mit konstantem Technologieparameter bilden sogenannte "`Learning-by-doing"' Modelle. Auch hier steigt die Produktivität der Faktoren an und das abnehmende Grenzprodukt des Kapitals wird durch Externalitäten unterbunden \cite{Arrow.1962}. Das hier thematisierte Learning-by-doing führt zu den positiven Externalitäten, dem informellen Lernen.


\subsubsection{Endogene Wachstumsmodelle mit variablem Technologieparameter}
Der Schwerpunkt dieser Modelle liegt auf der Endogenisierung des technischen Fortschritts. Indem die Annahme des vollständigen Wettbewerbs aufgehoben wird, sind die Unternehmen bestrebt durch Forschung und Entwicklung, das Gut oder den Produktionsprozess weiter zu entwickeln, um zusätzliche Gewinne durch Monopolmacht abschöpfen zu können. Demnach ist der Technologieparameter variabel und zurückzuführen auf innovationsbasierte Ansätze. 


\paragraph{Romer-Modell}
Ein Vertreter der innovationsbasierten Wachstumsmodelle, Paul Romer, verfolgt diesen Schwerpunkt, den des endogenen technischen Fortschritts, im Zwischengutsektor. Romer wurde 1955 in Denver geboren und begr{\"u}ndete die endogene Wachstumstheorie \citep{Lin.2007}, da er das Modell Solows um den Faktor Wissen erweitert und dadurch den Ansatz der Wissenschaft neu gestaltete. Er sieht den Motor des Wachstums im Wissen und der Ideenentwicklung, da Wissensvermehrung intertemporale externe Effekte mit sich bringt. Wissen als immaterielles Gut weist die Eigenschaft nicht abnehmender Grenzertr{\"a}ge auf und kann somit nicht aufgebraucht werden. Der technische Fortschritt als direkte Wachstumsquelle wurde bislang nicht in den theoretischen Modellen ber{\"u}cksichtigt und modelliert. Er galt als exogen und wurde als nichterkl{\"a}rbar gegeben hingenommen.
Romers Ansatz zeigte, dass der Faktor Wissen technologischen Fortschritt generierte und es gelang ihm diesen in die Modellwelt zu integrieren und dadurch letztendlich auch zu kalkulieren. In seinem Modell erhöhen horizontale Innovationen im Zwischengütersektor die Produktivität, was zu dauerhaftem Wachstum führt.\newline

Seine Gedanken formulierte \citet{Romer.1990} in seinem endogenen Wachstumsmodell des Aufsatzes "`Endogenous Technical Change"', indem er ein drei Sektoren Modell vorstellt bestehend aus dem Forschungs- und Entwicklungssektor, dem Zwischengutsektor und dem Endproduktsektor. Der stetige Wissenszuwachs durch Forschungsaktivitäten f{\"u}hrt zu zunehmender Produktvielfalt im Zwischengutsektor und bewirkt langfristig einen Anstieg des Einkommens, aufgrund der stärkeren Spezialisierung und Arbeitsteilung. Daf{\"u}r notwendig ist jedoch Humankapital, also F{\"a}higkeiten der Menschen, die dieses Wissen erzeugen. Desto mehr Humankapital im Forschungs- und Entwicklungssektor eingesetzt wird, desto mehr Produktvarianten der Zwischengüter, Innovationen, werden entwickelt und desto höher ist das Wachstum \citep{Romer.1990}.\newline


Der Produktionsprozess des technischen Fortschritts durch Innovationen regt zwar das Wirtschaftswachstum an, jedoch m{\"u}ssen auch die Kosten dieser ber{\"u}cksichtigt werden. Je aufwendiger und somit kostenintensiver ein Innovationsprozess ist, desto eher kann eine Innovation vor Nachahmern gesch{\"u}tzt werden.\footnote{Die Innovationen im Zwischengutsektor führen zu der Marktform der monopolistischen Konkurrenz. Ein patentunabhängiger Schutz der Monopolmacht sind die Kosten für die Entwickung bzw. Nachahmung der Innovation.} Ist eine Innovation jedoch zu kostenintensiv, {\"u}bersteigen die Kosten die m{\"o}glichen resultierenden Gewinne, dann wird sie nicht produziert und eingesetzt. \newline Ein formal detaillierterer Blick auf das Romer Modell zeigt den Prozess der Entwicklung von Produktvariationen durch Innovationen. Diese sind nicht zwingend qualitativ besser, f{\"u}hren jedoch zu einem h{\"o}heren Produktivit{\"a}tswachstum. \newline Die Produktionsfunktion \eqref{Produktionsfunktion Dixit} basiert auf der des Modells von \citet{Dixit.1977} und beschreibt die Produktion verschiedener Varianten $i$, mit $i=[0;N_t]$, eines Zwischenprodukts mit dem Produktionsfaktor Kapital $K_{it}$.

	\begin{equation}
		Y_t= \sum_{i=0}^{N_t} K_{it}^\alpha 
	\end{equation}\label{Produktionsfunktion Dixit}


Der Kapitalstock $K_t$, kann aufgrund der Symmetriebedingung gleichm{\"a}{\ss}ig auf $N_t$ Varianten aufgeteilt werden und f{\"u}hrt zu folgender Formulierung der Produktionsfunktion.

	\begin{equation}
		Y_t=N_t^{1-\alpha}K_t^\alpha
	\end{equation}

Laut dieser Gleichung ist hier der Produktivit{\"a}tsparameter der {\"O}konomie der Grad der Produktvielfalt $N_t$. Je gr{\"o}{\ss}er der Grad ist, desto gr{\"o}{\ss}er ist das Produktionspotenzial eines Landes. Der Kapitalstock wird auf eine gr{\"o}{\ss}ere Zahl von Produktvarianten aufgeteilt, wobei jede durch abnehmende Grenzertr{\"a}ge gepr{\"a}gt ist. Dauerhaftes Wachstum resultiert hier aus der stetigen Entwicklung neuer Produktvarianten. Das Modell zeigt die Rolle technologischer Spillover-Effekte im Sinne der Technologiediffusion.\\


In diesem Modell führt eine Innovation zu neuen Produktvarianten, dabei wird der Prozess der sch{\"o}pferischen Zerst{\"o}rung nicht ber{\"u}cksichtigt. Das Ersetzten "`alter"' Produkte durch neu entwickelte und qualitativ hochwertigere ist Kern, der schumpeterianischen Wachstumsmodelle.

\paragraph{Modelle nach dem Gedanken Schumpeters}\label{sec:schumpeter}
Neben dem Romer-Model z{\"a}hlen auch die Modelle zu den innovationsbasierten Modellen, die dem Gedanken Schumpeters folgen. Der Ansatz beruht auf dem Prozess der schöpferischen Zerstörung, deren Idee von ihm erstmals in seiner Monographie von 1912 entwickelt wurde \citep{Schumpeter.1934a}. Neue qualit{\"a}tsverbessernde Innovationen ersetzten vorherige und zerst{\"o}ren somit deren Bedeutung. Dabei steht die Entwicklung von Innovationen im Vordergrund und Wachstum entsteht als unbeabsichtigtes Nebenprodukt.\\


Zu dieser Gruppe endogener Modelle zählt auch das Wachstumsmodell von \citet{Aghion.1992}, das auch die vertikalen Innovationen betont. Es basiert auf dem Ansatz Schumpeters mit dem Konzept der Sch{\"o}pferischen~Zerst{\"o}rung. Sie untersuchen Wachstumseffekte, aus denen Innovationen resultieren, die auf Grund von Wissensakkumulation entstanden sind. Anders als im Romer-Modell ersetzt jede Innovation eine vorherige und es gibt keine zus{\"a}tzliche Variante des Gutes. Einerseits entmutigt dieser fortw{\"a}hrende Erneuerungsprozess die Unternehmer weitere Forschung zu betreiben, da sie einer st{\"a}ndigen Bedrohung der Veralterung ausgesetzt sind. Andererseits motiviert der anhaltende Wettbewerb die Unternehmen zu Entwicklung effizienterer Produktionsprozesse bzw. verbesserter Zwischeng{\"u}ter, um die Monopolstellung auf einem Markt zu erlangen \citep{Aghion.1992}.\\ 


Das folgende Ein-Sektor-Modell geht auf die bereits erwähnte Arbeit von \citet{Aghion.1992,Aghion.1998} zur{\"u}ck und ber{\"u}cksichtigt den dort angesprochenen Austausch von G{\"u}tern durch qualitativ hochwertigere Produktvarianten. Danach bleibt die Summe der Produkte gleich und weitet sich nicht mit jeder weiteren Innovation aus. Die Grundidee basiert auf der Betrachtung einzelner Industrieebenen $i$ mit der allgemeinen spezifischen Produktionsfunktion:

	\begin{equation}
		Y_{it}=A_{it}^{1-\alpha}K_{it}^\alpha \qquad \text{mit}\quad 0 < \alpha < 1 \label{Produktionsfunktion Industrien Schumpeter}
	\end{equation}

Auch hier ist $A_{it}$ wieder der Produktivit{\"a}tsparameter zum Zeitpunkt $t$ der Industrie $i$ und f{\"u}hrt neben einem Zwischengut $K_{it}$ zur Produktion des Gutes $Y_{it}$. Das Modell beschreibt die Herstellung eines Endproduktes durch den Einsatz eines Zwischengutes. Der technische Fortschritt liegt also im Zwischengutsektor.  Ein Zwischenprodukt wird von einem Innovator hergestellt und ersetzt die vorherige Innovation. Je schneller eine Volkswirtschaft in diesem Modell w{\"a}chst, desto h{\"o}her ist die Fluktuation bei den technologisch f{\"u}hrenden Firmen. \newline Wachstum entsteht somit durch die Verbesserung der Produktqualit{\"a}t. Formal bedeutet dies, dass der Produktivit{\"a}tsparameter $A_t$ von $A_{t-1}$ auf $A_t=\gamma A_{t-1}$, mit $\gamma>1$, steigt und somit direkt aus innovativen T{\"a}tigkeiten resultiert. F{\"u}r die Entwicklung dieser Neuerungen muss es, neben dem Produktionssektor, auch einen Forschungssektor geben. Die Kosten f{\"u}r die Forschung entsprechen den verwendeten Endprodukten, die als  Faktoreinsatz fungieren. Mit zunehmendem Forschungsaufwand, der zu steigenden Kosten führt, erhöht sich die Wahrscheinlichkeit einer erfolgreichen Innovation. 
%\textcolor[rgb]{1,0,0}{(und die Unsicherheit sinkt})
. \newline Die Motivation in Forschung zu investieren liegt in der M{\"o}glichkeit Monopolmacht zu erlangen und h{\"o}here Einnahmen zu generieren. Schumpeter war der erste, der die Rolle des Monopols thematisierte in Bezug auf Innovationen und der Entstehung im Forschungs- und Entwicklungssektor. \newline Unter der Annahme, dass alle Industrien eines Landes identisch sind, kann Gleichung \eqref{Produktionsfunktion Industrien Schumpeter} auch auf aggregierter Ebene formuliert werden.

	\begin{equation}
		Y_t=A_t^{1-\alpha}K_t^\alpha
	\end{equation}

Wird neben der Innovation auch die Möglichkeit einer Imitation berücksichtigt, wird davon ausgegangen, dass bereits ein gewisser Bestand an Innovationen vorhanden ist, das gegenwärtige technische Wissen. Die langfristige Wachstumsrate $g_t$ entspricht der Wachstumsrate des arbeitsvermehrenden Produktivit{\"a}tsfaktors $A_t$ und wird im folgenden genauer betrachtet. Das technische Wissen ist {\"o}ffentlich verf{\"u}gbar und kann durch erfolgreiche Innovatoren erweitert werden \citep{Aghion.1992,Aghion.1998}.  Bei einer Innovation ver{\"a}ndert sich der Technologieparameter $A_t$ um das $\gamma$-Fache und die Welttechnologiegrenze $\bar{A}_t$ wird um die Neuerung erweitert. Handelt es sich um eine Imitation, dann ver{\"a}ndert sich nur der lokale Technologiebestand, indem eine Produktvariante nachgeahmt wird, die bereits auf dem Weltmarkt existiert, nicht jedoch in dem betrachteten Land. Beide Prozesse bilden den lokalen technologischen Wissensstand eines Landes und k{\"o}nnen formal folgenderma{\ss}en ausgedr{\"u}ckt werden: 

	\begin{equation}
		\dot{A_t}= A_{t+1}-A_t=\mu_n(\gamma-1)A_t+\mu_m(\bar{A}_t-A_t)
	\end{equation}

Bei $\mu_n$ und $\mu_m$ handelt es sich um die Frequenz bzw. Intensitäten der Innovations-  bzw. Imitationsentwicklung, die exogen sind. Daraus l{\"a}sst sich die Wachstumsrate des technischen Fortschritts ableiten.

	\begin{equation*}
		g_t=\hat{A_t}= \frac{A_{t+1}-A_t}{A_t} = \mu_n(\gamma-1)+\mu_m(\frac{\bar{A}_t}{A_t}-1)
	\end{equation*}

Die Relation $A_t/\bar{A}_t$ beschreibt den Abstand zur Welttechnologiegrenze $a_t$ und l{\"a}sst somit Aussagen zum relativen technologischen Entwicklungsstand zu \citep{Aghion.1992,Aghion.1998}.

	\begin{equation}
		g_t=\hat{A_t}=\frac{A_{t+1}-A_t}{A_t} = \mu_n(\gamma-1)+\mu_m(a_t^{-1}-1)
	\end{equation}

Dieses schumpeterianische Grundmodel eignet sich besonders zur Analyse der Reaktion des Abstands zur WTG durch die jeweilige Wachstumsrate eines Landes. Interessant ist dabei der Aspekt der Konvergenz zur globalen Grenze, die sich durch verschiedene wirtschaftspolitische Ma{\ss}nahmen justieren l{\"a}sst. \newline 
Ein Fazit des Ein-Sektor-Modells nach Schumpeter ist, dass sich die langfristige Wachstumsrate aus den relativen H{\"a}ufigkeiten der entwickelten Innovationen ergibt, wobei die Reichweite oder auch Wirkungskreis der Innovation ebenfalls ber{\"u}cksichtigt werden muss. Bei dem Ein-Sektor-Modell wird nur ein Gut ersetzt, wohingegen im mehrsektoralen Modell mehrere Produkte durch Innovationen erneuert werden k{\"o}nnen. Der entscheidende Unterschied zum Ein-Sektor Modell liegt darin, dass eine Innovation nicht mehr bedingt durch Zufall entwickelt wird. Sofern in einem Sektor  nicht erfolgreich innoviert wird, kommt es in einem anderen Sektor zu einer erfolgreichen Innovation mit der entsprechenden Wahrscheinlichkeit von $\nu$. Daraus ergibt sich die durchschnittliche aggregierte Produktivit{\"a}t in der multisektoralen Variante von:

	\begin{equation}
		A_t=\nu A_{1t}+(1-\nu)A_{2t} \footnote{Die hier angeführten Indizes 1 und 2 unterscheiden die Produktivitäten zweier Sektoren 1 und 2 voneinander.}
	\end{equation}

Auch dieses schumpeterianische mehrsektorale Modell folgt dem Ansatz von \citet{Aghion.1998}. Ein anderes schumpeterianisches Modell von \citet{Reinganum.1985} beschreibt die andauernde Entwicklung von Innovationen als evolutions{\"a}hnlichen Prozess im Sinne der Sch{\"o}pferischen Zerst{\"o}rung.\\
Die hier kurz angerissenen Modelle sind Vorreiter des in Kapitel \ref{Papier1} behandelten Wachstumsmodells. Dieses ist demnach in die Gruppe der innovationsbasierten endogenen schumpeterianischen Wachstumsmodelle einzubetten. 


Zusammenfassend lässt sich festhalten, je mehr eine Innovation die Produktivit{\"a}t steigert, desto st{\"a}rker steigt die Wachstumsrate. Demzufolge sollte als wachstumsf{\"o}rdernde Ma{\ss}nahme vermehrt in den Forschungssektor investiert werden. Dies wiederum steigert die Nachfrage nach Wissenschaftlern in diesem Bereich, die nur durch die zus{\"a}tzliche Ausbildung der Arbeiter befriedigt werden kann. Ein weitsichtiges strategisches Vorgehen ist demnach der Ausbau des Bildungssektors, damit f{\"u}r wachstumsf{\"o}rdernde Ma{\ss}nahmen ausreichend qualifizierte Arbeit vorhanden ist \citep[Kapitel 4]{Aghion.2015}.\footnote{Diesem kausalem Zusammenhang folgt auch der Hauptteil dieser Arbeit, zunächst wird der Ausbau des Bildungssektors durch Au{\ss}enhandel stimuliert. Das dadurch entstehende erhöhte Angebot qualifizierter Arbeit ist für innovierende und imitierende Tätigkeiten notwendig, da andernfalls eine Weiterentwicklung des technischen Entwicklungsstandes gehemmt werden würde.} \\


Bildung und die damit einhergehende Humankapitalakkumulation steht im Vordergrund des folgenden Abschnitts, der die Vielfalt der unterschiedlichen Vorgehens- und Betrachtungsweisen darlegt. 


\subsubsection{Humankapitaltheorien}
Die Humankapitaltheorien stellen eine Unterkategorie der Wachstumstheorien dar, die sich mit der Akkumulation von Humankapital beschäftigen und dadurch Wirtschaftswachstum erklären. Hierzu zählt auch das Uzawa-Lucas-Modell, dass im Laufe dieser Arbeit bereits erwähnt wurde. Die verschiedenen Theorien begründen die Unterschiede von Bildung und zeigen wie deren Einfluss auf das Wirtschaftswachstum interpretiert werden kann.


\paragraph{Mincer Modell}
Die Humankapitaltheorie geht urspr{\"u}nglich zur{\"u}ck auf die Arbeiten von \citet{ Becker.1965} und \citet{Mincer.1974}, die zwei Schwerpunkte berücksichtigten. Zum einen die  produktionssteigernde Rolle des Humankapitals f{\"u}r den Produktionsprozess und zum anderen die Motivation der Arbeiter in Humankapital zu investieren. So unterscheiden sie zwischen einer Grundausbildung und einer berufsbegleitenden Ausbildung. Dabei gilt jegliche Bildung, die vor der ersten Besch{\"a}ftigung in einem Unternehmen genossen wurde als Grundausbildung. Die Opportunit{\"a}tskosten eines weiteren Schuljahres entsprechen dem entgangenen Verdienst durch eine Anstellung \citep{Mincer.1974}. Eine Ausbildung w{\"a}hrend eines Angestelltenverhältnisses als eine Art Zusatzausbildung neben dem Beruf wird auch als Training-on-the-Job bezeichnet \citep{Acemoglu.2009}. Den Schwerpunkt des Mincer Modells bildet dabei die Grundausbildung.


\paragraph{Ben-Porath-Modell}
Dieses Modell der Humankapitaltheorie unterscheidet sich von dem Mincer Modell, indem auch Bildungsm{\"o}glichkeiten w{\"a}hrend einer Berufst{\"a}tigkeit ausgef{\"u}hrt werden k{\"o}nnen und sich diese nicht ausschlie{\ss}lich auf die Zeit vor dem Berufsleben beschr{\"a}nken. Der Fokus der Arbeit von \citet{BenPorath.1967} liegt demnach auf dem Training-on-the-Job. Dabei wird auch eine Minderung des Humankapitals ber{\"u}cksichtigt, weil  davon ausgegangen wird, dass durch den Einsatz von Maschinen das vorher noch notwendige Humankapital obsolet wird \citep{BenPorath.1967,Heckman.1998,Guvenen.2012,Manuelli.2014}. Die Bedeutung des Modells ist vor allem darauf zur{\"u}ck zuf{\"u}hren, dass neben der Schulausbildung eine Vielzahl weiterer M{\"o}glichkeiten existieren in Humankapital zu investieren. Au{\ss}erdem kommt er zu der These, dass Volkswirtschaften mit hohen Ausgaben f{\"u}r Schulbildung ebenso hohe Anspr{\"u}che bez{\"u}glich der berufsbegleitenden Weiterbildungsm{\"o}glichkeiten haben und diese durch das System der gesicherten Grundausbildung nicht gemildert werden \citep{BenPorath.1967}.


\paragraph{Uzawa-Lucas-Modell}
Das Uzawa-Lucas-Modell, besch{\"a}ftigt sich ebenfalls mit wirtschaftlichem Wachstum, welches durch die Humankapitalakkumulation bedingt ist und deswegen als Motor des Wachstums bezeichnet wird \citep{Lucas.1988}. Im Rahmen des AK-Modells \citep{Rebelo.1991} betrachtet \citet{Lucas.1988}, inspiriert durch den Aufsatz von \citet{Becker.1964} und basierend auf dem Modell von \citet{Uzawa.1965}, Humankapital als einzigen Einsatzfaktor im Bildungssektor und untersucht das dadurch angeregte Wirtschaftswachstum. Sowohl im Uzawa-Lucas-Modell als auch im AK-Modell wird Wachstum durch Faktormehrung generiert. Im AK-Modell wird dauerhaftes Wachstum durch Kapitalakkumulation hervorgerufen, wohingegen Lucas zwischen Sach- und Humankapital differenziert und es wird neben Humankapital auch hier auch physisches Kapital akkumuliert. Beide, Sach- und Humankapital, verhalten sich komplement{\"a}r zueinander, denn durch den Anstieg von physischem Kapital steigt die Nachfrage nach qualifizierter Arbeit st{\"a}rker an, als nach relativ unqualifizierter Arbeit. Das bedeutet, dass die maximale Produktivit{\"a}t einer Volkswirtschaft dann erreicht wird, wenn beide ausgeglichen sind und gleichmä{\ss}ig ansteigen.\\


Die Haushalte müssen sich zwischen der Arbeit im Konsumgutsektor und Bildung entscheiden. Dadurch entsteht ein trade-off zwischen heutigem und morgigem Konsum, da neben der Erwerbstätigkeit in Bildung investiert werden kann. Durch einen gegenwärtigen Verzicht auf Lohneinkommen und stattdessen einer Investition in Bildung ist der zuk{\"u}nftige Konsum h{\"o}her. Mit diesem Zusammenhang wird sich in aller Ausführlichkeit in Kapitel \ref{Papier2} auseinander gesetzt und daher an dieser Stelle nicht näher beschrieben.


\paragraph{Modell von Nelson und Phelps}
Eine vollkommen andere Perspektive auf die Bedeutung des Humankapitals etablieren \citet{ Nelson.1966}. Zwar vertreten sie auch die Ansicht, dass eine korrekte Ergr{\"u}ndung von Wachstum mit der Einbeziehung von Bildung einhergehen muss, jedoch ist Humankapital  hier kein direkter Einsatzfaktor, der die Produktivität erhöht.\footnote{Eine ähnliche Idee ist auch auf die Arbeit von \citet{Schultz.1964,Schultz.1975} zurückzuführen.} In diesem Ansatz begünstigt Humankapital nicht die Produktivit{\"a}t bekannter Aufgaben, sondern erm{\"o}glicht zu der F{\"a}higkeit unbekannte Abl{\"a}ufe, Technologien und G{\"u}ter zu adaptieren. Wachstum wird erzeugt durch die produktivit{\"a}tssteigernde Implementierung von Imitationen. \\
Dieser Unterschied in der Auffassung ist auch in der Modellierung der darauf aufbauenden Theorie gut sichtbar. Denn Humankapital hat keinen direkten Einfluss auf die Produktionsfunktion und bedingt nur den technologischen Wissensstand eines Landes durch die Implementierung bereits vorhandener Technologien der Welttechnologiegrenze. Dabei differenzieren sie erstmals zwischen unterschiedlichen Fähigkeiten und lassen eine Gewichtung dieser zu. Bislang wurde davon ausgegangen, dass mit steigender Humankapitalausstattung die Produktivität aller Aufgaben zunimmt. Jedoch unterscheiden \citet{Nelson.1966} zwischen innovierenden und adaptierenden Tätigkeiten und Fähigkeiten. 
Dieses Modell beschreibt erstmals den direkten Einfluss von Humankapital auf das Wirtschaftswachstum \citep{Nelson.1966}.\\
Dargestellt in einem schlichteren Ansatz nach \citet{Nelson.1966} in einer Variation von \citet[Kapitel 10]{Acemoglu.2009}, ist die einzige ver{\"a}nderbare Gr{\"o}{\ss}e der lokale Technologieparameter $A$, die WTG ist exogen gegeben. Der lokale technologische Wissensstand und somit eine Verbesserung der Technologien ergibt sich aus zwei Komponenten, der intrinsischen Ver{\"a}nderung der Produktivit{\"a}t, welche Produktivitätswachstum wie beispielsweise durch learning-by-doing darstellt und durch die Nachahmung fortschrittlicherer neuer Technologien der WTG. Der Erfolg der Nachahmung wird dabei wesentlich von der durchschnittlichen Humankapitalausstattung eines Arbeiters beeinflusst. Ist der Arbeiter nicht ausreichend qualifiziert, dann k{\"o}nnen keine Technologien der WTG adaptiert und implementiert werden. Je besser die Unternehmer ausgebildet sind, desto eher kann adaptiert werden. Dadurch ergibt sich die M{\"o}glichkeit im Entwicklungsprozess zu anderen f{\"u}hrenden L{\"a}ndern aufzuschlie{\ss}en. Empirisch belegt wurde die Theorie unter anderem von \citet{Foster.1995} am Beispiel der Produktivit{\"a}t im Landwirtschaftssektor.


\paragraph{Modell von Benhabib und Spiegel}
In dem Aufsatz von \citet{Benhabib.1994} wird das Modell von \citet{Nelson.1966} erweitert und zeigt, dass neben imitativen T{\"a}tigkeiten auch die M{\"o}glichkeit besteht nahe der Welttechnologiegrenze Innovationen zu entwickeln. In ihrer Regressionsanalyse stellten sie einen positiv signifikanten Zusammenhang zwischen der Wachstumsrate und dem Humankapitalbestand fest. Humankapital beeinflusst nach \citet{Benhabib.1994} nicht nur das Wachstum des Pro-Kopf-Einkommens, sondern auch das Wachstum der totalen Faktorproduktivit{\"a}t positiv.  Au{\ss}erdem belegen sie, dass der Abstand zur Welttechnologiegrenze f{\"u}r das Wachstum relevant ist \citep{Benhabib.1994}. Dieser Ansatz zeigt also einen st{\"a}rkeren Zusammenhang zwischen Wirtschaftswachstum und Humankapitalniveaus als zwischen dem Wirtschaftswachstum und der Ver{\"a}nderung des Humankapitals. Denn die Adaption neuer Technologien beeinflusst die Produktivit{\"a}t deutlich st{\"a}rker als eine Produktivit{\"a}tserh{\"o}hung bereits bekannter Aufgaben \citep{Benhabib.1994}. Dieser Gedanke wird auch in Kapitel \ref{Papier1} aufgegriffen, in dem das Humankapitalniveau die Ver{\"a}nderung der Produktivit{\"a}t einer Volkswirtschaft bedingt, sowie deren Imitations- bzw. Innovationsm{\"o}glichkeiten. 


\paragraph{Modell von Krüger und Lindahl}
\citet{Krueger.2001} hingegen untersuchen in den OECD-L{\"a}ndern, dass Bildung zwar zum Aufholprozess, nicht jedoch zur Ausweitung der WTG beitr{\"a}gt.
Sie zeigen die Relevanz der Zusammensetzung des Humankapitalbestandes und der Lage zur WTG eines Landes f{\"u}r das Wirtschaftswachstum. Dabei widerlegten sie einige Ergebnisse von \citet{Benhabib.1994} und stellten fest, dass Wachstum und Humankapital nur innerhalb von OECD-L{\"a}ndern korreliert. In Ländern, die deutlich weniger weit entwickelt sind gilt dieser Zusammenhang nicht. Dies hebt zun{\"a}chst eine gewisse Bedeutungslosigkeit der Humankapitalakkumulation auf den Wachstumsprozess eines Landes hervor, was durch ein kleines Gedankenspiel aus \citet{Krueger.2001} veranschaulicht werden soll. Es soll verdeutlichen, dass nicht nur die Ausstattung mit Humankapital wichtig ist, sondern es vor allem auf deren Zusammensetzung innerhalb eines Landes und den Entwicklungsstand eines Landes ankommt.\footnote{Sofern diese beiden Aspekte unabhängig voneinander sind.} Es werden zwei Länder betrachtet, die dieselbe Humankapitalausstattung vorweisen, sich jedoch hinsichtlich ihrer Zusammensetzung, also hinsichtlich der Qualifikationen ihrer Arbeitskr{\"a}fte, unterscheiden. Land 1 sei in diesem Fall relativ reichlich mit sehr gut ausgebildeten Arbeitnehmern ausgestattet, wohingegen Land 2 relativ mehr traditionell weniger gut ausgebildete Arbeitskräfte vorweist. Je nach Lage zur WTG entwickelt sich das eine oder das andere Land schneller. Nahe der Technologiegrenze sind besser ausgebildete Arbeiter wichtiger, demnach wird Land 1 sich schneller entwickeln als Land 2, welches den gleichen Entwicklungsstand nahe der WTG hat. Handelt es sich bei beiden Ländern um weniger weit entwickelte Volkswirtschaften mit einem gro{\ss}en Abstand zur WTG, dann weist Land 2 das höhere Wachstumspotenzial auf mit einer reichlicheren Ausstattung weniger gut ausgebildeter Arbeitskräfte als Land 1. Also wird das Land, welches relativ reichlicher mit h{\"o}her qualifizierten Arbeitskr{\"a}ften ausgestattet ist, schneller wachsen, wenn der Abstand zur WTG relativ gering ist. Wohingegen das Land, welches relativ reichlicher mit unqualifizierte Arbeitskr{\"a}ften ausgestattet ist, ein h{\"o}heres Wachstum erreicht als das andere, wenn der Abstand zur WTG beider, relativ gro{\ss} ist \citep{Krueger.2001}. So h{\"a}ngt das Wachstumspotential ma{\ss}geblich von der Lage zur WTG, sowie von der Zusammensetzung des Humankapitals ab. Mit zunehmender N{\"a}he zur WTG nimmt die Bedeutung weniger qualifizierter Arbeitskr{\"a}fte ab, wohingegen die der hochqualifizierten zunimmt.\\


Somit ist in diesem Kontext das Humankapital lediglich f{\"u}r den catching-up Prozess ma{\ss}geblich, jedoch nicht f{\"u}r Innovationst{\"a}tigkeiten an der WTG. \citet{Krueger.2001} zeigen in ihrer Abhandlung, dass neben der Lage zur WTG  der Humankapitalbestand allein nicht ausreicht, um das Wachstum eines Landes prognostizieren zu k{\"o}nnen.\newline
Werden Innovationen mit relativ mehr ausgebildeter Arbeit hergestellt als Imitationen, dann hat ausgebildete Arbeit einen gr{\"o}{\ss}eren Effekt auf das Wachstum eines Landes, welches nahe der WTG liegt, als auf ein Land, mit einem gr{\"o}{\ss}eren Abstand zur WTG. Es wurde empirisch belegt, dass es einen positiven Zusammenhang zwischen dem anf{\"a}nglichen Bildungsniveau und dem anschlie{\ss}enden Wachstumsverlauf gibt \citep{Vandenbussche.2006}.\footnote{Dabei verwendeten sie Daten von 19 OECD L{\"a}nder zwischen den Jahren 1960-2000 \citep{Vandenbussche.2006}.}.


\paragraph{Humankapitalexternalitäten - Vorteile urbaner Regionen}
Eine gro{\ss}e Bedeutung kommt auch den Humankapitalexternalitäten zu. Neben dem Uzawa-Lucas-Modell, das Wissensexternalitäten durch Spillover-Effekte anführt, gibt es noch zahlreiche weiter positive Effekte. Ein weiteres Beispiel beschreibt das Modell von \citet{Jacobs.1970}, dass eine höhere Produktivität in urbanen Regionen begründet.
Wenn ein gesamtwirtschaftlich hoher Kapitalstock die Produktivit{\"a}t jedes einzelnen Arbeiters erh{\"o}ht, dann kann dies auf Wissens-Spillover-Effekte zurückgeführt werden. Denn ein Ideenaustausch innerhalb der erwerbst{\"a}tigen Bev{\"o}lkerung ist wahrscheinlicher und stimuliert das {\"o}konomsiche Wachstum eher in städtischen Regionen als in weniger stark besiedelten Regionen \cite{Azariades.1990,Lucas.1988}. Empirisch belegt hat die Existenz dieser Externalit{\"a}ten erstmals die Arbeit von \citet{Rauch.}. Gefolgt von \citet{Acemoglu.2000}, die diese nicht f{\"u}r die unterschiedliche Bildungsniveaus amerikanischer Gro{\ss}st{\"a}dte überprüften, sondern die These auf dadurch bedingte Bildungsunterschiede zwischen Staaten ausweiteten. Mit dem Ergebnis, dass die Humankapitalexternalit{\"a}ten relativ klein sind und eher weniger Bedeutung beigemessen werden sollte.\footnote{\citet{Duflo.2004} und \citet{Ciccone.Apr} belegen diese Ergebnisse ebenfalls anhand der Daten von Indonesien und den USA.} Eineindeutig sind diese Ergebnisse jedoch nicht, denn \citet{Moretti.2004} zeigt wiederum einen gro{\ss}en Effekt der Externalit{\"a}ten auf das {\"o}konomische Wachstum.\\