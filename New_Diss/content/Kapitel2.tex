\chapter[Wachstum durch technischen Fortschritt]{Wachstum durch technischen Fortschritt}\label{Wachstum}
\chaptermark{Wachstum}
Zun{\"a}chst werden terminologische und theoretische Grundlagen zum Wachstum durch technischen Fortschritt vorgestellt, die dem besseren Verständnis der folgenden Untersuchungen dienen sollen. Wirtschaftliches Wachstum kann sehr allgemein definiert werden, als den Anstieg der gegenwärtigen Gütermenge einer Volkswirtschaft oder nach \citet{Frenkel.1999} als die quantitative Zunahme eines volksiwrtschaftlich erwirtschafteten G{\"u}terbergs. Mit der Zunahme des G{\"u}terbergs einer Volkswirtschaft steigt das Volkseinkommen an. Etwas pr{\"a}ziser und empirisch zweckdienlicher formuliert \citet{Bofinger.2015} Wachstum als intertemporale Entwicklung des realen Bruttoinlandsprodukts pro Kopf. Dabei beschreibt das Bruttoinlandsprodukt (BIP) die Wirtschaftsleistung bestehend aus dem Gesamtwert der Waren und Dienstleistungen, die innerhalb eines Jahres von einer Volkswirtschaft erbracht werden. Gemessen wird die Rate des Wirtschaftswachstums durch den j{\"a}hrlichen Anstieg des realen Pro-Kopf-Einkommens eines Landes \citep{Bofinger.2015}.\\
Die Hauptursachen des Wirtschaftswachstums sieht \citet{Gandolfo.1998} im Anstieg der Faktor\-ausstattung und dem technischen Fortschritt, wodurch jedoch die Welt des {\"o}konomischen Wachstums sehr stark reduziert wird.\footnote{Je nach Auffassung würden dann bestimmte Einflussfaktoren auf das Wirtschaftswachstum nicht impliziert werden. Weitere mögliche Gründe für Wirtschaftswachstum ist der in Kapitel \ref{sec:Globalisierung} noch folgende Au{\ss}enhandel sowie Institutionen oder auch externe Effekte.} Bei der Faktormehrung resultiert Wachstum durch den zusätzlichen Einsatz von Produktionsfaktoren, wodurch insgesamt mehr produziert werden kann und der von \cite{Frenkel.1999} genannte "`Güterberg"' ansteigt. Technischer Fortschritt kann zu vollkommen neuen Technologien führen oder aber auch zu zusätzlichen Gütervariationen, die neue Märkte schaffen.\\


Eine eineindeutige Definition des \textbf{technischen Fortschritts} ist gemeinhin nicht zu finden und h{\"a}ngt von der Modellvariation ab. So kann als technischer Fortschritt die Folge vieler Innovationen verstanden werden, wobei auch je nach Entwicklungsstand eines Landes Imitationen zum lokalen technischen Fortschritt beitragen und als technischer Fortschritt aufgefasst werden können. Beides jedoch impliziert eine Weiterentwicklung und Ausweitung des Wissensstands. Der technische Fortschritt erhöht die \textbf{totale Faktorproduktivit{\"a}t} und wirkt somit wie eine Faktorvermehrung. Die totale Faktorproduktivität beschreibt die Erhöhung der Produktivität, die nicht durch eine Erhöhung der Produktionsfaktoren Kapital und Arbeit erklärt werden kann. Empirisch belegt wurde die Totale Faktorproduktivität durch das sogenannte Solow-Residuum und ist durch den technischen Fortschritt zu erklären \citep{Solow.1957}. Das Solow-Residuum beschreibt demnach das Wachstum der Produktivität, welches nicht durch das Wachstum aus dem Faktoreinsatz resultiert.\newline Das Ziel des technischen Fortschritts ist es die Wirtschaftlichkeit eines Unternehmens und letztendlich auch einer Volkswirtschaft zu verbessern. Dabei wirkt sich der technische Fortschritt auf die Technologie aus, die direkten Einfluss auf die Produktivit{\"a}t eines Unternehmens hat. Dies ist unabh{\"a}ngig davon, ob der Fortschritt im Produktionsprozess vorkommt oder es sich um das Gut an sich handelt.\\


Nach \citet[Kapitel 1]{Barro.2004} bestimmt sich eine \textbf{Technologie} durch das Verfahren, bei dem Produktionsfaktoren im Herstellungsprozess zu G{\"u}tern umgewandelt werden. \citet{Krugman.2015} verstehen unter einer Technologie eine Art systematische Methodik. Dabei bedienen sich immer dann zwei Unternehmen oder Volkswirtschaften derselben Technologie, wenn sie mit der gleichen Menge an Einsatzfaktoren den gleichen Output generieren k{\"o}nnen. Das Grenzprodukt beider Länder ist gleich gro{\ss}, eine Einheit Kapital oder Arbeit f{\"u}hrt dann in beiden L{\"a}ndern zu dem gleichen anteiligen Endprodukt.\\


In der theoretischen Modellwelt wird eine Technologie beschrieben durch die Produktionsfunktion, in der die Einsatzverh{\"a}ltnisse der Produktionsfaktoren fest vorgegeben sind. Bestandteil der Produktionsfunktion ist ein Technologieparameter, meist abgek{\"u}rzt mit $A$. Dieser Parameter beschreibt das technische Wissen, das im Produktionsprozess eingesetzt wird. Geht das Modell von konstanten Skalenertr{\"a}gen aus, dann ist dieser Parameter konstant und {\"u}ber die Zeit unver{\"a}nderlich. Werden jedoch steigende Skalenertr{\"a}ge angenommen, dann kann es zu einer Weiterentwicklung des technischen Wissens kommen, zu technischem Fortschritt, der dadurch in der Technologie abgebildet wird. Die beiden notwendigen Voraussetzungen für den technischen Fortschritt, das technische Wissen und Kapital wird in Abschnitt \ref{sec:TechnischesWissenHumankapital} genauer erläutert.\\


\citeauthor{Gandolfo.1998}s \citeyear{Gandolfo.1998} Ursachen für Wachstum, Faktorakkumulation und technischer Fortschritt hängen jedoch sehr eng miteinander zusammen, weil beispielsweise eine technische Neuerung den Faktoreinsatz mindern kann und somit dann insgesamt mehr produziert werden w{\"u}rde.\footnote{Immer dann, wenn beispielsweise Wirtschaftswachstum als unbeabsichtigtes Nebenprodukt steigender Skalenertr{\"a}ge bei der Kapitalakkumulation resultiert. Als ein Beispiel f{\"u}r diesen Effekt gilt learning-by-doing, das sich vor allem bei Gr{\"o}{\ss}eneffekten durch die Produktion gro{\ss}er Mengen auswirkt. Denn mit der Produktionsmenge steigen die Lerneffekte der Besch{\"a}ftigten. Das durch die zunehmende Erfahrung hinzugewonnene Wissen verbessert die Abl{\"a}ufe der Produktionsstruktur. Der Produktionsfaktor Arbeit wird produktiver und die Effizienz der Arbeit verbessert sich \citep{Acemoglu.2009}.} Trennt man jedoch beide Argumente strikt voneinander, dann l{\"a}sst dies eine Untergliederung der Wachstumsmodelle in exogene und endogene Modelle zu. Es handelt sich um exogene Wachstumsmodelle, wenn es zu einer Ausweitung der Produktionsfaktoren kommt, obwohl der technische Fortschritt als von au{\ss}en gegeben betrachtet wird und der Grund f{\"u}r sein Dasein ungewiss ist.\newline Endogen ist ein Wachstumsmodell, wenn der technische Fortschritt direkt hervorgerufen wurde, indem gezielt Forschung und Entwicklung betrieben wurde \citep{Gandolfo.1998}.\\


Beispielhaft dient das AK-Modell. Hier resultiert technischer Fortschritt, als Wissen das als ein Nebenprodukt der Kapitalakkumulation hervorgeht. Abweichend von anderen endogenen Wachstumsmodellen wird Wachstum hier nicht durch innovative T{\"a}tigkeiten angeregt, sondern ist ein Ergebnis von Sparentscheidung und Kapitalakkumulation \citep{Rebelo.1991}. Wohingegen \citet{Arrow.1969} den technischen Fortschritt als den Prozess der Reduktion der Unwissenheit beschreibt. Wieder anders verh{\"a}lt es sich im Romer-Modell, in dem das technologische Wachstum durch die Zunahme von Produktvarianten beschrieben wird \citep{Romer.1990}.\footnote{Nachdem hier zunächst Begrifflichkeiten und Grundlagen erörtert werden, werden in Kapitel \ref{sec:Wachstumstheorien} die genannten Modelle genauer erläutert.}\\
Unabhängig von der Interpretation des technischen Fortschritts führt dieser zu einer Ausweitung der Welttechnologiegrenze (WTG). Bei der Welttechnologiegrenze $\bar{A}_t$  handelt es sich um den maximal erzielbaren Wissensstand, der zu einem Zeitpunkt $t$ erreicht werden kann. Vergleicht man die WTG mit dem Wissensstand einer Volkswirtschaft erlaubt dies Aussagen über die relative Lage des Landes zur WTG. So ergibt sich der Abstand zur WTG $a_t$ aus der Relation der lokalen Technologiegrenze (LTG) oder auch der Produktivität eines Landes $A_t$ zu der WTG, somit gilt $a_t = A_t/\bar{A}_t$ \citep{Aghion.1992,Aghion.1998}.\\


In dieser Arbeit wird unter technischem Fortschritt ein Ausbau des technischen Wissensstandes gesehen und impliziert dabei sowohl Innovationen als auch Imitationen, die in der Volkswirtschaft zu einem Erkenntnisgewinn beitragen.\\
Dabei werden hier beide Gr{\"u}nde f{\"u}r Wachstum nach Gandolfo ausführlich behandelt. So geht das Wachstum des Humankapitalmodells in Kapitel \ref{Papier2} auf die Faktorakkumulation zur{\"u}ck, die dann den im zweiten Modell, Kapitel \ref{Papier1}, angef{\"u}hrten Grund f{\"u}r Wachstum, den technischen Fortschritt, beg{\"u}nstigt. Verstärkt wird der technische Fortschritt wesentlich durch die Offenheit der Volkswirtschaften und den sich daraus ergebenden Handelsmöglichkeiten. 


\section[Prämissen des technischen Fortschritts: technisches Wissen und Humankapital]{Prämissen des technischen Fortschritts: \\technisches Wissen und Humankapital \sectionmark{Prämissen des techn. Fortschritts}}\label{sec:TechnischesWissenHumankapital}
\sectionmark{Prämissen des techn. Fortschritts}
Für einen technischen Fortschritt sind sowohl technisches Wissen, als auch Humankapital notwendig. Wird eine Aneinanderreihung von Innovationen als technischer Fortschritt verstanden, sind technisches Wissen und Humankapitel für innovierende Tätigkeiten notwendig \citep{Howitt.2005}. Als technisches Wissen gelten Ideen und Informationen, welches nur in Verbindung mit Kapital verwendet werden kann. Dafür ist es zunächst unerheblich, an welche Kapitalart technisches Wissen gebunden ist. In Kombination mit physischem Kapital tritt technisches Wissen, beispielsweise in Form von Blaupausen, Maschinen oder Gütern, auf. Ist technisches Wissen an den Menschen, also hier den Produktionsfaktor Arbeit, gebunden, dann handelt es sich um Humankapital.

\subsection{Technisches Wissen}\label{sec:techn. Wissen}
Für die Entwicklung einer Innovation ist technisches Wissen zwingend notwendig und wird hervorgerufen durch eine Idee. Die Gestaltung und Ansatzpunkte attraktiver Ideen können sehr verschieden sein. Dazu z{\"a}hlen vor allem die Kostenreduktion durch die Effizienzsteigerung in der Produktion oder aber die Entwicklung vollkommen neuer G{\"u}ter.\newline
Das technische Wissen an sich und auch die Idee ist ungebunden und somit ein {\"O}ffentliches Gut bzw. hat dessen Eigenschaften \citep{Arrow.1962, Nelson.1959}. Öffentliche Güter sind durch die beiden Eigenschaften der Nicht-Rivalit{\"a}t und der Nicht-Ausschlie{\ss}barkeit im Konsum charakterisiert.\newline 


Sofern die Möglichkeit besteht, dass der Konsum von den Anbietern verhindert werden kann, lassen sich die Ertr{\"a}ge dem jeweiligen Produzenten eindeutig zuordnen und es gilt die Ausschlie{\ss}barkeit. Ist diese Eigenschaft nicht vorhanden, sind positive Externalit{\"a}ten die Folge. Im Fall der Ideen und dem technischen Wissen k{\"o}nnen diese von mehreren Unternehmen gleichzeitig umgesetzt werden, ohne dass es von konkurrierenden Unternehmen verhindert wird. Der Anreiz zur Ideengenerierung f{\"u}r das einzelne Wirtschaftssubjekt ist dadurch relativ gering. Verstärkt wird dieser Zusammenhang durch die Nicht-Rivalität im Konsum des technischen Wissens. Denn es kann ein und dieselbe Anleitung von einem weiteren Unternehmen verwendet werden, wodurch die Produktion ansteigt, ohne das erneute Kosten f{\"u}r technologisches Wissen entstehen \citep{.1968,Ostrom.1990}.\\


Die Entwicklung einer Idee kann kostspielig sein und der kostenfreie Zugriff einer m{\"o}glicherweise gewinnbringenden Idee das Interesse vieler wecken. Dabei handelt es sich beispielsweise um eine Neuerung im Produktionsprozess, die zur Beseitigung von Ineffizienzen f{\"u}hrt. Eine Idee kann von mehreren Wirtschaftssubjekten zur gleichen Zeit realisiert werden, wohingegen sich die Faktoren Arbeit und Kapital nur einmal an einem Ort einsetzen lassen. Demzufolge ist auch ein Anstieg der Produktivit{\"a}t durch eine Idee in mehreren Unternehmen gleichzeitig denkbar \citep{Romer.1986}. \newline 
Endogenisiert man das technologische Wissen, dann steigen die Skalenertr{\"a}ge der Produktion an. Eine Verdopplung aller rivalen Inputfaktoren f{\"u}hrt zu einer mehr als doppelt so gro{\ss}en Produktionsmenge. Dies liegt daran, dass nicht nur das technische Wissen nicht rival ist, sondern dadurch auch die Technologie des Produktionsprozesses. Sie kann von mehreren Unternehmen gleichzeitig genutzt werden, ohne den Nutzen eines Wirtschaftssubjekts einzuschr{\"a}nken, wodurch eine stark erh{\"o}hte Produktionsmenge resultiert \citep{Jones.2005}.\\


Dieser Zusammenhang zeigt, wie einflussreich die Nichtrivalit{\"a}t auf das {\"o}konomische Wachstum ist, da dies steigende Skalenertr{\"a}ge bedingt. Die steigenden Skalenertr{\"a}ge liefern einen Anreiz Monopolmacht zu erlangen, was wiederum die Motivation darstellt, Innovationen zu entwickeln \citep{Jones.2005,Romer.1993}.\footnote{Eine Ausführliche Erläuterung folgt in Kapitel \ref{sec:Anreize}}\\


Technisches Wissen birgt zwei Folgen. Einerseits die Motivation Innovationen zu entwickeln um Monopolmacht zu erlangen, andererseits die Gefahr der schnellen und kostenfreien Nachahmung der Konkurrenten. 
Gelöst werden kann dieses Problem durch Patente, die die kommerzielle Nutzung von Ideen durch Dritte verhindern. Dabei wird das innovierende Unternehmen geschützt und der Erhalt der geistigen Eigentumsrechte {\"u}ber einen bestimmten Zeitraum erm{\"o}glicht, somit mittelfristig auch die Gewinne. Jedoch können Patente nicht die Weiterverbreitung der Idee an sich verhindern.\newline Neben Patenten kann die Generierung von technischem Wissen auch durch die staatliche Förderung gewährleistet werden. Grundlagenforschung wird deswegen meist von {\"o}ffentlichen Einrichtungen betrieben. Der Schwerpunkt dieser Arbeit liegt jedoch auf der angewandten Forschung, die von privat finanzierten Unternehmen forciert wird.

\subsection{Humankapital}
Humankapital ist (personen-)gebundenes technisches Wissen wie die F{\"a}higkeiten und Fertigkeiten eines Menschen. \citet{Acemoglu.2009} präzisiert diese Definition und beschreibt Humankapital als jegliche Eigenschaften von Arbeitern, die die potentielle Produktivit{\"a}t aller oder einiger produktiver Aufgaben steigert. Wohingegen \citet{Lucas.1988}\footnote{Obwohl das Papier von \citet{Lucas.1988} mehrere Modelle vorstellt, wird gemeinhin und auch in dieser Arbeit von dem Humankapitalmodell des Kapitels 4 ausgegangen.} weniger zwischen einzelnen Fähigkeiten und Aufgaben differenziert, sondern Humankapital eher als ein "`skill-level"' definiert, also ein Niveau erreichter F{\"a}higkeiten.\footnote{In dieser Form wird Humankapital in Kapitel \ref{Papier1} abgebildet. In dem Modell steht die Humankapitalakkumulation nicht im Vordergrund. Bildung ist indirekter Bestandteil der Produktivität einer Volkswirtschaft. Demnach werden keine einzelnen Aufgaben und Tätigkeiten spezifiziert, sondern verschiedenen Tätigkeitsfelder bzw. Bildungsniveaus miteinander verglichen.}\\


Bei dem technischen Wissen handelt es sich formal, wie in Abschnitt \ref{sec:techn. Wissen} bereits erörtert wurde, um ungebundene theoretische Kenntnisse, die auch den nachfolgenden Generationen zur Verf{\"u}gung stehen \citep[Kapitel 10]{Frenkel.1999}. Dieser wesentliche Punkt unterscheidet das technische Wissen von Humankapital. Denn die an den Menschen gebundenen Kenntnisse und Fertigkeiten gehen mit dem Tod des Menschen verloren und stehen der Welt nicht weiter zur Verf{\"u}gung. Mit diesem Argument stellt \citet{Ha.2002} zur Diskussion, dass Humankapitalakkumulation nicht dauerhaft zum Wachstum beitr{\"a}gt, da Bildung und F{\"a}higkeiten an den Menschen gebunden sind und somit von der begrenzten Lebensdauer des Menschen abh{\"a}ngig sind.\footnote{Dabei wurde der Gedanke vernachl{\"a}ssigt, dass das Grenzprodukt des Wissens steigen k{\"o}nnte und dadurch steigende Wachstumsraten resultieren würden. Dieser Sonderfall steigender Grenzertr{\"a}ge des Humankapitals geht auf \citet{Romer.1986} zurück.}  Dem soll hier nicht direkt widersprochen werden, jedoch ist zu ber{\"u}cksichtigen, dass die Entwicklung von Innovationen humankapitalintensiv ist und diese wiederum langlebig sind und somit trotzdem zu dauerhaftem technologischem Wachstum führen. 
Ein anderer wichtiger Unterschied des Humankapitals zum technischen Wissen liegt in der Eigenschaft der Nicht-Rivalität, denn Humankapital ist rival. Ein Wissenschaftler oder qualifizierter Arbeiter kann nur an einem Projekt gleichzeitig arbeiten und ihm steht seine Zeit nicht mehrfach zur Verfügung \citep{Romer.1993}. Somit ist wie beim Produktionsfaktor Arbeit eine eindeutige monet{\"a}re Verg{\"u}tung m{\"o}glich, der Lohn.\\


In vielen Modellen, wie beispielsweise dem AK-Modell, wird Humankapital und physisches Kapital unter dem Oberbegriff Kapital zusammengefasst. Der Kapitalbegriff könnte sogar noch weiter differenziert werden, indem intellektuelles Kapital noch einmal von Humankapital abgegrenzt wird. Der Wert des produktiven Wissens, das durch Forschung und Entwicklung gewonnen wurde, ist das intellektuelle Kapital \citep{Dosi.1993}. Hier wird jedoch explizit zwischen beiden Kapitalarten unterschieden, da diese verschiedene Eigenschaften aufweisen und dadurch dauerhaftes Wachstum möglich ist. 

\subsubsection{Humankapitalakkumulation}
Bei dem Faktor Arbeit handelt es sich nicht um einen homogenen Produktionsfaktor. F{\"a}higkeiten, Fertigkeiten und Kenntnisse k{\"o}nnen durch die Akkumulation von Humankapital erh{\"o}ht werden \citep{Ha.2002}.  Bildung steigert das Humankapital eines einzelnen Individuums und kann somit als Entstehungsprozess des Humankapitals, als Humankapitalakkumulation, gesehen werden. Es können zwei Arten der Humankapitalakkumulation unterschieden werden, das formelle und das informelle Lernen. Mit dem formellen Lernen der Bildung gehen Kosten einher, die berücksichtigt werden müssen. Dabei handelt es sich um direkte Ausbildungskosten oder Opportunit{\"a}tskosten durch entgangenen Lohn. Wohingegen das informelle Lernen, das learning-by-doing, kostenlos ist \citep{Arrow.1969}.

\paragraph{informelles Lernen - learning-by-doing}
Im Jahr 1936 ver{\"o}ffentlichte \citet{Wright.1936} seine Beobachtungen zum Flugzeugbau. Dabei war besonders auff{\"a}llig, dass die Arbeitsstunden f{\"u}r die Produktion eines Flugwerks mit zunehmender Produktionszahl sinken.\newline Dies motivierte \citet{Arrow.1962} zu seinem Modell {\"u}ber das learning-by-doing. Es beschreibt den Zusammenhang zwischen der Produktivit{\"a}t eines Arbeiters und seine dadurch zunehmende Erfahrung. Dieser Produktivitätsgewinn wird als Lernen bezeichnet. Dabei geht es um die wiederkehrende und aktive L{\"o}sung von Problemen, die durch die st{\"a}ndige Wiederholung zu sinkenden Grenzertr{\"a}gen f{\"u}hrt \citep{Arrow.1962,Sheshinski.1967}. Denn je l{\"a}nger ein Gut hergestellt wird, desto kosteng{\"u}nstiger kann es produziert werden, bedingt unter anderem durch die Lernkurve des Herstellungsprozesses. Durch die Feststellung von Ineffizienzen, die Umstrukturierung von Organisationsformen und auch durch die zunehmende Erfahrung der Mitarbeiter steigt mittelfristig die Sicherheit im Umgang mit Techniken, Verfahren und Produkten. Sind die Lernm{\"o}glichkeiten ersch{\"o}pft, dann f{\"u}hrt erst die Entwicklung neuer Produkte und Prozesse zu neuen Lerneffekten. Andauernde Effekte des learning-by-doings sind demzufolge zwingend an die Innovationst{\"a}tigkeit der Unternehmen gekn{\"u}pft \citep{Arrow.1962}.\footnote{\citet{Sheshinski.1967} untersuchte als einer der Ersten empirisch die These Arrows, die den Produktivit{\"a}tszuwachs durch zunehmende Erfahrung beschreibt. Er belegt den Ansatz und zeigt, dass effizientes Wachstum und das Investitionslevel positiv korrelieren. Dabei misst er die Erfahrung als kumulierte Bruttoinvestitionen. Demzufolge steigt mit zunehmender Erfahrung das Wirtschaftswachstum eines Landes.}


\paragraph{formelles Lernen - Uzawa-Lucas-Modell}
Bei dem formellen Lernen werden die Produktionsfaktoren direkt für Bildung investiert. Am Beispiel des Uzawa-Lucas-Modells bedeutet dies, das die Wirtschaftssubjekte sich zwischen der entlohnten Konsumgüterproduktion oder der eigenen Ausbildung entscheiden können. Der Produktionsfaktor Humankapital wird zwischen den Sektoren aufgeteilt und geht nur anteilig in den Lernprozess ein.\footnote{Eine ausführliche Darstellung des Modells folgt in Kapitel \ref{Papier2}.}  


\subsubsection{Messung von Humankapital} 
Bei der Messung von Humankapital sind einige Hindernisse zu überwinden. Zum einen führt die Diskrepanz bezüglich einer eindeutigen Definition zu dem Problem einer geeigneten Bezugsgrö{\ss}e. Wurde diese gefunden dann ist immer noch fraglich, ob eine Vergleichbarkeit möglich ist und dadurch konkrete Aussagen getroffen werden können. Die Methoden, mit denen Humankapital gesch{\"a}tzt wird, sind sehr verschieden. Als Bezugsgrö{\ss}en bediente man sich beispielsweise der Anzahl an Bildungsjahren oder vergleicht Bildungsniveaus miteinander. So können die Grundkenntnisse der Bevölkerung einer Volkswirtschaft über die Alphabetisierungsrate aller abgedeckt werden, die das 15. Lebensjahr überschritten haben \citep{Romer.}. An der Einschreiberate oder der Messung von Absolventen einer weiterf{\"u}hrenden Schule orientierten sich \citet{Levine.1992} sowie \citet{Barro.2001}. \citet{Mankiw.1992} verwendetet eine L{\"a}nderquerschnittanalyse, dabei wurde die Zahl der Jugendlichen zwischen 12 und 17 die eine Schule besuchen mit dem Anteil der arbeitsf{\"a}higen Bev{\"o}lkerung zwischen 15 und 19 multipliziert. Kritisch ist bei dieser Methode jedoch, dass das Humankapital in Industrieländern tendenziell {\"u}bersch{\"a}tzt und in Entwicklungsländern untersch{\"a}tzt wurde.\newline 
\citet{Barro.2001} haben in ihrer Arbeit einen Datensatz aufbereitet, der Humankapital dahingehend quantifiziert, indem die Bev{\"o}lkerung mehrerer Länder von 1960 bis 2000 nach sieben verschiedenen Bildungsstufen kategorisiert wurde.\\
Problematisch bei allen genannten Methoden ist, dass keine Aussage {\"u}ber die Qualit{\"a}t der Bildung möglich ist und keine eindeutige Aussage {\"u}ber eine m{\"o}gliche Qualifizierung zugelassen wird. Internationale Leistungstests wie die PISA Studien oder mögliche Sammel\-indikatoren, die die l{\"a}nderspezifischen Systeme in einen einheitlichen Rahmen einordnen, können diesbezüglich Abhilfe schaffen. So wird mit Hilfe der Daten aus dem UNESCO Institute for Statistics anhand der Anzahl der Lehrkr{\"a}fte oder auch über die Anzahl der Sch{\"u}ler pro Klasse versucht eine internationale Vergleichbarkeit  bezüglich eines Jahres Bildung herzustellen. 


\section[Entwicklung des technischen Fortschritts: Innovation]{Entwicklung des technischen Fortschritts:\\ Innovation \sectionmark{Entwicklung des techn. Fortschritts}}
\sectionmark{Entwicklung des techn. Fortschritts}
Die für den technischen Fortschritt notwendigen Bestandteile wurden im vorangegangenen Kapitel ausführlich erläutert. Im folgenden Kapitel wird gezeigt, dass die Intelligenz, Kompetenz sowie die Ausbildung eines Individuums, f{\"u}r die Entwicklung und den Erfolg von Innovationen und Imitationen bedeutsam ist \citep{Hassler.2000}.\\


In der Regel handelt es sich bei Innovationen um neue Technologien. Die beiden Bestandteile einer Innovation sind eine Idee und eine Investition. Die Idee ist dabei zunächst der Engpass, den es zu überwinden gilt und ohne die eine Neuentwicklung nicht möglich ist. Die Investition ist notwendig, um die Idee umzusetzen, zu entwickeln und in den Markt einzuf{\"u}hren.\footnote{Als wesentliche Voraussetzung gilt dabei, dass eine Neuerung vom Markt erfolgreich angenommen wird und es somit bereits einen Bedarf gibt oder dieser noch geschaffen werden kann. Au{\ss}erdem müssen die notwendigen Rahmenbedingungen für die Markteinführung vorhanden sein. Bei einer medizinischen Innovation beispielsweise sollte den {\"A}rzten Fortbildungen angeboten werden, um diese auch anwenden zu können.}\newline F{\"u}r die Entwicklung einer Idee kann technisches Wissen notwendig sein, das an Humankapital gebunden ist, bei der Investition ist das technische Wissen hingegen erforderlich, da f{\"u}r die Entwicklung einer Idee in der Regel bereits bekannte Technologien verwendet werden. Dabei ist einerseits technisches Wissen, das an physisches Kapital gebunden ist, notwendig und andererseits ausgebildete Arbeitskr{\"a}fte, in denen das technische Wissen an Humankapital gebunden ist \citep{Scotchmer.2004}.\footnote{So zählen zu den Investitionen neben monet{\"a}rer Gr{\"o}{\ss}en auch die Produktionsfaktoren (Maschinen, Arbeit, Zwischengüter, Humankapital, Zeit) sowie spezifisch gebundene Investitionen in Forschungseinrichtungen.}\newline 


Der Innovationsprozess kann auch anders untergliedert werden, in die Abschnitte: Invention, Innovation und der folgenden Diffusion. Vergleicht man dies mit der erst genannten, dann würde die Idee der Invention, also der Erfindung entsprechen und die Investition gliedert sich auf in die Innovation an sich, also die physische Umsetzung der Idee, und der Diffusion, der Markteinführung und dem damit verbundenen Wissenstransfer für die Allgemeinheit \citep{Jones.2005}.\newline 
Als wesentliche Bestandteile einer Innovation lassen sich Technologie und Humankapital zusammenfassen. Mit genau diesen beiden Schwerpunkten befasst sich auch der Hauptteil dieser Arbeit. Zunächst wird die Entstehung des Humankapitals in Kapitel \ref{Papier2} untersucht und anschlie{\ss}end wird in Kapitel \ref{Papier1} analysiert wie durch dieses mit dem notwendigen technischen Wissen Innovationen entstehen können, die zusätzlich den Entwicklungsprozess eines Landes beschleunigen.\\


Jedoch ist der Begriff "`Innovation"' stark vom theoretischen Zusammenhang abhängig und in der Literatur gibt es eine Vielzahl von Differenzierungsmöglichkeiten verschiedener Innovationsformen. Eine Möglichkeit der Abgrenzung bezieht sich auf das Ausma{\ss} der Innovation. Bei der graduellen Innovation werden bestehende Produkte bzw. Prozesse weiter entwickelt und verbessert. Wohingegen bei der Basisinnovation ein komplett neues Produkt entsteht \citep{Schebesch.1992}.\footnote{Des weiteren wird zwischen einer drastischen und einer nicht-drastischen Innovation unterschieden, beide Fälle werden in Kapitel \ref{sec:LimitPreis} diskutiert.}\\


Modelle die den technischen Fortschritt beschreiben differenzieren h{\"a}ufig zwischen der Produktinnovation und der Prozessinnovation. Es handelt sich um eine Produktinnovation, wenn ein neues Gut entwickelt und auf dem Markt eingef{\"u}hrt wird.  Die neuen G{\"u}ter erhöhen die Konsumm{\"o}glichkeiten der Haushalte \citep{Grossman.1991a,Grossman.1990b}. Daraus resultiert ein h{\"o}herer Nutzen bei den Konsumenten, wenn davon ausgegangen wird, dass die Pr{\"a}ferenzen in der Vorliebe f{\"u}r die Auswahl m{\"o}glichst vieler G{\"u}ter liegt \citep{Krugman.79}. Auch denkbar ist die Erh{\"o}hung der Qualit{\"a}t der G{\"u}ter. In diesem Fall ersetzen die neuen Produktvarianten die fr{\"u}heren und es kommt nicht zu einem Anstieg der Anzahl der Produktvarianten \citep{Acemoglu.2009}. \\ 


Endogene Wachstumsmodelle, in denen die Vielfalt an Inputfaktoren durch den technischen Fortschritt zunimmt, beschreiben Prozessinnovationen. Durch die Erh{\"o}hung der Verschiedenartigkeit der Einsatzfaktoren kommt es zu einer Produktivit{\"a}tssteigerung. Bei einer Prozessinnovation liegt der Schwerpunkt auf Neuerungen im Herstellungsverfahren bereits existierender G{\"u}ter. Ziel der Prozessoptimierung ist eine Kostenreduktion und eine effizientere Produktion. Der Erfolg einer Prozessinnovation l{\"a}sst sich intuitiv durch das Wirtschaftlichkeitsprinzip erl{\"a}utern: Kann mit der gleichen Menge an Einsatzfaktoren eine h{\"o}here Produktionsmenge erzeugt werden, dann hat sich die Produktivit{\"a}t des Prozesses erh{\"o}ht. Dem Minimumprinzip folgend, kann dann mit einem geringeren Faktoreinsatz die gleiche G{\"u}termenge hergestellt werden. Aus makro{\"o}konomischer Perspektive w{\"u}rde in einem Modell mit den Einsatzfaktoren Arbeit, Kapital und Technologie ein h{\"o}heres Sozialprodukt bei konstanten Faktoreins{\"a}tzen folgen \citep[Kapitel 10]{Frenkel.1999}.
Handelt es sich bei einem Inputfaktor um Zwischengüter, dann werden bei Prozessinnovationen vom Zwischengut immer neue Varianten entwickelt, die direkt wieder in den Produktionsprozess eingesetzt werden. Denn es gilt, je mehr Varianten den Produktionsprozess mitbestimmen, desto st{\"a}rker ist die Arbeitsteilung und desto höher dadurch letztlich die Produktivit{\"a}t eines Unternehmens \citep{Romer.1987,Romer.1990}.\\


Innovationen nach \citet{Hicks.1932} f{\"u}hren zu Ersparnissen des Faktors Arbeit, da dieser nun effizienter eingesetzt werden kann. Dieser Effekt entsteht auch durch die Akkumulation von Humankapital, das den einzelnen Arbeiter dazu befähigt effizienter zu arbeiten \citep{Arrow.1969}.\newline


Die Unterscheidung zwischen Produkt- und Prozessinnovation wird in dieser Arbeit jedoch nicht vorgenommen, sondern beide Arten unter dem Oberbegriff "`Innovation"' subsumiert. In der Literatur ist diese Unterscheidung gerade dann sinnvoll, wenn im Anschluss die Forschungsergebnisse empirisch {\"u}berpr{\"u}ft werden. Da dies hier nicht der Fall ist, wird von einer Unterscheidung abgesehen \citep{Acemoglu.2009}.\\


Au{\ss}erdem kann zwischen der vertikalen und horizontalen Innovation differenziert werden \citep{vanLong.1997}. Dabei handelt es sich bei horizontalen Innovationen um zus{\"a}tzlichen Variantenreichtum, wodurch die Vielfalt an m{\"o}glichen G{\"u}tern und Prozessen zunimmt \citep{Romer.1990}. Hingegen bei vertikalen Innovationen werden G{\"u}ter und Prozesse weiterentwickelt \citep{vanLong.1997}. Ein nun hochwertigeres Gut bzw. verbesserter Prozess ersetzt den vorherigen. Bleibt die Summe der G{\"u}ter unver{\"a}ndert, dann handelt es sich um den Prozess der schöpferischen Zerstörung nach \citep{Schumpeter.1934a}. Schumpeter pr{\"a}gt den Begriff der sch{\"o}pferischen Zerst{\"o}rung der den strukturellen Wandel durch immer neue Erfindungen beschreibt.\footnote{Genauere Erläuterung des Prozesses in Kapitel \ref{sec:Wachstumstheorien}.} Er erkannte das Wechselspiel von Innovation und Imitation als Triebkraft des Wettbewerbs.\\


Einer anderen Auffassung bezüglich der Innovationsarten ist \citet{Mokyr.1990} und berücksichtigt die Reichweite einer Innovation. Dabei unterscheidet er in seiner Arbeit zwischen Makro- und Mikroinnovationen. Eine Makroinnovation ist ein technologischer Fortschritt, der zu weitreichenden strukturellen Ver{\"a}nderungen f{\"u}hren kann. Beispiele hierf{\"u}r sind die Erfindung der Elektrizit{\"a}t oder das Internet. Die Folgen solcher Innovationen sind enorm und wirken sich meist auf die Mehrheit von Herstellungsprozessen aus. Werden jedoch in der Forschung bislang weitestgehend nicht ber{\"u}cksichtigt. \newline Denn die meisten Modelle analysieren Mikroinnovationen, die das Wirtschaftswachstum st{\"a}rker f{\"o}rdern als Makroinnovationen. Dies scheint zun{\"a}chst etwas ungewöhnlich, wurde aber von \citet{Abernathy.1978,Freeman.1982} empirisch best{\"a}tigt. Unter Mikroinnovationen versteht man sowohl Produkt- als auch Prozessinnovationen, deren Wirkung auf das technologische Umfeld von geringerer Bedeutung ist, dem einzelnen Wirtschaftssubjekt jedoch Nutzen stiftet. Es kann sich dabei um eine Kostenreduktion im Produktionsprozess, eine qualitativ hochwertigere Variante eines bereits bekannten Gutes oder auch ein neues vorher unbekanntes Produkt handeln \citep{Mokyr.1990}. Diese Terminologie wird auch in Kapitel \ref{Papier1} aufgegriffen und beschreibt den Einfluss beider Innovationsm{\"o}glichkeiten auf die Ausweitung der Welttechnologiegrenze. Je nachdem ob es sich um eine Makro- oder eine Mikroinnovation handelt beeinflusst dies den relativen technologischen Entwicklungsstand eines Landes.


\paragraph{Anreize zur Innovationsentwicklung}\label{sec:Anreize}
In dem folgenden Abschnitt soll erörtert werden worin die Motivation besteht Technologien zu entwickeln oder zu verbessern. Dabei lassen sich zwei Meinungsbilder unterscheiden.  Nach \citet{Ceruzzi.2003} beispielsweise besteht der Anreiz zu Innovieren vor allem in der Wissbegierde der Forscher. Er beschreibt in seinem Werk "`History of Modern Computing"', dass es keinen Bedarf nach Computern f{\"u}r den pers{\"o}nlichen Gebrauch gab und es deshalb auch nicht die Nachfrage in dem tats{\"a}chlich resultierten Umfang erwartet wurde. Die Vielzahl unerkl{\"a}rter Ph{\"a}nomene und Fragen veranlassen Wissenschaftler deren Ursprung und Erkl{\"a}rung zu ergr{\"u}nden, ohne dabei mögliche Absatzmöglichkeiten und ökonomische Argumente einflie{\ss}en zu lassen. Der gleichen Meinung ist \citet{Arrow.1969}, denn Wissen entsteht durch die Suche nach L{\"o}sungsans{\"a}tzen und durch Beobachtungen realer Vorg{\"a}nge und Ereignisse. So können ähnliche Gegebenheiten dabei helfen Erkl{\"a}rungsans{\"a}tze zu finden und Erkenntnisse zu gewinnen. Der Mensch ist nur durch Neugier getrieben und versucht die Welt in der er lebt zu verstehen, dabei sind Innovationen Instrumente für Problemlösungsansätze. \newline Nach herrschender Meinung liegt die Motivation jedoch eher in Gewinnerzielungsabsichten \citep{Romer.1993,Grossman1989b.}. So auch bei der Entwicklung des iPads, dem ersten Tablet-PC. Der Markt und das damit einhergehende Bedürfnis nach diesem Gut wurde von dem Hersteller Apple herbeigeführt. Jedoch ist fraglich, ob tatsächlich der Forschungsdrang nach einer Problemlösung die Erfindung motiviert hat oder eher wirtschaftliche Aspekte. Durch eine Innovation wird der Anbieter zunächst zum Monopolisten und die damit einhergehende anfängliche Monopolmacht zeigt sich in Preissetzungsspielräumen, wodurch Gewinne abgeschöpft werden können. Langfristig werden konkurrierende Anbieter sich ebenfalls der Innovation bedienen, was durch die Nicht-Rivalität und die Nicht-Ausschlie{\ss}barkeit des technischen Wissens möglich ist \citep{Romer.1993}. Darin besteht auch das eigentliche Problem der Innovationsentwicklung. Zwar suggerieren Innovationen kurzfristige Gewinne, die Entwicklung ist jedoch aufwendig und teuer. Die Investitionen können ohne den Schutz der Eigentumsrechte nicht ausgeglichen werden, wodurch sich der Anreiz zur Innovationsentwicklung stark mindert.Grundsätzlich spornt die wirtschaftliche Bereicherung als Konsequenz erfolgreich integrierter Innovationen die Menschheit seit Jahrhunderten dazu an den technischen Fortschritt voran zu treiben. Daraus begründet sich die notwendige Einf{\"u}hrung von Patenten, die das technische Wissen sch{\"u}tzen und Alleinstellungsmerkmale schaffen. Die geschaffene Ausschlie{\ss}barkeit im Konsum f{\"u}hrt zu einer monet{\"a}ren Bemessung und Zuordnung \citep{Acemoglu.2009}. Am Beispiel der Innovationst{\"a}tigkeiten des Hufeisensektors erläutert \citet{Schmookler.1966} die wirtschaftliche Abhängigkeit von Innovationen. Die Innovationsrate stieg Ende des 19. Jahrhunderts bis ins 20. Jahrhundert solange stark an, bis zu dem Zeitpunkt, ab dem sich das Automobil immer weiter in der Gesellschaft durchsetzte und dadurch die Fortbewegung mit dem Pferd als unn{\"o}tig erachtet wurde. Somit liegt letztendlich der Anreiz in Forschung zu investieren in der Entwicklung von Innovationen, um als Vorreiter eines Marktes Monopolgewinne abschöpfen zu können.\footnote{Zudem entsteht indirekt ein Wissenszuwachs f{\"u}r die gesamte Branche, von dem alle Marktteilnehmer gleicherma{\ss}en gegenseitig profitieren k{\"o}nnen \citep{Cohen.1989}.}\\


Die industrie{\"o}konomische Literatur befasst sich mit der Rivalit{\"a}t der Unternehmen, um die technologische F{\"u}hrerschaft und den damit einhergehenden Einfluss auf den Entwicklungsprozess zu erklären. Da viele Unternehmen nach erfolgreichen Innovationen streben, also nach Innovationen, aufgrund derer Patente angemeldet werden k{\"o}nnen um Monopolgewinne abzuschöpfen, birgt dies zugleich eine Unsicherheit des Erfolgs. Demzufolge besteht auch ein Risiko den Wettstreit um die f{\"u}hrende Position zu verlieren und vom technologischen Fortschritt nicht profitieren zu k{\"o}nnen. Die Unsicherheit die mit dem technologischen Fortschritt einhergeht beeinträchtigt den technologischen Erfolg und den damit einhergehenden Entwicklungsprozess eines Landes \citep{Reinganum.1981}.\\ Ein weiterer Punkt der nur kurz angeschnitten werden soll, ist der wirtschaftliche Trade-off zwischen der Entwicklung von Produktinnovationen und Prozessinnovationen. Die Verbesserung der Effizienz von Produktionsprozessen ist nur dann sinnvoll, wenn das Gut eine gewisse Beständigkeit auf dem Markt hat und nicht zeitnah durch ein Neues ersetzt wird. Denn der Produktionsprozess kann nicht optimiert werden, solange es immer wieder neue Varianten und G{\"u}ter gibt, die ein anderes Herstellungsverfahren haben. Diesen Zusammenhang beschreibt \citet{Abernathy.1978} in der Automobilindustrie am Beispiel Ford.\\


Die Monopolmacht wird in Kapitel \ref{Papier1} aufgegriffen und der damit einhergehende  Anreiz Innovationen zu entwickeln. 



\sectionmark{Ausdehnung des techn. Fortschritts}
\section[Ausdehnung des technischen Fortschritts: Technologiediffusion durch Imitation]{Ausdehnung des technischen Fortschritts: Technologiediffusion durch Imitation}\sectionmark{Ausdehnung des techn. Fortschritts}



Nachdem sich eingehend mit der Entstehung und Entwicklung des technischen Fortschritts beschäftigt wurde, der Innovation, wird im folgenden Kapitel die Ausdehnung des technischen Fortschritts genauer betrachtet, die Imitation. Mit der Adaption von Gütern und Prozessen gilt der Diffusionsprozess als beendet und Wissen wurde erfolgreich transferiert. \\
Für die Adaption von Gütern und Prozessen bedarf es der gleichen Faktoren wie für Innovationen und zwar technischem Wissen, Sachkapital und Humankapital \citep{Cohen.1989,Griffith.2004}. Eine Imitation ist eine "`alte"' Innovation, die durch benannte Investitionen nachgeahmt werden können. Demnach handelt es sich bei Imitation um die gleiche technologische Neuerung, mit dem gleichen Erkenntnisgewinn wie bei der Innovation, jedoch zu einem sp{\"a}teren Zeitpunkt \citep{Schmookler.1966}. Für eine Imitation ist Humankapital ebenso notwendig wie für eine Innovation, jedoch unterscheiden sich beide durch die eingesetzten Humankapitalniveaus. Grundsätzlich ist für eine Innovation mehr Humankapital notwendig, da neben den Investitionen auch die Idee durch den Einsatz von Humankapital entsteht. Jedoch gibt die H{\"o}he des Humankapitals Aufschluss über die Absorptionsf{\"a}higkeit eines Unternehmens oder einer Volkswirtschaft. Denn je mehr Humankapital für die Nachahmung notwendig ist, desto besser und genauer kann adaptiert werden \citep{Nelson.1966}. Das Humankapital eines Landes kann demnach für innovierende und imitierende Prozesse gleicherma{\ss}en eingesetzt werden. \\
Von der Gesamtheit der globalen Volkswirtschaften ausgehend ist tatsächlich nur ein sehr geringer Anteil innovierend tätig. Die meisten L{\"a}nder importieren Technologien und ahmen diese nach statt selbst Innovationen zu entwickelt. In weniger weit entwickelten L{\"a}ndern beläuft sich die Wachstumsrate durch die Adaption ausl{\"a}ndischer Technologien auf ca. 65{\%}. In weiter entwickelten L{\"a}ndern wird der Gro{\ss}teil (ca. 75{\%}) hingegen durch innovierende T{\"a}tigkeiten der heimischen Unternehmen hervorgerufen \citep{Santacreu.2015}. Dies zeigt wie wichtig der Prozess der Imitation f{\"u}r die {\"O}konomen ist, da ein betr{\"a}chtlicher Anteil davon profitiert. Wohingegen die Bedeutung der Innovationsentwicklung von L{\"a}ndern wie Deutschland, USA oder Japan f{\"u}r das globale Wirtschaftswachstum mindestens ebenso wichtig ist, da nur hierdurch  dauerhaftes Wachstum gew{\"a}hrleistet wird und es immer neue Innovationen gibt, die imitiert werden können \citep{Acemoglu.2009}.\\


Wird der Innovationsbegriff etwas extensiver verstanden, dann beinhalten Innovationen auch nachahmende Prozesse unter Verwendung bereits bestehender Technologien. Es handelt sich dabei nicht um eine kostenlose Kopie von G{\"u}tern oder Prozessen, sondern um eine anpassende {\"U}bertragung dieser an lokale Gegebenheiten, f{\"u}r die ebenso Investitionen ben{\"o}tigt werden \citep{Arrow.1969,Evenson.1995}. Demzufolge handelt es sich bei diesem weiter gefassten Verständnis um eine Innovation, jedoch mit imitierenden Elementen.\\


Denn für beide T{\"a}tigkeitsfelder, der Innovation und Imitation, muss ein {\"a}hnlicher Aufwand im Sinne von Zeit und Geld aufgebracht werden \citep{Cohen.1989,Griffith.2004}. Au{\ss}erdem ist der Erfolg beider von Unsicherheit gepr{\"a}gt. Dies ist der Neuheit des Produktionsprozesses geschuldet, unabh{\"a}ngig ob es sich um die Entwicklung eines vollkommen neuen Gutes bzw. Prozesses handelt, oder ob ein  f{\"u}r das Unternehmen neues Gut oder Prozess nachgeahmt wird \citep{Segerstrom.1991}.\\


Die Imitation als technischer Fortschritt kann auch als Technologie{\"u}bertragung gesehen werden \citep{Nelson.1966,Cohen.1989,Griffith.2004}. Die Technologiediffusion beschränkt sich dabei nicht notwendigerweise auf die Verbreitung innerhalb einer Volkswirtschaft, sondern der Kerngedanke kann auch l{\"a}nder{\"u}bergreifend {\"u}bernommen werden. Dann wird Wissen durch Imitation in ein anderes Land {\"u}bertragen \citep{Nelson.1966}.\\




Wissen steigt auf zwei Arten: Zum einen durch die Verbreitung bereits bekannter G{\"u}ter und Verfahren, zum anderen durch die Entwicklung neuer G{\"u}ter und Verfahren. Im ersten Fall handelt es sich um Wissensdiffusion, die durch Imitationen umgesetzt wird. Bei dem zweiten Fall steigt der Wissensstock durch innovierende Tätigkeiten an \citep{Schmookler.1966}.
Als Technologiediffusion oder auch Technologietransfer wird die Verbreitung von technischem Wissen bzw. Technologien bezeichnet. Dies kann durch verschiedene Kan{\"a}le geschehen, wie beispielsweise durch Fachzeitschriften, ausl{\"a}ndische Direktinvestitionen oder aber auch durch die Migration qualifizierter Arbeitskr{\"a}fte. In dieser Arbeit liegt der Schwerpunkt auf dem internationalen Handel als Diffusionskanal von technischem Wissen und ber{\"u}cksichtigt die verschiedenen Absichten, Technologiediffusion gezielt hervorzurufen.\newline


Eine Technologie ist erst dann diffundiert, wenn sie adaptiert wurde. Dabei kann es sich sowohl um die Diffusion von Wissen innerhalb eines Landes zwischen Unternehmen, als auch um die grenz{\"u}berschreitende Diffusion zwischen L{\"a}ndern handeln \cite{Acemoglu.2009}.\\


Aus welchem Grund Technologiediffusion letztendlich beabsichtigt wird, h{\"a}ngt im Wesentlichen von der Perspektive ab. \citet{Arrow.1969} sieht die Motivation f{\"u}r die {\"U}bertragung von technischem Wissen in dem Anreiz der Gewinnerzielungsabsichten und beschreibt dabei eher die mikro{\"o}konomische Perspektive. Makro{\"o}konomisch liegt der Grund des Technologietransfers viel mehr in einem m{\"o}glichen Entwicklungspotential, das daraus resultieren kann.\\


Die Bedeutung des Technologietranfers f{\"u}r den Entwicklungsprozess eines Landes werden erstmals von \citet{Gerschenkron.1962} beschrieben. Dabei unterscheidet er zwischen horizontalem und vertikalem Technologietransfer. Bei der {\"U}bertragung und Implementierung technischer Neuerungen vom Forschungs- und Entwicklungsbereich in den Bereich praktischer Anwendung handelt es sich um den vertikalen Technologietransfer. Verl{\"a}sst man die mikro{\"o}konomische Perspektive, dann ist der horizontale Technologietransfer auf der makro{\"o}konomischer Ebene zu finden. Dieser wiederum beschreibt die {\"U}bertragung von technischem Wissen und Produktionsfertigkeiten {\"u}ber L{\"a}ndergrenzen hinweg.\newline In dieser Arbeit liegt der Fokus auf dem horizontalen Transfer und steht in einem engen Zusammenhang mit dem catching-up Effekt , dem Aufholprozess einer Volkswirtschaft. Zahlreiche Beispiele zu Zeiten der industriellen Revolution im 19. Jahrhundert untermauern die These von \citet{Gerschenkron.1962} und \citet{Veblen.1915}. So gelang es Deutschland durch Technologietransfer an das Pionier-Land Gro{\ss}britannien aufzuschlie{\ss}en. Der Technologie- und Wissenstransfer im 19. Jahrhundert erfolgte durch Kundschafterreisen von Unternehmern und Ingenieuren nach Gro{\ss}britannien, dem Anwerben britischer Fachkr{\"a}fte in das eigene Land sowie durch Akademien, wissenschaftliche Gesellschaften und Fachzeitschriften. Die technische L{\"u}cke konnte geschlossen werden und liefert Anhaltspunkte, dass dieser sogenannte Velben-Gerschenkron-Effekt auch auf die heutige Zeit und die Problematik der Entwicklungspolitik {\"u}bertragen werden kann. Dieser Effekt hebt dabei unter anderem Bildung, Staatseingriffe und Technologietransfer als wichtige Wachstumsfaktoren hervor \cite{Peri.2004}. \\


Ein Merkmal von Entwicklungsl{\"a}ndern ist der gro{\ss}e Abstand zur Welttechnologiegrenze und dem damit einhergehenden eingeschr{\"a}nkten Zugang und der Verf{\"u}gbarkeit {\"u}ber technisches Wissen. Kann das bereits vorhandene Wissen genutzt werden und zus{\"a}tzlich neues Wissen angeeignet werden, f{\"u}hrt dies zum catching-up Prozess. Neben dem Beispiel Deutschlands w{\"a}hrend der Industrialisierung, dienen f{\"u}r die neuere Zeit Japan und die "`Tigerstaaten"' als Musterbeispiele, die heute zu den f{\"u}hrenden Industrienationen z{\"a}hlen. Die Ursache f{\"u}r diese Aufholprozesse sieht Gerschenkron in der anf{\"a}nglichen R{\"u}ckst{\"a}ndigkeit eines Landes. Je r{\"u}ckst{\"a}ndiger ein Land entwickelte ist, desto h{\"o}her ist sein Entwicklungspotenzial. \citet{Nelson.1966}  schr{\"a}nken die These Gerschenkrons ein und halten die F{\"a}higkeiten der Arbeiter im Land f{\"u}r einen weiteren bedeutenden Faktor. Die R{\"u}ckst{\"a}ndigkeit allein helfe einem Land ohne Humankapital nicht die L{\"u}cke zum technologisch f{\"u}hrenden Land zu schlie{\ss}en. F{\"u}r sie gilt, dass je besser ein Land mit adaptiven F{\"a}higkeiten in der Bev{\"o}lkerung ausgestattet ist, umso schneller kann der Entwicklungsprozess stattfinden. Der Technologietransfer und die immitativen F{\"a}higkeiten im Land k{\"o}nnen auch als Absorptionsf{\"a}higkeit bezeichnet werden, dessen G{\"u}te durch die strukturellen Voraussetzungen im Land bedingt wird \citep{Abramovitz.1986}. {\"A}hneln sich die Strukturen beider interagierender L{\"a}nder des Technologietransfers, dann unterst{\"u}tzt dies den catching-up Prozess. Jedoch ist zu erw{\"a}hnen, dass Gerschenkron selbst die Quantifizierung der strukturellen Konstellationen und der Absorptionsf{\"a}higkeit als kritisch bewertet.\newline 


Zusammenfassend l{\"a}sst sich festhalten, dass jede Innovation einen Wissens- und Technologietransfer mit sich bringt, da eine uneingeschr{\"a}nkte Zug{\"a}nglichkeit zu Wissen und Ideen besteht und somit jegliche Ideen der Welt mit in den Entstehungsprozess einflie{\ss}en \cite{Gerschenkron.1962}.


\paragraph{Diffusion durch Handel}
Die Wirkung und Intensit{\"a}t eines Technologietransfers kann jedoch von au{\ss}en durch die politische F{\"o}rderung des Bildungssektor, des Forschungssektors oder auch durch den Au{\ss}enhandel beeinflusst werden.\newline Die Bedeutung des Forschungssektors betonen \citet{Griffith.2004} in ihrer empirischen Arbeit {\"u}ber den Einfluss von Forschung und Entwicklung auf das Wachstum eines Landes. Dabei verdeutlichen sie gleichzeitig den Einfluss der Offenheit eines Landes durch die damit verbundene Technologiediffusion auf das Wirtschaftswachstum. Denn die Forschung wirkt nur dann {\"u}ber beide Kan{\"a}le, wenn das tangierte Land bereits Au{\ss}enhandel aufgenommen hat. Zum einen steigt direkt die Innovationsrate und langfristig mit ihr auch die Wachstumsrate. Zum anderen kommt es zu einem indirekten Effekt auf die Wachstumsrate anderer L{\"a}nder durch den nun m{\"o}glichen Technologietransfer, jedoch nur in offenen Volkswirtschaften. Ihre Untersuchung bezieht sich auf die Erh{\"o}hung der Intensit{\"a}t des Technologietransfers, wenn L{\"a}nder ihren Forschungssektor f{\"o}rdern. Demzufolge ist es unabh{\"a}ngig vom technologischen Entwicklungsstand immer angebracht Investitionen in Forschung und Entwicklung zu t{\"a}tigen. Dieser Einfluss verst{\"a}rkt sich erneut  durch die Offenheit eines Landes. Laut \citet{Griffith.2004}  f{\"o}rdert ein Forschungssektor sowohl den Aufholprozess durch imitative Aktivit{\"a}ten, als auch den Entwicklungsprozess von Innovationen.\\


Hier soll gezeigt werden, welchen Einfluss der Bildungssektor auf die Technologiediffusion hat und in wie weit der Handel diese anregt.
Das weite Feld des "`Brain Drains"', die Abwanderung hochqualifizierter Arbeitskr{\"a}fte und Wissenschaftler, wird vernachl{\"a}ssigt, da in der folgenden Analyse von Migration abgesehen wird. Demzufolge findet ein Wissenstransfer nicht durch die {\"U}bertragung in Form von Zu- oder Abwanderung statt. Diesem Teilbereich der Wachstumstheorie widmen sich  Wissenschaftler wie \citet{Agrawal.2011,Docquier.Sept} und \citet{ONeil.WashingtonDC} mit dem Ergebnis, dass eine Abwanderung sehr gut ausgebildeter Arbeiter nicht den Wissensbestand einer Volkswirtschaft mindert oder sogar ersch{\"o}pft.  \citet{Docquier.Sept}  belegen in ihrer Untersuchung positive Einflussfaktoren bedingt durch den "`Brain Drain"', da beispielsweise neue Kontakte entstehen und ein Netzwerk aufgebaut werden kann. Ein optimales Einwanderungslevel qualifizierter Arbeiter und Wissenschaftler berechnen \citet{Docquier.Sept} f{\"u}r weniger weit entwickelte L{\"a}nder.\\
Das Modell von \citet{Grossman.1990c} geht von einem aktiven Informationsfluss zwischen Volkswirtschaften aus. Die Mehrheit der Handelsmodelle setzt gemeinhin voraus, dass mit der {\"O}ffnung eines Landes allen Wirtschaftsteilnehmern das gesamte Wissen des Weltmarktes zu Verf{\"u}gung steht, ohne dies zwingend zu beabsichtigen. \citet{Grossman.1990c} formulieren den Wissenstransfer hier als bewussten Prozess, der durch das Zusammentreffen von beispielsweise Wissenschaftlern oder Handelsvertretern, die als Bindeglied zwischen den M{\"a}rkten fungieren, zu Stande kommt \citep{Grossman.1990c}.\\


Findet Handel statt und werden Technologien, oder humankapitalreichere G{\"u}ter in das Land importiert, dann f{\"u}hrt dies nicht zwingend zu einem technologischen Fortschritt. Es ist durchaus denkbar, dass der Import zu diesem Land nicht "`passt"' und demzufolge keine Produktivit{\"a}tssteigerung hervorruft. So verhelfen neue Verfahrenstechniken der Pharmaindustrie einem Land ohne Pharmawesen nicht weiter, der Import ist demzufolge nicht zweckm{\"a}{\ss}ig. Denn ob eine Imitation erfolgreich ist h{\"a}ngt auch im Wesentlichen davon ab, ob ausreichend und vor allem angemessen qualifizierte Arbeitskr{\"a}fte vorhanden sind, die den Nachahmungsprozess durchf{\"u}hren. Auch das kann dazu f{\"u}hren, dass bestimmte G{\"u}ter oder Prozesse f{\"u}r eine Volkswirtschaft "`noch"' nicht geeignet sind, jedoch in einem sp{\"a}teren Entwicklungsstadium mit einem reformierten und angepassten Bildungssystem die Importe der selbigen Innovation die Produktivit{\"a}t steigern.
\bigskip

In dieser Arbeit wird klar zwischen Innovation und Imitation unterschieden. Als Imitationen werden implementierte ausl{\"a}ndische Technologien verstanden. Es wird hier nicht nur graduell zwischen beidem unterschieden, sondern klar differenziert anhand des eingesetzten Humankapitals \citep{Cohen.1989,Griffith.2004}.
