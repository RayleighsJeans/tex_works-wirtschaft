\chapter[Globalisierung durch Außenhandel]{Globalisierung durch Außenhandel}\label{sec:Globalisierung}
\chaptermark{Außenhandel}
 %
Unter dem Begriff Globalisierung kann im Allgemeinen die Ausdehnung geographischer Wirkungsbereiche und die Zunahme grenzüberschreitender Interaktionen verstanden werden \cite[Kapitel 3, S. 35]{Kessler.2009}. Er bezieht sich nicht ausschließlich auf die wirtschaftlichen Aspekte einer zusammenwachsenden Welt, sondern kann deutlich umfassender verstanden werden. So führt nicht nur der Ausbau der Infrastruktur, die Ausweitung der Institutionen und die Verbreitung der Informationstechnologien zu einem gemeinsamen wirtschaftlichen und politischen System, sondern auch zu einer Annäherung verschiedener Kulturkreise. Im engeren Sinne kann Globalisierung als wirtschaftlicher Integrationsraum verstanden werden, in dem frei gehandelt wird. Ausgehend von freien Märkten sollen in dem folgenden Kapitel die Vorzüge des Freihandels gezeigt werden und welche Wirkungen die Außenwirtschaft auf den Globalisierungsprozess hat.\\
%
Das Regelwerk des freien Marktes, wie wir es heute verstehen, wird vornehmlich auf zwei Ökonomen zurückgeführt: David Ricardo und Robert Malthus. Beide spielten eine entscheidende Rolle in der Hausbildung der britischen Gesellschaft des 19. Jahrhunderts, die als Vorreiter der Entwicklungsgeschichte der heutigen industrialisierten Volkswirtschaften gilt.\\
%
Dabei hat vor allem die Theorie des komparativen Vorteils in den letzten 40 Jahren als Marktlogik zu einer immer arbeitsteiligeren globalen Wirtschaft geführt, die einerseits tiefgreifende gesellschaftliche und politische Veränderungen mit dem steigenden Wohlstand vieler Volkswirtschaften mit sich gebracht hat, aber auch ebenso vielen Ländern schadete.\\
%
Unter der Annahme, dass der heutige Wettbewerb sich in einem ständigen Spannungsfeld zwischen wirtschaftlichen und politischen Interessen bewegt, soll in diesem Kapitel der Prozess der Globalisierung betrachtet und der Frage nachgegangen werden, ob sich die Entwicklung der letzten 40 Jahre auch tatsächlich auf die Theorie des frühen 19. Jahrhunderts zurückführen lässt oder aber eher durch politische und wirtschaftliche Interessen zu erklären ist? \\
%
Legt man hierbei die geschichtliche Entwicklung der unterschiedlichsten Volkswirtschaften zugrunde, wurde vor allem eine Entwicklungsstrategie favorisiert, die vielerlei Anwendung fand, sich dann aber recht spät als suboptimal herausstellte.\\ 
%
Die vorliegende Arbeit hat den Anspruch, nicht nur einen Eindruck von der Vielschichtigkeit möglicher Strategien zu vermitteln, sondern auch deren Motive und Konsequenzen zu analysieren. Dabei wird von der These ausgegangen, dass keine perfekte und allgemeingültige Entwicklungsstrategie existiert, sondern wirtschaftliche Entwicklungen einerseits zwar stark von den internen Rahmenbedingungen einer Volkswirtschaft abhängen, aber in gewissem Maße auch von externen Einflussgrößen, die nicht unmittelbar gesteuert werden können.\\
%
Für die Entwicklung zu den heute industrialisierten Volkswirtschaften und die weltweite Handelsstruktur spielte die Theorie David Ricardos mit ihrem Ansatz vom komparativen Vorteil eine entscheidende Rolle. Im letzten Jahrhundert fand seine Theorie mehrfach von weniger weit entwickelten Volkswirtschaften als Entwicklungsstrategie Anwendung. So zählt  der 1772 in England geborene Ricardo heute zu den bedeutendsten Vertretern der klassischen englischen Nationalökonomen. Er war Sohn einer holländisch-jüdischen Einwandererfamilie, die zu den wohlhabendsten Familien seiner Zeit zählten. \cite{Lin.2007} schildert weiter, dass er bereits ab seinem 14. Lebensjahr an der Börse zusammen mit seinem Vater arbeitete. Die Börse entsprach zu der damaligen Zeit eher einem losen Zusammenschluss von Menschen, die sich in Caféhäusern trafen. Der Aktienhändler Ricardo arbeitet schon in jungen Jahren auf eigenes Risiko und war bereits mit 20 Jahren ein erfolgreicher, gestandener und reicher Mann. \\
%
Im Jahre 1796 reiste er nach Bath, dort las er das erste Mal im Hauptwerk von Adam Smith dem "`Wohlstand der Nationen"' und begann anschließend mit seinen Studien über die Wirtschaftspolitik. Die Ansichten Adam Smiths prägten nicht nur David Ricardo, sondern veränderten die Weltanschauung der folgenden Generationen. Erst Jahre nach seinem Tod wurde der Einfluss und das Ausmaß der Güte Smiths' Hauptwerkes "`Wohlstand der Nationen"' deutlich \cite[S. 24]{Lin.2007}. Vor diesem Hintergrund ist zunächst eine Auseinandersetzung mit Adam Smith sinnvoll, um anschlie{/ss}end den komparativen Vorteil fundiert darstellen zu können.\\
%
Adam Smith war einer der ersten großen Wirtschaftsdenker und Begründer der klassischen Schule der Nationalökonomie. Er beschreibt als erster die Gesetze des Marktes, die zur Stabilität der Gesellschaft beitragen sollten.  Sein liberales Weltbild zeigt sich in diesen Gesetzen des Marktes, die eine selbstregulierende Gesellschaft hervorbringen sollen. Er setzte sich für die Abschaffung der zentralen Instanz ein, die die Wirtschaftsabläufe steuert, demnach ist er einer der ersten Kritiker der Institution "`Staat"'. Der absolutistische Staat soll sich in eine Welt mit eigenverantwortlichen selbstbestimmten Individuen verwandeln \cite[S.25]{Huther.2006,Lin.2007}.\\
%
Statt den Handel politisch zu lenken, propagiert Smith den freien Austausch von Waren und Dienstleistungen. Auch innenpolitisch war er der Meinung, dass die Kräfte des Wettbewerbs ausreichen, um die Wirtschaft zu steuern und dadurch staatliche Eingriffe nicht mehr notwendig sind. Diese eigenständigen Mechanismen seien nur funktionsfähig, wenn der Staat durch die unsichtbare Hand ersetzt werden würde. \\
%
Smith ist der Ansicht, dass dem Staat die Aufgaben zufallen, die für die ganze Gesellschaft nützlich sind. Dazu zählen die Bereiche der sozialen Sicherung wie das Rechtswesen und Aufbau bzw. Instandhaltung einer intakten Infrastruktur. Er verlangt beispielsweise auch produktive Staatsausgaben wie in Bildung \cite[S. 39-40]{Huther.2006}. \\
%
In seinem Hauptwerk "`An Inquiry into the Nature and Causes of the Wealth of Nations"' von 1776 analysiert er unter anderem die wohlfahrtsmehrenden Effekte von Arbeitsteilung und freien Märkten. Arbeitsteilung führt seiner Ansicht nach zur Erreichung des wichtig\-sten ökonomischen Ziels: Effizientes Arbeiten. Arbeitsteilung bedeutet die Untergliederung der zu verrichtenden Arbeit in kleinere Aufgaben und führt zu kleinen spezifischen Arbeitsschritten, auf die sich die Arbeiter spezialisieren. Es fördert die Geschicklichkeit der Menschen und ihre Fähigkeiten können leichter weiterentwickelt werden, wodurch ihre Arbeitsproduktivität ansteigt. Als Konsequenz kann in der gleichen Zeit mehr produziert werden. Ohne eine starke Arbeitsteilung wäre die Industrialisierung nicht denkbar gewesen - und sie löste ein vorher nie gekanntes Wirtschaftswachstum aus. Dass Arbeitsteilung und Spezialisierung die Menschen aus einer jahrhundertelangen ökonomischen Stagnation auf einen dauerhaften Wachstumspfad gebracht haben, wurde bereits in \ref{Unified} geschildert.\\
%
Außerdem ist es Smiths Verdienst, dass die Ökonomie sich zu einer eigenständigen wissenschaftlichen Disziplin entwickelt hat \cite{Lin.2007,Huther.2006}. \\
%
Die Welt zu Smiths Zeiten war wirtschaftlich und politisch im Wandel, denn die einsetzende industrielle Revolution machte die zuvor agrar-dominierte Wirtschaft immer komplexer. Es stellten sich Lohn-, Preis- und Verteilungsfragen und neue Phänomene wie Arbeitsteilung, Massenproduktion und ein wachsender Finanzsektor traten auf \cite[S. 41-42]{Huther.2006}.\\
%
Adam Smith stellte daher eine Verbindung der Wirtschaft mit dem Staat und dem Recht her. Ricardo hingegen erklärt die Wirtschaft durch die Wirtschaft und ermöglichte dadurch eine rein ökonomische Reflektion, indem er die Wirtschaftswissenschaften von den anderen Sozialwissenschaften isolierte. Hierdurch kam es zu einem wichtigen Wendepunkt in der Geschichte der Wirtschaftswissenschaften. David Ricardo wurde zu einem der ersten Globalisierungstheoretikern und einem führenden Vertreter der klassischen Nationalökonomie. \\
%
Anders als Adam Smith hatte Ricardo erkannt, dass die gesellschaftlichen Schichten unterschiedlich vom wirtschaftlichen Wachstum profitieren und dadurch ein Ungleichgewicht entstehen wird. Dies zeigte sich auch kurz darauf in der Umsetzung eines Getreidegesetzes, welches Schutzzölle für die Getreideeinfuhr vorsah. Die Durchsetzung dieses Gesetzes wurde vor allem durch die mächtigen Großgrundbesitzer ermöglicht, deren Wohlstand erheblich von den Getreidepreisen abhingen und somit dieses Gesetz befürworteten \cite[S. 134]{Kurz.2008}.\\
%
Anfang des 19 Jahrhunderts beobachtete Ricardo gewisse gesellschaftliche und politische Vorkommnisse mit Sorge. Dazu zählten, dass die Landflucht zunahm, ebenso wie das Bevölkerungswachstum und der immer weiter voranschreitende Prozess der Industrialisierung. Er fragte sich, wie sich die gesellschaftlichen Reichtümer langfristig verteilen werden. Nach seinen Schlussfolgerungen könnte der Preis für das Agrarland allen Wohlstand absorbieren und somit wären die Grundbesitzer, ohne staatliche Intervention von außen, irgendwann unermesslich reich. Den Grund sah Ricardo darin, dass fruchtbarer Boden durch das anhaltende Bevölkerungswachstum zu einem extrem seltenen und kostbaren Gut wird. Die Ländereien waren überbewertet und konnten die Bevölkerung nicht mehr ernähren. Die Grundbesitzer könnten ihren Lebensunterhalt allein durch die Verpachtung ihres Grund und Bodens bestreiten. \\
%
Mitverantwortlich für den aufkommenden Pessimismus ist ein Freund Ricardos, der bereits vorgestellte Thomas Malthus, der im Jahre 1766 in der englischen Grafschaft Surrey südlich von London geboren wurde. Er befasste sich intensiv mit der Problematik des Bevölkerungswachstums und teilte größtenteils die Ansichten Ricardos .\\

\cite{Lin.2007} bezeichnet Malthus als den ersten professionellen Nationalökonomen, der den weltweit ersten Lehrstuhl für Geschichte und politische Ökonomie in England inne hatte. Bedingt durch seine pessimistische Einstellung wurde er als der am meisten gehasste Mann seiner Epoche beschrieben. \\
%
Sein erstes Werk "`An Essay on the Principle of Population as It Affects the Future Improvement of Society"' veröffentlichte er 1798 anonym. Es handelt von dem Bevölkerungswachstum und der damit einhergehenden Prognose drohender Hungersnöte und Verelendung. Er beschreibt den Zusammenhang zwischen dem Bevölkerungswachstum und der Nahrungsmittelproduktion.\\
%
Seiner These nach wächst die Bevölkerung Englands schneller, als die Fähigkeit genügend Lebensmittel zu produzieren. Dem exponentiellen Bevölkerungswachstum zur Folge verdoppelt sich die Menschheit etwa alle 25 Jahre, wohingegen die Lebensmittelproduktion im selbigen Zeitraum nur linear wächst. Demnach wird ein Zeitpunkt eintreten, zu dem die Ressourcen der Erde nicht mehr ausreichen würden, um die Bevölkerung ausreichend zu ernähren. Die damit einhergehenden Probleme wie Krankheit, Elend und Tod erhöhen die Sterblichkeitsrate und korrigieren damit das Bevölkerungswachstum nach unten. Als erster beschreibt er dabei die Bevölkerungsfalle und dessen Folgen \cite{Malthus.1798}. \\

Vor seiner Arbeit ging man davon aus, dass in einem Land mit der wachsenden Bevölkerung auch eine wachsende wirtschaftliche Leistungsfähigkeit einhergeht. Laut seiner Bevölkerungstheorie kommt es aber zu Verarmung und Verelendung des betrachteten Landes. Außerdem hinterfragt \cite{Malthus.1798} in seiner Arbeit wodurch die Zahl der Menschen begrenzt wird und was letztlich zu dem beobachteten Rückgang der Sterblichkeitsrate geführt hat. Seine politische Besorgnis lässt die Angst vor Überbevölkerung und pessimistische Grundausrichtung nachvollziehen, denn Zeit seines Lebens herrschte die französische Revolution mit der ein Großteil der Probleme einhergingen.\\
%
 Erst Jahre später zeigte sich, dass er vor allem den Menschen selbst und seinen Erfindergeist unterschätzte. Er war skeptisch hinsichtlich der Geschwindigkeit des technischen Fortschritts, die vor allem in der Landwirtschaft die Produktivität erheblich erhöhte und damit die Ernten vergrößerte. Diese Aspekte wurden seinerseits vernachlässigt. \\
%
 Hinterfragt man noch einmal die Vorhersagen von Malthus, dann lag er hinsichtlich England mit seiner Analyse sicherlich falsch. Allerdings besteht weiterhin das Problem wachsender Hungersnöte vor allem in Entwicklungsländern. Die Nahrungsmittelproduktion überholte das Bevölkerungswachstum um ein Vielfaches und der Hungertod ist heute seltener als Ende des 18. Jahrhunderts. Die Ursache ist in weniger weit entwickelten Ländern jedoch eine andere, als die von Malthus beschriebene, denn hier beruhen Hungersnöte vornehmlich auf sozialer Ungerechtigkeit und nicht auf dem Unvermögen ausreichend Nahrungsmittel zu produzieren \cite[S.119]{Hesselbein.2000}.

Ricardo hinterfragte zu seinerzeit ebenfalls die Thesen von Thomas Malthus und beschäftigte sich mit seinen pessimistischen Auffassungen. Das ansteigende Bevölkerungswachstum würde langfristig zu einer Bestellung qualitativ schlechterer Böden und zu einem Anstieg der Nahrungsmittelpreise führen. Er begann, wie Adam Smith, die Gesetzmäßigkeiten des Wirtschaftswachstums zu erforschen, dabei erarbeitete er eine Reihe konkreterer Vorschläge zur Liberalisierung des Marktes und zur Förderung des privaten Unternehmertums. Es entstand eines der ersten Wirtschaftsmodelle: Die Theorie des komparativen Kostenvorteils.\\
%
  Kern dieser Theorie ist, dass jeder das macht, was er am besten kann und jeder von dem Wissen und der Erfahrung des anderen profitiert. In Anbetracht der damaligen politischen Lage und einer unsicheren Zukunft produzierten alle Länder aus Angst alles was sie brauchten selbst und erhoben hohe Zölle auf ausländische Waren. In seiner Arbeit suggerierte Ricardo (\citeyear{Ricardo.1817}) erstmals ein Interesse die Märkte für freien Warenaustausch zu öffnen.\\
%
 So bemühte er sich auch um die Aufhebung des Getreidegesetzes, welches die lobbyistisch mächtige Stellung der Grundbesitzer veranlasst hatte den Getreidepreis künstlich zu regeln. Ricardo ging es im wesentlichen um die Verringerung bzw. Abschaffung der Zölle auf Getreide. Dies würde die Dynamik des Wettbewerbs steigern und zu einem geringeren Brotpreise führen. \\
%
 David Ricardo veranschaulichte seine Theorie anhand des Beispiels von Tuch und Wein, die in den beiden Ländern England und Portugal hergestellt wurden. Er hinterfragte, warum ein Land beide Güter herstellen sollte, wenn sich ein Land auch spezialisieren kann und dann mehr von einem Gut herstellt, welches es gegen das andere Gut eintauschen kann. Er zeigte in seinem Modell, dass Handel in allen beteiligten Ländern den Wohlstand erhöht, auch wenn ein Land absolute Kostenvorteile aufweist. Der Kern seiner Theorie liegt im komparativen Kostenvorteil. Dieser geht aus den technologischen Gegebenheiten und den damit verbundenen Produktivitätsunterschieden beider Länder hervor und führt zur Vorteilhaftigkeit von Freihandel. \cite{Ricardo.1817}s (\citeyear{Ricardo.1817}) daraus resultierende theoretische Schlussfolgerung war nun, dass die Theorie universal gültig sei für alle Länder.\\
%
 So wurde diese im 21. Jahrhundert von der Welthandelsorganisation (WTO) in die Praxis umgesetzt. Dabei wurde das Prinzip des komparativen Vorteils genutzt, um den Freihandel populär zu machen. Eindeutig war es für die meisten Wirtschaftswissenschaftler aber noch nicht, ob es sich tatsächlich um einen Motor für Wachstum und Wohlstand handelt. Seine Arbeit über den komparativer Vorteil setzte nämlich die Annahme der Vollbeschäftigung voraus und dass alle Länder Zugang zu allen Technologien haben, somit kein Technologietransfer stattfindet. \\
%
 Im Nachhinein  lässt sich sagen, dass die Theorie sich heute als effizienter erwiesen hat, als sie es in der Vergangenheit jemals war. Damals waren Entfernungen in der Welt wichtig und stellten eine Hemmschwelle für den derzeitigen internationalen Handelsaustausch da. Die fremden Länder und mögliche Handelspartner waren weit entfernt und der Transport war somit kostspielig und zeitaufwendig. Zweihundert Jahre später sind die Transport\-kosten deutlich geringer. Der technologische Fortschritt und eine ausgeprägte Infrastruktur erleichtern die Überwindung von Distanzen. Einschlägige Beispiele hierfür sind technologische Errungenschaften, wie das Internet und eine verbesserte Verkehrsanbindung und Transportmöglichkeit durch Containerschiffe \cite[S. 191-196]{Rosner.2012}. \\
%
 Problematisch bei der realen Anwendung ist jedoch, dass sein Modell die Möglichkeit der Arbeitslosigkeit ausschließt. Jeder der einen Arbeitsplatz verliert bekommt einen Arbeitsplatz in der anderen Branche. Im Modell von Tuch und Wein steckt die Annahme dahinter, dass alle Arbeitsplätze in der jeweils anderen Branche finden, unabhängig von Qualifikationsvoraussetzungen \cite{Ricardo.1817}.\\
%
 Außerdem lehnt er Faktormobilität ab und schließt aus, dass ein Kapitalist mit seiner Technologie nicht in einem anderen Land zu billigeren Löhnen produzieren kann und in dem ursprünglichen Herkunftsland sein Gut nur noch verkauft.\\
%
 Diese Auswirkungen der Modellrestriktionen werden durch das Beispiel von General Motors (GM) verdeutlicht. Der Grund für die Schließung des Standorts in Lynn von GM in den USA war nicht der Freihandel oder die Theorie Ricardos, sondern nur die Suche nach billigerer Arbeitskraft. \\
%
 Wie aber vereinbarte Ricardo die faktische Suche nach billigen Arbeitskräften mit seinem Anspruch nach Wohlstandsgewinn für alle? \\
%
 Steht das wirtschaftliche Interesse über dem sozialen Interesse, dem Wohlfahrtsgewinn, dann kann bei dem Gedanken vom freien Handel ein wesentlicher Bestandteil sein, die Interessen bestimmter Gruppen zu fördern. Dies ist meist der Fall bei großen Unternehmen, die ihre Standorte schnell verlegen können, wie es bei GM der Fall war. Der Öffentlichkeit wird ein bestimmtes Gesellschaftsmodell dargeboten, das mit der Argumentation wissenschaftlicher Erkenntnisse untermauert wird.\\ 
%
So verlagerte General Motors in Flint, Michigan USA, ab 1978 ihre Produktionsstätten nach Mexiko und später nach China. Es ist durchaus denkbar, dass der Verlust von ca. 40 000 Arbeitsplätzen die Folge vom Freihandel ist und den damit einhergehenden politischen Entscheidungen.\\
%
Befasst man sich nur mit der Geschichte des freien Welthandels ungeachtet der Wissenschaft und Ricardos Grundthese des komparativen Vorteils, so gründen sehr frühe Handelsbeziehungen auf Zwang durch Waffengewalt. \\
%
 England war im 18. Jahrhundert das Handelszentrum der Welt und in mehrere weltweite Handelskriege verwickelt. Der weltweite freie Markt wurde vom Imperialismus geschaffen, denn ohne Kolonisation hätte das britische Empire kaum Märkte für seine Produkte schaffen können. So kam es letztlich in China Ende des 18. Jahrhunderts zum Opiumkrieg. \\
%
 \cite{Straubhaar.2011} sieht das Motiv des Krieges zunächst in einer bis ca. 1820 unausgeglichenen bilateralen Handelsbilanz, zugunsten der Chinesen. Die Europäer hatten den begehrten chinesischen Exportartikeln, wie Tee und Seide, meist wenig entgegenzusetzen. Die Briten wollten Textilien aus Baumwolle und Wolle auf dem chinesischem Markt verkaufen, scheiterten jedoch mit beidem, weil das Material für die dortigen Verhältnisse zu warm war und es schon eine fortschrittlichere Textilindustrie gab. Um Tee, Rohseide und andere Produkte in China zu kaufen, musste Großbritannien große Mengen Silber ausgeben. Die damit verbundenen Devisenabflüsse nach China führten in Europa zu einer spürbaren Silberverknappung, die wiederum fatale Auswirkungen auf die dortigen Volkswirtschaften hatte. Um dem zu begegnen, gingen die Engländer dazu über, im von ihnen beherrschten Indien mehr Opium produzieren zu lassen. Es sollte eine Nachfrage nach Opium erzeugt werden, um somit Silber als Zahlungsmittel zu umgehen.  Dieses Opium, für das es in China einen sehr aufnahmefähigen Markt gab, wurde dann mit Unterstützung bestochener Hafen- und Verwaltungsbehörden auf dem chinesischen Markt verkauft. Den andauernden Handel mit China konnte England nur gewaltsam und durch den Verkauf von Opium ausbalancieren. Jetzt kehrten sich der {\dq}Silberfluss{\dq} und die Handelsbilanz zugunsten der Europäer um.\\
%
 Hierin begründete sich eine der ersten dokumentierten {\dq}freien{\dq} Handelsbeziehungen der Geschichte \cite[S.2]{Straubhaar.2011}.\\
%
 Chinas Kaiser ließ das Opium verbrennen und vernichten. Er verbot die Einfuhr, sowie den Verkauf und Konsum. Fraglich ist jedoch, ob diese Maßnahmen wirtschaftlich motiviert waren oder die massenhafte Opiumsucht, die inzwischen auch die oberen Gesellschaftsschichten ergriffen hatte, den Grund für das Verbot darstellten. \\
%
 Dem Verbot des Kaisers begegnete Großbritannien mit dem ersten Opiumkrieg. Queen Victoria setzte mit dem Zwang Chinas zu freiem Handel ein Exempel und verdeutlichte den anderen Ländern die Konsequenzen bei Zuwiderhandlungen. Sie wollte verhindern, dass auch die anderen Kolonialländer sich ebenfalls weigern Freihandel zu betreiben.  Im Jahre 1840 griffen britische Truppen den Freihandelshafen Kanton an. Es entwickelte sich der fast drei Jahre dauernde Opiumkrieg. In dessen Verlauf besiegten die überlegen ausgerüsteten britischen Landungstruppen, unter dem Schutz der modernen englischen Kriegsschiffe, die chinesischen Truppen \cite[S.2]{Straubhaar.2011}.\\
%
 Nach der Niederlage musste China den Opiumhandel wieder zulassen, Hongkong an England abtreten und weitere Handelspunkte öffnen. Mit dem Nanjing-Vertrag und anderen ungleichen Verträgen verlor China seine politische Unabhängigkeit. Als Folge eines weiteren Opiumkriegs erzwangen 1844 die USA und Frankreich weitere Verträge. Mit denen verlor China seine Zollautonomie, also das Recht, Zölle zu erheben. China war gezwungen, seinen wirtschaftlichen Protektionismus aufzugeben \cite[S.5]{Schliemann.1984}.
Neben China liefert Haiti ein weiteres Beispiel für die Anwendung Ricardos Theorie. Die Entwicklung, die Haiti durchlief, ist in vielerlei Hinsicht bemerkenswert. Alexander \cite{King.2005} beschreibt am Beispiel Haiti den Einfluss der Globalisierung auf die Entwicklung eines Landes. Der Entdeckung im 15. Jahrhundert durch Christoph Columbus folgte die Auslöschung der indigenen Bevölkerung und die Wiederbevölkerung durch die Kolonialmächte mit aus Afrika stammenden Sklaven im 17. Jahrhundert. Zu Zeiten der französischen Kolonialisierung galt Haiti als eines der reichsten Länder Lateinamerikas und zählt heute zu den am wenigsten weit entwickelten Ländern der Welt \cite[S. 44]{Beck.2008,IBP.2013,Stauber.2014}.\\
%
 Die Industrialisierung wurde in dem Mutterland Frankreich erheblich unterstützt, jedoch wurde dies gleichzeitig in der Kolonie unterbunden. Dies geschah beispielsweise durch ein Verbot von verarbeitendem Gewerbe in der Kolonie selbst, wodurch die Wirtschaft zusätzlich abhängig  von dem~Mutterland wurde und die Instabilität gefördert wurde. Die ökonomischen Potenzen einer Kolonie wurden nur hinsichtlich des Nutzens für die Kolonialmächte gefördert, jedoch nicht, um langfristig die Entwicklung Haitis zu unterstützen \cite{King.2005}.\\
%
Unmittelbar nach der Liberalisierung 1980 ließ Haiti Handel mit der übrigen Welt zu. Zurückzuführen ist dies auf eine Bedingung der WTO, um internationale Anleihen zu erhalten. Dies brachte jedoch schwere Folgen für den landwirtschaftlichen Sektor Haitis mit sich. Ein Land, dass sich zuvor noch selbst versorgen konnte, verzeichnete nun Hungersnöte in der Bevölkerung. Der vorhandene fruchtbare Boden wurde unter der Bevölkerung aufgeteilt und die Agrarstruktur bestand nun aus kleinen Parzellen, deren Produktivität deutlich geringer war, als die der Großplantagen. Das Problem der Bodenerosion verstärkte diesen Effekt und die Abholzung des beinah gesamten Regenwalds führte zusätzliche zur Desertifikation. Die Übernutzung des übrigen fruchtbaren Bodens war die Folge. Dennoch galt das Land als Exporteur von Kaffee, Kakao, Häuten und Bauholz \cite[S. 76-77]{King.2005}.\\
%
 Vor Haitis Unabhängigkeit war die~Bevölkerung noch fähig die eigene Ernährung durch Reisanbau zu sichern. Landesweit führte der Verlust an landwirtschaftlichen Flächen für den eigenen Verbrauch zu sozialer Destabilisierung des Landes.\\
%
 Auch die Vergabe von Krediten durch den Internationalen Währungsfond (IWF) war an die Bedingung des Freihandels gekoppelt. Diese begründeten ihr Vorgehen darin, dass offene Märkte als Wachstumsfaktor gefördert werden sollten und versuchten damit Ricardos Theorie in die Realität zu übertragen \cite[S. 104]{Weiss.2008,InternationalMonetaryFund.2007}.\\
%
 Der komparative Vorteil Haitis lag in den günstigen Arbeitskräften und den natürlichen Umweltbedingungen in der Landwirtschaft. Daraus leitete sich eine Entwicklungsstrategie ab, die den Schwerpunkt Haitis auf exportorientierte Landwirtschaft und Montageindustrie legte. Doch liefert gerade Haiti ein~Negativbeispiel für Ricardos Ansichten. Die Ernährungssicherung in Haiti wurde durch die Verdrängung der Kleinproduktion in den 1980er und 1990er Jahren gefährdet, weil Importe vom subventionierten US-amerikanischen Reis und Zucker den heimischen Markt dominierten. Der Reisanbau lohnte für viele Bauern nicht mehr und sie waren gezwungen ihr Land aufzugeben. Zeitgleich wurden Kaffee- und Mangoplantagen durch Gelder der US-amerikanischen Entwicklungszusammenarbeit gefördert.\footnote{Möglicherweise liegt in diesem Spezialanbau tatsächlich ein komparativer Kostenvorteil Haitis.} Doch konnte der steigende Nahrungsmittelbedarf, der durch das Bevölkerungswachstum bedingt ist, nicht durch die kleiner werdende Lebensmittelproduktion gedeckt werden. Das Einkommen aus dem Anbau von Kaffee und Mangos ist zu gering, um eine importbasierte Sicherung der Ernährung zu gewährleisten. Die Entwicklungsstrategie sah vor, den Zugang zu lebensnotwendigen Gütern über den Importmarkt sicher zu stellen, was jedoch unvereinbar mit der Selbstversorgung des Landes war. Da der haitianische Binnenmarkt zu klein erschien, wurde für den internationalen Markt produziert. Der zweite Schwerpunkt, die Montageindustrie, sollte dabei die Kaufkraft für die importierten Güter sicherstellen. Trotz erheblicher Steuernachlässe, die der Unterstützung der Montageindustrie dienten, waren deren Entwicklungspotenziale beschränkt. Im Jahr 1984 befanden sich 96 Montagebetriebe auf Haiti und erreichten damit ihren Höhepunkt \cite[S. 69]{King.2005}. \\
%
 Die Hälfte der Bevölkerung ist arm und unterernährt. Belegt ist diese Aussage durch FAO-Angaben von 2010 und durch Daten des auswärtigen Amts aus dem Jahre 2007, die besagen, dass die Hälfte der Bevölkerung mit weniger als 1 US-Dollar pro Tag auskommen muss. Dieser Wert liegt laut WTO unter der Armutsgrenze von 1 US Dollar am Tag. Bei einer Gesamtbevölkerung von 9,4 Millionen Einwohnern entspricht dies 5,5 Millionen Haitanern.  \\
%
 Die UNO sieht die Schuld  am Scheitern der Agrarproduktion bei den Liberalisierungsprogrammen.\footnote{\cite[S. 74-77]{King.2005} nennt hier bei Haiti ein 1995 beschlossenes Strukturanpassungsprogramm, das neben wettbewerbspolitischen Maßnahmen, wie der Privatisierung der neun größten Staatsbetriebe, unter anderem auch die Verringerung von Importzöllen regelte.} Die Entwicklungsländer sind nicht industrialisierter als zuvor. Wie dieses Beispiel zeigt ist das Gegenteil der Fall: Viele Länder können heute ihr eigenes Volk nicht mehr ernähren.\\
%
 Die Liberalisierung wurde stark durch die USA befürwortet. Seit 1981 verfolgte die amerikanische Politik den Standpunkt den weniger entwickelten Ländern niedrigere Güter wie Nahrungsmittel zu liefern, um sich auf die Industrialisierung zu konzentrieren und den Sprung ins industrielle Zeitalter zu schaffen.  Wie Präsident Clinton in dieser Zeit öffentlich zugab, habe diese Strategie nicht funktioniert und bekannte diese Vorgehensweise als folgenschweren Fehler.\\
%
Das Beispiel Haitis zeigt die Probleme auf, die nach Ansicht der Globalisierungskritiker durch die Inanspruchnahme von Krediten des IWF entstehen können. \\
%
 Weltweit steigende Grundnahrungsmittelpreise führten dazu, dass sich die Regierung Haitis 1986 an den IWF wandte, um Kredite aufzunehmen. Die Grundidee des IWF basiert auf der Stärkung des politischen Friedens und dem weltweiten Wohlergehen. Die Ziele des 1944 gegründeten Weltwährungsfonds sind unter anderem die Förderung der internationalen Zusammenarbeit in der Währungspolitik, die Stabilisierung der internationalen Finanzmärkte und die Überwachung der Geldpolitik.  Ein weiteres und für die vorliegende Arbeit zentrales Ziel, ist eine Analyse der Konsequenzen einer Ausweitung des Welthandels.
 %
 Mit der Unabhängigkeit vieler Länder in den  1950er und insbesondere in den 1960iger Jahren, wurde die Notwendigkeit dieser Sonderorganisation deutlich. Das Wachstumspotential der meist weniger weit entwickelten Länder, konnte nur durch weitere Investitionen ausgeschöpft werden. Die finanzielle Hilfe gewährte der IWF, jedoch unter strengen Auflagen und Bedingungen, die der Ideologie des freien Marktes folgen. Das Beispiel Haiti zeigt, dass es zur Ausplünderung von Rohstoffen durch transnationale Konzerne kommen kann und die sozialen Auswirkungen von Krisen und Hilfsmaßnahmen nicht bedacht wurden \cite{IBP.2013}.\\
%
Ebenfalls unter dem Einfluß der Kolonialmächte stand Ghana. Trotz der bedeutenden wirtschaftlichen Stellung des Landes, aufgrund der Goldvorkommen, zählt auch Ghana zu den ärmsten Ländern der Welt. Im~Jahr 2003 belief sich der Anteil der Bevölkerung mit einem Einkommen von weniger als einem US-Dollar pro Tag auf 45 {\%} \cite{Regeher.2013}. In den 80er Jahren wurden Ghana Darlehen zur Schuldenreduzierung der großen Organisationen WTO und IWF gewährt, unter der Auflage eines Strukturanpassungsprogramms. Dieses beinhaltete wieder die Öffnung des Marktes für ausländische Investoren und hatte Massenarbeitslosigkeit, eine wachsende Schattenwirtschaft und einen Rückgang lokaler landwirtschaftlicher Erzeugnisse zur Folge. Ebenso wie bei Haiti führte die Wirtschaftsliberalisierung zu Monokulturen und Reisimporten. \\
%
 Auch dieses Beispiel verdeutlicht kritische Anmerkungen am Globalisierungsgedanken. Wird ein Wettbewerb zwischen armen und reichen bzw. zwischen strukturschwachen und -starken Ländern zugelassen, dann wird voraussichtlich das weniger weit entwickelte Land den Kürzeren ziehen. Investiert ein relativ reiches Land Kapital in ein weniger weit entwickeltes Land, dann garantiert dieses Vorgehen noch nicht die gesellschaftliche und politische Entwicklung des weniger weit entwickelten Landes. \\
%
 Diese beiden Beispiele zeigen die negativen Aspekte des Freihandels. Blickt man jedoch auf die vergangenen 50 Jahre Wirtschaftsgeschichte zurück, so gibt es auch zahlreiche positive Beispiele. Die positive Wendung trat im Fall Ghana relativ spät ein, wie der politische Sonderbericht Ghanas zeigt. Im Jahr 2014 sank der Anteil der Bevölkerung mit einem Einkommen von weniger als einem US-Dollar pro Tag von 45 {\%} (2003) auf 28,5{\%} und konnte ein Wirtschaftswachstum von 7,43{\%} pro Jahr verzeichnen.\footnote{Die im Jahr 2007 entdeckten Ölvorkommen stellen eine weitere Entwicklungschance für Ghana dar. Jedoch zeigt das bisher tendenziell schleppende Wachstum, dass die Gefahr des "`Ressourcenfluchs"' besteht \cite{Regeher.2013}.}\\
%
  Im Schnitt liefern Länder, die Handel zulassen bessere Wirtschaftsdaten als Länder die nicht oder dies nur im beschränkten Masse getan haben.\\
%
Ein Musterbeispiel für den Erfolg von Freihandel liefert die Koreanische Halbinsel. Anhand der Entwicklungsprozesse der letzten 60 Jahre lassen sich durch einen Vergleich von Nord- und Südkorea Rückschlüsse über die Wirkungsweise politischer Entscheidungen ziehen. \\
%
 Die Grundvoraussetzungen auf der Koreanischen Halbinsel waren die gleichen, wie Rohstoffvorkommen, Kultur, Militär und die wirtschaftlichen Institutionen. Vor dem zweiten Weltkrieg stand Korea unter japanischer Herrschaft und wurde bedingt durch den japanischen Einfluss gegen Ende des 19. Jahrhunderts zur Öffnung von drei Handelshäfen gezwungen \cite{Engelhard.2004,Lee.1999}.\\
%
Südkorea ist heute durch seine stete Handelsoffenheit, die nach dem zweiten Weltkrieg ausgedehnt wurde, gut entwickelt, während Nordkorea in einem wirtschaftlich desolaten Zustand ist, weil es weitestgehend verschlossen agierte und sich damit weiter isolierte.\\
%
 Die Erfolgsfaktoren und Ereignisse der südlichen Halbinsel werden im folgenden ausführlicher dargelegt.
Der Entwicklungsprozess Südkoreas wurde zunächst bis Ende der 80er Jahre strengen Grundsätzen folgend von der Regierung gesteuert. Die Wirtschaftsplanung erfolgte flexibel und ideologisch ungebunden, strebte jedoch weiterhin einen exportorientierten Ausbau des Industriesektors an. Dabei war der Staat vor allem kontrollierend tätig. Die staatlichen Investitionen wurden wachstums- und exportfördernd eingesetzt und es war dem Staat gestattet in die Führung privater Unternehmen einzugreifen, wie beispielsweise größere Investitionsentscheidungen mitzutragen. Dieser staatlich bestimmte Entwicklungsprozess lässt sich nach  \cite[S. 130-140]{Engelhard.2004} in drei  Phasen gliedern. \\
%
 Die erste Phase umreißt den Zeitraum von 1962-1973. Der Schwerpunkt lag in der arbeitsintensiven Exportförderung der Leichtindustrie sowie dem Aufbau einer modernen physischen Infrastruktur. Es gelang Südkorea in kurzer Zeit, dass die Textilindustrie 38{\%} des Gesamtexportwerts ausmachte. Der aus dem Wohlstandsgewinn darauf folgende rasche Bevölkerungszuwachs wurde durch eine Auflage für die Familienplanung reguliert und der Ausbau der Verkehrsinfrastruktur ebnete die Basis für die folgende wirtschaftliche Entwicklung. \\
%
 So folgte in der Zeit von 1973 bis 1982 der Übergang von der Leichtindustrie zur Schwerindustrie. Trotz starker Handelsorientierung wurden Importzölle erlegt, um große Branchen, wie beispielsweise die Stahlindustrie zu schützen. Der Staat investierte in dieser Zeit 70{\%} der verfügbaren finanziellen Mittel in die schwer- und petrochemische Industrie. Jedoch zeigten sich auch große Probleme, beispielsweise stiegen die Einkommensunterschiede an. Dies war vor allem darauf zurückzuführen, dass nun eine deutlich größere Nachfrage nach qualifizierten Arbeitskräften herrschte, die einen Lohnanstieg entsprechender Branchen mit sich führte. Außerdem wurde durch den andauernden Import neuer Technologien die Entwicklung eigener Technologien vernachlässigt, was die Wettbewerbsfähigkeit minderte. \\
%
 Die Lösung dieser Schwachpunkte bildeten den Beginn der dritten Phase. Man wendete sich von der arbeitsintensiven Produktion ab und konzentrierte sich von 1980-1987 auf die Industrialisierung kapitalintensiver Investitionsgüter. Schwerpunkte stellten dabei der Ausbau der Maschinen- und  Automobilindustrie dar. Deren Exporte summierten sich auf 50{\%} der gesamten Exportmenge. Außerdem stellte diese Phase auch eine politische Wende dar, da nun ein Großteil der Wirtschaftsprozesse liberalisiert und Handelsbeschränkungen reduziert wurden. Die Kontrollfunktion des Staates wurde zudem herabgesetzt. \\
%
 Auf diese drei Phasen folgte Ende der 80er Jahre der Umschwung hin zur Demokratisierung und der Förderung von Hochtechnologiebranchen. Um diese zu erweitern konzentrierten sich staatliche und private Investitionen auf den Forschungs- und Enwicklungssektor. Durch den technologischen Fortschritt musste das Bildungswesen reformiert werden, um eine fortführende qualitative Ausbildung der Bevölkerung zu gewährleisten. Der Beitritt zur WTO und der OECD führten zu einem andauernden Abbau protektionistischer Handelsbarrieren und der Wettbewerb wurde immer stärker den Marktkräften überlassen \cite[S. 135-140]{Engelhard.2004}.\\
%
 Der enorme Aufschwung brachte jedoch nicht nur Nutznießer zu Tage, der Verlierer der Entwicklungsstrategie war in erster Linie die Bevölkerung. Eine so stark wachstumsorientierte Strategie ging nicht mit sozialer Gerechtigkeit einher. Politische Gegenströmungen wurden unterdrückt \cite[S. 111]{Engelhard.2004}.\\
%
 Auch andere offene asiatische Länder wie Taiwan, Japan und China erreichen westliches Produktionsniveau, vor allem weil sie ihre Produktivität verbessert und technologisch aufgeholt haben. Dies ist nicht nur durch massive Investitionen des Westens zu begründen. Korea hat binnen 40 Jahren eine der schnellsten sozioökonomischen Transformationen in der Geschichte der Menschheit gehabt. Die wirtschaftliche Veränderung des Landes entspricht der Entwicklung Englands von der Kolonialisierung\footnote{Als zeitlicher Rahmen dient hier die Regentschaft von George des Dritten, als die vereinigten Staaten noch britische Kolonie war.} bis heute. Erreicht wurde dieses enorme Wachstum durch den Schutz junger Wirtschaftszweige \cite[S. 20]{Lee.1999}.\\
%
Mit Wachstumsraten zwischen 8 und 9 {\%} bis 1995 ist die Entwicklung Südkoreas ein beispielhafter Aufholprozess. Es übersprang den langwierigen Prozess der technologischen Entwicklung, indem es jegliche Technologien importierte. Die relativ reichlich vorhandene qualitativ hochwertige Arbeit wurde genutzt und beschleunigte die Entwicklung.\\
%
 Der Humankapitalreichtum befähigte Südkorea sich nur auf rohstoffsparende Technologien zu beschränken und somit ihre eigenen Rohstoffe gezielt einsetzten zu können, ohne diese zwingend aus der übrigen Welt importieren zu müssen. \\
%
 Ein weiterer Erfolgsfaktor war das Intervenieren und Lenken des Staates.  Mit General Park Chung Hee übernahm 1961 das Militär die staatliche Führung Südkoreas mit einer klar formulierten Entwicklungsstrategie: "`growth first /export first"' \cite[S. 111]{Engelhard.2004}. Der Staat hatte erheblichen Einfluss auf die wirtschaftlichen Prozesse und agierte eher wie ein Unternehmen. Dazu zählten die gezielte Lenkung von Investitionen, die Aufteilung der Branchenstrukturen, Anreizregulierung oder auch die betriebliche Standortwahl der Unternehmen, um nur einige der Maßnahmen zu nennen. Diese stark wachstumsorientierte Strategie ging einher mit einer Exportorientierung. Sich dem Außenhandel zu öffnen, sollte nicht nur das eigene Wirtschaftswachstum begünstigen, sondern war außerdem notwendig, um die Vorhaben im eigenen Land zu ermöglichen. Neben Technologien mussten auch ergänzende Rohstoffe für die heimischen Industriezweige importiert werden, außerdem war das Potential des inländischen Binnenmarkt, d.h. nicht genug Käufer bzw. Nachfrager, zu gering um die Kapazitäten vollständig ausnutzen zu können. Man erhoffte sich aus dem durch Handel resultierenden Marktgrößeneffekt eine Ausnutzung der vorhandenen Kapazitäten. \\
%
 Die zentrale Rolle des Staates äußerte sich in dem Instrument der Kontrolle. Der Kreditmarkt unterlag strengen Vergabekriterien, sowie auch der Einsatz der genehmigten Gelder streng kontrolliert wurde, damit diese nicht für nicht produktive Absichten eingesetzt wurden. Eine weitere Maßnahme war trotz Handelsoffenheit der Schutz bestimmter heimischer Industrien. So wurde beispielsweise die  Automobilindustrie durch Importzölle geschützt \cite{Engelhard.2004}. \\
%
Die Beispiele verdeutlichen, das David Ricardos Theorie in der realen Welt vielfach angewandt wurde. Das prinzipielle Konzept, das dahinter stand, funktionierte zwar, jedoch waren die weitreichenden negativen Folgen nicht absehbar. Ricardos Argumente waren durch seine Arbeit zu stark an das theoretische Model gebunden. Er stellte die Welt so dar, als basierte die gesamte Wirtschaft nur auf Handel. Er berücksichtigt weder Schulden, Arbeitslosigkeit noch Geld. Er gilt als Begründer unserer heutigen Mathematisierung der Wirtschaftswissenschaft. Er lieferte Konzepte, die sich geschickt mathematisch umsetzen lassen und zeigen, dass es zu einem Gleichgewicht kommt, auch wenn es in der Realität nicht der Fall ist. Er erkannte, dass die schlechte Anwendbarkeit vor allem auf der Annahme der Vollbeschäftigung beruhte. Um sich diesem Aspekt anzunähern trat er sehr für den vermeintlichen Segen der Arbeitsfreizügigkeit ein \cite{Huther.2006}. \\
%
 Die Situation zu Lebzeiten Ricardos verdeutlichten ihm den Handlungsbedarf. Die englischen Städte des 18. Jahrhunderts waren überfüllt mit notleidenden Bauern. Dabei sollte jeder Mensch vor äußerster Not geschützt sein, denn das Armengesetz garantierte jedem ein Recht auf Unterstützung durch die Gemeinde. Dazu lieferte David Ricardo die Grundlage für ein national einheitliches System der Unterstützung bedürftiger Menschen und wird heute als einer der ersten sozialpolitischen Eingriffe des Staates gesehen. Der dortige frühere Zustand müsste mit dem heutigen Port-au-Prince der Hauptstadt Haitis vergleichbar sein. Nur, dass es dort kein Wohlfahrtssystem gibt wie in England. Vor allem Thomas Malthus hielt nicht viel von Wohlfahrtssystemen, da es den Menschen die Motivation zum arbeiten nimmt. Die Armengesetze produzierten Armut statt diese zu lindern. Sie ermöglichten dem Einzelnen, trotz finanzieller Schwierigkeiten und Grundversorgungsproblemen, zu heiraten und Kinder zu bekommen. Finanziell schlechter gestellte erhielten finanzielle Unterstützung gemäß der Anzahl ihrer Kinder. Dies war laut Malthus ein~Anreiz mehr Kinder in die Welt zu setzen, als von den Eltern ernährt werden konnten. \cite{Lin.2007} schildert weiter, dass mit Beginn des 18. Jahrhunderts Arbeitshäuser eingerichtete wurden,  in die die Armen eingewiesen wurden. Dort sollten sie auf ihre Arbeitswilligkeit hin getestet werden und ihre finanziellen Zuwendungen wurden mit Arbeitsleistung ausgeglichen. \\
%
 Bis zum Ende des 19. Jahrhunderts konnten die Arbeiter auf der Suche nach einer Beschäftigung nicht ohne Weiteres in eine andere Stadt ziehen. Es galt das Herkunftsprinzip, bei dem einem Bürger nur dann staatliche Unterstützung zustand, wenn die Personen in der Gemeinde geboren, verheiratet oder ausgebildet wurden. Das führte zu einem sehr unflexiblen Arbeitsmarkt \cite[S. 511]{Wende.2001,Hesse.2001}.\\
%
 Die Industrialisierung und das Wohlfahrtssystem führten zu ansteigendem Bevölkerungswachstum und der Zunahme der Verstädterung. Dadurch entstanden erhebliche Kosten für die Armenunterstützung, die das System ineffektiv machten. Malthus setzte sich gemeinsam mit David Ricardo für den freien Wettbewerb ein. Die Setzung der Löhne wurde der Kontrolle des Gesetzgebers entzogen. Sie waren der Ansicht, dass die öffentliche Fürsorge den Gesetzen des Marktes schadet \cite[Kapitel 4]{Baek.2010,Fischer.1972}.\\
%
 Durch ihren Einsatz wurde 1834 ein neues Gesetz zum Armenrecht erlassen, darin wurde unter Berücksichtigung der Argumente von Malthus und Ricardo über die verpflichtende Einweisung in Arbeitshäuser verfügt. Der starke Andrang führte zu deutlich verschlechterten~Lebensbedingungen in den Arbeitshäusern. Ziel der Gesetzesänderung war die Kostensenkung durch die Kürzung sozialer Zuwendungen. Jedoch waren die Zustände in den überfüllten Arbeitshäusern so schlecht, dass in den Bedürftigen die Motivation geweckt wurde, ihren Lebensunterhalt eigenständig durch Arbeit zu verdienen, um nicht länger auf das Wohlfahrtssystem angewiesen sein zu müssen. Die Arbeiter mussten eine Beschäftigung finden und das Lohnniveau wurde durch die Kräfte des Marktes bestimmt. Nach der gesetzlichen Änderung konnten sie sich auch wieder frei bewegen, da eine interessante Unterstützung nicht mehr existierte. Beide Wissenschaftler verhalfen der britischen Gesellschaft dazu eine reine kapitalistische Marktwirtschaft zu werden \cite{Wende.2001}.\\
%
Die Befreiung der Arbeitskraft führte jedoch zu weitreichenden Folgen. In Großbritannien fand eine Entwicklung weg vom landwirtschaftlichen, hin zum Industriesektor statt. Diesen~Strukturwandel unterzog sich auch China in den vergangenen 30 - 40 Jahren und zeigt noch deutlicher welche zusätzlichen Konsequenzen dies für den Arbeitsmarkt hatte.  Vor ca. 30 Jahren lebte ein Großteil der Bevölkerung auf dem Land und China war weitgehend eine bäuerliche Gesellschaft. Wenn bei einer überwiegend ländlichen, landwirtschaftlich geprägten Bevölkerung, Land das Gemeinbesitz war zum Privatbesitz gemacht wird, führt es langfristig zu einer Struktur von wenigen Großgrundbesitzern und wenigen kleinen Landbesitzern. Viele der ehemaligen Bauern besitzen gar kein Land mehr und  sind somit potenzielle Arbeiter für den Industriesektor. Die hinzugewonnenen Arbeiter machten in einem Land wie China mit seiner sehr hohen Bevölkerungszahl einen beträchtlichen Anteil aus. Dank David Ricardo und Thomas Malthus konnte sich diese Arbeitskraft auf der Suche nach einer Beschäftigung frei bewegen. Den Großteil der ehemaligen Bauern führte ihr Weg vom Land in die Städte und konnten ihre Arbeitskraft auf einem globalen kapitalistischem Markt anbieten \cite[Kapitel 1, S. 34]{Franke.2013,Menzel.2013,Reisach.1997}.\\
%
Die zunehmende Verstädterung und das gewachsene Potential an Arbeitskräften bot den westlichen Industriestaaten die Möglichkeit die Produktionsstätten in weniger entwickelte Länder auszulagern, in denen das Arbeitsangebot hoch und der Lohn somit gering war. Dies geschah auch bei General Motors. Die amerikanischen Arbeiter in Flint wurden arbeitslos, da sie im Wettbewerb mit den chinesischen und mexikanischen Arbeitern nicht mithalten konnten. In den vereinigten Staaten wiederholten sich gewisse Züge der britischen Geschichte. Der Staat Michigan entwarf eine Art Neuauflage des Armutsgesetze ganz im Stil von Malthus.\\
%
 Der Einfluss Ricardos und Malthus ist auch in der heutigen Zeit noch spürbar. Je höher die Mindestlöhne sind, desto besser können die Grundbedürfnisse befriedigt werden und desto mehr Macht bekommen die Arbeiter. Im globalen Kontext wird dies als großes Problem gesehen. \\
%
 David Ricardo starb am 11.09.1823 im Alter von 51 Jahren. In der Öffentlichkeit ist der Theoretiker kaum bekannt, dabei hat seine Lehre die globale Wirtschaftsgeschichte nachhaltig beeinflusst. Ricardo und Malthus hatten großen Anteil an einer Umstrukturierung der Gesellschaft entsprechend der Logik des Marktes. Ihre Theorien und Ansichten schufen Reichtum und Armut gleichermaßen \cite{Heilbroner.2011}.\\
%

Die eingangs gestellte Frage nach den Motiven für Handelsbeziehungen lässt sich zusammenfassend als ein Problemlösungsansatz der damaligen Zeit sehen bzw. beantworten. Die angeführten Beispiele zeigen, dass in vielen Fällen die Anwendung der Theorie Ricardos und Malthus auf wirtschaftliche und politische Interessen zurückzuführen sind. Der Kerngedanke zielte jedoch auf die Erhöhung der Wohlfahrt aller beteiligter Länder ab. Ihnen schwebte eine ausgeglichene Gesellschaft mit geringen Standesunterschieden vor, ein noch immer zeitgemäßes Ideal im andauernden Prozess der Globalisierung.
%
\section{Grundlagen und Handelstheorien}\label{Handelstheorien}
Die Diskussion über den aktuellen Nutzen und die zukünftig möglichen Entwicklungspotenziale durch Freihandel wurde im vorherigen Kapitel \ref{sec:Globalisierung} sowohl anhand historischer als auch aktueller Beispiele bereits ausführlich vorgestellt. \\
%
 Dabei konnte festgestellt werden, dass für die Öffnung eines Landes  verschiedenste Argumente sprechen, die sich zwar unterschiedlicher Analysen bedienen und dabei aber die Motive, Blickwinkel und Intensionen der jeweiligen Betrachter berücksichtigen. In diesem Zusammenhang stellte sich aber die Frage nach einem richtigen Maß für die jeweilige Ausprägung von Freihandel bzw. Protektionismus. Ab wann überwiegen die Nachteile bzw.  bis wann kann der Nutzen diese aufwägen? Globalisierungsbefürworter gewichten eine Handelsliberalisierung stärker als beispielsweise Politiker, die einerseits innenpolitische Probleme lösen müssen, andererseits die Interessen derer Vertretern, die ihnen zu einer Wiederwahl verhelfen.\\
%
 Zunächst wird auf der Ebene der Wohlfahrtsanalyse das Effizienzargument für Freihandel angeführt, weil der durch den Außenhandel entstandene Wohlfahrtsanstieg durch Protektionismus gemindert werden würde. Demnach wäre es effizient auf Eingriffe zu verzichten und den Marktkräften zu vertrauen.
Handelt es sich jedoch um ein ökonomisch großes Land, dann kann theoretisch die Wohlfahrt darüber hinaus durch protektionistische Maßnahmen gesteigert werden. Dies besagt z.B. das Terms of Trade Argument und zeigt, dass dies bei einem Optimalzoll zwar zutrifft, in der Realität aber selten Anwendung findet \cite{Ventura.1997,  Acemoglu.2002}.\footnote{Dies ist zum Einen dadurch bedingt, dass er nur dann wohlfahrtssteigernd wirkt, wenn sich die übrige Welt nicht widersetzt und ebenfalls den Handel beschränkt. Zum anderen mangelt es häufig an der politischen Durchsetzbarkeit.}\\
%
 Die Intention des Staates den Handel einzuschränken kann auch dadurch bedingt sein ein bestehendes inländische Marktversagen ausgleichen zu wollen. Dies ist meist dann der Fall, wenn ein zusätzlicher nicht erfasster Nutzen, der aus der heimischen Produktion hervorgeht, den gesellschaftlichen Gesamtnutzen steigert \cite[Kapitel 10]{Krugman.2015}.\\
%
Neben zusätzlicher Wohlfahrt kann es durch Freihandel noch zu weiteren Gewinnen durch die Sondierung produktiver und weniger produktiver Unternehmen durch den Wettbewerbseffekt kommen. Der erhöhte Wettbewerb setzt Anreize innovativ tätig zu sein und verdrängt weniger produktive Unternehmen vom Markt, so dass lang\-fri\-stig die volkswirtschaftliche Produktivität steigt.\footnote{Der Wettbewerbseffekt wird in Kapitel \ref{WirkungHandel} ausführlich diskutiert.} \\
%
Der Wettbewerbseffekt stärkt zwar die davon profitierenden größeren Unternehmen, jedoch wird dieser Effekt auch häufig als Argument gegen die Öffnung eines Marktes verwendet. Da durch Außenhandel die weniger effiziente Unternehmen vom Markt verdrängt werden, befürchten Unternehmen aus technologisch weniger weit entwickelten Ländern, nicht zu unrecht dem erhöhten Wettbewerbsdruck nicht standhalten zu können. Die Vielfalt an klein- und mittelständischen~Unternehmen sinkt. Jedoch würden nicht nur einzelne Unternehmen unten den Konsequenzen leiden, sondern ganze Branchen eines Landes könnten betroffen sein.\\
%
Das dritte Argument für Freihandel betrifft die politische Durchsetzbarkeit. So scheitern  Handelshemmnisse selbst dann schon, wenn  es politisch durchaus sinnvoll ist den Handel einzuschränken. Letztlich sichert Freihandel dem Politiker die Wiederwahl und ist häufig der Weg des geringsten Wiederstandes. \\
%
Somit bedingt die Gunst des Freihandels bei den potentiellen Wählern die politische Durchsetzbarkeit. Dabei tritt das Problem im Rahmen der politischen Ökonomie auf, denn häufig werden die Anliegen mächtiger Interessensgruppen eher vertreten, als die dem Gemeinwohl dienlichen. Auch werden tendenziell in technologisch relativ weiter entwickelten Ländern eher Bedenken bezüglich einer Öffnung angeführt, hinsichtlich möglicher Einkommensdefizite. Wird beispielsweise Handel mit arbeitskräfteintensiven Gütern aus weniger weit entwickelten Volkswirtschaften betrieben, kann dies zu einer Anpassung des Lohnniveaus und letztlich zu einem geringeren Lebensstandard führen. Diese Option weckt das Begehren nach protektionistischen Maßnahmen um diese Einkommensanpassung zu mindern, bzw. zu verhindern \cite[Kapitel 1]{Krugman.2015}.\\
%
Auch wenn durch Außenhandel auf gesamtwirtschaftlicher Ebene die Wohlfahrt ansteigt, führt er innenpolitische Probleme herbei. Dazu zählt auch das Verteilungsproblem des Einkommens, weil nicht jede Gruppe gleichermaßen begünstigt bzw. einige sogar benachteiligt werden. Die Einkommensschwerpunkte verlagern sich beispielsweise von den Arbeitnehmern zu den Kapitaleignern. In diesem Fall können durch staatliche Regulierung Wohlfahrtsgewinne zugunsten schlechter gestellter Bevölkerungsgruppen umverteilt werden \cite{Dixit.1980}. Außerdem können importkonkurrierende Branchen, in denen spezifische Faktoren eingesetzt werden, unter Außenhandel leiden, da es nur sehr schlecht bis gar nicht möglich ist diese Faktoren in anderen Bereichen einzusetzen.
Die Argumente basieren auf theoretischen Modellen und empirischen Überprüfungen \cite[Kapitel 1]{Krugman.2015}.
%
\subsubsection{Außenwirtschaftstheorien}
Grundsätzlich lässt sich in Bezug auf die derzeit bekannten Außenwirtschaftstheorien feststellen, dass sie sich überwiegend mit wirtschaftlichen Interaktionen zwischen den Volkswirtschaften befassen.
Entsprechend den verschiedenen Erklärungsansätzen nach denen die Gründe, warum Länder miteinander Handel betreiben, recht unterschiedlich sind, werden im folgenden diese möglichen Gründe vorgestellt.
%
Dabei liegt der Schwerpunkt weniger auf intertemporalen Entscheidungen, da davon ausgegangen wird, dass alle Wirtschaftsteilnehmer zu jedem Zeitpunkt alles haben können. Gerade hinsichtlich der Koordination von Produktionsprozessen ist eine intertemporale Optimierung nicht notwendig, da die Güter und auch die Produktionsfaktoren jederzeit aus der übrigen Welt bezogen werden können.
%
\subsubsection*{Ricardo - Technologieunterschiede}
Die Darlegung der Beweggründe für ökonomischen Handel liefert einen kurzen Überblick über die Hauptströmungen der Handelstheorien, denen die Leitfrage aller traditionellen Handelstheorien zugrunde liegt: Welches Land exportiert welches Gut? \\
%
Die klassische Theorie des Außenhandels wurde vor allem durch David Ricardos Arbeit von 1817 geprägt. Seine Idee basiert auf dem gleichen Konzept, dass Robert \cite{Torrens.1815} in seinem Aufsatz über den Getreidehandel verfasste. Dabei liegt der hier angeführte Grund für Außenhandel in der Verschiedenheit der Technologien und den damit verbundenen Produktivitätsunterschieden. Das Ursprungsmodell beschreibt den Handel zwischen den beiden Ländern Portugal und England mit den Gütern Wein und Tuch. Produziert werden beide Güter nur mit dem Einsatzfaktor Arbeit.  Allerdings unterscheiden sich die jeweils notwendigen Einsatzmengen für die Produktion eines Gutes, bedingt durch den Einsatz unterschiedlicher Produktionstechnologien. Somit führen die Produktivitätsunterschiede zwischen den Ländern zu unterschiedlichen Produktionskosten. Dieser komparative Kostenvorteil beschreibt den relativen Vorteil eines Landes, der durch den Einsatz verschiedener Technologien zu Stande kommt und stellt hier den Grund für Außenhandel dar. Dabei stellen sich die teilnehmenden Wirtschaftssubjekte durch die Aufnahme von Außenhandel besser, weil jedes Land immer einen komparativen Vorteil in irgendeinem Sektor hat \cite{Ricardo.1817}. \\
%
 Ricardo widersprach damit den Annahmen Adam Smiths, dass absolute Vorteile einer Ökonomie zwingend notwendig sind, damit absolute Arbeitsteilung, also Handel im weiteren Sinne, für beide Seiten sinnvoll ist.\\
%
Kann ein Land in allen Branchen effizienter produzieren, dann geht dies nicht zwangsläufig mit einer kostengünstigeren Produktion einher, denn vergleicht man die Opportunitätskosten der beteiligten Länder in den entsprechenden Branchen, dann zeichnet sich allein schon dadurch der absolute vom komparativen Vorteil ab. So können weniger effiziente Länder schon durch niedrigere Löhne ihre Konkurrenzfähigkeit erhalten und zu geringeren Opportunitätskosten produzieren. Somit ist ihr komparativer Vorteil dann durch den produktiveren Einsatz des Faktors Arbeit bedingt, also durch günstige Arbeitskraft. Dieser Zusammenhang beschreibt den Unterschied zwischen dem absoluten und dem komparativen Vorteil \cite{Ricardo.1817}.\\
%
Auf Ricardos grundlegende Arbeit "`On the Principles of Political Economy and Taxation{\dq} von  (1817) stützen sich eine Vielzahl von empirischen Untersuchungen und Modellvariationen, von denen hier nur einige wenige vorgestellt werden.\\
%
Bei der Variation des Modells des komparativen Vorteils von \cite{Dornbusch.1977} handelt sich um eine vereinfachte Version des Ricardo Modells. Jedoch werden nicht nur zwei Güter produziert und gehandelt, sondern sehr viele Güter, sodass sich ein Kontinuum an handelbaren Gütern ergibt. Dies führt Ricardos These mit der realen Welt ein wenig näher zusammen.\\
%
So ist ein Vergleich der Produktivitäten der USA mit denen von Großbritannien Gegenstand vieler empirischer Untersuchungen, in denen die Theorie Ricardos dahingehend bestätigt wurde, dass die theoretischen komparativen Vorteile mit den tatsächlichen übereinstimmen \cite{MacDougall.1952,Stern.1962,Balassa.1963}.\\
%
Ebenfalls empirisch ist die Arbeit von \cite{Golub.2000}. Sie untersuchen den Zusammenhang zwischen den Verhältnissen der relativen Produktivitäten und bilateralen Handelsstrukturen der USA. Dabei stellen sie fest, dass die Struktur nicht komplett durch den komparativen Vorteil erklärt werden kann, aber diese dennoch in Teilen erklärt.\\
%
Beschränkt man die Betrachtung des Handels ausschließlich auf Industrieprodukte, dann liegt der Grund für Handel mit diesen in der technologischen Ausstattung der Länder bzw. dem technischen Entwicklungsstand eines Landes. Empirische Beobachtungen, die Aufschluss über die Handelsstruktur geben, bestätigen ebenfalls Ricardos Aussagen \cite{Dosi.1988}.\\
%
Die Hauptaussage der Theorie, dass jedes Land bei der Produktion eines Gutes einen komparativen Vorteil hat, klingt gerade für weniger weit entwickelte Länder vielversprechend. Auch ein Vergleich, mit beispielsweise den USA, betont die relativ schlechte Situation dieser Länder, aufgrund fehlender absoluter Vorteile. Jedoch ändert sich dieses Bild sobald die komparativen Vorteile hinzugezogenen werden. Diese können auf unterschiedliche Argumente zurückgeführt werden. Dazu zählen Faktoren wie das Klima, natürliche Ressourcen, besondere akkumulierte Fähigkeiten, Überschussangebote an günstigen Arbeitskräften oder auch gezielt hervorgerufene komparative Vorteile durch staatliche Förderungsmaßnahmen eines bestimmten Sektors. Den komparativen Vorteil können entweder Faktoren bedingen, die relativ fest und über die Zeit unveränderlich sind, oder auch andere Faktoren, die sich erst noch über die Zeit entwickeln werden.\\
%
Diesen Aspekt greift auch \cite{Helpman.2011} auf. Demnach ist es einzelnen Unternehmen möglich einen komparativen Vorteil für ein Land zu generieren. Dies zeigt, dass mikroökonomische Entscheidungen beträchtlichen Einfluss auf das makroökonomische Gleichgewicht haben können. In dem ausführlich dargelegten Modell in Kapitel \ref{Papier1} wird ein ähnlicher Ansatz verfolgt. Die technologischen Entwicklungen einzelner Unternehmen erhöhen nicht nur die Produktivität eines Landes, sondern im offenen Modell sogar die der übrigen Welt.\\
%
Kritiker Ricardos bezeichnen seine Theorie als überholt, da er die Produktionsfaktormobilität und den~Technologietransfer nicht berücksichtigt \cite{Irwin.2009}. Allerdings wird in der herrschenden Meinung die Ansicht vertreten, dass seine Hauptaussagen auch heute immer noch aktuell sind.
%
\subsubsection*{Heckscher-Ohlin - Ausstattungsunterschiede}
%
Ein weiteres Modell geht davon aus, dass Handel auch dann vorteilhaft ist, wenn verschiedene Länder zwar die gleiche Technologie verwenden, sich aber in ihrer Ausstattung mit Produktionsfaktoren unterscheiden. Die Vertreter dieser neoklassischen Theorie des Außenhandels sind Eli Filip Heckscher und Bertil Ohlin, die Begründer des nach ihnen benannten Heckscher-Ohlin-Modells. Ausgehend von technologisch ähnlichen oder gleichen Ländern, stellten sie einen komparativen Preisvorteil bei Volkswirtschaften fest. Dabei führt die Aufnahme von Freihandel zu einer Spezialisierung des gesamtwirtschaftlichen Produktionsvolumens, hin zu einem Gut. Genau zu dem Gut, bei dem der bei der Produktion relativ reichlicher vorhandene Produktionsfaktor intensiver genutzt wird. Dieses Faktorproportionentheorem ist der Kern des Heckscher-Ohlin-Modells und veranschaulicht welche Handelsstruktur sich bilden wird.\\
%
Das Heckscher-Ohlin-Modell wurde erstmals von \cite{Jones.1965} algebraisch formuliert und liefert damit den Ausgangspunkt zahlreicher Modellvarianten \cite{Davis.2001,Trefler.1993,Deardorff.1984}\footnote{Einen weiteren allgemeinen Überblick über die Außenhandelstheorien, wie das Faktorproportionentheorem liefern \cite{Jones.1984} im "`Handbook of International Economics"'.}.\\
%
\cite{Leontief.1953} beschäftigte sich als einer der Ersten mit der empirischen Überprüfung des Heckscher-Ohlin-Modells. Er zeigte am Beispiel der USA, das relativ reichlicher mit Kapital ausgestattet ist, dass dort nicht die Handelsstruktur besteht, die das Faktorproportionentheorem vorhersagt.  Die Handelsströme der USA sind überwiegend durch relativ arbeitsintensive Exporte und kapitalintensive Importe geprägt \cite{Leontief.1953}. Diese Ergebnissen widerlegten schließlich die Theorie von Heckscher und Ohlin und wurde als das Leontief Paradoxon bekannt. Für weitere industrialisierte Länder konnten ähnliche Ergebnisse belegt werden \cite{Gruber.1970,Maskus.1985}.\\
%
\cite{Trefler.1993} widerspricht dem Leontief Paradoxon und zeigt anhand einer modifizierten Variante des Heckscher-Ohlin-Modells, dass dieses bestätigt werden kann, sofern Produktivitätsunterschiede zwischen den beteiligten Ländern zugelassen werden. Ebenso widerlegt auch \cite{Leamer.1980} Leontiefs Untersuchungen, indem er einen Test anwendete, der auf dem Vergleich der Faktorintensitäten der produzierten und konsumierten Güter gründet. Die Allgemeingültigkeit wurde jedoch nicht belegt, da das Leontief-Paradoxon nur in bestimmten Jahren Anwendung fand \cite{Stern.1981}.\\
%
Auch \cite{Davis.1995} beschäftigen sich mit der Anwendbarkeit der Theorie. Sie vertreten die Meinung, dass trotz fehlender empirischer Bestätigung der Theorie von Heckscher und Ohlin der Kerngedanke und das Ergebnis des Modells anwendbar ist. Ähnlich wie \cite{Trefler.1993} modifizieren sie es, indem sie die Grundannahmen anpassen und erhalten für die Daten Japans die Theorie stützende Ergebnisse. Dabei sehen sie zum einen von der Annahme ab, dass die Technologien für die betrachteten Länder gleich sein sollten und sich somit nicht ausschließlich durch ihre Ausstattung unterscheiden. Zum anderen analysieren sie die Produktions- und Konsumstruktur separat, ohne die direkten Handelsdaten zu nutzen.  Beides zusammen führt dazu, dass sie das Heckscher-Ohlin-Modell empirisch für Japan bestätigen können.\\
%
Einen anderen Ansatz wählen \cite{Bond.}, die eine dynamische Version des Heckscher-Ohlin Modells graphisch lösen und stellen dabei neben der Existenz, die Dynamik und Stabilität möglicher Gleichgewichte dar.\\
%
Es ist auch durchaus üblich verschiedene Ansätze miteinander zu kombinieren, wodurch der Handel zwischen Ländern mit ähnlicher Ressourcenausstattung erklärt werden kann. Dafür wurde die Idee des komparativen Vorteils Ricardos in das Heckscher-Ohlin Modell implementiert. Bei ähnlichen Faktoreinsatzverhältnissen in ähnlichen Ländern ist der technische Unterschied der Länder von Bedeutung und bestimmt die Handelsstruktur \cite{Davis.1995b}.\\
%
Eine Kombination mit dem Ansatz der Neuen Handelstheorien bestätigt die Faktorproportionentheorie, sowie das Rybczynski Theorem weitestgehend, vor allem jedoch für humankapitalreiche Länder. Diese Erweiterung des Heckscher-Ohlin-Modells nahm \cite{Romalis.2004} vor, indem er es um Transportkosten und den Ansatz der monopolistischen Konkurrenz nach \cite{Krugman.1980} erweiterte.\\
%
\subsubsection*{Faktorpreisausgleichstheorem}
%
In einem engen Zusammenhang mit dem Faktorpropotionentheorem bzw. Heckscher-Ohlin-Theorem steht das Faktorpreisausgleichstheorem bzw. Stolper-Samuelson-Theorem. Nachdem zunächst die Reaktionen auf den Gütermärkten betrachtet wurden, werden hier die sich ergebenden Konsequenzen auf den Faktormärkten dargelegt. Das Faktorpreisausgleichstheorem geht auf die Arbeit von \cite{Samuelson.1941} zurück, in der sie die Wirkung durch die Aufnahme von Handel auf die Faktorpreise zeigen. Dabei greifen sie die Idee ihres Kollegen \cite{Ohlin.1933} auf, der ebenso wie \cite{Heckscher.1919}, den Zusammenhang zwischen der Handelsstruktur und der Resourccenausstattung eines Landes thematisiert. Das daraus resultierende Heckscher-Ohlin-Theorem besagt, dass ein Land stets das Gut exportieren wird, das den relativ reichlicher vorhandenen Produktionsfaktor intensiver bei der Herstellung verwendet. \\
%
 Samuelsons weiterführenden Überlegungen basieren auf den beiden Regionen USA und Europa, die sich seinerzeit hinsichtlich ihrer Bevölkerungsdichte und dem verfügbaren fruchtbaren Boden deutlich unterschieden. Demzufolge werden durch Handel die relativ hohen Löhne im eher dünn besiedelten Amerika sinken und der Bodenpreis in Europa  wird ansteigen. Somit werden sich langfristig die Faktorpreise auf dem Weltmarkt angleichen. Es ist dann in der theoretischen Welt nicht mehr kostengünstiger Produktionsfaktoren zu im- oder exportieren, um diese dann weiter zu verarbeiten, wenn durch den Preisausgleich ein direkter Güteraustausch zum gleichen Ergebnis führt \cite{Samuelson.1948}.\\
 %
Ein weiteres Papier von \cite{Samuelson.1949} knüpft an seine vorherige Arbeit an und beschäftigt sich wieder mit dem Faktorpreisausgleichstheorem. Auch hier formuliert er die Gedanken Ohlins formal und bestätigt erneut das Stolper-Samuelson-Theorem. \\
%
Dem Zusammenhang zwischen der Faktormobilität und Handel widmet sich \cite{Mundell.1957} in seiner theoretischen Arbeit. Dabei geht er zunächst von immobilen Produktionsfaktoren aus und zeigt, dass mit der Zunahme protektionistischer handelseinschränkender Maßnahmen die Motivation zur Mobilität der Faktoren ansteigt. Weiterhin kommt er zu dem umgekehrten Ergebnis, dass mit der Einschränkung der Faktormobilität der Handel mit Gütern zunimmt. Somit bestätigt auch er, dass Faktormobilität und Gütermobilität substituierbar sind \cite{Mundell.1957}.\\
%
Sobald jedoch ein Modell von der Grundannahme, die der gleichen bzw. ähnlichen Technologien, abweicht, werden sich die Faktorpreise nicht mehr vollständig angleichen \cite{Jones.1970,Davis.2001}.\footnote{Interessant ist hier vor allem der Aspekt, dass in empirischen Überprüfung verschiedener Handelstheorien festgestellt wurde und, dass vollkommene Spezialisierung, tendenziell realistischer ist, bzw. häufiger vorkommt, als Autarkie oder der hier thematisierte Faktorpreisausgleich \cite{Cunat.2001}.}\\
%
Nur indirekt mit dem technischen Entwicklungsstand beschäftigt sich \cite{Trefler.1993}. Er widerspricht zunächst dem Leontief-Paradoxon und zeigt dann anhand einer modifizierten Variante des Heckscher-Ohlin-Modells, dass dieses bestätigt werden kann, sofern Produktivitätsunterschiede zwischen den beteiligten Ländern zugelassen werden. Dabei handelt es sich um einen bedingten Faktorpreisausgleich. In dem ursprünglichen Stolper-Samuelson-Theorem gleichen sich die Faktorpreise, wie der Lohn $w$ an. Bei Treflers bedingter Variante steht der Lohn jedoch im Verhältnis zum technologischen Wissen\footnote{Das technische Wissen ist hier durch den Parameter $A$ gekennzeichnet.}, somit gleicht sich nur das Verhältnis $w/A$ beider an.\bigskip\\
%
Für beide aufeinander aufbauenden Theorien gilt: Spezialisierung und Handel lohnen sich umso mehr, je verschiedener die Handelspartner sind. Der interindustrielle Handel, erklärt durch das Heckscher-Ohlin oder Ricardo Modell, nimmt zu, je unterschiedlicher sich die Länder hinsichtlich ihrer Ausstattung sind. Wohingegen intraindustrieller Handel auf Skaleneffekte bei monopolistischer Konkurrenz zurückzuführen ist. Dabei sind die Handelsbeziehungen umso intensiver je ähnlicher die Länder sich einander sind \cite{Dosi.1993}. Dieser Erklärungsansatz wird im Rahmen der Neuen Handelstheorien behandelt.
%
\subsubsection*{Krugman - interne Skalenerträge}
Eine weitere Handelstheorie basiert auf dem Ansatz der internen Skalenerträge von Paul \cite{Krugman.79}. Die Arbeit von Robert \cite{Solow.1956} hatte indirekten Einfluss auf seine Außenhandelstheorien. In Solows Theorie über unvollständigen Wettbewerb wurde eine sehr realistische Welt dargestellt, in der Unternehmen durch steigende Skalenerträge Gewinne erwirtschaften können. Denn auf eine große Produktionsmenge können die fixen Kosten stärker umgelegt werden. 
Der Grundgedanke der Größenvorteile, die internen Skalenerträge, geht auf die Ideen Ricardo und Smith zurück. Danach führt das Konzept der Arbeitsteilung zu fallenden Stückkosten, aufgrund der Größenvorteile. Demzufolge ist es für die Unternehmen und die gesamte Volkswirtschaft lohnend sich zu spezialisieren und Handel zu betreiben, und zwar unabhängig von Ausstattungs- oder Technologieunterschieden. Interne Skalen\-erträge führen jedoch zu einer Marktmacht, die nicht mit vollkommenem Wettbewerb vereinbar ist. Diese Bedingung setzt Krugman mit Hilfe des Modells von \cite{Dixit.1977} um, die ein formales Modell zur monopolistischen Konkurrenz entwickelt hatten.\\
%
Hinzu kommt ein weiterer Punkt, der von Krugman berücksichtigt wurde. Die Produktvielfalt ist den Unternehmungen eher unwichtig, denn bei ihnen steht die Massenproduktion im Vordergrund. Aus Sicht der Konsumenten gilt allerding das Umgekehrte: Sie bevorzugen eine möglichst große Auswahl und legen Wert darauf, möglichst viele verschiedene Produkte zu haben und ihnen ist dies wichtiger, als von einem einzigen Produkt eine große Menge zu erhalten.\\
%
\cite{Krugman.79} zeigt, dass der durch Handel induzierte Marktgrößeneffekt die Bedürfnisse beider befriedigen kann. Bei den Unternehmen entsteht durch den Zugewinn des ausländischen Marktes eine größere Nachfrage, für den nun ebenfalls produziert werden kann und die Konsumenten können durch ausländische Anbieter ein vielfältigeres Angebot nutzen. \\
%
Lohnender Außenhandel basiert aber in diesem Fall nicht auf dem klassischen Argument des Produktivitätsvorteils, sondern zeigt hier auf warum einander ähnliche Industrieländer miteinander handeln und machen zudem auch deutlich warum sie dies  gerade innerhalb derselben Branchen tun. Krugman lieferte damit die wirtschaftstheoretische Erklärung für die Handelsströme des Europäischen Binnenmarktes.\\
%
Mit Hilfe der bisherigen theoretischen Modelle konnten allerdings einige der bis hier angeführten empirischen Beobachtungen noch nicht zutreffend vorhergesagt werden, denn die Handelsmodelle von Ricardo und Heckscher-Ohlin reichten nicht aus, um die derzeitige weltweite Handelsstruktur vollständig erklären zu können. So wurde weder der Außenhandel zwischen den sich ähnelnden Industrieländern begründet, noch die Möglichkeit der Gütervielfalt als Wohlfahrtsgewinn wahrgenommen. Diese beiden Erklärungsdefizite, Größenvorteile und Produktdifferenzierung sowie der damit einhergehende unvollkommene Wettbewerb wurden bereits von \cite{Balassa.1967} sowie \cite{Grubel.1967,Grubel.1970} als Kernbestandteile der sogenannten Neuen Handelstheorien angedeutet. Krugman gab dem Erklärungsansatz der aufkeimenden Neuen Wachstumstheorie in seiner Arbeit von 1980 einen formalen Rahmen.
Sein Ansatz begründet damit den Handel zwischen Ländern, die sich nicht drastisch unterscheiden. In den bisherigen Theorien wurde der Austausch von unterschiedlichen Gütern zwischen verschiedenen Ländern erklärt. Es handelte sich dabei um interindustriellen Handel. In diesem Ansatz geht es um die Erklärung von Handel mit ähnlichen Gütern zwischen ähnlichen Ländern, dem intraindustriellem Handel. \\
%
 Eine weitere Neuerung ist die Annahme bezüglich der Präferenzen der Konsumenten. Nicht mehr die absolute Gesamtmenge von~Gütern steht im Vordergrund, sondern deren Vielfalt. Unter der Voraussetzung, dass alle Güter den selben Preis haben, möchten die  Nachfrager eher so viele unterschiedliche Güter wie möglich beziehen, statt ausschließlich ein Gut zu konsumieren. \\
%
 In den Neuen Handelstheorien lässt sich keine eindeutige Handelsstruktur zuordnen.\footnote{Als Ausnahme gelten hier die sogenannten Nord-Süd Modelle, die Wachstum und Handel miteinander kombinieren. In dieser Modellart wird Handel zwischen der Region des relativ weniger weit entwickelten Süden mit dem relativ weit entwickelten Norden beschrieben. Diese Einteilung geht auf die Beobachtung zurück, dass auf der Nordhalbkugel ein Großteil der entwickelten bzw. industrialisierten Länder zu finden ist, wohingegen auf der Südhalbkugel viele der weniger weit entwickelten Länder liegen. Dabei muss unter anderem von den Pazifikstaaten Australien und Neuseeland abstrahiert werden. Aus dieser regionalen Aufteilung bestimmt sich die Handelsstruktur.  Der weniger weit entwickelte Süden importiert die neu entwickelten Güter \cite{Grossman.1991a,Krugman.1990}.
 %
Dieser Modellaufbau zeigt wie der technische Fortschritt in die Neuen Handelstheorien integriert werden kann. 
 } Aufgrund der Ähnlichkeit der Länder wird sich diese durch Zufall ergeben. Dabei erhöht sich die Produktvielfalt aller beteiligter Länder durch Außenhandel. Die weltweite Nachfrage nach einem Gut ist dann so groß, dass die sich bei der Produktion ergebenden Größenvorteile die Produktionskosten pro Stück reduzieren und das Gut günstiger angeboten werden kann. Die internen Skalenerträge können ausgenutzt werden und ermöglichen eine Spezialisierung auf einige wenige Güter. Die absolute Anzahl der Produkte auf dem Weltmarkt ist zwar geringer, als die Summe aller im Autarkiefall, jedoch besteht eine höhere Produktvielfalt in allen beteiligten Ländern. Dadurch steigt die Wohlfahrt, weil die Konsumenten die Vielfalt der Güter schätzen. Krugmans Theorie hebt die Rolle großer heimischer Märkte als künftige aufstrebende Exportzweige hervor. Dabei profitieren alle beteiligten Ländern von internen Skalenerträgen und es ist wirtschaftlich und wohlfahrtstheoretisch sinnvoll sich zu spezialisieren und miteinander Handel zu betreiben \cite{Krugman.79,Krugman.1983,Melvin.1969}.\\
%
Die Neuen Handelstheorien unterscheiden sich von den bisherigen der Neoklassik dahingehend, dass die grundlegenden Bedingungen, wie die Voraussetzung des vollkommenen Wettbewerbs und die Annahme über die Homogenität der Güter nicht zwingend Gültigkeit finden und nur höchstens eine von beiden Voraussetzungen noch zutrifft. Weitere Charakteristika sind zum einen der Erklärungsansatz des intra-industriellen Handels und zum anderen die Möglichkeit der Einbeziehung von steigenden Skalenerträgen. \\
%
In einem nachfolgendem Papier \cite{Krugman.1979b} formuliert Krugman ein weiteres Handelsmodell, dass eine Kombination aus dem Ansatz von Hecker-Ohlin und einem intrasektoralen Ansatz ist, der mit steigenden Skalenerträgen einhergeht. Dabei hinterfragt er, welches Handelsmuster sich ergibt, wenn sich Länder zwar ähneln, sich aber dennoch in ihrer Ausstattung unterscheiden. Je ähnlicher sich Länder  auch hinsichtlich ihrer Ausstattung sind, desto eher ergibt sich die Handelsstruktur gemäß dem Ansatz der Skaleneffekte \cite{Krugman.79}.\\
%
Dies verdeutlicht im allgemeinen, dass interindustrieller Handel und intraindustrieller Handel  nicht komplett voneinander getrennt werden sollten, denn es besteht ein Zusammenhang dergestalt, dass je ähnlicher sich Länder werden, desto eher entwickelt sich intraindustrieller Handel \cite{Krugman.1981}. Mit der Entwicklung eines Landes ändert sich der Grund für Handel.\\
%
Eine zusätzliche Modellerweiterung berücksichtigt nun auch die Transport\-ko\-sten und zeigt dadurch welche Wirkung Zölle und politische Eingriffe haben können \cite{Krugman.1980}.\footnote{Weitere Modellvariationen und theoretische Arbeiten, die zu den Neuen Handelstheorien zählen liefern zum Beispiel \cite{Grossman.1991b}.
Bestätigt wird der Ansatz Krugmans durch zahlreiche empirische Untersuchungen \cite{Antweiler.2002}.}\\
\cite{Lancaster.1980} analysiert das Ausmaß von Handelsvolumen, die durch monopolistische Konkurrenz bedingt sind. Auch wenn Länder hinsichtlich Technologie und Ausstattung identisch sind, jedoch die Marktform der monopolistischen Konkurrenz vorliegt, handeln sie intraindustriell miteinander. Er vergleicht jetzt das hypothetische Handelsvolumen durch einen komparativen Vorteil mit dem des möglichen  intraindustriellen Handels und kommt zu dem Ergebnis, dass das Volumen deutlich höher ist, wenn sich die Länder nicht zwingend ihrer komparativen Vorteile spezialisieren.\\
%
Im späteren Verlauf der Arbeit wird das in Kapitel \ref{Papier2} erörterte Modell dem Ansatz Ricardos folgen. Anschließend in Kapitel \ref{Papier1} wird Handel durch Ausstattungsunterschiede nach Heckscher-Ohlin begründet.
%
\section{Wirkung von Handel auf Wachstum}\label{WirkungHandel}
Die Ansätze der Neuen Handelstheorien haben gezeigt, dass die Forschungszweige Handel und Wachstum eng miteinander verbunden sind. In diesem Rahmen wurden immer mehr Faktoren in die Modelle implementiert, die erst durch Außenhandel in ein Land kommen und dann langfristig Einfluss auf das Wachstum der Volkswirtschaft haben. Die Handelsgewinne beeinflussen das ökonomische Wachstum und verdeutlichen die Bedeutung des Freihandels für den Entwicklungsprozess eines Landes. Zu den Hauptvertretern dieser zusammenführenden Ansätze zählen Gene Grossman, Elhanan Helpman und Alwyn Young.\\
%
Sie beschreiben die dynamischen Effekte des internationalen Handels auf das Wirtschaftswachstum \cite{Young.1991,Grossman.1995}. Dabei kann grundsätzlich zwischen exogenen Wachstumsmodellen unterschieden zwerden, die den Handel implementiert haben \cite{Dixit.1980,Ethier.1982, Krugman.1979ab,Krugman.1981,Lancaster.1980} und endogenen Wachstumsmodellen offener Volkswirtschaften \cite{Dinopoulos.,Feenstra.,Grossman.1989a,Grossman1989b.,Grossman.1990d,Grossman.1991c, Krugman.1990,Segerstrom.1990,Young.1991,Backus.} unterschieden werden. Die Hauptergebnisse der bisherigen endogenen Wachstumsmodelle konnten auch in Verbindung mit Handel bestätigt werden \cite{vanLong.1997}. 
%
\cite{Atkeson.2000} sowie \cite{Cunat.2001} kombinieren den Handel nach dem Heckscher-Ohlin-Model mit einem Wachstumsmodell. 
%
Bei der Kombination der Wachstumsökonomie mit den Handelstheorien, gibt es zwei mögliche Betrachtungsweisen. Zum einen wird die Wirkung von Außenhandel, also der Offenheit eines Landes, auf das Wirtschaftswachstum untersucht. Zum anderen wird der Einfluss wachstumsstimullierender Faktoren, wie beispielsweise der technische Fortschritt, auf die Handelsstruktur, das Handelsvolumen oder die Terms of Trade\footnote{Die Wirkung des technischen Fortschritts auf die Terms of Trade hängt in erster Linie von der Art des technischen Fortschritts ab und in welchem Sektor dieser angewendet wird. So würde beispielsweise ein arbeitsvermehrender technischer Fortschritt in dem relativ arbeitsintensiven Importsektor zu einem Anstieg der Term of Trade führen und das innovierende Land besser stellen \cite{Gandolfo.1998}.} analysiert. Dabei werden in erster Linie die Abweichungen und Veränderung der genannten Größen in bereits offenen Volkswirtschaften ermittelt, wohingegen bei der zuerst angeführten Betrachtungsweise erstmalig eine Handelsstruktur mit einem dazugehörigen Handelsvolumen entsteht und diese neuen Wechselwirkungen das Wachstum beeinflussen. Der Schwerpunkt liegt auf der Analyse von offenen Wachstumsmodellen, bei denen die Entwicklung und das Wachstum eines Landes untersucht wird.
%
\subsection{Effekte des Außenhandels}\label{Effekte Handel}
Die in der Forschung vertretenen Herangehensweisen der Wirkungsmechanismen von Außenhandel gehen auf die Unterscheidung der Handelsgewinne zurück. Es wird unterschieden zwischen direkten und indirekten Handelsgewinnen. Alle Wirtschaftsteilnehmer profitieren durch Außenhandel, also durch Arbeitsteilung, die zur Spezialisierung führt, und Tausch. Dies wurde in den vorangegangen Kapiteln erläutert und geht zurück auf die Überlegungen von Adam Smith und David Ricardo. Aus Arbeitsteilung und Tausch resultiert ein Handelsgewinn, der als direkt bezeichnet wird \cite{Mill.1909}. Die indirekten Handelsgewinnen entstehen durch die folgenden drei Effekte und begründen dass Handel das Einkommen in der Welt steigert, da das Produktivitätswachstum gefördert wird.\footnote{Ein Land profitiert von Handelsliberalisierungen auf zwei verschiedene Arten: statisch und dynamisch \cite{Grossman.1989a,Grossman.1991b,Grossman.1991c,RiveraBatiz.,RiveraBatiz.1991a}. Dies ist eine andere Möglichkeit die Handelsgewinne zu untergliedern. Der statische Gewinn fasst höhere Produktqualitäten oder auch ein größeres Variantenreichtum zusammen. Dynamischer Gewinn beschreibt hingegen eine höhere Innovationsrate, die den stetigen Prozess neuer Produktentwicklungen eines Landes meint. Grossman und Helpman beschreiben dabei sowohl den Prozess der Innovation als auch den der Imitation. Beide Prozesse benötigen finanzielle Ressourcen, physisches Kapital und Arbeitskräfte. Ferner muss bei beiden mit der Unsicherheit des Erfolgs gerechnet werden. In ihren Beiträgen beschreiben sie eine Modellwelt, in der die Länder mit einem relativ hohen Lohnniveau einen komparativen Vorteil im Forschungssektor haben und somit günstiger Innovationen entwickeln können. Niedriglohnländer hingegen sind befähigt diese nachzuahmen und sich somit ebenfalls weiter zu entwickeln. Ausgehend von einer Nord-Süd Handelswelt werden sich die Produktionsstätten der Güter langfristig vom Norden in den Süden verlagern \cite{Grossman.1991c}.}
\begin{itemize}
\item [1.] Marktgrößeneffekt
\item [2.] Wissens-Spillover-Effekt
\item [3.] Wettbewerbseffekt
\end{itemize}
Bei dem \textit{Marktgrößeneffekt} führt die Öffnung eines Landes zu neuen Märkten, also zu einem insgesamt größeren Absatzmarkt, dem Weltmarkt. Je größer ein Markt ist, desto höhere Gewinne können erwartet werden. Durch die Öffnung der Grenzen steigt der Absatzmarkt eines Landes um die übrige Welt an. Es können insgesamt höhere Stückzahlen produziert und abgesetzt werden, wovon alle Produzenten gleichermaßen profitieren.\footnote{Diese Größeneffekte beschreibt \cite{Jones.1995a}, indem er allgemein endogene Wachstumsmodelle empirisch testet.} Dadurch nimmt die Bedeutung steigender Skaleneffekte und learning-by-doing Externalitäten deutlich zu \cite[Kapitel 15]{Aghion.1998}.\\
%
Der zweite Effekt, der \textit{Wissens-Spillover-Effekt}, bezieht sich nicht mehr auf die Gütermärkte, sondern beschreibt Wissensströme zwischen Regionen bzw. Ländern. Er beschreibt die Wissens- und Technologiediffusion, die unmittelbar aus internationalem Handel resultiert. In der Regel kommt es zu einem Austausch von technischem Wissen zu weniger weit entwickelten Regionen der relativ weiter entwickelten Regionen \cite{Sachs.1995}. 
Den Diffusionsprozess, bedingt durch die industrielle Revolution beschreibt \cite{Lucas.2007}, indem er untersucht, ob sich Unterschiede hinsichtlich der Offenheit von Ländern feststellen lassen. Dabei legt er die Kriterien für Offenheit\footnote{Bei diesen Kriterien handelt es sich um die Regelung der maximalen Höhe von Handelsbeschränkungen, sowie institutioneller und wettbewerbspolitischer Art.} von \cite{Sachs.1995} zugrunde, die erfüllt sein müssen, damit eine Volkswirtschaft als offen kategorisiert werden kann. Er stellt fest, dass mit der Offenheit eines Landes auch die Diffusionsdurchlässigkeit zunimmt.
%
Den Einfluss des Außenhandels auf eine Branche beschreibt der dritte Effekt, der \textit{Wettbewerbseffekt}. Die Öffnung eines Landes ist mit einer Vergrößerung des Marktes verbunden, wodurch der Wettbewerb zwischen den Produzenten steigt. In der theoretischen Modellwelt wird meist angenommen, dass es ein repräsentatives Unternehmen gibt und sich somit die Gesamtheit aller Unternehmer gemäß der Symmetrie der Unternehmen nach diesem richtet. In der Realität ist die Gesamtheit der Unternehmen aber nicht homogen. Somit ist auch der Einfluss von Handel auf die Unternehmen verschieden. Diese verhalten sich gerade nicht komplett gleich und weisen unterschiedliche Produktivitäten auf. Zwar eröffnen die hinzugewonnenen Absatzmöglichkeiten allen Marktteilnehmern neue Möglichkeiten, jedoch führt der gestiegene Wettbewerbsdruck dazu, dass die am wenigsten leistungsfähigen Unternehmen aus dem Markt gedrängt werden. Was wiederum dazu führt, dass die akkumulierte Produktivität einer Volkswirtschaft ansteigt.\footnote{Dieses Argument untermauert auch \cite{Trefler.2004} in seinem Aufsatz über die Produktivitätssteigerung Kanadas durch Handelsliberalisierung.}\\
%
Der Wettbewerbseffekt äußert sich demnach in einem Selektionseffekt. \cite{Melitz.2003} betont in seiner Arbeit diesen Selektionseffekt. Außenhandel ermöglicht den Zugang zu neuen Märkten und vergrößert somit das Absatzgebiet eines jeden Unternehmens. Neben der Nachfrage weitet sich jedoch auch das Feld der Anbieter aus, durch die der Wettbewerb des Marktes ansteigt. Die Marktkräfte führen dazu, dass die weniger effizienten Produzenten aus dem Markt ausscheiden, da sie nun durch ausländische Mitstreiter verdrängt wurden. Dieser Selektionseffekt beschränkt sich nicht nur auf die lokalen Unternehmen, sondern setzt sich auch im internationalen Wettbewerb zwischen den Unternehmen fort.\\Insgesamt werden jetzt nur noch die heimischen Unternehmen am Markt bleiben, die ein bestimmtes Effizienzniveau erfüllen. Das gestiegenen Effizienzniveau eines Landes wirkt sich direkt positiv auf das gesamtwirtschaftliche Einkommen aus \cite[Kapitel 15]{Aghion.2015}.
%
\subsection{Auswirkung der Effekte}
Die Kernfragen, die sich daraus ergeben lauten:  Welche Wirkung hat Handel auf das ökonomische Wachstum? Welche Folgen ergeben sich aus den genannten Effekten? Dies hängt im wesentlichen von der Modellierung des Handelsmodells ab und letztlich auch von den Gründen für ökonomisches Wachstum.\\
%
Die wissenschaftlichen Meinungen über den Einfluss von Handel auf das ökonomische Wachstum gehen auseinander. Vorherrschend ist, dass Außenhandel Wachstum fördert und somit ein positiver Zusammenhang zwischen Handel und Wachstum besteht \cite{Dollar.1992,Sachs.1995}. Auch empirisch wurde nachgewiesen, dass mit zunehmenden Handelsbeziehungen das Pro-Kopf-Einkommen ansteigt und somit auch das Wirtschaftswachstum \cite{Frankel.1999}.\footnote{Um dies zeigen zu können wurde ein meßbarer und berechenbarer Grad der Offenheit eines Landes entwickelt, mit dem sich die Länder einzeln katalogisieren lassen.}\\
%
Mikroökonomisch basierte Ansätze, wie von \cite{Bernard.2003} und \cite{Bernard.2004} zeigen, dass Unternehmen, die für den Exportsektor produzieren, produktiver sind. \\
Mit Außenhandel und der Heterogenität von Unternehmen beschäftigt sich die Arbeit von \cite{Melitz.2003}. Bei ihm führt Handelsliberalisierung zu einer dem Produktivitätsgrad entsprechenden Unternehmensstruktur. Nur die produktivsten Unternehmen produzieren für den Export, weniger produktive Unternehmen befriedigen die heimische Nachfrage und die schwächsten Unternehmen scheiden aus dem Markt aus.\footnote{Dabei handelt es sich um den angeführten Selektionseffekt.} Dies führt zu unternehmensinternen Umstrukturierungsprozessen, die der zusätzliche Wettbewerb fordert. Jedoch berücksichtigt er in seinem Modell nicht den Einfluss von Handel auf die Innovationstätigkeit.\\
%
Berücksichtigt man den Entwicklungsstand eines Landes, wird der Einfluss von Handel in weniger weit entwickelten Länder hervorgehoben \cite{Pavcnik.2002}. Hier zeigt sich die Wirkung des Wissens-Spillover-Effekts, denn der Import von Technologien aus relativ weiter entwickelten Volkswirtschaften erhöht die Produktivität der weniger weit entwickelten Ländern durch den Technologietransfer. \\
%
Die Nord-Süd-Modelle berücksichtigen ebenfalls den technologischen Fortschritt durch Innovationsentwicklung. Der weniger weit entwickelte Süden profitiert dabei vom Technologietransfer durch den Import von Innovationen. Neben diesem Spillover-Effekt verstärkt der Außenhandel weiterhin den technische Fortschritt durch die nun vorhandenen Imitationsmöglichkeiten. Der Import von Gütern erlaubt es dem Süden mit einer zeitlichen Verzögerung diese Güter nachzuahmen, währenddessen wieder neu entwickelte importiert werden \cite{Grossman.1991a,Krugman.1990}.\\
Vertreter der Mindermeinung  hinterfragen die positive Wirkung durchaus kritisch und zeigen teilweise, dass Außenhandel sogar die Wachstumsraten einiger Länder mindern kann \cite{RodriguezCaballero.2000,Matsuyama.,Young.1991,Galor.2008}.\bigskip\\
Grundsätzlich wirkt Außenhandel auf das Wachstum über die beiden von \cite{Gandolfo.1998} genannten Kanäle, der Faktorvermehrung und dem technischen Fortschritt.\\
Bei der \textbf{Faktorvermehrung}, dem ersten Wirkungskanal, stehen den Volkswirtschaften durch die Zunahme der Marktgröße insgesamt mehr Produktionsfaktoren zur Verfügung und die Technologiediffusion offener Volkswirtschaften erhöht die Effizienz des Faktoreinsatzes \cite{Gandolfo.1998}. Die Größenvorteile können unternehmensintern ausgenutzt werden. Aus mikroökonomischer Sicht können die Produktionsfaktoren effizienter genutzt werden und mit der gleichen Einsatzmenge kann nun eine höhere Ausbringungsmenge produziert werden. Das Grenzprodukt steigt an und somit steigt auch die Wachstumsrate. Der Wettbewerbseffekt bedingt ebenfalls die volkswirtschaftliche Produktivität, da er zur Selektion nur der konkurrenzfähigsten Unternehmen führt. Dies zeigt makroökonomisch, dass Unternehmen aus dem Markt austreten und nur die produktivsten Unternehmen eines Landes verbleiben. Somit liegt jetzt eine produktivere Gesamtheit aller Unternehmen vor, als in der geschlossenen Volkswirtschaft. Außerdem wirkt der Marktgrößeneffekt und die dadurch implizierte Unternehmensselektion durch den Wettbewerbseffekt auf den Forschungs- und Entwicklungssektor. Unternehmen streben stärker nach monopolistischer Marktmacht und das erhöht den Anreiz zur Innovationsentwicklung.  Dies bildet den Übergang zu dem zweiten Wirkungskanal des Wachstums, dem technischen Fortschritt. Er resultiert nicht nur aus der erhöhten Innovationstätigkeit in einer Volkswirtschaft, sondern auch durch den Wissens-Spillover-Effekt, der die internationale Technologiediffusion ermöglicht.\\
%
Der \textbf{technische Fortschritt} wird im folgenden durch die Intensität der \textit{Innovationstätigkeit} eines Landes untersucht. Dabei wird die Vorteilhaftigkeit von Außenhandel für die Wohlfahrt eines Landes durch den Einfluss der Offenheit auf die Innovationstätigkeit eines Landes gezeigt.\footnote{Gleichwohl ist auch eine hemmende Wirkung von Außenhandel auf die Innovationstätigkeit und letztlich das Wirtschaftswachstum möglich. Denn unterscheiden sich beide Länder durch ihre ursprünglichen Produktivitätsniveaus bei Autarkie, dann kann Außenhandel die Innovationstätigkeit hemmen, sofern es sich um anfänglich relativ weniger weit entwickelte Länder handelt \cite{Devereux.1994,RiveraBatiz.1991a}.} In diesem Zusammenhang werden die drei Effekte des Außenhandels nochmals verdeutlicht. \\
%
 Erfolgreiche Innovatoren profitieren vom Marktgrößeneffekt, da sich nun eher Innovationen finanzieren lassen, so dass diese auch tatsächlich produziert werden können. \\
%
Der Wissens-Spillover-Effekt, wirkt sich durch den nun möglichen internationalen Wissens- und Technologietransfer aus. Dieser Diffussionsprozess ist jedoch nicht zwingend notwendig, um die Vorteilhaftigkeit des Handels zu zeigen. Werden mögliche zusätzliche Wissens-Spillover-Effekte ausgeschlossen, da die geöffneten Volkswirtschaften über die gleichen Technologien verfügen, dann führen trotzdem dynamische Effekte zu höheren Wachstumsraten in geöffneten Ländern \cite{Grossman.1991b}. Der Ursprung des Wachstums liegt ebenfalls in der Entwicklung von Innovationen.\\
%
 Wird Wissenstransfer jedoch zugelassen, dann ist ein Zugewinn von technologischem Wissen über die Grenze hinweg möglich und  Pioneerunternehemen können das Wissen verwenden, um damit neue Innovationen zu entwickeln.\footnote{Nicht nur Innovationen, sondern auch die Imitationsrate wird durch Freihandel gefördert. Dies kann ebenfalls durch den Wissenstransfer im Zuge des Wissens-Spillover-Effektes begründet werden.} \\
%
Auch der aus Außenhandel resultierende Wettbewerbseffekt steigert die Innovationsrate, da insgesamt die Produktivität eines Unternehmens, einer Branche und letztlich eines Landes zunimmt. Demzufolge wird ein innovierendes Unternehmen in einer offenen Volkswirtschaft theoretisch mehr Innovationen entwickeln als in einer geschlossenen.
Denn der zusätzliche Wettbewerb stellt einen Anreiz zur Innovationsentwicklung dar, damit Unternehmen sich von den konkurrierenden Anbietern abheben können, um Marktmacht zu erlangen. Dabei fördert der durch Handelsliberalisierung verstärkte Wettbewerb den Innovationsprozess, welcher sich in dauerhaften Produktverbesserungen äußert \cite{Segerstrom.1990}.\\
%
Es bleibt festzuhalten, dass sich über alle drei Wirkungskanäle, Marktgrößeneffekt, Wissens-Spillover-Effekt und Wettbewerbseffekt, die Offenheit eines Landes positiv auf die Innovationsrate auswirkt.\\
%
Doch kann die Reaktion eines Landes nicht immer eindeutig vorhergesagt werden. Denn erörtert man diese Situation für technologisch kleine Länder mit einem großen Abstand zur WTG, dann kann ein Entmutigungseffekt bezüglich der Innovationstätigkeit auftreten. Die hohe Rückständigkeit lässt die aufzuholende Lücke hinreichend groß erscheinen, sodass es wenige Bestrebungen gibt an den technologischen Entwicklungsstand anzuschließen. Die Folge wäre ein Rückgang der Wachstumsrate und Handel würde in diesem Fall das Wachstum sogar reduzieren \cite{Aghion.2015}. \cite{Hicks.1968} behandelt diesen Zusammenhang zwischen Handel und Wachstum, basierend auf der Thematik nach dem zweiten Weltkrieg bezüglich Deutschland und den USA. Der Entwicklungsunterschied war so groß, dass es Bedenken gab, die Lücke nicht mehr schließen zu können. Auch wenn die Thematik nicht mehr zeitgemäß ist, lassen sich die Bedenken und Ansätze auf heutige Konstellationen anwenden.\\
%
Ebenfalls in technologisch kleinen Länder, deren Strategie nicht darin liegt Innovationen zu entwickeln ist es denkbar, dass die Wachstumsrate geschmälert wird, da von einem Flucht-Eintritts-Effekt\footnote{Dieser Effekt beschreibt einen zusätzlichen Impuls innovativ tätig zu sein, da in offenen Volkswirtschaften die Konkurrenzsituation zwischen den innovierenden Unternehmen deutlich stärker ist. Es besteht die Möglichkeit, dass ausländische Unternehmen schneller sind und somit eher eine Neuerung am Markt ansiedeln.} abstrahiert werden kann. Auch der Größeneffekt bezüglich der Innovationsrate kann vernachlässigt werden, sofern es sich um ein ökonomisch kleines Land handelt. Diese beiden Faktoren können dazu führen, dass sich ein Land  durch die außenwirtschaftliche Öffnung verschlechtert und die Wachstumsrate sinkt. \\
%
  Die beiden zuletzt genannten Argumente lassen eine politische Empfehlung für ökonomisch und technologisch kleine Länder ableiten. Denn wenn zunächst die Innovationstätigkeit gefördert wird, so dass es sektoral eine technologische Führerschaft gibt, und anschließend Handelsliberalisierug zugelassen wird, dann werden sich die Wachstumsaussichten verbessern \cite[Kapitel 15]{Aghion.2015}.
%
Der Import von Innovationen kann die heimischen Innovationsbestrebungen mindern bzw. sogar ersetzen. Selbst Länder die nicht innovieren, können ein höheres Produktivitätswachstum erreichen, indem sie Handel betreiben \cite[Kapitel 15]{Aghion.2015}. Die Anpassung und die damit einhergehende Konvergenz zum Weltmarkt stellen dann die Vorteilhaftigkeit von Außenhandel dar und nicht die Steigerung der Innovationsintensität.\\
%
Die Wirkung des Handels über den technischen Fortschritt durch die Innovationstätigkeit auf das Wachstum wurde ausführlich erläutert. Weitere Effekte des Handels werden  durch \textit{learning-by-doing Externalitäten}, die in einigen Sektoren auftreten, bedingt \cite{Young.1991,Matsuyama.}.\\
%
Learning-by-doing steht im engen Zusammenhang mit Größeneffekten, da mit steigender Ausbringungsmenge die Effizienz der Produktionsverfahren zunimmt. Bei kleineren Stückzahlen wirkt sich die Erfahrung durch leaning-by-doing noch nicht hinreichend positiv auf die Produktivität aus. Wohingegen durch Außenhandel die Bedeutung dieses Effekts durch den erweiterten Markt ansteigt \cite{Arrow.1962}.\\
Wird von Skaleneffekten abstrahiert, steigt zwar zunächst die Innovationsrate an, steigert jedoch nicht langfristig die Wachstumsrate. Dies ist nur ein Argument, dass aufzeigt, warum junge Industriezweige anfänglich vor der internationalen Konkurrenz geschützt werden sollen und wird in Kapitel \ref{Entwicklungsstrategien} nochmals aufgegriffen, um mögliche Entwicklungsstrategien aufzuzeigen.\\
%
 Außenhandel kann sich auch bei der Herausbildung der Handelsstruktur negativ auswirken. So kann internationaler Handel dazu führen, sich entgegen der tatsächlichen komparativen Vorteile zu spezialisieren und somit Branchen zu fördern, die ein vergleichsweise geringes Wachstumspotential aufweisen \cite[Kapitel 5, S.277-278]{Acemoglu.2009}.\\
%
Hinzu kommt die Möglichkeit des Ausbleibens von learning-by-doing-Effekten. Sind in weniger weit entwickelten Volkswirtschaften die Produktionsverfahren sehr traditionell geprägt und haben sich bereits über einen langen Zeitraum hinweg optimiert, dann werden learning-by-doing-Effekte die Produktivität nicht maßgeblich verbessern, da diese bereits weitestgehend ausgeschöpft wurden \cite[S.403]{Young.1991}. Eine Öffnung für Handel, die wiederum zu grenzübergreifendem learning-by-doing führt, hätte keinen oder sogar einen hemmenden Einfluss auf das Wachstum der Volkswirtschaft. \\
%
 Diesen Zusammenhang zeigt \cite{Young.1991}, indem er ein Land jeweils vor und nach der Einführung von Außenhandel analysiert. Handelt es sich um ein technologisch weniger weit entwickeltes Land, dann ist die Wachstumsrate des Einkommens in der geschlossenen Volkswirtschaft größer oder gleich der einer geöffneten Volkswirtschaft. Umgekehrt verhält sich ein relativ weit entwickeltes Land, das seine Situation durch Handel verbessert. So steigt nicht nur der technologische Entwicklungsstand an, sondern auch das Wirtschaftswachstum. Somit ist in dieser Konstellation die Wirkung des Außenhandels von dem Entwicklungsstand des Landes abhängig.\\
%
Die Veränderung des langfristigen Wachstums einer nun offenen Volkswirtschaft, verglichen mit derselbigen im geschlossenen Zustand untersuchen ebenfalls \cite{Grossman.1995} in ihrem Modell. Sie arbeiten dabei zwei Einflussfaktoren heraus, die zu einer relativen Veränderung der Wachstumsgeschwindigkeit beitragen. Von Bedeutung ist zum einen die Reichweite der learning-by-doing-Effekte, denn es ist fraglich ob diese nur national oder gar international wirken. Zum anderen beeinflusst die Spezialisierung der Produktion auf einzelne Sektoren, induziert durch Handel, die Wachstumsgeschwindigkeit. Je nach Gestaltung der Produktionsschwerpunkte und der daraus resultierenden Handelsstruktur kann das Wachstum eines Landes langfristig beschleunigt oder verlangsamt werden.\\
%
Ähnliche Ansichten wie \cite{Grossman.1995} teilt auch \cite{Krugman.1987}, der ebenfalls die Reichweite und damit das Wirkungsgebiet von learning-by-doing analysiert. Sein Handelsmodell beschreibt die dynamische Entwicklung des komparativen Vorteils, der durch learning-by-doing hervorgeht.  Wirkt learning-by-doing nur national, führt Handel nicht automatisch zur Konvergenz von Wachstumsraten der am Handel beteiligter Länder. Jedoch lösen durch Handel hervorgerufene internationale learning- by-doing-Effekte gegenseitige Reaktionen aus, die die Wachstumsraten konvergieren lassen. \\
%
Es wurde dargelegt, dass die Öffnung eines Landes die beiden Wirkungskanäle des Wachstums beeinflusst. Es verändert sich die inländische Produktivität einer Volkswirtschaft, durch die drei angeführten Effekte. Dabei wurde die Rolle des technologischen Fortschritts besonders betont und wird im folgenden nochmals am Beispiel Japans hervorgehoben.\\
\cite{Grossman.1990} entwickelte ein dynamisches Modell des komparativen Vorteils, bei dem Innovationen endogen sind. Es werden zwei Länder betrachtet und zwei Güter produziert, ein Hightech Konsumgut und ein gewöhnliches Konsumgut. In diesem Modell nimmt Japan die Rolle des Landes ein, das relativ reichlich mit sehr gut ausgebildeten Arbeitskräften, aber weniger gut mit natürlichen Ressourcen ausgestattet ist. In beiden Ländern verwenden Unternehmen ihre Ressourcen für Forschung und Entwicklung, um die Qualität der Güter zu verbessern. Motiviert sind sie durch bestehende Profitmöglichkeiten auf dem Weltmarkt durch Innovationen bzw. der daraus resultierenden Monopolstellung. Demzufolge kann Japan neue Technologien besser entwickeln und Hightech Produkte günstiger produzieren.  Der komparative Vorteil Japans liegt in der Herstellung von Hightech Gütern, was auch durch die Daten bestätigt werden konnte. \cite{Grossman.1990} ergründet neben der Handelsstruktur auch die Wirkung handelspolitischer Maßnahmen wie Importzölle und Exportsubventionen. Diese politischen Eingriffe erhöhen zwar die Wettberwerbsfähigkeit der Hightech Branche in dem jeweiligen Land, mindern jedoch die Innovationsquote, da weniger Anreize bestehen zu innovieren. Für die Entwicklung des technische Fortschritts bedeutet dies laut \cite{Grossman.1990}, dass die Wachstumsrate sinkt, sofern das Land die protektionistischen Maßnahmen einführt, bei dem der komparative Vorteil in der Hightech Produktion liegt, also hier Japan. Hat das sich schützende Land einen komparativen Nachteil in der Hightech Produktion, dann steigt die Rate des technischen Fortschritts an \cite[S. 30]{Grossman.1990}.\\
%
Die Richtung des technischen Fortschritts lässt sich insofern bestimmen, dass bei unvollständigen geistigen Eigentumsrechten mit Handel deutlich mehr Fachkräfte notwendig sind und ein höherer technologischer Entwicklungsstand realisiert werden kann als in einer geschlossenen Volkswirtschaft \cite{Acemoglu.2003,Thoenig.2003,Epifani.2006}.
