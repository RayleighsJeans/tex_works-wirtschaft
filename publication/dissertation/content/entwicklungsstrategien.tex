\chapter[Außenwirtschaftliche Entwicklungsstrategien]{Außenwirtschaftlich orientierte Entwicklungsstrategien}\label{Entwicklungsstrategien}
Ein Ziel dieser Arbeit ist es eine Entwicklungsstrategie zu bestimmen, die den Aufholprozess weniger weit entwickelter Volkswirtschaften durch Außenhandel bedingt. Dazu werden anhand der Handelseffekte mögliche Entwicklungsstrategien diskutiert.\\
%
Der Begriff Entwicklung beschreibt den Prozess einer positiven Veränderung von Zielgrößen \cite[Kapitel 1]{Wagner.1995}. In wirtschaftswissenschaftlichen Zusammenhängen wird er häufig nicht klar vom Begriff des Wachstums unterschieden. \cite{Findlay.1984} grenzt beide Begriffe hinsichtlich des Resultats des Prozesses ab. Er begreift den Begriff Wachstum als einen eher unspezifischen Ausdruck, der in vielen Bereichen Anwendung findet. Wohingegen er bei Entwicklung von einer strukturellen oder qualitativen Verbesserung ausgeht \cite{Findlay.1984}.\\
%
In diesem Zusammenhang wird eine Entwicklungsstrategie mit einem Entwicklungsdefizit bzw. dem Entwicklungspotenzial eines Landes assoziiert und somit nur von relativ weniger weit entwickelten Volkswirtschaften verfolgt. Eine eindeutige Klassifikation der weniger weit entwickelten Länder ist seit den 60er Jahren deutlich komplexer geworden, da solche Länder nach einem Strategiewechsel weg von der Importsubstitution unterschiedliche Wege mit unterschiedlichem Erfolg gegangen sind.\footnote{Die Erläuterung der Importsubstitutionsstrategie und weiterer Alternativen folgt nachstehend. Außerdem liefert die Arbeit von \cite{Stern.1973} einen Überblick des Forschungsstandes bis in die 70er Jahre hinsichtlich protektionistischer Handelspolitik.} Nach \cite{Krugman.2015} trifft jedoch meist eins der folgenden strukturellen Merkmale auf die weniger weit entwickelten Länder zu. Die staatliche Kontrolle, wie beispielsweise Handelsbeschränkungen die den Außenhandel steuern, ist ein charakteristisches Merkmal weniger weit entwickelter Länder. Ein weiteres Merkmal ist die Steuerung der Wechselkurse sowie eine sehr hohe Inflationsrate. Kennzeichnend ist auch, dass häufig liberale Finanzmärkte von eher schwachen Kreditinstituten geführt werden. Die Exporte werden vor allem durch landwirtschaftliche Erzeugnisse geprägt und des weiteren ist eine hohe Rate der Korruption bezeichnend.\\
%
Handelspolitik und die damit verbundenen protektionistischen Maßnahmen wurden und werden als Entwicklungsstrategie vieler weniger weit entwickelter Volkswirtschaften genutzt. Dabei traten schon Smith und Ricardo nicht nur für freie Märkte innerhalb einer Volkswirtschaft ein, sondern sahen auch die Regulierung und Steuerung des Weltmarktes als wohlfahrtsmindernd an. Jedoch bestätigen die Arbeiten von \cite{Dollar.1992,BenDavid.1993,Sachs.1995,Frankel.1999} und \cite{Edwards.1993} sowie \cite{RodriguezCaballero.2000} empirisch anhand länderübergreifender Untersuchungen, dass Handelspolitik sich positiv auf das ökonomische Wachstum auswirkt. Die dahinterliegende Intention ist der Schutz vor dem Wettbewerbseffekt durch Freihandel. Bestimmte Märkte und Branchen sollten zunächst die Möglichkeit haben sich aufzubauen und zu etablieren, bevor diese sich gegenüber der weltweiten Konkurrenz behaupten können. \\
%
Eine verbreitete Entwicklungsstrategie nach dem zweiten Weltkrieg war die \textbf{Importsubstitution} im Industriesektor. Ziel dieser Strategie war es Entwicklungsdefizite aufzuholen und die Selbstversorgung eines Landes zu sichern, damit die Unabhängigkeit vom Weltmarkt gewahrt wird. Dafür wird der Import von Gütern reduziert, um eine heimische Produktion anzustreben und zu unterstützen. Zu dem Instrumentarium der Importsubstituierung gehören einerseits protektionistische Maßnahmen der Außenhandelspolitik wie Zölle, Subventionen und Kontingente, andererseits binnenwirtschaftliche Richtlinien, wie Steuer- und Innovationsanreize, sowie auch die Befreiung von Markteintrittsbarrieren \cite{Muller.2005,Lachmann.1994}. Diese Regulierungsmaßnahme sollte den importkonkurrierenden Industrien helfen, sich vorübergehend vor dem weltweiten Wettbewerb zu schützen \cite{Lewis.1955}. Der Außenhandel wird aktiv vom Staat reduziert, um die eigene Produktion zu fördern. Somit geht eine Importsubstitution mit dem Rückgang des Handelsvolumens einher und es kommt ebenfalls zu geringeren Exporten, denn es ist nicht möglich importkonkurrierende Sektoren zu fördern, ohne dabei das Exportwachstum zu mindern.\footnote{Jedoch gibt es durchaus Rahmenbedingungen, bei denen ein Importzoll das Handelsvolumen erhöhen kann. Wird ein Zoll in einem Sektor auferlegt, der durch einen komparativen Nachteil geprägt ist, dann steigt dadurch der Güterhandel an \cite{Lancaster.1980}.}\\
%
Indien verfolgte diese Strategie so konsequent, dass in den 1970er Jahren anteilig nur 3{\%} des Bruttoinlandsproduktes durch Handel erwirtschaftet wurde. Dabei zeigte sich, dass trotz des geringen Importanteils kein überdurchschnittliches Wachstum folgte.\footnote{In Indien kam es in den 1990er Jahren zu einer Reform umfassender Handelsliberalisierungen, woraus anschließend starkes Wachstum resultierte.}
Begründet ist dies durch Preisverzerrungen, die die internationale Konkurrenzfähigkeit nicht wiederspiegeln \cite{Lachmann.1994}. Des Weiteren ist der Grundgedanke dieser Strategie, die Branche zu schützen, bei der zukünftig ein komparativer Vorteil erwartet wird, der bei der Strategieentwicklung jedoch nicht immer berücksichtigt wurde. Die künstliche Begrenzung des Wettbewerbseffekts sowie das Ausbleiben des Marktgrößeneffekts durch Außenhandel bedingte weiterhin das Bestehen von Ineffizienzen und Überkapazitäten, die wohlfahrtsmindernd wirken. \\
%
Bei der Importsubstitution kommt es durch den ausbleibenden Import neuer Technologien nicht zum Wissenstransfer und Spillover-Effekten.\footnote{Dies trifft nur dann zu, wenn es sich um eine vollständig geschlossene Volkswirtschaft handelt und keine Importe in das Land eingeführt werden.} Der Philosophie der Strategie entsprechend sollen Güter selbst produziert werden. Dies trifft dann auch auf Innovationen zu. Ein Import notwendiger Technologien, um diese nachahmen zu können, wird nicht angestrebt. Somit liegt der Schwerpunkt dieser Entwicklungsstrategie auf der Innovation von Gütern und Prozessen, auch wenn diese eventuell bereits existieren und somit nicht zu einer Ausweitung der Welttechnologiegrenze beitragen. Diese Realisierung des technischen Fortschritts ist jedoch sehr teuer und aufwendig. Außerdem ist qualifizierte Arbeit notwendig für erfolgreiche Innovationen. Dies begründet, warum eine Entwicklungsdefizit nicht nur bezüglich des Einkommens besteht, sondern auch hinsichtlich des technologischen Entwicklungsstandes.\\
%
Weitere Gründe für den ausbleibenden Erfolg der Strategie der Importsubstitution waren unter anderem fehlende Institutionen beispielsweise im Finanzsektor oder der Mangel qualifizierter Arbeit aufgrund eines unzureichenden Bildungssystems.\footnote{Die fehlende Implementierung von Institutionen sehen auch \cite{Collier.1999} als Ursachen des schwachen Wachstums afrikanischer Länder und konstatieren, dass die Integration in den Welthandel einfacher ist, als die Errichtung von Institutionen oder einer Infrastruktur \cite{Collier.1999}.}\\
%
Die Euphorie der Importsubsitution nahm letztendlich ab, da selbst Länder, die beinahe 100{\%} ihrer Güter selbst produzierten, keinen überdurchschnittlichen Entwicklungserfolg verzeichnen konnten, was zu einer abflachenden Beliebtheit dieser Strategie in den 1970er Jahren führte \cite{LittleIanMalcolmDavid.1970}.\\
%
Mitte der 1980er Jahre kam es zu einem allgemeinen Umschwung hin zu liberalerer Handelspolitik. Teilweise wurden die Handelsbeziehungen stark fokussiert und zusätzlich gefördert. Die \textbf{Exportförderungsstrategie} zielt auf eine vollständige Integration eines Landes in die Handelsbeziehungen der übrigen Welt ab. Diese Strategie basiert auf dem Ansatz des Freihandels, der auch politisch angestrebt und unterstützt wird. Zölle und andere Handelsbeschränkungen haben eine verzerrende Wirkung auf die Dynamik multisektoraler Modelle, hier den Märkten, die zur Wohlfahrtsminderung führen \cite{Ortigueira.2002}. Somit werden gezielt Anreize gesetzt den Export auszuweiten, indem Protektionismus reduziert wird. Auch wenn man damit dem Leitbild des Freihandels folgt, bleibt noch die Frage der Spezialisierung eines Landes. \\
%
Zunächst geriet der Fokus vieler Länder auf den primären Sektor. Der Export landwirtschaftliche Erzeugnisse wurde angestrebt. Jedoch kam es in den vergangenen Jahrzehnten zu Veränderungen der Struktur der weltweiten Nachfrage des primären Sektors. Es zeigte sich, dass diese einseitige Ausrichtung keine langfristige und nachhaltige Entwicklung sichert, da die Nachfrage nach landwirtschaftlichen Erzeugnissen  kontinuierlich abnimmt. Die weiter entwickelten Länder zeigen bereits, dass es zu einer Tendenz der Bevölkerungsabnahme kommt und demnach die Nachfrage nach Gütern des täglichen Bedarfs sinkt. Außerdem ist der Markt bereits weitestgehend gesättigt und weitere hinzukommende Anbieter, würden den Ertrag jedes einzelnen Anbieters zusätzlich schmälern, sodass es nicht mehr lohnend ist in den Markt einzudringen \cite{Muller.2005,Lachmann.1994}.\\
%
Bei der Strategie des exportorientierten Wachstums steigt das Handelsvolumen an, da die Vorteile des \textit{Marktgrößeneffekts} ausgenutzt werden, indem die vorhandenen Kapazitäten durch die neu hinzukommenden Märkte beim Freihandel ausgelastet werden. 
So bringt Außenhandel zwar größere Märkte mit sich und eröffnet die Möglichkeit von Größenvorteilen zu profitieren, doch aufgrund der meist fehlenden Produktionsfaktoren Kapital und qualifizierte Arbeit kann dieser Umschwung von Entwicklungsländern nicht gleichermaßen genutzt werden, wie dies bei industrialisierten Ländern der Fall ist. Bedingt ist der Mangel an den entsprechenden Produktionsfaktoren durch institutionelle Defizite. Dazu zählen beispielsweise unsichere Eigentumsverhältnisse, politische Instabilität und auch bei dieser Strategie wieder ein unzureichendes Bildungssystem.\\
%
Der \textit{Wettbewerbseffekt} äußert sich durch zusätzliche Anbieter, die die Konkurrenz verstärken, wodurch nun insgesamt ein höheres Bestreben nach effizienterer Produktion besteht. \cite{Trefler.2004} zeigt an dem konkreten Fall des Freihandelsabkommens zwischen den USA und Kanada den Wettbewerbseffekt durch Handelsliberalisierung. Er reflektierte die Wirkung des Abkommens auf die Unternehmensstruktur und zeigte, dass die Produktivität ganzer Branchen durch die Öffnung zum Weltmarkt zugenommen hat, weil die am wenigsten leistungsfähigen Unternehmen vom Markt verdrängt wurden \cite{Trefler.2004}.\\
%
Auch der \textit{Spillover-Effekt} äußert sich in der Strategie der Exportförderung.  Technologietransfer begünstigt nur dann weniger weit entwickelte Länder in ihrem Entwicklungsprozess, wenn Freihandel angestrebt wird. Denn catching up hängt wesentlich von den Handels- bzw. Markteintrittsbarrieren ab. Werden diese erhöht, ist eher eine Stagnation der Volkswirtschaft wahrscheinlich statt eines Aufholprozesses \cite{Stokey.2015}. Denn der mit Freihandel einhergehende importbedingte Technologietransfer führt zur Modernisierung der eigenen inländischen Technologien und letztlich zu technischem Fortschritt.
Neben Qualitätsverbesserungen werden auch neue Anreize  an inländische Unternehmen gesetzt innovierend und imitierend tätig zu werden. Denn eine Möglichkeit zur übrigen Welt aufzuschließen liegt darin, Innovationen, die dem Sektor einen Konkurrenzvorteil verschaffen, zu entwickeln und dadurch die Attraktivität der Handelsbeziehung zu steigern \cite{Muller.2005}. Doch auch für Imitationen liefert die Exportförderungsstrategie gute Voraussetzungen. Auch wenn keine Innovationen entwickelt werden, ist es möglich die Entwicklungsdefizite zu mindern, indem der Technologietransfer genutzt wird, um weiter entwickelte Technologien zu adaptieren.\\
%
Dies führte in vielen weniger weit entwickelten Ländern zu einem stärkeren Anstieg der Wachstumsraten als durch die Importsubstitution \cite{Krugman.2015}. Grund dafür ist die zunehmende Qualität importierter Zwischenprodukte. Die Innovationen des Auslandes werden indirekt importiert und wirken in dieser Form als direkter Technologietransfer. Demzufolge ist es nicht zwingend notwendig innovativ tätig zu sein, da der Handel mit technologisch entwickelten Gütern ein Substitut dafür sein kann \cite{Keller.2004}. Vor allem asiatische Länder, wie die sogenannten Tiger-Staaten zeigen, dass es möglich ist mit dieser Strategie zu den Industrieländern aufzuschließen. Die positive Wirkung von Handel auf den technologischen Entwicklungsstand eines Landes zeigen auch \cite{Bloom.2011} am Beispiel Chinas.\\
%
Bei der \textbf{Strategie der Integration} steht die Eingliederung in ein handelspolitisch geprägtes System im Vordergrund. Es geht bei dem sogenannten Integrationsraum um einen Wirtschaftsraum, der eine gemeinschaftliche Wohlfahrtsteigerung durch gleichartige Außenpolitik innerhalb, sowie nach außen gegenüber der übrigen Welt, anstrebt. Demnach werden in einem Integrationsraum intensive Handelsbeziehungen gepflegt, um gemeinschaftlich wirtschaftliche Probleme zu lösen. Zu unterscheiden ist dabei zwischen einem weniger weit entwickelten Land, das in einen entwickelten Wirtschaftsraum integriert wird oder aber gemeinsam mit ähnlichen weniger weit entwickelten Ländern einen Wirtschaftsraum zu bilden.\\
%
Je nach innen- und außenpolitischen Maßnahmen werden die drei Effekte des Außenhandels mehr oder weniger genutzt, woraus sich auch dann erst eine Tendenz der technologischen Weiterentwicklung ableiten lässt, ob diese eher innovativ oder imitativ geprägt ist. 
\cite{Glass.1999} sieht in der Imitation die Möglichkeit für weniger weit entwickelte Volkswirtschaften, meist auf der Südhalbkugel, sich dem Entwicklungsstand des Nordens anzupassen. Erst nachdem im Süden eine Basis an Wissen geschaffen wurde, ist ein Wechsel zur Innovationsstrategie sinnvoll. \\
%
Diese Überlegung hebt die Möglichkeit der Mischung beider zuerst genannter Strategien, der Importsubtitutions und der Exportförderung, hervor. Der Integrationsraum schützt zunächst vor der übrigen Welt und eröffnet die Möglichkeit die Vorteile des Freihandels innerhalb des geschlossenen Wirtschaftsraums auszunutzen. Ob ein Land der Strategie der Exportförderung oder der Integration folgt, determiniert noch nicht eindeutig welche Strategie im Forschungs- und Entwicklungssektor angestrebt wird.
\bigskip
%
Als Entwicklungsstrategie wird hier eine Strategie vorgeschlagen, die den technischen Fortschritt fördert. Im Zusammenhang mit Außenhandel spielt dabei der Wissens-Spillover-Effekt eine bedeutende Rolle. Im Sinne einer Exportförderungsstrategie wird Freihandel angestrebt, durch den eine negative Beeinträchtigung des Entwicklungsprozesses wegfallen soll. Das institutionelle Defizit eines unzureichend ausgebauten Bildungssystems, bzw. des fehlenden Anreizes zur Weiterbildung, wird daher durch die Öffnung eines Landes eingedämmt. Denn es wurde gezeigt, dass die unmittelbaren produktiven Auswirkungen des Humankapitals erheblichen Einfluss auf die Entwicklungspolitik haben. Die Förderung des Wachstums weniger weit entwickelter Länder wurde früher vermehrt durch physische Investitionsprojekte unterstützt. Nach herrschender Meinung ist der erhebliche Einfluss des Humankapitals auf das Wachstum bekannt und dadurch ist der Auf- bzw. Ausbau des Bildungssektors entwicklungspolitisch in den Vordergrund gerückt.\\
%
In den beiden folgenden Kapiteln \ref{Papier2} und \ref{Papier1} soll eine Strategie entwickelt werden, die das Wachstum eines weniger weit entwickelten Landes fördert. Dabei wird ein Modell zu Grunde gelegt, dessen Ursache für Wachstum im technischen Fortschritt liegt.\footnote{Die beiden Wirkungskanäle wurden bereits in Kapitel \ref{sec:Wachstumstheorien} genauer erläutert. Die noch folgenden ausführlich dargestellten Modelle unterscheiden sich genau hinsichtlich dieser beiden Gründe. Das in Kapitel \ref{Papier2} beschriebene Modell widmet sich der Faktorvermehrung und fokussiert dabei die Humankapitalakkumulation und den Ausbau des Bildungssektors, wodurch es indirekt zur Erhöhung des technologischen Entwicklungsstandes kommt. Der technische Fortschritt als direkte Ursache für Wachstum wird in dem darauf folgenden Modell in Kapitel \ref{Papier1} der Schwerpunkt sein. Das Modell spiegelt den engen Zusammenhang des technischen Fortschritts mit der Innovationstätigkeit eines Landes wieder. Dieser wurde bereits im theoretischen Grundlagenteil, Kapitel \ref{sec:techn. Wissen}, genauer dargelegt.} Der notwendige aber häufig hemmende Umstand mangelnder qualifizierter Arbeit soll dafür in einem ersten Schritt reduziert werden. \\
%
In einem weiteren Schritt wird die nun stärker vorhandene qualifizierte Arbeit für den Ausbau des technischen Entwicklungsstandes eines Landes eingesetzt. Dabei wird der gegenwärtige Entwicklungsstand eines Landes berücksichtigt und eine Empfehlung ausgesprochen, ob eine Imitations- oder Innovationsstrategie zu stärkerem Wachstum führt. Bei beiden folgenden Analysen wird außerdem auf die Wirkung möglicher Handelshemmnisse eingegangen.
