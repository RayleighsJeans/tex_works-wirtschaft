\chapter{Rest}
%
Etwas anders verh{\"a}lt es sich in weniger entwickelten L{\"a}ndern, die sich der {\"u}brigen Welt ge{\"o}ffnet haben. Die langfristigen Lernm{\"o}glichkeiten sinken nicht, sondern halten durch den Austausch mit ausl{\"a}ndischem Wissen weiter an \cite{Arnold.1997}.
%
\textcolor[rgb]{0.2,0.8,0.2}{SCHIEBEN zu offenen Modelle??:\\
In Modellen offener Volkswirtschaften werden weitere Einflussfaktoren auf den technischen Fortschritt hinzugezogen. Au{\ss}erdem ist der \textcolor[rgb]{1,0,0}{Wirkungskreis/Reichweite} einer einzelnen Innovation deutlich weiter. Die totale Faktorproduktivit{\"a}t h{\"a}ngt nicht allein von den inl{\"a}ndischen Investitionen in Forschung und Entwicklung ab, sondern auch von denen der {\"u}brigen Welt.  Je st{\"a}rker ein Land dem internationalen Handel ge{\"o}ffnet ist, desto st{\"a}rker ist auch der Einfluss der ausl{\"a}ndischen Investitionen auf den inl{\"a}ndischen Wissenszuwachs aufgrund von Wissens-Spillover-Effekten \cite{Coe.1995}.\footnote{Die Effekte des Außenhandels werden in Kapitel \ref{???} erläutert.}\\}\\
Hier werden drei wesentliche Gründe für langfristiges Wachstum behandelt. Die Innovationskraft einer Volkswirtschaft (P1), Bildung (P2) und Außenhandel (beide)
%
\textcolor[rgb]{0.2,0.8,0.2}{
QUELLE: Die Untersuchung von Farmer und Lahiri (2005) basiert auf einer Erweiterung des Uzawa-Lucas-Modells um ein weiteres Land und untersucht die Externalit{\"a}ten des Humankapitals.~\cite{FarmerRogerE.A..2005}}
%
Zu Handel SCHIEBEN:
Krugman beschreibt in seinem zweiten Papier von 1979 eine Modellwelt \footnote{Das hier beschrieben Modell geht zurück auf Krugman (1982) und ist eine Weiterentwicklung seiner Arbeit von 1979}, in der die L{\"a}nder der Nordhalbkugel als innovativ kategorisiert werden, wohingegen der S{\"u}den vom Technologietransfer des Nordens profitiert. Dabei ist es in den jungen Branchen wichtig konstant technologische Neuerungen zu entwickeln, um deren Lebensstandart aufrecht zu erhalten. Dieser ist langfristig durch die Anpassung der Lohnniveaus ungewiss, da der Norden in st{\"a}ndiger Konkurrenz zu den Arbeitern des S{\"u}dens steht, f{\"u}hrt ein R{\"u}ckgang an Innovationen zu n{\"a}her beieinander liegenden Lohnniveaus. Dieses Handelsmodel ist nicht durch verschiedene Produktivit{\"a}ten oder Ausstattung bedingt und es ergibt sich auch nicht direkt ein Handelsmuster wie in der beispielsweise in den neoklassischen Modellen.  Handel wird bestimmt durch die Entwicklung von Innovationen auf der Nordhalbkugel und dem Technologietransfer zur S{\"u}dhalbkugel. \\ \\ \\ Seine Arbeit liefert viele parallelen zu der Vorliegenden. \\ Stichpunkte:\\ aus P2: Es werden technologisch h{\"o}her entwickelte G{\"u}ter von der relativ weiter entwickelten Region in die weniger weit entwickelte Region exportiert. \\ aus P1. Hinzu kommt noch die Zuordnung einer Strategie gem{\"a}{\ss} dem Entwicklungsstand. \\
%
Eine daraus resultierende Handelsstruktur führte \cite{Krugman.1979ab} an. Er geht davon aus, dass in weniger weit entwickelten L{\"a}ndern adaptierte G{\"u}ter exportiert werden. Wohingegen in industrialisierteren L{\"a}ndern der Exportsektor durch Innovationen gepr{\"a}gt ist. \cite{Krugman.1979ab}--> brauch ich das??
%
\textcolor[rgb]{0.2,0.8,0.2}{SCHIEBEN\\
 Au{\ss}erdem kommt er zu dem Schluss, dass auch die {\"o}konomische Gr{\"o}{\ss}e eines Landes den Wachstumsprozess beeinflusst. Nach seinen Berechnungen wachsen kleine L{\"a}nder tendenziell langsamer als {\"o}konomisch gro{\ss}e L{\"a}nder. \\ \\ richtige Kategorie???~\cite{Romer.1986}}\\
%
\textcolor[rgb]{0.2,0.8,0.2}{EINBAUEN: Im Kapitel Handel P1 oder P2.\\
Es kommt zu einem Technologietransfer durch Handel zwischen einem relativ weiter entwickelten Land und einem vergleichsweise weniger weit entwickelten Land \cite{Findlay.1978}.}
%
SCHIEBEN Modelle Handel und Wachstum Krugman\\
Krugman wiederum beschreibt in seinem zweiten Papier von 1979 eine Modellwelt\footnote{Das hier beschriebene Handelsmodell geht zurück auf Krugman (1982) und ist eine Weiterentwicklung seiner Arbeit von 1979}, in der die L{\"a}nder der Nordhalbkugel als innovativ kategorisiert werden, wohingegen der S{\"u}den vom Technologietransfer des Nordens profitiert. Dabei ist es in den jungen Branchen wichtig konstant technologische Neuerungen zu entwickeln, um den Lebensstandart der Beschäftigten aufrecht zu erhalten. Dieser ist langfristig durch die Anpassung der Lohnniveaus ungewiss, da der Norden in st{\"a}ndiger Konkurrenz zu den Arbeitern des S{\"u}dens steht, f{\"u}hrt ein R{\"u}ckgang an Innovationen zu n{\"a}her beieinander liegenden Lohnniveaus. Dieses Handelsmodell ist nicht durch verschiedene Produktivit{\"a}ten oder Ausstattung bedingt und es ergibt sich auch nicht direkt ein Handelsmuster wie beispielsweise in den neoklassischen Handelsmodellen.  Handel wird bestimmt durch die Entwicklung von Innovationen auf der Nordhalbkugel und dem Technologietransfer zur S{\"u}dhalbkugel. \textcolor[rgb]{1,0,0}{--> hier doch noch gar kein Handel ??}\\
\textcolor[rgb]{1,0,0}{Seine Arbeit liefert viele parallelen zu der Vorliegenden. \\ Stichpunkte:\\ aus P2: Es werden technologisch h{\"o}her entwickelte G{\"u}ter von der relativ weiter entwickelten Region in die weniger weit entwickelte Region exportiert. \\ aus P1. Hinzu kommt noch die Zuordnung einer Strategie gem{\"a}{\ss} dem Entwicklungsstand. }
\\ Krugman (1979b) geht davon aus, dass in weniger weit entwickelten L{\"a}ndern adaptierte G{\"u}ter exportiert werden. Zumal in industrialisierteren L{\"a}ndern der Exportsektor durch Innovationen gepr{\"a}gt ist. \\ \textcolor[rgb]{1,0,0}{--> nochmal nachsehen: welchen Effekt hat Handel? wie entwickeln sich die VWL durch Handel weiter? oder genau das L{\"u}cke, die ich versuche zu schlie{\ss}en?} \cite{Krugman.1979}
