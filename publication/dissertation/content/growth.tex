\section[Wachstumstheorien beruhend auf technischem Fortschritt]{Wachstumstheorien beruhend auf\\ technischem Fortschritt \sectionmark{Wachstumstheorien}}\label{sec:Wachstumstheorien}
\sectionmark{Wachstumstheorien}
%
Das folgende Unterkapitel befasst sich mit der Entwicklung der Wachstumstheorien, die sich vornehmlich mit den Ursachen des Wirtschaftswachstums beschäftigen. Beginnend mit der relativ jungen Wirtschaftstheorie der "`unified growth theory"', zu deutsch die Theorie des einheitlichen Wachstums, wird anschließend wieder die Struktur \cite{Gandolfo.1998}s aufgegriffen, die die Wirtschaftstheorien gemäß ihrer Gründe für Wachstum untergliedert. Gandolfo sah als direkte Ursachen von Wachstum zum einen die Akkumulation von Produktionsfaktoren und zum anderen den technischen Fortschritt. Die Akkumulation von physischem Kapital wird unter anderem im neoklassischen Solow-Modell thematisiert. Darauf folgt die Abgrenzung zu den endogenen Wachstumstheorien, wie beispielsweise dem Romer-Modell. Anschließend wird der technische Fortschritt in schumpeterianischen Modellen genauer analysiert, bevor abschließend anhand des Uzawa-Lucas-Modells die Akkumulation von Humankapital als Voraussetzung für den technischen Fortschritt behandelt wird. \\
In diesem Rahmen werden die verschiedenen Ansätze und Modelle kurz vorgestellt, um die im Hauptteil folgenden Modellvariationen darin einordnen zu können. 
%
\subsubsection*{unified growth thoery}\label{Unified}
Die "`unified growth theory"' wurde von Oded Galor begründet und versucht einen zeitlich allumfassenden Erklärungsansatz für das Wirtschaftswachstum zu finden. Dabei wird das langfristige Wachstums vor der Zeit der Industrialisierung mit einbezogen, wodurch eine stärkere Bedeutung des demographischen Wandels bedingt wird \cite{Galor.2011}.
%
	\begin{figure}[h!]
 		\centering 
		 \begin{tabular}{@{}r@{}} 
%			\includegraphics[width=0.95\textwidth]{figure/Abbildungen/Karte.eps}\\
%			\epsfig{file=figures/Abbildungen/Karte.eps}
 		\end{tabular}  
		\quelle{\textbf{Quelle:} Galor (2011)}
		\caption[Pro-Kopf-Einkommen der Welt 2010]{Pro-Kopf-Einkommen der Welt im Jahre 2010}\label{KarteEinkommen}
	\end{figure}
%
Abbildung \ref{KarteEinkommen} zeigt das Pro-Kopf-Einkommen der Weltbevölkerung aus dem Jahr 2010. Das BIP ist das Maß für das wirtschaftliche Wachstum, wobei eine Pro-Kopf-Betrachtung eine internationale Vergleichbarkeit ermöglicht. Dabei wird ersichtlich, dass auf der Nordhalbkugel und in den Pazifikstaaten Australien und Neuseeland das durchschnittliche Einkommen pro Kopf bei mindestens 15.000 US-Dollar pro Jahr liegt. Führend sind Nordamerika, Europa, sowie Australien und Neuseeland. Das durchschnittliche Einkommen dieser Länder ist größer als 30.000 US-Dollar. Mit weniger als 3.000 US-Dollar im Jahr müssen die Einwohner im Norden Sub-Sahara-Afrikas auskommen \cite[Kapitel 1]{Galor.2014}.\\
%		
		\begin{figure}[h!]
			\centering 
				\begin{tabular}{@{}r@{}}  
%				\includegraphics[width=0.95\textwidth]{figure/Abbildungen/BIP.eps}
				\end{tabular}  
			\quelle{\textbf{Quelle:} Galor (2011)}
			\caption[Pro-Kopf-Einkommen 1820-2010]{Pro-Kopf-Einkommen von 1820-2010 -~zu den Western Offshoots zählen Australien, Kanada, Neuseeland und USA~-}\label{BIP200Jahre}
		\end{figure}
%		
Abbildung \ref{BIP200Jahre} zeigt das BIP pro Kopf im Zeitverlauf der letzten 200 Jahre. Es sind immer noch deutliche regionale Unterschiede zu verzeichnen, doch viel auffälliger ist, dass gegen Ende des 19. Jahrhunderts, in den heute relativ weit entwickelten Ländern, eine Phase der Stagnation endete. Außerdem gab es weltweit nach dem zweiten Weltkrieg einen erneuten Wachstumsschub.
%
		\begin{figure}[h!]
			\centering 
			\begin{tabular}{@{}r@{}} 
				\psfrag{A}{Asien} 
%				\includegraphics[width=0.95\textwidth]{figure/Abbildungen/200JahreBIP.eps}\\
			\end{tabular}
		 	\quelle{\textbf{Quelle:} Galor (2011)} 
			\caption{Pro-Kopf-Einkommen von 1810-2010}\label{BIP2000Jahre}
		\end{figure}
%		
Ein erweiterter Blick auf Schätzungen\footnote{Diese wurden von \cite{Galor.2011} vorgenommen und gehen zurück auf die Daten von \cite{Maddison.2001}.} der letzten 2000 Jahre in Abbildung \ref{BIP2000Jahre} zeigt, dass die Phase der Stagnation seit Beginn unserer Zeitrechnung andauert. Der erste Wachstumsschub gegen Ende des 19. Jahrhunderts  gründet auf der Erfindung der Dampfmaschine und der damit einhergehenden industriellen Revolution. Zunächst in England, dann in ganz Westeuropa, Japan und in den USA kam es zu dem Übergang von der Agrar- zur Industriegesellschaft. Die Industrialisierung bedingte eine stark beschleunigte Entwicklung von Technologie, Produktivität und Wissenschaft.\\
%
In der vorliegenden Arbeit soll aufgezeigt werden, dass es sich hierbei um wesentliche Einflussfaktoren wirtschaftlichen Wachstums handelt. Jedoch ist der Grenzertrag dieser Neuerungen abnehmend und somit für die Industrieländer von geringerer Bedeutung. Auf das wirtschaftliche Wachstum noch relativ wenig entwickelter Länder üben diese Faktoren heute aber einen deutlichen Einfluss aus. Die regionale Ausbreitung der industriellen Entwicklung, der Technologietransfer, erfolgt entweder durch Migration oder durch den Güterhandel, dem zweiten Schwerpunkt dieser Arbeit.\\ 
%
In dem Bereich der "`unified growth theory"' beschäftigt sich Oded Galor vornehmlich mit  Forschungsfragen über den Ursprung der sozialen Ungleichheit zwischen den Ländern:\footnote{Die soziale Ungleichheit hat sich in den vergangenen 2000 Jahren enorm verändert. Wird nur Westeuropa betrachtet, so ist der Faktor 40 Mal so groß, als zu Beginn unserer Zeitrechnung. In Ländern Afrikas, hat sich die Ungleichheit hingegen "`nur"' vervierfacht. Die Folge des hohen Wirtschaftswachstums ist eine größere Kluft zwischen den armen und reichen Bevölkerungsschichten \cite{Galor.2011}.} Welche Faktoren hemmten die Konvergenz armer Länder an reichere in den letzten Jahrzehnten? Welche Rolle spielen die originären Faktoren, wie kulturelle, geologische und geographische  Eigenschaften eines Landes bei der Erklärung der beobachteten komparativen Vorteile?\\
%
Die Bevölkerungsfalle oder auch Malthusianische Katastrophe genannt, bildet die Grundlage der einheitlichen Wachstumstheorie und stellt ein Hemmnis für Entwicklung und Wachstum dar. Der Grundgedanke geht auf Thomas Malthus (1798) zurück. Er behauptete, dass langfristiges Wachstum des Lebensstandards nicht möglich sei. \cite{Galor.2011} greift seine Theorie auf und unterteilt dabei die letzten 2000 Jahre in drei verschiedene Epochen. Die Malthusianische Epoche, die Post-Malthusianische Epoche und die Zeit des Modernen Wachstums.\footnote{Neben \cite{Galor.2006} befassen sich ebenso die Aufsätze von \cite{Hansen.2002}, sowie \cite{Ashraf.2008} mit dieser zeitlich allumfassenden Wachstumstheorie.}\\

		\begin{figure}[htbp]
			\centering 
			\begin{tabular}{@{}r@{}}  
%				\includegraphics[width=0.90\textwidth]{figure/Abbildungen/EpochenMalthus.eps}
			\end{tabular}
			\quelle{\textbf{Quelle:} Galor (2011)} 
			\caption[Entwicklungsphasen des Wachstums]{Entwicklungsphasen des Wachstums (der "`unified growth theory"')}\label{Epochen}
		\end{figure}
%		
Die Malthusianische Epoche nimmt 99,8{\%} der letzten 2000 Jahre ein und endet in den 50er Jahren des 18. Jahrhunderts. Die verbleibenden 0,2{\%} bilden die Post-Malthusianische Epoche, welche ca. 120 Jahre andauerte und durch die Industrielle Revolution eingeleitet wurde, sowie die anschließende Zeit des Modernen Wachstums. Diese begann in den 1870ern und dauert bis heute an \cite{Galor.2014}.\\
%
\subsubsection*{Malthusianische Epoche}
\cite{Ashraf.2011} charakterisieren die Malthusianische Epoche vor allem durch den sehr langsamen Prozess des technischen Fortschritts. Dieser war nicht das Ergebnis organisierter Wissensakkumulation, wie es seit der Industrialisierung und in den Industrie\-ländern üblich war, sondern basierte auf Erkenntnissen, Erfahrungen und Experimenten des Alltags sowie der Notwendigkeit Probleme zu lösen, um das Überleben zu sichern. Jedoch wurde in Anbetracht des sehr langen Zeitraums von knapp 2000 Jahren relativ wenig Neuerungen eingeführt und es resultierte laut der Schätzungen von \cite{Maddison.2001} nur eine jährliche Wachstumsrate von $\frac{1}{19}\%$ des Pro-Kopf-Einkommens. In diesem Zeitabschnitt entspricht das Pro-Kopf-Einkommen ungefähr dem Existenzminimum. Der geografisch begrenzte Produktionsfaktor Land stellt die Haupteinnahmequelle der Bevölkerung dar. Der fruchtbare Boden konnte nur bedingt bewirtschaftet werden und führte langfristig zu abnehmenden Grenzerträgen des Bodens und der Arbeit. Geht man von einem Land aus, das nur landwirtschaftliche Güter herstellt, dann benötigt die Volkswirtschaft fruchtbares Land $X$, Arbeit $L$ und den Produktivitätsparameter $A$ um das Gut $Y$ herzustellen.
%
	\begin{equation}
		Y=AX^\beta L^{1-\beta},  \qquad \text{mit}\quad 0< \beta < 1
	\end{equation}
%	
Wenn davon ausgegangen wird, dass jedes Mitglied der Bevölkerung arbeitet und das fruchtbare Land  auf $X=1$ normiert wird, dann ergibt sich für das Pro-Kopf-Einkommen $y$ folgende Gleichung. 
%
	\begin{equation}
		y=\frac{Y}{L}=AL^{-\beta}
	\end{equation}
%
Hier lässt sich formal darstellen, dass ein positiver Zusammenhang zwischen der Produktivität $A$ und dem Pro-Kopf-Einkommen $y$ besteht und ein negativer mit der Bevölkerungsgröße $L$. Damals wie heute bestimmt das Einkommen die Familienplanung. Ein hohes Pro-Kopf-Einkommen geht mit einem hohen Lebensstandard einher. Je stärker das Pro-Kopf-Einkommen wächst, desto schneller wächst die Bevölkerung.\\
%
Drei externe Effekte beeinflussen in diesem Zeitabschnitt das Pro-Kopf-Einkommen positiv: der technologische Fortschritt, die Ausweitung des bestellbaren Bodenbestands und ein Rückgang der Bevölkerung durch exogene Schocks, wie Krankheiten oder Hungersnöte. Diese führen kurzfristig zu einem positiven Pro-Kopf-Einkommenseffekt. Der Wohlstandsanstieg der Bevölkerung bedingt dann wiederum ein höheres Bevölkerungswachstum. Langfristig bedeutet das jedoch, dass das Niveau des Pro-Kopf Einkommens wieder sinkt. Beispielhaft für das Verhalten des Einkommens auf einen exogenen Schock ist die Pest, die in England von 1250 bis 1270 wütete. Die Bevölkerungszahl sank sehr stark, wodurch der Produktionsfaktor Arbeit knapper und dadurch teurer wurde. Ein stark ansteigendes reales Lohnniveau war die Folge. Erst mit dem Anstieg der Bevölkerung sank auch das Lohnniveau wieder. \\ Ein weiterer Zusammenhang besteht zwischen der Bodenproduktivität und der Bevölkerungsdichte. Je produktiver das Land ist und je mehr Lebensmittel angebaut und geerntet werden können, desto stärker wächst die Bevölkerung. In diesem Fall vornehmlich in Volkswirtschaften, denen relativ viel bestellbares Land zur Verfügung steht. Jedoch hat die Zunahme der Produktivität des Bodens keinen direkten Einfluss auf das Pro-Kopf-Einkommen, weil der anfängliche Einkommenszuwachs durch den Produktivitätsgewinn, durch die Bevölkerungszunahme ausgeglichen wird. \\ Bei dem dritten positiven Effekt dieser Zeit, dem Technologischen Fortschritt verhält es sich ähnlich. Anfänglich steigert dieser die Produktivität und somit das Einkommen, aber ein höheres Einkommen führt zu einer höheren Geburtenrate und gleicht somit den kurzfristigen Effekt wieder aus. Ansonsten lässt sich zwischen technologischem Fortschritt und Pro-Kopf-Einkommen nur ein geringer positiver Zusammenhang feststellen \cite{Galor.2014}.
%
\subsubsection*{Post-Malthusische Epoche}
%
Der Übergang der Malthusischen zu der Post-Malthusischen Epoche ist durch den Startpunkt, den "`take-off-point"', des wirtschaftlichen Wachstums charakterisiert. Dabei wird die Phase der Stagnation durch Wachstum abgelöst. 
Laut der Theorie nach \cite{Hansen.2002} sowie \cite{Ashraf.2008}\footnote{Das Papier von \cite{Ashraf.2008} bestätigt den Wandel des Bevölkerungswachstums in der malthusischen Epoche empirisch.} wurde der technische Fortschritt  durch die Industrielle Revolution im 18. Jahrhundert stark beschleunigt.\footnote{Eine andere Theorie besagt, dass die Humankapitalakkumulation im Vordergrund steht und letztlich zur Industrialisierung, dem Übergang von der Stagnation zum Wachstum, geführt hat \cite{Galor.2000}. Die Ansammlung von Humankapital führt zu technischem Fortschritt, der somit durch einen Skaleneffekt der Bevölkerungsgröße entsteht. Andererseits führt erst der Produktivitätsfortschritt zu einer Nachfrage nach Humankapital und es kommt zu ständigen positiven Wechselwirkungen zwischen der Humankapitalakkumulation und dem technischen Fortschritt.} Dadurch kam es zu einem sehr starken Anstieg des totalen Outputs und auch des Pro-Kopf-Einkommens. Das Pro-Kopf-Einkommen hatte noch immer einen positiven Effekt auf das Bevölkerungswachstum, jedoch ist dieser nun im Vergleich zur Malthusischen Epoche  abnehmend. Es herrschte also ein vergleichsweise schnelles Wachstum des Pro-Kopf-Einkommens und der Bevölkerung. \\ Dieser Wachstumsstartpunkt ist jedoch regional verschieden, vor allem, weil es schon regionale Entwicklungsunterschiede gab. So vor allem in technologisch weiter entwickelten Volkswirtschaften und auch in Ländern, die sehr reichlich mit dem Faktor Boden ausgestattet waren. Hier gab es grundsätzlich eine höhere Bevölkerungsdichte und größtenteils ähnliche Einkommenslevel in den Bevölkerungsschichten. Somit waren diese schon in der Malthusianischen Epoche relativ weiter entwickelt, was wiederum einen früheren "`take-off point"' mit einem relativ stärker andauernden Wachstum bedingte.\\ Werden die Regionen anhand der Industrialisierung pro Kopf \footnote{Dies kann als die Arbeitsleistung pro Kopf gesehen werden, die durch den Einsatz fortschrittlicherer Verfahren ansteigt und wird gemessen an der industriellen Produktion pro Kopf.} miteinander verglichen, verdeutlicht dies, dass die Industrialisierung in Großbritannien ihren Ursprung hat \cite{Galor.2014}.\\
%
Durch Migration und Handel bedingt, kam es erst über 50 Jahre später in den übrigen europäischen Ländern, wie Frankreich und Deutschland, sowie Nordamerika zu einem Anstieg der Pro-Kopf-Industrialisierung.  In den heutigen Entwicklungsländern sank sogar in der Zeit der Malthusischen Epoche die Industrialisierung pro Kopf aufgrund des starken Bevölkerungswachstums. Erst in der Zeit des modernen Wachstums, ab dem Jahre 1920, gelangte ein Wachstumsimpuls in die Länder der dritten Welt. Ein deutlich stärkerer Wachstumsimpuls auf deren Industrialisierung folgte mit etwas zeitlicher Verzögerung nach dem zweiten Weltkrieg im Jahre 1960. Jedoch handelt es sich hierbei um einen deutlich geringeren Wachstumsschub, als er durch die Industrialisierung in den heutigen Industrieländern hervorgerufen wurde \cite{Galor.2014}.\\
%
\subsubsection*{Epoche des modernen Wachstums}
%
Die Epoche des modernen Wachstums beschreibt den Zeitabschnitt in dem das anhaltende ökonomische Wachstum beginnt. Der technische Fortschritt war in dieser Zeit so intensiv, dass es eine starke Nachfrage nach Humankapital gab. Die Bevölkerung begann daher in ihre Ausbildung zu investieren und baute Humankapital auf. Die Menschen mussten aber Prioritäten setzen und ihre Zeit zwischen Erwerbstätigkeit, Kindererziehung und ihrer eigenen Weiterbildung aufteilen. Dies geschah zu Lasten der Geburtenrate, welche mit steigenden Humankapital schließlich sank. Qualifizierte Mitarbeiter förderten von nun an den andauernden Industrialisierungsprozess und der technische Fortschritt nahm weiterhin zu. Die gesunkene Geburtenrate führte letztlich zu einem geringeren Bevölkerungswachstum. Von diesem Zeitpunkt an war das ökonomische Wachstum unabhängig von den Bevölkerungsbewegungen und es kam zu keiner Kompensation positiver wachstumsfördernder Effekte durch Bevölkerungszunahme. Die drei angeführten Punkte, technologischer Fortschritt, gemindertes Bevölkerungswachstum und Humankapitalakkumulation generierten von da an langfristiges gleichmäßiges ökonomisches Wachstum. \\ Werden die Wachstumsraten der verschiedenen Volkswirtschaften betrachtet, so handelt es sich seit 1950 bis zum heutigen Zeitpunkt um größtenteils gleichmäßiges positives Wachstum. Die unterschiedlichen Entwicklungsstände werden durch die verschiedenen Niveaus des BIPs pro Kopf deutlich. Diese resultieren aus den unterschiedlichen Startsituationen in der Malthusianischen Epoche und den daraus folgenden "`take off points"' induziert durch die Industrialisierung \cite{Galor.2014}.\\
%
Die Entwicklung der Geburtenrate greift \cite{Galor.2014} erneut auf und analysiert in seiner Wachstumstheorie deren Rückgang. Die Daten zeigen, dass nicht nur die Entwicklung der Länder zeitlich versetzt ist, sondern auch die Geburtenraten ähnlich reagieren. Länder mit relativ schlechteren Anfangsbedingungen und somit einem späteren "`take off"' verzeichnen auch einen verzögerten Anstieg und späterem Absinken der Geburtenrate. Die Geburtenrate wächst zunächst durch das zusätzliche Einkommen aus der industrialisierten Wirtschaft und sinkt mit zunehmenden Bildungsstand der Bevölkerung. Werden die asiatischen oder afrikanischen Volkswirtschaften betrachtet, so stieg dort die Geburtenrate erst im Jahr 1870 an. Fünfzig Jahre später begann in den Ländern der westlichen Welt zu diesem Zeitpunkt die Geburtenrate bereits wieder zu sinken \cite{Galor.2014}. Oder Galors "`unified growth theory"' fand viele Anhänger, die ihre Aufgabe darin sahen die Entwicklung rückblickend zu erörtern.\\
%
Das Malthusische Modell zeigt, dass die Produktion mit einem fixen Faktor, dem Land bzw. dem fruchtbaren Boden, und zunehmenden Bevölkerungswachstum von der Pro-Kopf-Output-Rate abhängt. Dabei führt ein hohes Pro-Kopf-Einkommen zu einem Anstieg der Bevölkerung, was wiederum die Pro-Kopf-Rate mindert und die Bevölkerungszahl sinkt.  Langfristig ergibt sich eine Stagnation der Wachstumsrate. Wird der Ansatz von \cite{Malthus.1798} um eine AK-Produktionstechnologie erweitert, dann simuliert dies die Zeit des 1900 Jahrhundert, in der die industrielle Revolution zu grundlegenden Veränderungen führte. Diese Modellerweiterung nach \cite{Hansen.2002}, sowie \cite{Ashraf.2008} veranschaulicht, dass sofern der Wissensparameter groß genug ist, es zu einem Strukturwandel vom primären Landwirtschaftssektor zum sekundären Industriesektor kommt. Somit wird die Kompetenz und Qualifiziertheit der Unternehmer in Zeiten struktureller Veränderungen, wie beispielsweise dem Wandel im Zuge der Industrialisierung betont \cite{Galor.1997}.
Die Volkswirtschaft bewegt sich damit aus der Stagnation heraus und die Wirtschaft wächst langfristig. Sie sehen den Grund für den Entwicklungsprozess stagnierender zu wachsenden Volkswirtschaften in dem Wandel von landintensiver Produktion hin zu technologieintensiver Produktion, auch als Folge der Industrialisierung. Dieser Zusammenhang ebnet den Übergang zur neoklassischen Wachstumstheorie, dessen führender Vertreter Robert Solow ist \cite{Hansen.2002}.
%
\subsection{Exogene Wachstumsmodelle}
Die folgenden traditionellen Wirtschaftstheorien beschäftigen sich vornehmlich mit der Erklärung des Wachstums seit dem Industrialisierungsprozess. Ein Wachstumsmodell wird immer dann als exogen bezeichnet, wenn die Ursachen des technischen Fortschritts nicht hinterfragt werden und per Annahme in das Modell eingehen. Dies belegt das Solow-Modell, indem Kapitalakkumulation zu einem Anpassungswachstum hin zum Gleichgewicht führt und technischer Fortschritt als exogene Annahme einer langfristigen Stagnation entgegenwirkt.
%
\subsubsection*{Solow-Modell}
Robert Merton Solow wurde 1924 in New York City geboren und fand, nach dem zweiten Weltkrieg während eines volkswirtschaftlichen Studiums in Harvard, in Wassily Leontief seinen Lehrer \cite{Lin.2007}. Aus seinem bedeutendsten Papier "` A Contribution to the Theory of Economic Growth"` von 1956 entwickelte er ein Wachstumsmodell basierend auf einer gesamtwirtschaftlichen Produktionsfunktion. Die beiden Produktionsfaktoren Arbeit und Kapital werden in einem flexiblen Verhältnis eingesetzt und führen zu einer gleichgewichtig wachsenden Wirtschaft. Dabei zeigt das sogenannte Solow-Modell in seiner Einfachheit die Bedeutung des technischen Fortschritts für die ökonomische Entwicklung eines Landes und beschreibt den gleichgewichtigen Zustand einer Volkswirtschaft, bei dem die Abschreibung und das Bevölkerungswachstum genau durch die Ersparnis kompensiert wird. In diesem Gleichgewicht verändert sich die Kapitalintensität nicht mehr.\footnote{Nach einem Anpassungswachstum verändert sich die Kapitalintensität pro Kopf $k(t)$ nicht mehr über die Zeit, deshalb gilt $\dot{k}=0$.} Das Modell setzt sich zunächst aus einer Produktionsfunktion und einem Bewegungsgesetz zusammen.
%
	\begin{equation}
		Y=A K^\alpha L^{1-\alpha}
	\end{equation}
%
Das Gut bzw. Volkseinkommen $Y$ wird mit den Produktionsfaktoren Kapital und Arbeit hergestellt. Die Produktionselastizität $\alpha <1$ beschreibt abnehmende Grenzerträge des Kapitals und $A$ ist ein Produktionsparameter. \\ Das Bewegungsgesetz beschreibt die Abhängigkeit der Kapitalakkumulation von den Investitionen, die sich aus der Ersparnis $sY$ ergibt, und der Abschreibung auf das Kapital.
%
	\begin{equation}
		\dot{K}=sY-\delta K
	\end{equation}\label{Bewegungsgesetz Solow}
%
Dabei ist $\dot{K}$ das aggregierte Sparen und entspricht der aggregierten Investition, $\delta K$ beschreibt die aggregierte Abschreibung \cite{Solow.1956}.\\ Die Kernaussage des Solow-Modells ist, dass langfristiges Wirtschaftswachstum nicht durch ökonomische Bedingungen herbeigeführt wird. Das Pro-Kopf-Einkommen ${Y}/{L}$ kann nur dann wachsen, wenn auch der Produktivitätsparameter $A$ wächst. Dieser wird auch als technischer Fortschritt bezeichnet, der jedoch weder erklärt noch begründet wird. Langfristig ist Wirtschaftswachstum nur dann möglich, wenn es zu technischem Fortschritt kommt. Neben diesem Ergebnis zeigt Robert Solow erstmals, dass eine Volkswirtschaft intrinsisch bestrebt ist Stabilität zu erreichen.\\ Bis zur Entwicklung seines Modells galt der Faktor Kapital als limitierend für das Wirtschaftswachstum.\footnote{Als Beispiel dient hier das Harrod-Domar Wachstumsmodell \cite{Harrod.1939,Domar.1946}.} Basierend auf den Gedanken Ricardos zeigt Solow, dass ohne technischen Fortschritt eine Kapitalsättigung und somit eine Stagnation eintreten wird \cite{Solow.1956} \\ Die Ergebnisse seiner Arbeit belegte Solow selbst im Jahr 1957 empirisch am Beispiel der USA. Er argumentiert, dass nicht der erhöhte Einsatz von Kapital und Arbeit die Entwicklung förderten, sondern rund 90 Prozent des Wachstums durch technischen Fortschritt verursacht wurden. Dies gelang ihm mit Hilfe des Solow-Residuums. Dieser Term, auch als Totale Faktorproduktivität bezeichnet, beschreibt den Zuwachs der Produktivität, der weder durch eine erhöhte Kapitalzufuhr, noch durch zusätzliche Arbeit hervorgerufen wird und sich demnach nur auf den technischen Fortschritt zurückführen lässt.\\ 
%
Das Solow-Model ist der Ausgangspunkt vieler weiterer Wachstumstheorien und Strömungen, die auf den folgenden Seiten skizziert werden \cite{Aghion.2015}.
%
\subsubsection*{Ramsey-Modell}
In seinem dynamischen Model maximiert \cite{Ramsey.1928} die Wohlfahrt über einen unendlichen Zeithorizont intertemporal. Dabei unterscheidet sich seine Arbeit von der Solows durch die Annahme hinsichtlich der Beschaffenheit der Sparquote. Im Solow-Modell ist diese konstant und somit exogen, wohingegen Ramsey sie endogenisiert. Darin liegt auch der Kern seines Modells: die Konsum- bzw. Sparentscheidung der Haushalte. Sein endogenes Wachstumsmodell bestimmt den optimalen Konsumpfad in Form der Keynes-Ramsey-Regel, indem der Nutzen intertemporal maximiert wird, ergibt sich die optimale Wachstumsrate des Konsums \cite{Ramsey.1928}.
\bigskip


\cite{Solow.1956} und \cite{Ramsey.1928} stehen stellvertretend für die exogenen Wachstumsmodelle, die die Ursachen des technischen Fortschirtts vernachlässigen. Diese vorhandenen Erklärungsdefizite der exogenen Modelle versuchen die endogenen Modelle zu beheben. 


\subsection{Endogene Wachstumsmodelle}
Bis zu den neueren Wachstumstheorien oder auch endogenen Wachstumstheorien wurden weder die Möglichkeit unvollständiger Konkurrenz noch Externalitäten als Einflussfaktoren auf das Wirtschaftswachstum berücksichtigt. Externe Effekte durch Investitionen in Human- oder Sachkapital können zu einem gesamtwirtschaftlich langfristigen Wachstum führen, unabhängig davon, ob der Effekt intraindustriell eine Branche betrifft, oder aber interindustriell branchenübergreifend wirkt. Das hier vorherrschende Beispiel für einen positiven externen Effekt entsteht durch zunehmende Bildung, denn ein höherer Bildungsstand verbessert nicht nur die eigene Produktivität im Berufsleben, sondern trägt auch zur Verbreitung von Wissen bei, wie durch die Weitergabe an die nächste Generation.\\
Wird in der Theorie von unvollständigem Wettbewerb ausgegangen, birgt dies für Unternehmen Anreize den technischen Fortschritt zu beschleunigen, um von Monopolmacht profitieren zu können. \\
%
Eine weiteres Charakteristika endogener Wachstumsmodelle ist, dass sie nicht von abnehmenden Grenzerträgen des Kapitals ausgehen.\\
%
\cite{Gandolfo.1998}s \citeyear{Gandolfo.1998} Struktur, Wachstumsmodelle hinsichtlich ihrer Wachstumsursachen, Faktorakkumulation und technischem Fortschritt, zu untergliedern kann auch bei den endogenen Modellen angewandt werden. Die folgenden Abbildung \ref{endoWachstumsmodelle} spezifiziert die Ursache und ordnet entsprechend charakterisierende Modelle zu \cite{Frenkel.1999}.\\
%
	\begin{figure}[h!]
		\centering 
		\begin{tabular}{@{}r@{}} 
			\psfrag{e}{X} 
		%	\includegraphics[width=0.78\textwidth]{figure/Abbildungen/uebersichtEndogene.eps}
		\end{tabular}  
		\quelle{\textbf{Quelle:} Entwurf in Anlehnung an Frenkel (1999)}
		\caption{Übersicht endogener Wachstumsmodelle}\label{endoWachstumsmodelle}
	\end{figure}
%
Endogenen Wachstumsmodelle werden von \cite{Frenkel.1999} in zwei Strömungen untergliedert. Wird der Technologieparameter als konstant angenommen, ist Wachstum auf die Kapitalakkumulation zurückzuführen. Diese Modelle zeigen, dass auch ohne technischen Fortschritt das Grenzprodukt des Kapitals nicht abnimmt. Die Zweite Strömung endogenisiert den technischen Fortschritt, indem Innovationen aktiv angestrebt werden \cite{Frenkel.1999}. Beiden Strömungen ist jedoch gemein, dass in diesen Modellen  die Wissenschaftler die Möglichkeit haben auf das Wissen vorangegangener Generationen zurückzugreifen, aus diesen zu lernen und das Wissen weiter aufzubauen. Der endogene Faktor besteht in der Weitergabe des Wissens, also dem daraus resultierenden augenblicklichen Wissensstand und nicht in einer erhöhten Investitionstätigkeit in den Forschungssektor \cite{Romer.1990,Rebelo.1991}.
%
\subsubsection{Endogene Wachstumsmodelle mit konstantem Technologieparameter}
%
Wird von einer Linearität zwischen dem Kapital und dem Volkseinkommen ausgegangen, dann handelt es sich um eine konstante Kapitalproduktivität, die ein abnehmendes Grenzprodukt des Kapitals ausschließt, so wie im AK-Modell. 
%
\subsubsection*{AK-Modell}
%
Das AK-Modell ist ein weiteres richtungsweisendes Modell, eines der er\-sten endogenen Wachstumsmodelle in Hinblick auf den technischen Fortschritt und basiert auf dem Papier von \cite{Rebelo.1991}. Es unterscheidet sich dahingehend vom Solow-Modell, dass der technische Fortschritt den abnehmenden Grenzerträgen entgegenwirkt und diesen "`Wachstumshemmer"' unterbindet.  Der technische Fortschritt wird nicht einzeln aufgeführt, sondern bedingt die Akkumulation von Humankapital, die ein Bestandteil der allgemeinen Kapitalakkumulation ist. Die Produktionsfunktion besteht, wie der Name des Modells bereits sagt, aus Kapital $K$ und der Konstanten $A$, jedoch ohne abnehmende Erträge.
%
	\begin{equation}
		Y=AK
	\end{equation}
%
Er modelliert ein endogenes Wachstumsmodell, obwohl er von konstanten Skalenerträgen ausgeht. Denn \cite{Rebelo.1991} erachtet, anders als \cite{Romer.1990}, steigende Skalenerträge als nicht notwendig, um Wachstum zu generieren, sofern für die Investitionsgüterproduktion nur akkumulierbare Einsatzfaktoren eingebracht werden \cite{Rebelo.1991}. \\
%
Die Kapitalakkumulation entspricht der des Solow-Modells und ist demnach der Gleichung \eqref{Bewegungsgesetz Solow} zu entnehmen. Die Wachstumsrate $g$ der Ökonomie beschreibt das langfristige Wachstum und wird durch eine hohe Ersparnis des BIPs hervorgerufen.
%
	\begin{equation}
		g=\frac{\dot{K}}{K}=s\frac{Y}{K}-\delta=sA-\delta
	\end{equation}
%
Das Modell kann sowohl auf industrialisierte Länder als auch auf Entwicklungsländer angewendet werden. Der beschriebene  Wachstumsprozess ist unabhängig von der Entwicklung der übrigen Welt und schließt zunächst den Handel mit anderen Volkswirtschaften aus. Wird dieser berücksichtigt, dann verändern sich die Bedingungen der Kapitalakkumulation und das Modell müsste modifiziert werden.\\ Das AK-Modell ist immer dann hilfreich, wenn die Unterscheidung von Innovation und Akkumulation irrelevant ist. Da in diesem Rahmen unter anderem der Einfluss von Innovationen untersucht werden soll, werden im folgenden die innovationsbasierten Wachstumsmodelle genauer betrachtet \cite{Aghion.2015}.
%
\subsubsection*{Uzawa-Lucas-Modell} In diesem Modell verhindert die Akkumulation von Sach- und Humankapital ein abnehmendes Grenzprodukt, jedoch nicht durch eine Ausweitung des Kapitals, wie dies zuvor bei der Faktormehrung exogener Modelle der Fall war, sondern durch eine Erhöhung der Produktivität des Kapitals. Bildung stellt in dem Modell von Uzawa-Lucas den Hauptgrund für die Akkumulation von Humankapital dar und erklärt damit langfristiges Wachstum.\footnote{Eine ausführliche Darstellung folgt in Kapitel \ref{Papier2}.}
%
\subsubsection*{Learning-by-doing}
Die dritte Strömung endogener Modelle mit konstantem Technologieparameter bilden sogenannte "`Learning-by-doing"' Modelle. Auch hier steigt die Produktivität der Faktoren an und das abnehmende Grenzprodukt des Kapitals wird durch Externalitäten unterbunden \cite{Arrow.1962}. Das hier thematisierte Learning-by-doing führt zu den positiven Externalitäten, dem informellen Lernen.
%
\subsubsection{Endogene Wachstumsmodelle mit variablem Technologieparameter}
Der Schwerpunkt dieser Modelle liegt auf der Endogenisierung des technischen Fortschritts. Indem die Annahme des vollständigen Wettbewerbs aufgehoben wird, sind die Unternehmen bestrebt durch Forschung und Entwicklung, das Gut oder den Produktionsprozess weiter zu entwickeln, um zusätzliche Gewinne durch Monopolmacht abschöpfen zu können. Demnach ist der Technologieparameter variabel und zurückzuführen auf innovationsbasierte Ansätze. 
%
\subsubsection*{Romer-Modell}
Ein Vertreter der innovationsbasierten Wachstumsmodelle, Paul Romer, verfolgt diesen Schwerpunkt, den des endogenen technischen Fortschritts, im Zwischengutsektor. Romer wurde 1955 in Denver geboren und begründete die endogene Wachstumstheorie \cite{Lin.2007}, da er das Modell Solows um den Faktor Wissen erweitert und dadurch den Ansatz der Wissenschaft neu gestaltete. Er sieht den Motor des Wachstums im Wissen und der Ideenentwicklung, da Wissensvermehrung intertemporale externe Effekte mit sich bringt. Wissen als immaterielles Gut weist die Eigenschaft nicht abnehmender Grenzerträge auf und kann somit nicht aufgebraucht werden. Der technische Fortschritt als direkte Wachstumsquelle wurde bislang nicht in den theoretischen Modellen berücksichtigt und modelliert. Er galt als exogen und wurde als nichterklärbar gegeben hingenommen.
Romers Ansatz zeigte, dass der Faktor Wissen technologischen Fortschritt generierte und es gelang ihm diesen in die Modellwelt zu integrieren und dadurch letztendlich auch zu kalkulieren. In seinem Modell erhöhen horizontale Innovationen im Zwischengütersektor die Produktivität, was zu dauerhaftem Wachstum führt.\\
%
Seine Gedanken formulierte \cite{Romer.1990} in seinem endogenen Wachstumsmodell des Aufsatzes "`Endogenous Technical Change"', indem er ein drei Sektoren Modell vorstellt bestehend aus dem Forschungs- und Entwicklungssektor, dem Zwischengutsektor und dem Endproduktsektor. Der stetige Wissenszuwachs durch Forschungsaktivitäten führt zu zunehmender Produktvielfalt im Zwischengutsektor und bewirkt langfristig einen Anstieg des Einkommens, aufgrund der stärkeren Spezialisierung und Arbeitsteilung. Dafür notwendig ist jedoch Humankapital, also Fähigkeiten der Menschen, die dieses Wissen erzeugen. Desto mehr Humankapital im Forschungs- und Entwicklungssektor eingesetzt wird, desto mehr Produktvarianten der Zwischengüter, Innovationen, werden entwickelt und desto höher ist das Wachstum \cite{Romer.1990}.\\
%
Der Produktionsprozess des technischen Fortschritts durch Innovationen regt zwar das Wirtschaftswachstum an, jedoch müssen auch die Kosten dieser berücksichtigt werden. Je aufwendiger und somit kostenintensiver ein Innovationsprozess ist, desto eher kann eine Innovation vor Nachahmern geschützt werden.\footnote{Die Innovationen im Zwischengutsektor führen zu der Marktform der monopolistischen Konkurrenz. Ein patentunabhängiger Schutz der Monopolmacht sind die Kosten für die Entwickung bzw. Nachahmung der Innovation.} Ist eine Innovation jedoch zu kostenintensiv, übersteigen die Kosten die möglichen resultierenden Gewinne, dann wird sie nicht produziert und eingesetzt. \\ Ein formal detaillierterer Blick auf das Romer Modell zeigt den Prozess der Entwicklung von Produktvariationen durch Innovationen. Diese sind nicht zwingend qualitativ besser, führen jedoch zu einem höheren Produktivitätswachstum. \\ Die Produktionsfunktion \eqref{Produktionsfunktion Dixit} basiert auf der des Modells von \cite{Dixit.1977} und beschreibt die Produktion verschiedener Varianten $i$, mit $i=[0;N_t]$, eines Zwischenprodukts mit dem Produktionsfaktor Kapital $K_{it}$.
%
	\begin{equation}
		Y_t= \sum_{i=0}^{N_t} K_{it}^\alpha 
	\end{equation}\label{Produktionsfunktion Dixit}
%
Der Kapitalstock $K_t$, kann aufgrund der Symmetriebedingung gleichmäßig auf $N_t$ Varianten aufgeteilt werden und führt zu folgender Formulierung der Produktionsfunktion.
%
	\begin{equation}
		Y_t=N_t^{1-\alpha}K_t^\alpha
	\end{equation}
%
Laut dieser Gleichung ist hier der Produktivitätsparameter der Ökonomie der Grad der Produktvielfalt $N_t$. Je größer der Grad ist, desto größer ist das Produktionspotenzial eines Landes. Der Kapitalstock wird auf eine größere Zahl von Produktvarianten aufgeteilt, wobei jede durch abnehmende Grenzerträge geprägt ist. Dauerhaftes Wachstum resultiert hier aus der stetigen Entwicklung neuer Produktvarianten. Das Modell zeigt die Rolle technologischer Spillover-Effekte im Sinne der Technologiediffusion.\\
%
In diesem Modell führt eine Innovation zu neuen Produktvarianten, dabei wird der Prozess der schöpferischen Zerstörung nicht berücksichtigt. Das Ersetzten "`alter"' Produkte durch neu entwickelte und qualitativ hochwertigere ist Kern, der schumpeterianischen Wachstumsmodelle.
%
\subsubsection*{Modelle nach dem Gedanken Schumpeters}\label{sec:schumpeter}
Neben dem Romer-Model zählen auch die Modelle zu den innovationsbasierten Modellen, die dem Gedanken Schumpeters folgen. Der Ansatz beruht auf dem Prozess der schöpferischen Zerstörung, deren Idee von ihm erstmals in seiner Monographie von 1912 entwickelt wurde \cite{Schumpeter.1934a}. Neue qualitätsverbessernde Innovationen ersetzten vorherige und zerstören somit deren Bedeutung. Dabei steht die Entwicklung von Innovationen im Vordergrund und Wachstum entsteht als unbeabsichtigtes Nebenprodukt.\\
%
Zu dieser Gruppe endogener Modelle zählt auch das Wachstumsmodell von \cite{Aghion.1992}, das auch die vertikalen Innovationen betont. Es basiert auf dem Ansatz Schumpeters mit dem Konzept der Schöpferischen~Zerstörung. Sie untersuchen Wachstumseffekte, aus denen Innovationen resultieren, die auf Grund von Wissensakkumulation entstanden sind. Anders als im Romer-Modell ersetzt jede Innovation eine vorherige und es gibt keine zusätzliche Variante des Gutes. Einerseits entmutigt dieser fortwährende Erneuerungsprozess die Unternehmer weitere Forschung zu betreiben, da sie einer ständigen Bedrohung der Veralterung ausgesetzt sind. Andererseits motiviert der anhaltende Wettbewerb die Unternehmen zu Entwicklung effizienterer Produktionsprozesse bzw. verbesserter Zwischengüter, um die Monopolstellung auf einem Markt zu erlangen \cite{Aghion.1992}.\\ 
%
Das folgende Ein-Sektor-Modell geht auf die bereits erwähnte Arbeit von \cite{Aghion.1992,Aghion.1998} zurück und berücksichtigt den dort angesprochenen Austausch von Gütern durch qualitativ hochwertigere Produktvarianten. Danach bleibt die Summe der Produkte gleich und weitet sich nicht mit jeder weiteren Innovation aus. Die Grundidee basiert auf der Betrachtung einzelner Industrieebenen $i$ mit der allgemeinen spezifischen Produktionsfunktion:
%
	\begin{equation}
		Y_{it}=A_{it}^{1-\alpha}K_{it}^\alpha \qquad \text{mit}\quad 0 < \alpha < 1 \label{Produktionsfunktion Industrien Schumpeter}
	\end{equation}
%
Auch hier ist $A_{it}$ wieder der Produktivitätsparameter zum Zeitpunkt $t$ der Industrie $i$ und führt neben einem Zwischengut $K_{it}$ zur Produktion des Gutes $Y_{it}$. Das Modell beschreibt die Herstellung eines Endproduktes durch den Einsatz eines Zwischengutes. Der technische Fortschritt liegt also im Zwischengutsektor.  Ein Zwischenprodukt wird von einem Innovator hergestellt und ersetzt die vorherige Innovation. Je schneller eine Volkswirtschaft in diesem Modell wächst, desto höher ist die Fluktuation bei den technologisch führenden Firmen. \\ Wachstum entsteht somit durch die Verbesserung der Produktqualität. Formal bedeutet dies, dass der Produktivitätsparameter $A_t$ von $A_{t-1}$ auf $A_t=\gamma A_{t-1}$, mit $\gamma>1$, steigt und somit direkt aus innovativen Tätigkeiten resultiert. Für die Entwicklung dieser Neuerungen muss es, neben dem Produktionssektor, auch einen Forschungssektor geben. Die Kosten für die Forschung entsprechen den verwendeten Endprodukten, die als  Faktoreinsatz fungieren. Mit zunehmendem Forschungsaufwand, der zu steigenden Kosten führt, erhöht sich die Wahrscheinlichkeit einer erfolgreichen Innovation. \\
%
Die Motivation in Forschung zu investieren liegt in der Möglichkeit Monopolmacht zu erlangen und höhere Einnahmen zu generieren. Schumpeter war der erste, der die Rolle des Monopols thematisierte in Bezug auf Innovationen und der Entstehung im Forschungs- und Entwicklungssektor. \\ Unter der Annahme, dass alle Industrien eines Landes identisch sind, kann Gleichung \eqref{Produktionsfunktion Industrien Schumpeter} auch auf aggregierter Ebene formuliert werden.
%
	\begin{equation}
		Y_t=A_t^{1-\alpha}K_t^\alpha
	\end{equation}
%
Wird neben der Innovation auch die Möglichkeit einer Imitation berücksichtigt, wird davon ausgegangen, dass bereits ein gewisser Bestand an Innovationen vorhanden ist, das gegenwärtige technische Wissen. Die langfristige Wachstumsrate $g_t$ entspricht der Wachstumsrate des arbeitsvermehrenden Produktivitätsfaktors $A_t$ und wird im folgenden genauer betrachtet. Das technische Wissen ist öffentlich verfügbar und kann durch erfolgreiche Innovatoren erweitert werden \cite{Aghion.1992,Aghion.1998}.  Bei einer Innovation verändert sich der Technologieparameter $A_t$ um das $\gamma$-Fache und die Welttechnologiegrenze $\bar{A}_t$ wird um die Neuerung erweitert. Handelt es sich um eine Imitation, dann verändert sich nur der lokale Technologiebestand, indem eine Produktvariante nachgeahmt wird, die bereits auf dem Weltmarkt existiert, nicht jedoch in dem betrachteten Land. Beide Prozesse bilden den lokalen technologischen Wissensstand eines Landes und können formal folgendermaßen ausgedrückt werden: 
%
	\begin{equation}
		\dot{A_t}= A_{t+1}-A_t=\mu_n(\gamma-1)A_t+\mu_m(\bar{A}_t-A_t)
	\end{equation}
%
Bei $\mu_n$ und $\mu_m$ handelt es sich um die Frequenz bzw. Intensitäten der Innovations-  bzw. Imitationsentwicklung, die exogen sind. Daraus lässt sich die Wachstumsrate des technischen Fortschritts ableiten.
%
	\begin{equation*}
		g_t=\hat{A_t}= \frac{A_{t+1}-A_t}{A_t} = \mu_n(\gamma-1)+\mu_m(\frac{\bar{A}_t}{A_t}-1)
	\end{equation*}
%
Die Relation $A_t/\bar{A}_t$ beschreibt den Abstand zur Welttechnologiegrenze $a_t$ und lässt somit Aussagen zum relativen technologischen Entwicklungsstand zu \cite{Aghion.1992,Aghion.1998}.
%
	\begin{equation}
		g_t=\hat{A_t}=\frac{A_{t+1}-A_t}{A_t} = \mu_n(\gamma-1)+\mu_m(a_t^{-1}-1)
	\end{equation}
%
Dieses schumpeterianische Grundmodel eignet sich besonders zur Analyse der Reaktion des Abstands zur WTG durch die jeweilige Wachstumsrate eines Landes. Interessant ist dabei der Aspekt der Konvergenz zur globalen Grenze, die sich durch verschiedene wirtschaftspolitische Maßnahmen justieren lässt. \\ 
Ein Fazit des Ein-Sektor-Modells nach Schumpeter ist, dass sich die langfristige Wachstumsrate aus den relativen Häufigkeiten der entwickelten Innovationen ergibt, wobei die Reichweite oder auch Wirkungskreis der Innovation ebenfalls berücksichtigt werden muss. Bei dem Ein-Sektor-Modell wird nur ein Gut ersetzt, wohingegen im mehrsektoralen Modell mehrere Produkte durch Innovationen erneuert werden können. Der entscheidende Unterschied zum Ein-Sektor Modell liegt darin, dass eine Innovation nicht mehr bedingt durch Zufall entwickelt wird. Sofern in einem Sektor  nicht erfolgreich innoviert wird, kommt es in einem anderen Sektor zu einer erfolgreichen Innovation mit der entsprechenden Wahrscheinlichkeit von $\nu$. Daraus ergibt sich die durchschnittliche aggregierte Produktivität in der multisektoralen Variante von:
%
	\begin{equation}
		A_t=\nu A_{1t}+(1-\nu)A_{2t} \footnote{Die hier angeführten Indizes 1 und 2 unterscheiden die Produktivitäten zweier Sektoren 1 und 2 voneinander.}
	\end{equation}
%
Auch dieses schumpeterianische mehrsektorale Modell folgt dem Ansatz von \cite{Aghion.1998}. Ein anderes schumpeterianisches Modell von \cite{Reinganum.1985} beschreibt die andauernde Entwicklung von Innovationen als evolutionsähnlichen Prozess im Sinne der Schöpferischen Zerstörung.\\
%
Die hier kurz angerissenen Modelle sind Vorreiter des in Kapitel \ref{Papier1} behandelten Wachstumsmodells. Dieses ist demnach in die Gruppe der innovationsbasierten endogenen schumpeterianischen Wachstumsmodelle einzubetten. 
%
Zusammenfassend lässt sich festhalten, je mehr eine Innovation die Produktivität steigert, desto stärker steigt die Wachstumsrate. Demzufolge sollte als wachstumsfördernde Maßnahme vermehrt in den Forschungssektor investiert werden. Dies wiederum steigert die Nachfrage nach Wissenschaftlern in diesem Bereich, die nur durch die zusätzliche Ausbildung der Arbeiter befriedigt werden kann. Ein weitsichtiges strategisches Vorgehen ist demnach der Ausbau des Bildungssektors, damit für wachstumsfördernde Maßnahmen ausreichend qualifizierte Arbeit vorhanden ist \cite[Kapitel 4]{Aghion.2015}.\footnote{Diesem kausalem Zusammenhang folgt auch der Hauptteil dieser Arbeit, zunächst wird der Ausbau des Bildungssektors durch Außenhandel stimuliert. Das dadurch entstehende erhöhte Angebot qualifizierter Arbeit ist für innovierende und imitierende Tätigkeiten notwendig, da andernfalls eine Weiterentwicklung des technischen Entwicklungsstandes gehemmt werden würde.} \\
%
Bildung und die damit einhergehende Humankapitalakkumulation steht im Vordergrund des folgenden Abschnitts, der die Vielfalt der unterschiedlichen Vorgehens- und Betrachtungsweisen darlegt. 
%
\subsubsection{Humankapitaltheorien}
Die Humankapitaltheorien stellen eine Unterkategorie der Wachstumstheorien dar, die sich mit der Akkumulation von Humankapital beschäftigen und dadurch Wirtschaftswachstum erklären. Hierzu zählt auch das Uzawa-Lucas-Modell, dass im Laufe dieser Arbeit bereits erwähnt wurde. Die verschiedenen Theorien begründen die Unterschiede von Bildung und zeigen wie deren Einfluss auf das Wirtschaftswachstum interpretiert werden kann.
%
\subsubsection*{Mincer Modell}
Die Humankapitaltheorie geht ursprünglich zurück auf die Arbeiten von \cite{Becker.1965} und \cite{Mincer.1974}, die zwei Schwerpunkte berücksichtigten. Zum einen die  produktionssteigernde Rolle des Humankapitals für den Produktionsprozess und zum anderen die Motivation der Arbeiter in Humankapital zu investieren. So unterscheiden sie zwischen einer Grundausbildung und einer berufsbegleitenden Ausbildung. Dabei gilt jegliche Bildung, die vor der ersten Beschäftigung in einem Unternehmen genossen wurde als Grundausbildung. Die Opportunitätskosten eines weiteren Schuljahres entsprechen dem entgangenen Verdienst durch eine Anstellung \cite{Mincer.1974}. Eine Ausbildung während eines Angestelltenverhältnisses als eine Art Zusatzausbildung neben dem Beruf wird auch als Training-on-the-Job bezeichnet \cite{Acemoglu.2009}. Den Schwerpunkt des Mincer Modells bildet dabei die Grundausbildung.
%
\subsubsection*{Ben-Porath-Modell}
Dieses Modell der Humankapitaltheorie unterscheidet sich von dem Mincer Modell, indem auch Bildungsmöglichkeiten während einer Berufstätigkeit ausgeführt werden können und sich diese nicht ausschließlich auf die Zeit vor dem Berufsleben beschränken. Der Fokus der Arbeit von \cite{BenPorath.1967} liegt demnach auf dem Training-on-the-Job. Dabei wird auch eine Minderung des Humankapitals berücksichtigt, weil  davon ausgegangen wird, dass durch den Einsatz von Maschinen das vorher noch notwendige Humankapital obsolet wird \cite{BenPorath.1967,Heckman.1998,Guvenen.2012,Manuelli.2014}. Die Bedeutung des Modells ist vor allem darauf zurück zuführen, dass neben der Schulausbildung eine Vielzahl weiterer Möglichkeiten existieren in Humankapital zu investieren. Außerdem kommt er zu der These, dass Volkswirtschaften mit hohen Ausgaben für Schulbildung ebenso hohe Ansprüche bezüglich der berufsbegleitenden Weiterbildungsmöglichkeiten haben und diese durch das System der gesicherten Grundausbildung nicht gemildert werden \cite{BenPorath.1967}.
%
\subsubsection*{Uzawa-Lucas-Modell}
Das Uzawa-Lucas-Modell, beschäftigt sich ebenfalls mit wirtschaftlichem Wachstum, welches durch die Humankapitalakkumulation bedingt ist und deswegen als Motor des Wachstums bezeichnet wird \cite{Lucas.1988}. Im Rahmen des AK-Modells \cite{Rebelo.1991} betrachtet \cite{Lucas.1988}, inspiriert durch den Aufsatz von \cite{Becker.1964} und basierend auf dem Modell von \cite{Uzawa.1965}, Humankapital als einzigen Einsatzfaktor im Bildungssektor und untersucht das dadurch angeregte Wirtschaftswachstum. Sowohl im Uzawa-Lucas-Modell als auch im AK-Modell wird Wachstum durch Faktormehrung generiert. Im AK-Modell wird dauerhaftes Wachstum durch Kapitalakkumulation hervorgerufen, wohingegen Lucas zwischen Sach- und Humankapital differenziert und es wird neben Humankapital auch hier auch physisches Kapital akkumuliert. Beide, Sach- und Humankapital, verhalten sich komplementär zueinander, denn durch den Anstieg von physischem Kapital steigt die Nachfrage nach qualifizierter Arbeit stärker an, als nach relativ unqualifizierter Arbeit. Das bedeutet, dass die maximale Produktivität einer Volkswirtschaft dann erreicht wird, wenn beide ausgeglichen sind und gleichmäßig ansteigen.\\
%
Die Haushalte müssen sich zwischen der Arbeit im Konsumgutsektor und Bildung entscheiden. Dadurch entsteht ein trade-off zwischen heutigem und morgigem Konsum, da neben der Erwerbstätigkeit in Bildung investiert werden kann. Durch einen gegenwärtigen Verzicht auf Lohneinkommen und stattdessen einer Investition in Bildung ist der zukünftige Konsum höher. Mit diesem Zusammenhang wird sich in aller Ausführlichkeit in Kapitel \ref{Papier2} auseinander gesetzt und daher an dieser Stelle nicht näher beschrieben.
%
\subsubsection*{Modell von Nelson und Phelps}
Eine vollkommen andere Perspektive auf die Bedeutung des Humankapitals etablieren \cite{Nelson.1966}. Zwar vertreten sie auch die Ansicht, dass eine korrekte Ergründung von Wachstum mit der Einbeziehung von Bildung einhergehen muss, jedoch ist Humankapital  hier kein direkter Einsatzfaktor, der die Produktivität erhöht.\footnote{Eine ähnliche Idee ist auch auf die Arbeit von \cite{Schultz.1964,Schultz.1975} zurückzuführen.} In diesem Ansatz begünstigt Humankapital nicht die Produktivität bekannter Aufgaben, sondern ermöglicht zu der Fähigkeit unbekannte Abläufe, Technologien und Güter zu adaptieren. Wachstum wird erzeugt durch die produktivitätssteigernde Implementierung von Imitationen. \\
%
Dieser Unterschied in der Auffassung ist auch in der Modellierung der darauf aufbauenden Theorie gut sichtbar. Denn Humankapital hat keinen direkten Einfluss auf die Produktionsfunktion und bedingt nur den technologischen Wissensstand eines Landes durch die Implementierung bereits vorhandener Technologien der Welttechnologiegrenze. Dabei differenzieren sie erstmals zwischen unterschiedlichen Fähigkeiten und lassen eine Gewichtung dieser zu. Bislang wurde davon ausgegangen, dass mit steigender Humankapitalausstattung die Produktivität aller Aufgaben zunimmt. Jedoch unterscheiden \cite{Nelson.1966} zwischen innovierenden und adaptierenden Tätigkeiten und Fähigkeiten. 
Dieses Modell beschreibt erstmals den direkten Einfluss von Humankapital auf das Wirtschaftswachstum \cite{Nelson.1966}.\\
%
Dargestellt in einem schlichteren Ansatz nach \cite{Nelson.1966} in einer Variation von \cite[Kapitel 10]{Acemoglu.2009}, ist die einzige veränderbare Größe der lokale Technologieparameter $A$, die WTG ist exogen gegeben. Der lokale technologische Wissensstand und somit eine Verbesserung der Technologien ergibt sich aus zwei Komponenten, der intrinsischen Veränderung der Produktivität, welche Produktivitätswachstum wie beispielsweise durch learning-by-doing darstellt und durch die Nachahmung fortschrittlicherer neuer Technologien der WTG. Der Erfolg der Nachahmung wird dabei wesentlich von der durchschnittlichen Humankapitalausstattung eines Arbeiters beeinflusst. Ist der Arbeiter nicht ausreichend qualifiziert, dann können keine Technologien der WTG adaptiert und implementiert werden. Je besser die Unternehmer ausgebildet sind, desto eher kann adaptiert werden. Dadurch ergibt sich die Möglichkeit im Entwicklungsprozess zu anderen führenden Ländern aufzuschließen. Empirisch belegt wurde die Theorie unter anderem von \cite{Foster.1995} am Beispiel der Produktivität im Landwirtschaftssektor.
%
\subsubsection*{Modell von Benhabib und Spiegel}
In dem Aufsatz von \cite{Benhabib.1994} wird das Modell von \cite{Nelson.1966} erweitert und zeigt, dass neben imitativen Tätigkeiten auch die Möglichkeit besteht nahe der Welttechnologiegrenze Innovationen zu entwickeln. In ihrer Regressionsanalyse stellten sie einen positiv signifikanten Zusammenhang zwischen der Wachstumsrate und dem Humankapitalbestand fest. Humankapital beeinflusst nach \cite{Benhabib.1994} nicht nur das Wachstum des Pro-Kopf-Einkommens, sondern auch das Wachstum der totalen Faktorproduktivität positiv.  Außerdem belegen sie, dass der Abstand zur Welttechnologiegrenze für das Wachstum relevant ist \cite{Benhabib.1994}. Dieser Ansatz zeigt also einen stärkeren Zusammenhang zwischen Wirtschaftswachstum und Humankapitalniveaus als zwischen dem Wirtschaftswachstum und der Veränderung des Humankapitals. Denn die Adaption neuer Technologien beeinflusst die Produktivität deutlich stärker als eine Produktivitätserhöhung bereits bekannter Aufgaben \cite{Benhabib.1994}. Dieser Gedanke wird auch in Kapitel \ref{Papier1} aufgegriffen, in dem das Humankapitalniveau die Veränderung der Produktivität einer Volkswirtschaft bedingt, sowie deren Imitations- bzw. Innovationsmöglichkeiten. 
%
\subsubsection*{Modell von Krüger und Lindahl}
\cite{Krueger.2001} hingegen untersuchen in den OECD-Ländern, dass Bildung zwar zum Aufholprozess, nicht jedoch zur Ausweitung der WTG beiträgt.
Sie zeigen die Relevanz der Zusammensetzung des Humankapitalbestandes und der Lage zur WTG eines Landes für das Wirtschaftswachstum. Dabei widerlegten sie einige Ergebnisse von \cite{Benhabib.1994} und stellten fest, dass Wachstum und Humankapital nur innerhalb von OECD-Ländern korreliert. In Ländern, die deutlich weniger weit entwickelt sind gilt dieser Zusammenhang nicht. Dies hebt zunächst eine gewisse Bedeutungslosigkeit der Humankapitalakkumulation auf den Wachstumsprozess eines Landes hervor, was durch ein kleines Gedankenspiel aus \cite{Krueger.2001} veranschaulicht werden soll. Es soll verdeutlichen, dass nicht nur die Ausstattung mit Humankapital wichtig ist, sondern es vor allem auf deren Zusammensetzung innerhalb eines Landes und den Entwicklungsstand eines Landes ankommt.\footnote{Sofern diese beiden Aspekte unabhängig voneinander sind.} Es werden zwei Länder betrachtet, die dieselbe Humankapitalausstattung vorweisen, sich jedoch hinsichtlich ihrer Zusammensetzung, also hinsichtlich der Qualifikationen ihrer Arbeitskräfte, unterscheiden. Land 1 sei in diesem Fall relativ reichlich mit sehr gut ausgebildeten Arbeitnehmern ausgestattet, wohingegen Land 2 relativ mehr traditionell weniger gut ausgebildete Arbeitskräfte vorweist. Je nach Lage zur WTG entwickelt sich das eine oder das andere Land schneller. Nahe der Technologiegrenze sind besser ausgebildete Arbeiter wichtiger, demnach wird Land 1 sich schneller entwickeln als Land 2, welches den gleichen Entwicklungsstand nahe der WTG hat. Handelt es sich bei beiden Ländern um weniger weit entwickelte Volkswirtschaften mit einem großen Abstand zur WTG, dann weist Land 2 das höhere Wachstumspotenzial auf mit einer reichlicheren Ausstattung weniger gut ausgebildeter Arbeitskräfte als Land 1. Also wird das Land, welches relativ reichlicher mit höher qualifizierten Arbeitskräften ausgestattet ist, schneller wachsen, wenn der Abstand zur WTG relativ gering ist. Wohingegen das Land, welches relativ reichlicher mit unqualifizierte Arbeitskräften ausgestattet ist, ein höheres Wachstum erreicht als das andere, wenn der Abstand zur WTG beider, relativ groß ist \cite{Krueger.2001}. So hängt das Wachstumspotential maßgeblich von der Lage zur WTG, sowie von der Zusammensetzung des Humankapitals ab. Mit zunehmender Nähe zur WTG nimmt die Bedeutung weniger qualifizierter Arbeitskräfte ab, wohingegen die der hochqualifizierten zunimmt.\\
%
Somit ist in diesem Kontext das Humankapital lediglich für den catching-up Prozess maßgeblich, jedoch nicht für Innovationstätigkeiten an der WTG. \cite{Krueger.2001} zeigen in ihrer Abhandlung, dass neben der Lage zur WTG  der Humankapitalbestand allein nicht ausreicht, um das Wachstum eines Landes prognostizieren zu können.\\
%
Werden Innovationen mit relativ mehr ausgebildeter Arbeit hergestellt als Imitationen, dann hat ausgebildete Arbeit einen größeren Effekt auf das Wachstum eines Landes, welches nahe der WTG liegt, als auf ein Land, mit einem größeren Abstand zur WTG. Es wurde empirisch belegt, dass es einen positiven Zusammenhang zwischen dem anfänglichen Bildungsniveau und dem anschließenden Wachstumsverlauf gibt \cite{Vandenbussche.2006}.\footnote{Dabei verwendeten sie Daten von 19 OECD Länder zwischen den Jahren 1960-2000 \cite{Vandenbussche.2006}.}.
%
\subsubsection*{Humankapitalexternalitäten - Vorteile urbaner Regionen}
Eine große Bedeutung kommt auch den Humankapitalexternalitäten zu. Neben dem Uzawa-Lucas-Modell, das Wissensexternalitäten durch Spillover-Effekte anführt, gibt es noch zahlreiche weiter positive Effekte. Ein weiteres Beispiel beschreibt das Modell von \cite{Jacobs.1970}, dass eine höhere Produktivität in urbanen Regionen begründet.
Wenn ein gesamtwirtschaftlich hoher Kapitalstock die Produktivität jedes einzelnen Arbeiters erhöht, dann kann dies auf Wissens-Spillover-Effekte zurückgeführt werden. Denn ein Ideenaustausch innerhalb der erwerbstätigen Bevölkerung ist wahrscheinlicher und stimuliert das ökonomsiche Wachstum eher in städtischen Regionen als in weniger stark besiedelten Regionen \cite{Azariades.1990,Lucas.1988}. Empirisch belegt hat die Existenz dieser Externalitäten erstmals die Arbeit von \cite{Rauch.}. Gefolgt von \cite{Acemoglu.2000}, die diese nicht für die unterschiedliche Bildungsniveaus amerikanischer Großstädte überprüften, sondern die These auf dadurch bedingte Bildungsunterschiede zwischen Staaten ausweiteten. Mit dem Ergebnis, dass die Humankapitalexternalitäten relativ klein sind und eher weniger Bedeutung beigemessen werden sollte.\footnote{\cite{Duflo.2004} und \cite{Ciccone.Apr} belegen diese Ergebnisse ebenfalls anhand der Daten von Indonesien und den USA.} Eineindeutig sind diese Ergebnisse jedoch nicht, denn \cite{Moretti.2004} zeigt wiederum einen großen Effekt der Externalitäten auf das ökonomische Wachstum.
