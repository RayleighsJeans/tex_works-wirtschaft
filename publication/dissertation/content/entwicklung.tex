\sectionmark{Entwicklungsstrategien}
\chapter{Au{\ss}enwirtschaftlich orientierte Entwicklungsstrategien}\label{Entwicklungsstrategien}
\chaptermark{Entwicklungsstrategien}

Ein Ziel dieser Arbeit ist es eine Entwicklungsstrategie zu bestimmen, die den Aufholprozess weniger weit entwickelter Volkswirtschaften durch Au{\ss}enhandel bedingt. Dazu werden anhand der Handelseffekte m{\"o}gliche Entwicklungsstrategien diskutiert.\\

Der Begriff Entwicklung beschreibt den Prozess einer positiven Ver{\"a}nderung von Zielgr{\"o}{\ss}en \citep[Kapitel 1]{Wagner.1995}. In wirtschaftswissenschaftlichen Zusammenh{\"a}ngen wird er h{\"a}ufig nicht klar vom Begriff des Wachstums unterschieden. \citet{Findlay.1984} grenzt beide Begriffe hinsichtlich des Resultats des Prozesses ab. Er begreift den Begriff Wachstum als einen eher unspezifischen Ausdruck, der in vielen Bereichen Anwendung findet. Wohingegen er bei Entwicklung von einer strukturellen oder qualitativen Verbesserung ausgeht \citep{Findlay.1984}.\\


In diesem Zusammenhang wird eine Entwicklungsstrategie mit einem Entwicklungsdefizit bzw. dem Entwicklungspotenzial eines Landes assoziiert und somit nur von relativ weniger weit entwickelten Volkswirtschaften verfolgt. Eine eindeutige Klassifikation der weniger weit entwickelten L{\"a}nder ist seit den 60er Jahren deutlich komplexer geworden, da solche L{\"a}nder nach einem Strategiewechsel weg von der Importsubstitution unterschiedliche Wege mit unterschiedlichem Erfolg gegangen sind.\footnote{Die Erl{\"a}uterung der Importsubstitutionsstrategie und weiterer Alternativen folgt nachstehend. Au{\ss}erdem liefert die Arbeit von \citet{Stern.1973} einen {\"U}berblick des Forschungsstandes bis in die 70er Jahre hinsichtlich protektionistischer Handelspolitik.} Nach \citet{Krugman.2015} trifft jedoch meist eins der folgenden strukturellen Merkmale auf die weniger weit entwickelten L{\"a}nder zu. Die staatliche Kontrolle, wie beispielsweise Handelsbeschr{\"a}nkungen die den Au{\ss}enhandel steuern, ist ein charakteristisches Merkmal weniger weit entwickelter L{\"a}nder. Ein weiteres Merkmal ist die Steuerung der Wechselkurse sowie eine sehr hohe Inflationsrate. Kennzeichnend ist auch, dass h{\"a}ufig liberale Finanzm{\"a}rkte von eher schwachen Kreditinstituten gef{\"u}hrt werden. Die Exporte werden vor allem durch landwirtschaftliche Erzeugnisse gepr{\"a}gt und des weiteren ist eine hohe Rate der Korruption bezeichnend.\newline


Handelspolitik und die damit verbundenen protektionistischen Ma{\ss}nahmen wurden und werden als Entwicklungsstrategie vieler weniger weit entwickelter Volkswirtschaften genutzt. Dabei traten schon Smith und Ricardo nicht nur f{\"u}r freie M{\"a}rkte innerhalb einer Volkswirtschaft ein, sondern sahen auch die Regulierung und Steuerung des Weltmarktes als wohlfahrtsmindernd an. Jedoch best{\"a}tigen die Arbeiten von \citet{Dollar.1992,BenDavid.1993,Sachs.1995,Frankel.1999} und \citet{Edwards.1993} sowie \citet{RodriguezCaballero.2000} empirisch anhand l{\"a}nder{\"u}bergreifender Untersuchungen, dass Handelspolitik sich positiv auf das {\"o}konomische Wachstum auswirkt. Die dahinterliegende Intention ist der Schutz vor dem Wettbewerbseffekt durch Freihandel. Bestimmte M{\"a}rkte und Branchen sollten zun{\"a}chst die M{\"o}glichkeit haben sich aufzubauen und zu etablieren, bevor diese sich gegen{\"u}ber der weltweiten Konkurrenz behaupten k{\"o}nnen. \newline


Eine verbreitete Entwicklungsstrategie nach dem zweiten Weltkrieg war die \textbf{Importsubstitution} im Industriesektor. Ziel dieser Strategie war es Entwicklungsdefizite aufzuholen und die Selbstversorgung eines Landes zu sichern, damit die Unabh{\"a}ngigkeit vom Weltmarkt gewahrt wird. Daf{\"u}r wird der Import von G{\"u}tern reduziert, um eine heimische Produktion anzustreben und zu unterst{\"u}tzen. Zu dem Instrumentarium der Importsubstituierung geh{\"o}ren einerseits protektionistische Ma{\ss}nahmen der Au{\ss}enhandelspolitik wie Z{\"o}lle, Subventionen und Kontingente, andererseits binnenwirtschaftliche Richtlinien, wie Steuer- und Innovationsanreize, sowie auch die Befreiung von Markteintrittsbarrieren \citep{Muller.2005,Lachmann.1994}. Diese Regulierungsma{\ss}nahme sollte den importkonkurrierenden Industrien helfen, sich vor{\"u}bergehend vor dem weltweiten Wettbewerb zu sch{\"u}tzen \cite{Lewis.1955}. Der Au{\ss}enhandel wird aktiv vom Staat reduziert, um die eigene Produktion zu f{\"o}rdern. Somit geht eine Importsubstitution mit dem R{\"u}ckgang des Handelsvolumens einher und es kommt ebenfalls zu geringeren Exporten, denn es ist nicht m{\"o}glich importkonkurrierende Sektoren zu f{\"o}rdern, ohne dabei das Exportwachstum zu mindern.\footnote{Jedoch gibt es durchaus Rahmenbedingungen, bei denen ein Importzoll das Handelsvolumen erh{\"o}hen kann. Wird ein Zoll in einem Sektor auferlegt, der durch einen komparativen Nachteil gepr{\"a}gt ist, dann steigt dadurch der G{\"u}terhandel an \cite{Lancaster.1980}.} 
Indien verfolgte diese Strategie so konsequent, dass in den 1970er Jahren anteilig nur 3{\%} des Bruttoinlandsproduktes durch Handel erwirtschaftet wurde. Dabei zeigte sich, dass trotz des geringen Importanteils kein {\"u}berdurchschnittliches Wachstum folgte.\footnote{In Indien kam es in den 1990er Jahren zu einer Reform umfassender Handelsliberalisierungen, woraus anschlie{\ss}end starkes Wachstum resultierte.}
Begr{\"u}ndet ist dies durch Preisverzerrungen, die die internationale Konkurrenzf{\"a}higkeit nicht wiederspiegeln \citep{Lachmann.1994}. Des Weiteren ist der Grundgedanke dieser Strategie, die Branche zu sch{\"u}tzen, bei der zuk{\"u}nftig ein komparativer Vorteil erwartet wird, der bei der Strategieentwicklung jedoch nicht immer ber{\"u}cksichtigt wurde. Die k{\"u}nstliche Begrenzung des Wettbewerbseffekts sowie das Ausbleiben des Marktgr{\"o}{\ss}eneffekts durch Au{\ss}enhandel bedingte weiterhin das Bestehen von Ineffizienzen und {\"U}berkapazit{\"a}ten, die wohlfahrtsmindernd wirken. \\


Bei der Importsubstitution kommt es durch den ausbleibenden Import neuer Technologien nicht zum Wissenstransfer und Spillover-Effekten.\footnote{Dies trifft nur dann zu, wenn es sich um eine vollst{\"a}ndig geschlossene Volkswirtschaft handelt und keine Importe in das Land eingef{\"u}hrt werden.} Der Philosophie der Strategie entsprechend sollen G{\"u}ter selbst produziert werden. Dies trifft dann auch auf Innovationen zu. Ein Import notwendiger Technologien, um diese nachahmen zu k{\"o}nnen, wird nicht angestrebt. Somit liegt der Schwerpunkt dieser Entwicklungsstrategie auf der Innovation von G{\"u}tern und Prozessen, auch wenn diese eventuell bereits existieren und somit nicht zu einer Ausweitung der Welttechnologiegrenze beitragen. Diese Realisierung des technischen Fortschritts ist jedoch sehr teuer und aufwendig. Au{\ss}erdem ist qualifizierte Arbeit notwendig f{\"u}r erfolgreiche Innovationen. Dies begr{\"u}ndet, warum eine Entwicklungsdefizit nicht nur bez{\"u}glich des Einkommens besteht, sondern auch hinsichtlich des technologischen Entwicklungsstandes.\\


Weitere Gr{\"u}nde f{\"u}r den ausbleibenden Erfolg der Strategie der Importsubstitution waren unter anderem fehlende Institutionen beispielsweise im Finanzsektor oder der Mangel qualifizierter Arbeit aufgrund eines unzureichenden Bildungssystems.\footnote{Die fehlende Implementierung von Institutionen sehen auch \citet{Collier.1999} als Ursachen des schwachen Wachstums afrikanischer L{\"a}nder und konstatieren, dass die Integration in den Welthandel einfacher ist, als die Errichtung von Institutionen oder einer Infrastruktur \cite{Collier.1999}.}\newline 


Die Euphorie der Importsubsitution nahm letztendlich ab, da selbst L{\"a}nder, die beinahe 100{\%} ihrer G{\"u}ter selbst produzierten, keinen {\"u}berdurchschnittlichen Entwicklungserfolg verzeichnen konnten, was zu einer abflachenden Beliebtheit dieser Strategie in den 1970er Jahren f{\"u}hrte \cite{LittleIanMalcolmDavid.1970}.\\


Mitte der 1980er Jahre kam es zu einem allgemeinen Umschwung hin zu liberalerer Handelspolitik. Teilweise wurden die Handelsbeziehungen stark fokussiert und zus{\"a}tzlich gef{\"o}rdert. Die \textbf{Exportf{\"o}rderungsstrategie} zielt auf eine vollst{\"a}ndige Integration eines Landes in die Handelsbeziehungen der {\"u}brigen Welt ab. Diese Strategie basiert auf dem Ansatz des Freihandels, der auch politisch angestrebt und unterst{\"u}tzt wird. Z{\"o}lle und andere Handelsbeschr{\"a}nkungen haben eine verzerrende Wirkung auf die Dynamik multisektoraler Modelle, hier den M{\"a}rkten, die zur Wohlfahrtsminderung f{\"u}hren \cite{Ortigueira.2002}. Somit werden gezielt Anreize gesetzt den Export auszuweiten, indem Protektionismus reduziert wird. Auch wenn man damit dem Leitbild des Freihandels folgt, bleibt noch die Frage der Spezialisierung eines Landes. 
Zun{\"a}chst geriet der Fokus vieler L{\"a}nder auf den prim{\"a}ren Sektor. Der Export landwirtschaftliche Erzeugnisse wurde angestrebt. Jedoch kam es in den vergangenen Jahrzehnten zu Ver{\"a}nderungen der Struktur der weltweiten Nachfrage des prim{\"a}ren Sektors. Es zeigte sich, dass diese einseitige Ausrichtung keine langfristige und nachhaltige Entwicklung sichert, da die Nachfrage nach landwirtschaftlichen Erzeugnissen  kontinuierlich abnimmt. Die weiter entwickelten L{\"a}nder zeigen bereits, dass es zu einer Tendenz der Bev{\"o}lkerungsabnahme kommt und demnach die Nachfrage nach G{\"u}tern des t{\"a}glichen Bedarfs sinkt. Au{\ss}erdem ist der Markt bereits weitestgehend ges{\"a}ttigt und weitere hinzukommende Anbieter, w{\"u}rden den Ertrag jedes einzelnen Anbieters zus{\"a}tzlich schm{\"a}lern, sodass es nicht mehr lohnend ist in den Markt einzudringen \citep{Muller.2005,Lachmann.1994}.\\


Bei der Strategie des exportorientierten Wachstums steigt das Handelsvolumen an, da die Vorteile des \textit{Marktgr{\"o}{\ss}eneffekts} ausgenutzt werden, indem die vorhandenen Kapazit{\"a}ten durch die neu hinzukommenden M{\"a}rkte beim Freihandel ausgelastet werden. 
So bringt Au{\ss}enhandel zwar gr{\"o}{\ss}ere M{\"a}rkte mit sich und er{\"o}ffnet die M{\"o}glichkeit von Gr{\"o}{\ss}envorteilen zu profitieren, doch aufgrund der meist fehlenden Produktionsfaktoren Kapital und qualifizierte Arbeit kann dieser Umschwung von Entwicklungsl{\"a}ndern nicht gleicherma{\ss}en genutzt werden, wie dies bei industrialisierten L{\"a}ndern der Fall ist. Bedingt ist der Mangel an den entsprechenden Produktionsfaktoren durch institutionelle Defizite. Dazu z{\"a}hlen beispielsweise unsichere Eigentumsverh{\"a}ltnisse, politische Instabilit{\"a}t und auch bei dieser Strategie wieder ein unzureichendes Bildungssystem.\\


Der \textit{Wettbewerbseffekt} {\"a}u{\ss}ert sich durch zus{\"a}tzliche Anbieter, die die Konkurrenz verst{\"a}rken, wodurch nun insgesamt ein h{\"o}heres Bestreben nach effizienterer Produktion besteht. \citet{Trefler.2004} zeigt an dem konkreten Fall des Freihandelsabkommens zwischen den USA und Kanada den Wettbewerbseffekt durch Handelsliberalisierung. Er reflektierte die Wirkung des Abkommens auf die Unternehmensstruktur und zeigte, dass die Produktivit{\"a}t ganzer Branchen durch die {\"O}ffnung zum Weltmarkt zugenommen hat, weil die am wenigsten leistungsf{\"a}higen Unternehmen vom Markt verdr{\"a}ngt wurden \cite{Trefler.2004}.\\


Auch der \textit{Spillover-Effekt} {\"a}u{\ss}ert sich in der Strategie der Exportf{\"o}rderung.  Technologietransfer beg{\"u}nstigt nur dann weniger weit entwickelte L{\"a}nder in ihrem Entwicklungsprozess, wenn Freihandel angestrebt wird. Denn catching up h{\"a}ngt wesentlich von den Handels- bzw. Markteintrittsbarrieren ab. Werden diese erh{\"o}ht, ist eher eine Stagnation der Volkswirtschaft wahrscheinlich statt eines Aufholprozesses \cite{Stokey.2015}. Denn der mit Freihandel einhergehende importbedingte Technologietransfer f{\"u}hrt zur Modernisierung der eigenen inl{\"a}ndischen Technologien und letztlich zu technischem Fortschritt.
Neben Qualit{\"a}tsverbesserungen werden auch neue Anreize  an inl{\"a}ndische Unternehmen gesetzt innovierend und imitierend t{\"a}tig zu werden. Denn eine M{\"o}glichkeit zur {\"u}brigen Welt aufzuschlie{\ss}en liegt darin, Innovationen, die dem Sektor einen Konkurrenzvorteil verschaffen, zu entwickeln und dadurch die Attraktivit{\"a}t der Handelsbeziehung zu steigern \citep{Muller.2005}. Doch auch f{\"u}r Imitationen liefert die Exportf{\"o}rderungsstrategie gute Voraussetzungen. Auch wenn keine Innovationen entwickelt werden, ist es m{\"o}glich die Entwicklungsdefizite zu mindern, indem der Technologietransfer genutzt wird, um weiter entwickelte Technologien zu adaptieren.
Dies f{\"u}hrte in vielen weniger weit entwickelten L{\"a}ndern zu einem st{\"a}rkeren Anstieg der Wachstumsraten als durch die Importsubstitution \cite{Krugman.2015}. Grund daf{\"u}r ist die zunehmende Qualit{\"a}t importierter Zwischenprodukte. Die Innovationen des Auslandes werden indirekt importiert und wirken in dieser Form als direkter Technologietransfer. Demzufolge ist es nicht zwingend notwendig innovativ t{\"a}tig zu sein, da der Handel mit technologisch entwickelten G{\"u}tern ein Substitut daf{\"u}r sein kann \citep{Keller.2004}. Vor allem asiatische L{\"a}nder, wie die sogenannten Tiger-Staaten zeigen, dass es m{\"o}glich ist mit dieser Strategie zu den Industriel{\"a}ndern aufzuschlie{\ss}en. Die positive Wirkung von Handel auf den technologischen Entwicklungsstand eines Landes zeigen auch \citet{Bloom.2011} am Beispiel Chinas.\\


Bei der \textbf{Strategie der Integration} steht die Eingliederung in ein handelspolitisch gepr{\"a}gtes System im Vordergrund. Es geht bei dem sogenannten Integrationsraum um einen Wirtschaftsraum, der eine gemeinschaftliche Wohlfahrtsteigerung durch gleichartige Au{\ss}enpolitik innerhalb, sowie nach au{\ss}en gegen{\"u}ber der {\"u}brigen Welt, anstrebt. Demnach werden in einem Integrationsraum intensive Handelsbeziehungen gepflegt, um gemeinschaftlich wirtschaftliche Probleme zu l{\"o}sen. Zu unterscheiden ist dabei zwischen einem weniger weit entwickelten Land, das in einen entwickelten Wirtschaftsraum integriert wird oder aber gemeinsam mit {\"a}hnlichen weniger weit entwickelten L{\"a}ndern einen Wirtschaftsraum zu bilden.\\


Je nach innen- und au{\ss}enpolitischen Ma{\ss}nahmen werden die drei Effekte des Au{\ss}enhandels mehr oder weniger genutzt, woraus sich auch dann erst eine Tendenz der technologischen Weiterentwicklung ableiten l{\"a}sst, ob diese eher innovativ oder imitativ gepr{\"a}gt ist. 
\citet{Glass.1999} sieht in der Imitation die M{\"o}glichkeit f{\"u}r weniger weit entwickelte Volkswirtschaften, meist auf der S{\"u}dhalbkugel, sich dem Entwicklungsstand des Nordens anzupassen. Erst nachdem im S{\"u}den eine Basis an Wissen geschaffen wurde, ist ein Wechsel zur Innovationsstrategie sinnvoll. \newline 


Diese {\"U}berlegung hebt die M{\"o}glichkeit der Mischung beider zuerst genannter Strategien, der Importsubtitutions und der Exportf{\"o}rderung, hervor. Der Integrationsraum sch{\"u}tzt zun{\"a}chst vor der {\"u}brigen Welt und er{\"o}ffnet die M{\"o}glichkeit die Vorteile des Freihandels innerhalb des geschlossenen Wirtschaftsraums auszunutzen. Ob ein Land der Strategie der Exportf{\"o}rderung oder der Integration folgt, determiniert noch nicht eindeutig welche Strategie im Forschungs- und Entwicklungssektor angestrebt wird.
\bigskip

Als Entwicklungsstrategie wird hier eine Strategie vorgeschlagen, die den technischen Fortschritt f{\"o}rdert. Im Zusammenhang mit Au{\ss}enhandel spielt dabei der Wissens-Spillover-Effekt eine bedeutende Rolle. Im Sinne einer Exportf{\"o}rderungsstrategie wird Freihandel angestrebt, durch den eine negative Beeintr{\"a}chtigung des Entwicklungsprozesses wegfallen soll. Das institutionelle Defizit eines unzureichend ausgebauten Bildungssystems, bzw. des fehlenden Anreizes zur Weiterbildung, wird daher durch die {\"O}ffnung eines Landes einged{\"a}mmt. Denn es wurde gezeigt, dass die unmittelbaren produktiven Auswirkungen des Humankapitals erheblichen Einfluss auf die Entwicklungspolitik haben. Die F{\"o}rderung des Wachstums weniger weit entwickelter L{\"a}nder wurde fr{\"u}her vermehrt durch physische Investitionsprojekte unterst{\"u}tzt. Nach herrschender Meinung ist der erhebliche Einfluss des Humankapitals auf das Wachstum bekannt und dadurch ist der Auf- bzw. Ausbau des Bildungssektors entwicklungspolitisch in den Vordergrund ger{\"u}ckt.\\


In den beiden folgenden Kapiteln \ref{Papier2} und \ref{Papier1} soll eine Strategie entwickelt werden, die das Wachstum eines weniger weit entwickelten Landes f{\"o}rdert. Dabei wird ein Modell zu Grunde gelegt, dessen Ursache f{\"u}r Wachstum im technischen Fortschritt liegt.\footnote{Die beiden Wirkungskan{\"a}le wurden bereits in Kapitel \ref{sec:Wachstumstheorien} genauer erl{\"a}utert. Die noch folgenden ausf{\"u}hrlich dargestellten Modelle unterscheiden sich genau hinsichtlich dieser beiden Gr{\"u}nde. Das in Kapitel \ref{Papier2} beschriebene Modell widmet sich der Faktorvermehrung und fokussiert dabei die Humankapitalakkumulation und den Ausbau des Bildungssektors, wodurch es indirekt zur Erh{\"o}hung des technologischen Entwicklungsstandes kommt. Der technische Fortschritt als direkte Ursache f{\"u}r Wachstum wird in dem darauf folgenden Modell in Kapitel \ref{Papier1} der Schwerpunkt sein. Das Modell spiegelt den engen Zusammenhang des technischen Fortschritts mit der Innovationst{\"a}tigkeit eines Landes wieder. Dieser wurde bereits im theoretischen Grundlagenteil, Kapitel \ref{sec:techn. Wissen}, genauer dargelegt.} Der notwendige aber h{\"a}ufig hemmende Umstand mangelnder qualifizierter Arbeit soll daf{\"u}r in einem ersten Schritt reduziert werden. 
In einem weiteren Schritt wird die nun st{\"a}rker vorhandene qualifizierte Arbeit f{\"u}r den Ausbau des technischen Entwicklungsstandes eines Landes eingesetzt. Dabei wird der gegenw{\"a}rtige Entwicklungsstand eines Landes ber{\"u}cksichtigt und eine Empfehlung ausgesprochen, ob eine Imitations- oder Innovationsstrategie zu st{\"a}rkerem Wachstum f{\"u}hrt. Bei beiden folgenden Analysen wird au{\ss}erdem auf die Wirkung m{\"o}glicher Handelshemmnisse eingegangen. 