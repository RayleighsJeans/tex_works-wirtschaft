\chapter[Wachstum durch technischen Fortschritt]{Wachstum durch technischen Fortschritt}\label{Wachstum}
\chaptermark{Wachstum}

Zunächst werden terminologische und theoretische Grundlagen zum Wachstum durch technischen Fortschritt vorgestellt, die dem besseren Verständnis der folgenden Untersuchungen dienen sollen. Wirtschaftliches Wachstum kann sehr allgemein definiert werden, als Anstieg der gegenwärtigen Gütermenge einer Volkswirtschaft oder nach \cite[S.1]{Frenkel.1999} als die quantitative Zunahme eines volkswirtschaftlich erwirtschafteten "`Güterbergs"'. Mit der Zunahme des Güterbergs einer Volkswirtschaft steigt das Volkseinkommen an. Etwas präziser und empirisch zweckdienlicher formuliert \cite[Kapitel 16,S.273]{Bofinger.2015} Wachstum als intertemporale Entwicklung des realen Bruttoinlandsprodukts pro Kopf. Dabei beschreibt das Bruttoinlandsprodukt (BIP) die Wirtschaftsleistung bestehend aus dem Gesamtwert der Waren und Dienstleistungen, die innerhalb eines Jahres von einer Volkswirtschaft erbracht werden. Gemessen wird die Rate des Wirtschaftswachstums durch den jährlichen Anstieg des realen Pro-Kopf-Einkommens eines Landes \cite[Kapitel 16,S.273]{Bofinger.2015}.\\
%
Die Hauptursachen des Wirtschaftswachstums sieht \cite[S.269]{Gandolfo.1998} im Anstieg der Faktor\-ausstattung und dem technischen Fortschritt, wodurch jedoch die Welt des ökonomischen Wachstums sehr stark reduziert wird.\footnote{Je nach Auffassung würden dann bestimmte Einflussfaktoren auf das Wirtschaftswachstum nicht impliziert werden. Weitere mögliche Gründe für Wirtschaftswachstum ist der in Kapitel \ref{sec:Globalisierung} noch folgende Außenhandel sowie Institutionen oder auch externe Effekte.} Bei der Faktormehrung resultiert Wachstum durch den zusätzlichen Einsatz von Produktionsfaktoren, wodurch insgesamt mehr produziert werden kann und der von \cite[S.1]{Frenkel.1999} genannte Güterberg ansteigt. Technischer Fortschritt kann zu vollkommen neuen Technologien führen oder aber auch zu zusätzlichen Gütervariationen, die neue Märkte schaffen.\\
%
Eine eineindeutige Definition des \emph{technischen Fortschritts} ist gemeinhin nicht zu finden und hängt von der Modellvariation ab. So kann als technischer Fortschritt die Folge vieler Innovationen verstanden werden, wobei auch je nach Entwicklungsstand eines Landes Imitationen zum lokalen technischen Fortschritt beitragen und als technischer Fortschritt aufgefasst werden können. Beides jedoch impliziert eine Weiterentwicklung und Ausweitung des Wissensstands. Der technische Fortschritt erhöht die \emph{totale Faktorproduktivität} und wirkt somit wie eine Faktorvermehrung. Die totale Faktorproduktivität beschreibt die Erhöhung der Produktivität, die nicht durch eine Erhöhung der Produktionsfaktoren Kapital und Arbeit erklärt werden kann. Empirisch belegt wurde die Totale Faktorproduktivität durch das sogenannte Solow-Residuum und ist durch den technischen Fortschritt zu erklären \cite{Solow.1957}. Das Solow-Residuum beschreibt demnach das Wachstum der Produktivität, welches nicht aus dem Wachstum des Faktoreinsatzes resultiert.\\
%
Das Ziel des technischen Fortschritts ist es, die Wirtschaftlichkeit eines Unternehmens und letztendlich auch einer Volkswirtschaft zu verbessern. Dabei wirkt sich der technische Fortschritt auf die Technologie aus, die direkten Einfluss auf die Produktivität eines Unternehmens hat. Dies ist unabhängig davon, ob sich der Fortschritt im Produktionsprozess oder in Form einer Produktneuentwicklung äußert.\\
%
Nach \cite[Kapitel 1]{Barro.2004} bestimmt sich eine \emph{Technologie} durch das Verfahren, bei dem Produktionsfaktoren im Herstellungsprozess zu Gütern umgewandelt werden. \cite[Kapitel 5,S.139]{Krugman.2015} verstehen unter einer Technologie eine Art systematische Methodik. Dabei bedienen sich immer dann zwei Unternehmen oder Volkswirtschaften derselben Technologie, wenn sie mit der gleichen Menge an Einsatzfaktoren den gleichen Output generieren können. Das Grenzprodukt beider Länder ist gleich groß, eine Einheit Kapital oder Arbeit führt dann in beiden Ländern zu dem gleichen anteiligen Endprodukt.\\
%
In der theoretischen Modellwelt wird eine Technologie beschrieben durch die Produktionsfunktion, in der die Einsatzverhältnisse der Produktionsfaktoren fest vorgegeben sind. Bestandteil der Produktionsfunktion ist ein Technologieparameter, meist abgekürzt mit $A$. Dieser Parameter beschreibt das technische Wissen, das im Produktionsprozess eingesetzt wird. Geht das Modell von konstanten Skalenerträgen aus, dann ist dieser Parameter konstant und über die Zeit unveränderlich. Werden jedoch steigende Skalenerträge angenommen, dann kann es zu einer Weiterentwicklung des technischen Wissens kommen, zu technischem Fortschritt, der dadurch in der Technologie abgebildet wird. Die beiden notwendigen Voraussetzungen für den technischen Fortschritt, das technische Wissen und Humankapital werden in Abschnitt \ref{sec:TechnischesWissenHumankapital} genauer erläutert.\\
%
\cite{Gandolfo.1998}s (\citeyear{Gandolfo.1998}) Ursachen für Wachstum, Faktorakkumulation und technischer Fortschritt, hängen jedoch sehr eng miteinander zusammen, weil beispielsweise eine technische Neuerung den Faktoreinsatz mindern kann und somit dann insgesamt mehr produziert werden würde.\footnote{Dies gilt immer dann, wenn beispielsweise Wirtschaftswachstum als unbeabsichtigtes Nebenprodukt steigender Skalenerträge bei der Kapitalakkumulation resultiert. Als ein Beispiel für diesen Effekt gilt learning-by-doing, das sich vor allem bei Größeneffekten durch die Produktion großer Mengen auswirkt. Denn mit der Produktionsmenge steigen die Lerneffekte der Beschäftigten. Das durch die zunehmende Erfahrung hinzugewonnene Wissen verbessert die Abläufe der Produktionsstruktur. Der Produktionsfaktor Arbeit wird produktiver und die Effizienz der Arbeit verbessert sich \cite[Kapitel 12, S.413]{Acemoglu.2009}.} Trennt man jedoch beide Argumente strikt voneinander, dann lässt dies eine Untergliederung der Wachstumsmodelle in exogene und endogene Modelle zu. Es handelt sich um exogene Wachstumsmodelle, wenn es zu einer Ausweitung der Produktionsfaktoren kommt, bei denen der technische Fortschritt als von außen gegeben betrachtet wird und der Grund für sein Dasein ungewiss ist.\\
%
Endogen ist ein Wachstumsmodell, wenn der technische Fortschritt direkt hervorgerufen wird, indem gezielt Forschung und Entwicklung betrieben wird \cite[S.269]{Gandolfo.1998}.\\
%
Als Beispiel dient das AK-Modell nach \cite{Rebelo.1991}. Hier ist technischer Fortschritt, Wissen, das als ein Nebenprodukt der Kapitalakkumulation hervorgeht. Abweichend von anderen endogenen Wachstumsmodellen wird Wachstum hier nicht durch innovative Tätigkeiten angeregt, sondern ist ein Ergebnis von Sparentscheidung und Kapitalakkumulation. Dagegen beschreibt \cite{Arrow.1969} technischen Fortschritt als den Prozess der Reduktion der Unwissenheit. Wieder anders verhält es sich im Romer-Modell, siehe dazu \cite{Romer.1990}, in dem das technologische Wachstum durch die Zunahme von Produktvarianten beschrieben wird.\footnote{Nachdem hier zunächst Begrifflichkeiten und Grundlagen erörtert werden, werden in Kapitel \ref{sec:Wachstumstheorien} die genannten Modelle genauer erläutert.}\\
%
Unabhängig von der Interpretation des technischen Fortschritts führt dieser zu einer Ausweitung der Welttechnologiegrenze (WTG). Bei der Welttechnologiegrenze $\bar{A}_t$  handelt es sich um den maximal erzielbaren Wissensstand, der zu einem Zeitpunkt $t$ erreicht werden kann. Vergleicht man die WTG mit dem Wissensstand einer Volkswirtschaft, erlaubt dies Aussagen über die relative Lage des Landes zur WTG. So ergibt sich der Abstand zur WTG $a_t$ aus der Relation der lokalen Technologiegrenze (LTG) oder auch der Produktivität eines Landes $A_t$ zu der WTG, somit gilt $a_t = A_t/\bar{A}_t$ \cite{Aghion.1992,Aghion.1998}.\\
%
In dieser Arbeit wird unter technischem Fortschritt ein Ausbau des technischen Wissensstandes gesehen und impliziert dabei sowohl Innovationen als auch Imitationen, die in der Volkswirtschaft zu einem Erkenntnisgewinn beitragen.\\
%
Demzufolge werden hier beide Gründe für Wachstum nach Gandolfo ausführlich behandelt. So geht das Wachstum des Humankapitalmodells in Kapitel \ref{Papier2} auf die Faktorakkumulation zurück, die dann den im zweiten Modell, Kapitel \ref{Papier1}, angeführten Grund für Wachstum, den technischen Fortschritt, begünstigt. Verstärkt wird der technische Fortschritt wesentlich durch die Offenheit der Volkswirtschaften und die sich daraus ergebenden Handelsmöglichkeiten. 
%
\section{Prämissen des technischen Fortschritts:\\technisches Wissen und Humankapital}\label{sec:TechnischesWissenHumankapital}
\sectionmark{Prämissen des techn. Fortschritts}
%
Für technischen Fortschritt sind sowohl technisches Wissen als auch Humankapital notwendig. Wird technischer Fortschritt als eine Aneinanderreihung von Innovationen  verstanden, bedarf die Durchführung innovierender Tätigkeiten die beiden Komponenten technisches Wissen und Humankapitel \cite{Howitt.2005}. Als technisches Wissen gelten Ideen und Informationen, welche nur in Verbindung mit Kapital verwendet werden können. Dafür ist es zunächst unerheblich, an welche Kapitalart technisches Wissen gebunden ist. In Kombination mit physischem Kapital tritt technisches Wissen, beispielsweise in Form von Blaupausen, Maschinen oder Gütern auf. Ist Wissen an den Menschen, also hier den Produktionsfaktor Arbeit, gebunden, dann handelt es sich um Humankapital.
%
\subsection{Technisches Wissen}\label{sec:techn. Wissen}
%
Zunächst wird die Komponente technisches Wissen erläutert, bevor anschließend Humankapital genauer analysiert wird, um die Entstehung des technischen Fortschritts zu verdeutlichen.\\
%
Für die Entwicklung einer Innovation ist technisches Wissen zwingend notwendig und wird hervorgerufen durch eine Idee. Die Gestaltung und Ansatzpunkte attraktiver Ideen können sehr verschieden sein. Dazu zählen vor allem die Kostenreduktion durch die Effizienzsteigerung in der Produktion oder aber die Entwicklung vollkommen neuer Güter.\\
%
Das technische Wissen an sich und auch die Idee ist ungebunden und somit ein öffentliches Gut bzw. hat dessen Eigenschaften \cite{Arrow.1962,Nelson.1959}. Öffentliche Güter sind durch die beiden Eigenschaften der Nicht-Rivalität und der Nicht-Ausschließbarkeit im Konsum charakterisiert.\\
%
Sofern die Möglichkeit besteht, dass der Konsum von den Anbietern verhindert werden kann, lassen sich die Erträge dem jeweiligen Produzenten eindeutig zuordnen und es gilt die Ausschließbarkeit. Ist diese Eigenschaft nicht vorhanden, sind positive Externalitäten die Folge. Im Fall der Ideen und des technischen Wissens können diese von mehreren Unternehmen gleichzeitig umgesetzt werden, ohne dass es von konkurrierenden Unternehmen verhindert wird. Der Anreiz zur Ideengenerierung für das einzelne Wirtschaftssubjekt ist dadurch relativ gering. Verstärkt wird dieser Zusammenhang durch die Nicht-Rivalität im Konsum des technischen Wissens. Denn es kann ein und dieselbe Anleitung von einem weiteren Unternehmen verwendet werden, wodurch die Produktion ansteigt, ohne dass erneute Kosten für technologisches Wissen entstehen \cite[S.60]{.1968,Ostrom.1990}.\\
%
Die Entwicklung einer Idee kann kostspielig sein und der kostenfreie Zugriff einer möglicherweise gewinnbringenden Idee das Interesse vieler wecken. Dabei handelt es sich beispielsweise um eine Neuerung im Produktionsprozess, die zur Beseitigung von Ineffizienzen führt. Eine Idee kann von mehreren Wirtschaftssubjekten zur gleichen Zeit realisiert werden, wohingegen sich die Faktoren Arbeit und Kapital nur einmal an einem Ort einsetzen lassen. Demzufolge ist auch ein Anstieg der Produktivität durch eine Idee in mehreren Unternehmen gleichzeitig denkbar \cite[S.1020]{Romer.1986}. \\ 
%
Endogenisiert man das technologische Wissen, dann steigen die Skalenerträge der Produktion an. Eine Verdopplung aller rivalisierender bzw. konkurrierender Inputfaktoren führt zu einer mehr als doppelt so großen Produktionsmenge. Dies liegt daran, dass nicht nur das technische Wissen nicht konkurrierend ist, sondern dadurch auch die Technologie des Produktionsprozesses. Sie kann von mehreren Unternehmen gleichzeitig genutzt werden, ohne den Nutzen eines Wirtschaftssubjekts einzuschränken, wodurch eine erhöhte Produktionsmenge resultiert.\\
%
Dieser Zusammenhang zeigt, wie einflussreich die Nichtrivalität auf das ökonomische Wachstum ist, da dies steigende Skalenerträge bedingt. Die steigenden Skalenerträge liefern einen Anreiz Monopolmacht zu erlangen, was wiederum die Motivation darstellt, Innovationen zu entwickeln \cite[S.556]{Jones.2005,Romer.1993}.\footnote{Eine Ausführliche Erläuterung folgt in Kapitel \ref{sec:Anreize}}\\
%
Technisches Wissen birgt zwei Folgen: Einerseits die Motivation Innovationen zu entwickeln um Monopolmacht zu erlangen, andererseits die Gefahr der schnellen und kostenfreien Nachahmung der Konkurrenten. 
Gelöst werden kann dieses Problem durch Patente, die die kommerzielle Nutzung von Ideen durch Dritte verhindern. Dabei wird das innovierende Unternehmen geschützt und der Erhalt der geistigen Eigentumsrechte über einen bestimmten Zeitraum ermöglicht, somit mittelfristig auch die Gewinne. Jedoch können Patente nicht die Weiterverbreitung der Idee an sich verhindern.\\
%
Neben Patenten kann die Generierung von technischem Wissen auch durch die staatliche Förderung gewährleistet werden. Grundlagenforschung wird deswegen meist von öffentlichen Einrichtungen betrieben. Der Schwerpunkt dieser Arbeit liegt jedoch auf der angewandten Forschung, die von privat finanzierten Unternehmen forciert wird.
%
\subsection{Humankapital}
%
Humankapital ist (personen-)gebundenes Wissen wie die Fähigkeiten und Fertigkeiten eines Menschen. \cite[Kapitel 7,S.259]{Acemoglu.2009} präzisiert diese Definition und beschreibt Humankapital als jegliche Eigenschaften von Arbeitern, die die potentielle Produktivität aller oder einiger produktiver Aufgaben steigert. Wohingegen \cite{Lucas.1988}\footnote{Obwohl das Papier von \cite{Lucas.1988} mehrere Modelle vorstellt, wird gemeinhin und auch in dieser Arbeit von dem Humankapitalmodell des Kapitels 4 ausgegangen.} weniger zwischen einzelnen Fähigkeiten und Aufgaben differenziert, sondern Humankapital eher als ein "`skill-level"' definiert, also ein Niveau erreichter Fähigkeiten.\footnote{In dieser Form wird Humankapital in Kapitel \ref{Papier1} abgebildet. In dem Modell steht die Humankapitalakkumulation nicht im Vordergrund. Bildung ist indirekter Bestandteil der Produktivität einer Volkswirtschaft. Demnach werden keine einzelnen Aufgaben und Tätigkeiten spezifiziert, sondern verschiedene Tätigkeitsfelder bzw. Bildungsniveaus miteinander verglichen.} \\
%
Bei dem technischen Wissen handelt es sich formal, wie in Abschnitt \ref{sec:techn. Wissen} bereits erörtert wurde, um ungebundene theoretische Kenntnisse, die auch den nachfolgenden Generationen zur Verfügung stehen \cite[Kapitel 10]{Frenkel.1999}. Dieser wesentliche Punkt unterscheidet das technische Wissen von Humankapital. Denn die an den Menschen gebundenen Kenntnisse und Fertigkeiten gehen mit dem Tod des Menschen verloren und stehen der Welt nicht weiter zur Verfügung. Mit diesem Argument stellt \cite{Ha.2002} zur Diskussion, dass Humankapitalakkumulation nicht dauerhaft zum Wachstum beiträgt, da Bildung und Fähigkeiten an den Menschen gebunden sind und somit von der begrenzten Lebensdauer des Menschen abhängig sind.\footnote{Dabei wurde der Gedanke vernachlässigt, dass das Grenzprodukt des Wissens steigen könnte und dadurch steigende Wachstumsraten resultieren würden. Dieser Sonderfall steigender Grenzerträge des Humankapitals geht auf \cite{Romer.1986} zurück.}  Dem soll hier nicht direkt widersprochen werden, jedoch ist zu berücksichtigen, dass die Entwicklung von Innovationen humankapitalintensiv ist und diese wiederum langlebig sind und somit trotzdem zu dauerhaftem technologischem Wachstum führen. \\
%
Ein anderer wichtiger Unterschied des Humankapitals zum technischen Wissen liegt in der Eigenschaft der Nicht-Rivalität, denn Humankapital ist rivalisierend. Ein Wissenschaftler oder qualifizierter Arbeiter kann nur an einem Projekt gleichzeitig arbeiten und ihm steht seine Zeit nicht mehrfach zur Verfügung \cite{Romer.1993}. Somit ist wie beim Produktionsfaktor Arbeit eine eindeutige monetäre Vergütung möglich, der Lohn.\\
%
In vielen Modellen, wie beispielsweise dem AK-Modell, wird Humankapital und physisches Kapital unter dem Oberbegriff Kapital zusammengefasst. Hier wird jedoch explizit zwischen beiden Kapitalarten unterschieden, da diese verschiedene Eigenschaften aufweisen und dadurch dauerhaftes Wachstum möglich ist. Der Kapitalbegriff könnte sogar noch weiter differenziert werden, indem intellektuelles Kapital noch einmal von Humankapital abgegrenzt wird. Der Wert des produktiven Wissens, das durch Forschung und Entwicklung gewonnen wurde, ist das intellektuelle Kapital \cite[S. 81]{Dosi.1993}. 
%
\subsubsection{Humankapitalakkumulation}
%
Bei dem Faktor Arbeit handelt es sich nicht um einen homogenen Produktionsfaktor. Fähigkeiten, Fertigkeiten und Kenntnisse können durch die Akkumulation von Humankapital erhöht werden \cite[S.205]{Aghion.2015}.  Bildung steigert das Humankapital eines einzelnen Individuums und kann somit als Entstehungsprozess des Humankapitals, als Humankapitalakkumulation, gesehen werden. Es können zwei Arten der Humankapitalakkumulation unterschieden werden, das formelle und das informelle Lernen. Mit dem formellen Lernen der Bildung gehen Kosten einher, die berücksichtigt werden müssen. Dabei handelt es sich um direkte Ausbildungskosten oder Opportunitätskosten durch entgangenen Lohn. Wohingegen das informelle Lernen, das learning-by-doing nach \cite{Arrow.1969}, kostenlos ist.
%
\subsubsection*{Informelles Lernen - learning-by-doing}
%
Im Jahr 1936 veröffentlichte \cite{Wright.1936} seine Beobachtungen zum Flugzeugbau. Dabei war besonders auffällig, dass die Arbeitsstunden für die Produktion eines Flugwerks mit zunehmender Produktionszahl sinken.\\
%
Dies motivierte \cite{Arrow.1962} zu seinem Modell über das learning-by-doing. Es beschreibt den Zusammenhang zwischen der Produktivität eines Arbeiters und seiner dadurch zunehmenden Erfahrung. Dieser Produktivitätsgewinn wird als Lernen bezeichnet. Dabei geht es um die wiederkehrende und aktive Lösung von Problemen, die durch die ständige Wiederholung zu sinkenden Grenzkosten führt \cite[S.155]{Sheshinski.1967,Arrow.1962}. Denn je länger ein Gut hergestellt wird, desto kostengünstiger kann es produziert werden, bedingt unter anderem durch die Lernkurve des Herstellungsprozesses. Durch die Feststellung von Ineffizienzen, die Umstrukturierung von Organisationsformen und auch durch die zunehmende Erfahrung der Mitarbeiter steigt mittelfristig die Sicherheit im Umgang mit Techniken, Verfahren und Produkten. Sind die Lernmöglichkeiten erschöpft, dann führt erst die Entwicklung neuer Produkte und Prozesse zu neuen Lerneffekten. Andauernde Effekte des learning-by-doings sind demzufolge nach \cite{Arrow.1962} zwingend an die Innovationstätigkeit der Unternehmen geknüpft.\footnote{\cite{Sheshinski.1967} untersuchte als einer der Ersten empirisch die These Arrows, die den Produktivitätszuwachs durch zunehmende Erfahrung beschreibt. Er belegt den Ansatz und zeigt, dass effizientes Wachstum und das Investitionslevel positiv korrelieren. Dabei misst er die Erfahrung als kumulierte Bruttoinvestitionen. Demzufolge steigt mit zunehmender Erfahrung das Wirtschaftswachstum eines Landes.}
%
\subsubsection*{Formelles Lernen - Uzawa-Lucas-Modell}
%
Bei dem formellen Lernen werden die Produktionsfaktoren direkt für Bildung investiert. Am Beispiel des Uzawa-Lucas-Modells bedeutet dies, das die Wirtschaftssubjekte sich zwischen der entlohnten Konsumgüterproduktion oder der eigenen Ausbildung entscheiden müssen. Der Produktionsfaktor Humankapital wird zwischen den Sektoren aufgeteilt und geht nur anteilig in den Lernprozess ein.\footnote{Eine ausführliche Darstellung des Modells folgt in Kapitel \ref{Papier2}.}  
%
\subsubsection{Messung von Humankapital}
%
Bei der Messung von Humankapital sind einige Hindernisse zu überwinden. Zum einen führt die Unstimmigkeit bezüglich einer eindeutigen Definition zu dem Problem einer geeigneten Bezugsgröße. Wurde diese gefunden, dann ist immer noch fraglich, ob eine Vergleichbarkeit möglich ist und dadurch konkrete Aussagen getroffen werden können. Die Methoden, mit denen Humankapital geschätzt wird, sind sehr verschieden. Als Bezugsgrößen bediente  \cite{Romer.} sich beispielsweise der Anzahl an Bildungsjahren oder vergleicht Bildungsniveaus miteinander. So können die Grundkenntnisse der Bevölkerung einer Volkswirtschaft über die Alphabetisierungsrate aller erfasst werden, die das 15. Lebensjahr überschritten haben. An der Einschreiberate oder der Messung von Absolventen einer weiterführenden Schule orientierten sich \cite{Levine.1992} sowie \cite{Barro.2001}. \cite{Mankiw.1992} verwendeten eine Länderquerschnittanalyse, dabei wurde die Zahl der Jugendlichen zwischen 12 und 17, die eine Schule besuchen, mit dem Anteil der arbeitsfähigen Bevölkerung zwischen 15 und 19 multipliziert. Kritisch ist bei dieser Methode jedoch, dass das Humankapital in Industrieländern tendenziell überschätzt und in Entwicklungsländern unterschätzt wurde.\\
%
\cite{Barro.2001} haben in ihrer Arbeit einen Datensatz aufbereitet, der Humankapital quantifiziert, indem die Bevölkerung mehrerer Länder von 1960 bis 2000 nach sieben verschiedenen Bildungsstufen kategorisiert wird.\\
%
Problematisch bei allen genannten Methoden ist, dass keine Aussage über die Qualität der Bildung möglich ist und keine eindeutige Aussage über eine mögliche Qualifizierung zugelassen wird. Internationale Leistungstests wie die PISA-Studien oder mögliche Sammel\-indikatoren, die die länderspezifischen Systeme in einen einheitlichen Rahmen einordnen, können diesbezüglich Abhilfe schaffen. So wird mit Hilfe der Daten aus dem UNESCO Institute for Statistics anhand der Anzahl der Lehrkräfte oder auch über die Anzahl der Schüler pro Klasse versucht, eine internationale Vergleichbarkeit  bezüglich eines Jahres Bildung herzustellen. 
\newpage
\section[Technologieentwicklung durch Innovation]{Entstehung des technischen Fortschritts:\\Technologieentwicklung durch Innovation }
\sectionmark{Entstehung des techn. Fortschritts}
%
Die für den technischen Fortschritt notwendigen Bestandteile wurden im vorangegangenen Kapitel ausführlich erläutert. Im folgenden Kapitel wird gezeigt, dass die Intelligenz, Kompetenz sowie die Ausbildung eines Individuums für die Entwicklung und den Erfolg von Innovationen und Imitationen bedeutsam sind, was bereits von \cite{Hassler.2000} diskutiert wurde.\\
%
In der Regel handelt es sich bei Innovationen um neue Technologien. Die beiden Bestandteile einer Innovation sind eine Idee und eine Investition. Die Idee ist dabei zunächst der Engpass, den es zu überwinden gilt und ohne die eine Neuentwicklung nicht möglich ist. Die Investition ist notwendig, um die Idee umzusetzen, zu entwickeln und in den Markt einzuführen.\footnote{Als wesentliche Voraussetzung gilt dabei, dass eine Neuerung vom Markt erfolgreich angenommen wird und es somit bereits einen Bedarf gibt oder dieser noch geschaffen werden kann. Außerdem müssen die notwendigen Rahmenbedingungen für die Markteinführung vorhanden sein. Bei einer medizinischen Innovation beispielsweise sollten den Ärzten Fortbildungen angeboten werden, um die Neuerungen in den Berufsalltag einzubinden und auch anwenden zu können.}\\
%
Für die Entwicklung einer Idee kann technisches Wissen notwendig sein, das an Humankapital gebunden ist, bei der Investition ist das technische Wissen hingegen erforderlich, da für die Entwicklung einer Idee in der Regel bereits bekannte Technologien verwendet werden. Dabei ist einerseits technisches Wissen, das an physisches Kapital gebunden ist, notwendig und andererseits ausgebildete Arbeitskräfte, an die Humankapital gebunden ist \cite[S.39]{Scotchmer.2004}.\footnote{So zählen zu den Investitionen neben monetärer Größen auch die Produktionsfaktoren (Maschinen, Arbeit, Zwischengüter, Humankapital, Zeit) sowie spezifisch gebundene Investitionen in Forschungseinrichtungen.}\\
%
Der Innovationsprozess kann nach \cite{Jones.2005} auch anders untergliedert werden, in die Abschnitte: Invention, Innovation und die folgende Diffusion. Vergleicht man dies mit der erst genannten Aufteilung, dann würde die Idee der Invention, also der Erfindung entsprechen und die Investition gliedert sich auf in die Innovation an sich, also die physische Umsetzung der Idee, und der Diffusion, der Markteinführung und dem damit verbundenen Wissenstransfer für die Allgemeinheit.\\
%
Als wesentliche Bestandteile einer Innovation lassen sich Technologie und Humankapital zusammenfassen. Mit genau diesen beiden Schwerpunkten befasst sich auch der Hauptteil dieser Arbeit. Zunächst wird die Entstehung des Humankapitals in Kapitel \ref{Papier2} untersucht und anschließend wird in Kapitel \ref{Papier1} analysiert, wie durch dieses mit dem notwendigen technischen Wissen Innovationen entstehen können, die zusätzlich den Entwicklungsprozess eines Landes beschleunigen.\\
%
Jedoch ist der Begriff "`Innovation"' stark vom theoretischen Zusammenhang abhängig und in der Literatur gibt es eine Vielzahl von Differenzierungsmöglichkeiten verschiedener Innovationsformen. Eine Möglichkeit der Abgrenzung von \cite{Schebesch.1992} bezieht sich auf das Ausmaß der Innovation. Bei der graduellen Innovation werden bestehende Produkte bzw. Prozesse weiter entwickelt und verbessert. Wohingegen bei der Basisinnovation ein komplett neues Produkt entsteht.\footnote{Des weiteren wird zwischen einer drastischen und einer nicht-drastischen Innovation unterschieden, beide Fälle werden in Kapitel \ref{sec:LimitPreis} diskutiert.}\\
%
Modelle, die den technischen Fortschritt beschreiben, differenzieren häufig zwischen der Produktinnovation und der Prozessinnovation. Es handelt sich um eine Produktinnovation, wenn ein neues Gut entwickelt und auf dem Markt eingeführt wird.  Die neuen Güter erweitern die Konsummöglichkeiten der Haushalte \cite{Grossman.1991a,Grossman.1990b}. Daraus resultiert laut \cite{Krugman.79} ein höherer Nutzen bei den Konsumenten, wenn davon ausgegangen wird, dass es eine Vorliebe für die Auswahl möglichst vieler Güter gibt. Auch denkbar ist die Erhöhung der Qualität der Güter. In diesem Fall ersetzen die neuen Produktvarianten die früheren und es kommt nicht zu einem Anstieg der Anzahl der Produktvarianten \cite[Kapitel 12, S. 411]{Acemoglu.2009}. \\
%
Endogene Wachstumsmodelle, in denen die Vielfalt an Inputfaktoren durch den technischen Fortschritt zunimmt, beschreiben Prozessinnovationen. Durch die Erhöhung der Verschiedenartigkeit der Einsatzfaktoren kommt es zu einer Produktivitätssteigerung. Bei einer Prozessinnovation liegt der Schwerpunkt auf Neuerungen im Herstellungsverfahren bereits existierender Güter. Ziel der Prozessoptimierung ist eine Kostenreduktion und eine effizientere Produktion. Der Erfolg einer Prozessinnovation lässt sich intuitiv durch das Wirtschaftlichkeitsprinzip erläutern: Kann mit der gleichen Menge an Einsatzfaktoren eine höhere Produktionsmenge erzeugt werden, dann hat sich die Produktivität des Prozesses erhöht. Dem Minimumprinzip folgend, kann dann mit einem geringeren Faktoreinsatz die gleiche Gütermenge hergestellt werden. Aus makroökonomischer Perspektive würde in einem Modell mit den Einsatzfaktoren Arbeit, Kapital und Technologie ein höheres Sozialprodukt bei konstanten Faktoreinsätzen folgen \cite[Kapitel 10]{Frenkel.1999}. Handelt es sich bei einem Inputfaktor um Zwischengüter, dann werden bei Prozessinnovationen vom Zwischengut immer neue Varianten entwickelt, die direkt wieder in den Produktionsprozess eingesetzt werden. Denn es gilt wie \cite{Romer.1987,Romer.1990}  zeigt, je mehr Varianten den Produktionsprozess mitbestimmen, desto stärker ist die Arbeitsteilung und desto höher dadurch letztlich die Produktivität eines Unternehmens.\\
%
Innovationen nach \cite{Hicks.1932} führen zu Ersparnissen des Faktors Arbeit, da dieser nun effizienter eingesetzt werden kann. Dieser Effekt entsteht auch durch die Akkumulation von Humankapital, das den einzelnen Arbeiter dazu befähigt, effizienter zu arbeiten \cite[S.29]{Arrow.1969}.\\
%
Die Unterscheidung zwischen Produkt- und Prozessinnovation wird in dieser Arbeit jedoch nicht vorgenommen, sondern beide Arten unter dem Oberbegriff "`Innovation"' subsumiert. In der Literatur ist diese Unterscheidung gerade dann sinnvoll, wenn im Anschluss die Forschungsergebnisse empirisch überprüft werden. Da dies hier nicht der Fall ist, wird von einer Unterscheidung abgesehen \cite[Kapitel 12,S.411]{Acemoglu.2009}.\\
%
Außerdem kann zwischen der vertikalen und horizontalen Innovation differenziert werden \cite[S.20]{Grossman.1989a,vanLong.1997}. Dabei handelt es sich bei horizontalen Innovationen um zusätzlichen Variantenreichtum, wodurch die Vielfalt an möglichen Gütern und Prozessen zunimmt, wie es im Modell von \cite{Romer.1990} der Fall ist. Bei vertikalen Innovationen hingegen werden Güter und Prozesse weiterentwickelt \cite[S.20]{vanLong.1997}. Ein nun hochwertigeres Gut bzw. verbesserter Prozess ersetzt den vorherigen. Bleibt die Summe der Güter unverändert, dann handelt es sich um den Prozess der schöpferischen Zerstörung nach \cite{Schumpeter.1934a}. Schumpeter prägt den Begriff der schöpferischen Zerstörung, der den strukturellen Wandel durch immer neue Erfindungen beschreibt.\footnote{Genauere Erläuterung des Prozesses folgen in Kapitel \ref{sec:Wachstumstheorien}.} Er erkannte das Wechselspiel von Innovation und Imitation als Triebkraft des Wettbewerbs.\\
%
Einer anderen Auffassung bezüglich der Innovationsarten ist \cite{Mokyr.1990} und berücksichtigt die Reichweite einer Innovation. Dabei unterscheidet er in seiner Arbeit zwischen Makro- und Mikroinnovationen. Eine Makroinnovation ist ein technologischer Fortschritt, der zu weitreichenden strukturellen Veränderungen führen kann. Beispiele hierfür sind die Erfindung der Elektrizität oder das Internet. Die Folgen solcher Innovationen sind enorm und wirken sich meist auf die Mehrheit von Herstellungsprozessen aus, sie werden jedoch in der Forschung bislang weitestgehend noch nicht berücksichtigt.\\
%
Die meisten Modelle analysieren hingegen Mikroinnovationen, die das Wirtschaftswachstum stärker fördern als Makroinnovationen. Dies scheint zunächst etwas überraschend, wurde aber von \cite{Abernathy.1978} und \cite{Freeman.1982} empirisch bestätigt. Unter Mikroinnovationen versteht man sowohl Produkt- als auch Prozessinnovationen, deren Wirkung auf das technologische Umfeld von geringerer Bedeutung ist, dem einzelnen Wirtschaftssubjekt jedoch Nutzen stiftet. Es kann sich dabei nach \cite{Mokyr.1990} um eine Kostenreduktion im Produktionsprozess, eine qualitativ hochwertigere Variante eines bereits bekannten Gutes oder auch ein neues vorher unbekanntes Produkt handeln. Diese Terminologie wird auch in Kapitel \ref{Papier1} aufgegriffen und beschreibt den Einfluss beider Innovationsmöglichkeiten auf die Ausweitung der Welttechnologiegrenze. Je nachdem ob es sich um eine Makro- oder eine Mikroinnovation handelt beeinflusst dies den relativen technologischen Entwicklungsstand eines Landes unterschiedlich.
%
\subsubsection*{Anreize zur Innovationsentwicklung}\label{sec:Anreize}
%
In dem folgenden Abschnitt soll erörtert werden worin die Motivation besteht Technologien zu entwickeln oder zu verbessern. Dabei lassen sich zwei Meinungsbilder unterscheiden.  Nach \cite{Ceruzzi.2003} beispielsweise besteht der Anreiz zu innovieren vor allem in der Wissbegierde der Forscher. Er beschreibt in seinem Werk "`History of Modern Computing"', dass es keinen Bedarf nach Computern für den persönlichen Gebrauch gab und es deshalb auch nicht die Nachfrage in dem tatsächlich resultierten Umfang erwartet wurde. Die Vielzahl unerklärter Phänomene und Fragen veranlassen Wissenschaftler deren Ursprung und Erklärung zu ergründen, ohne dabei mögliche Absatzmöglichkeiten und ökonomische Argumente einfließen zu lassen. Der gleichen Meinung ist \cite[S.30]{Arrow.1969}, denn Wissen entsteht durch die Suche nach Lösungsansätzen und durch Beobachtungen realer Vorgänge und Ereignisse. So können ähnliche Gegebenheiten dabei helfen Erklärungsansätze zu finden und Erkenntnisse zu gewinnen. Der Mensch ist nur durch Neugier getrieben und versucht die Welt in der er lebt zu verstehen, dabei sind Innovationen Instrumente für Problemlösungsansätze. \\
%
Nach herrschender Meinung liegt die Motivation jedoch eher in Gewinnerzielungsabsichten \cite{Romer.1993,Grossman1989b.}. So auch bei der Entwicklung des iPads, dem ersten Tablet-PC. Der Markt und das damit einhergehende Bedürfnis nach diesem Gut wurde von dem Hersteller Apple herbeigeführt. Jedoch ist fraglich, ob tatsächlich der Forschungsdrang nach einer Problemlösung die Erfindung motiviert hat oder eher wirtschaftliche Aspekte. Durch eine Innovation wird der Anbieter zunächst zum Monopolisten und die damit einhergehende anfängliche Monopolmacht zeigt sich in Preissetzungsspielräumen, wodurch Gewinne abgeschöpft werden können. Langfristig werden konkurrierende Anbieter sich ebenfalls der Innovation bedienen, was durch die Nicht-Rivalität und die Nicht-Ausschließbarkeit des technischen Wissens möglich ist \cite{Romer.1993}. Darin besteht auch das eigentliche Problem der Innovationsentwicklung. Zwar suggerieren Innovationen kurzfristige Gewinne, die Entwicklung ist jedoch aufwendig und teuer. Die Investitionen können ohne den Schutz der Eigentumsrechte nicht ausgeglichen werden, wodurch sich der Anreiz zur Innovationsentwicklung stark mindert. Grundsätzlich spornt die wirtschaftliche Bereicherung als Konsequenz erfolgreich integrierter Innovationen die Menschheit seit Jahrhunderten dazu an, den technischen Fortschritt voran zu treiben. Daraus begründet sich die notwendige Einführung von Patenten, die das technische Wissen schützen und Alleinstellungsmerkmale schaffen. Die geschaffene Ausschließbarkeit im Konsum führt zu einer monetären Bemessung und Zuordnung \cite[Kapitel 12,S.414]{Acemoglu.2009}. Am Beispiel der Innovationstätigkeiten des Hufeisensektors erläutert \cite{Schmookler.1966} die wirtschaftliche Abhängigkeit von Innovationen. Die Innovationsrate stieg Ende des 19. Jahrhunderts bis ins 20. Jahrhundert solange stark an, bis zu dem Zeitpunkt, ab dem sich das Automobil immer weiter in der Gesellschaft durchsetzte und dadurch die Fortbewegung mit dem Pferd als unnötig erachtet wurde. Somit liegt letztendlich der Anreiz in Forschung zu investieren in der Entwicklung von Innovationen, um als Vorreiter eines Marktes Monopolgewinne abschöpfen zu können.\footnote{Zudem entsteht indirekt ein Wissenszuwachs für die gesamte Branche, von dem alle Marktteilnehmer gleichermaßen gegenseitig profitieren können \cite{Cohen.1989}.}\\
%
Die industrieökonomische Literatur befasst sich mit der Rivalität der Unternehmen, um die technologische Führerschaft und den damit einhergehenden Einfluss auf den Entwicklungsprozess zu erklären. Da viele Unternehmen nach erfolgreichen Innovationen streben, also nach Innovationen, aufgrund derer Patente angemeldet werden können um Monopolgewinne abzuschöpfen, birgt dies zugleich eine Unsicherheit des Erfolgs. Demzufolge besteht auch ein Risiko den Wettstreit um die führende Position zu verlieren und vom technologischen Fortschritt nicht profitieren zu können. Die Unsicherheit, die mit dem technologischen Fortschritt einhergeht, beeinträchtigt den technologischen Erfolg und den damit einhergehenden Entwicklungsprozess eines Landes \cite[S. 22]{Reinganum.1981}.\\
%
Ein weiterer Punkt der nur kurz angeschnitten werden soll, ist der wirtschaftliche Trade-off zwischen der Entwicklung von Produktinnovationen und Prozessinnovationen. Die Verbesserung der Effizienz von Produktionsprozessen ist nur dann sinnvoll, wenn das Gut eine gewisse Beständigkeit auf dem Markt hat und nicht zeitnah durch ein neues ersetzt wird. Denn der Produktionsprozess kann nicht optimiert werden, solange es immer wieder neue Varianten und Güter gibt, die ein anderes Herstellungsverfahren haben. Diesen Zusammenhang beschreibt \cite{Abernathy.1978} in der Automobilindustrie am Beispiel Ford.\\
%
Die Monopolmacht wird in Kapitel \ref{Papier1} aufgegriffen und der damit einhergehende  Anreiz Innovationen zu entwickeln. 
%
\section[Technologiediffusion durch Imitation]{Ausdehnung des technischen Fortschritts:\\Technologiediffusion durch Imitation}
\sectionmark{Ausdehnung des techn. Fortschritts}
%
Nachdem eingehend die Entstehung und Entwicklung des technischen Fortschritts betrachtet wurde, die Innovation, wird im folgenden Kapitel die Ausdehnung des technischen Fortschritts genauer betrachtet, die Imitation. Mit der Adaption von Gütern und Prozessen gilt der Diffusionsprozess als beendet und Wissen wurde erfolgreich transferiert. \\
%
Für die Adaption von Gütern und Prozessen bedarf es nach \cite{Cohen.1989} sowie \cite{Griffith.2004} der gleichen Faktoren wie für Innovationen und zwar technisches Wissen, Sachkapital und Humankapital. Eine Imitation ist eine "`alte"' Innovation, die durch benannte Investitionen nachgeahmt werden kann. Demnach handelt es sich gemäß \cite{Schmookler.1966} bei Imitation um die gleiche technologische Neuerung, mit dem gleichen Erkenntnisgewinn wie bei der Innovation, jedoch zu einem späteren Zeitpunkt. Für eine Imitation ist Humankapital ebenso notwendig wie für eine Innovation, jedoch unterscheiden sich beide durch die eingesetzten Humankapitalniveaus. Grundsätzlich ist für eine Innovation mehr Humankapital notwendig, da neben den Investitionen auch die Idee durch den Einsatz von Humankapital entsteht. Jedoch gibt das Niveau des Humankapitals Aufschluss über die Absorptionsfähigkeit eines Unternehmens oder einer Volkswirtschaft. Denn \cite{Nelson.1966} zeigen, dass je mehr Humankapital für die Nachahmung notwendig ist, desto besser und genauer kann adaptiert werden. Das Humankapital eines Landes kann demnach für innovierende und imitierende Prozesse gleichermaßen eingesetzt werden. \\
%
Von der Gesamtheit der globalen Volkswirtschaften ausgehend ist tatsächlich nur ein sehr geringer Anteil innovierend tätig. Die meisten Länder importieren Technologien und ahmen diese nach statt selbst zu innovieren. In weniger weit entwickelten Ländern beläuft sich die Wachstumsrate durch die Adaption ausländischer Technologien auf ca. 65{\%}. In weiter entwickelten Ländern wird der Großteil (ca. 75{\%}) hingegen durch innovierende Tätigkeiten der heimischen Unternehmen hervorgerufen \cite{Santacreu.2015}. Dies zeigt, wie wichtig der Prozess der Imitation für die Ökonomie ist, da ein beträchtlicher Anteil davon profitiert. Wohingegen die Bedeutung der Innovationsentwicklung von Ländern wie Deutschland, USA oder Japan für das globale Wirtschaftswachstum mindestens ebenso wichtig ist wie die Imitation, da nur hierdurch dauerhaftes Wachstum gewährleistet wird und es somit immer neue Innovationen gibt, die imitiert werden können \cite[Kapitel 18,S.642]{Acemoglu.2009}.\\
%
Sowohl \cite{Arrow.1969} als auch \cite{Evenson.1995} definieren den Innovationsbegriff etwas weiter. Ihrer Ansicht nach beinhalten Innovationen auch nachahmende Prozesse unter Verwendung bereits bestehender Technologien. Es handelt sich dabei nicht um eine kostenlose Kopie von Gütern oder Prozessen, sondern um eine anpassende Übertragung dieser an lokale Gegebenheiten, für die ebenso Investitionen benötigt werden. Demzufolge handelt es sich bei diesem weiter gefassten Verständnis um eine Innovation, jedoch mit imitierenden Elementen.\\
%
Es muss für beide Tätigkeitsfelder, Innovation und Imitation, ein ähnlicher Aufwand im Sinne von Zeit und Produktionsfaktoren aufgebracht werden \cite[S. 826]{Cohen.1989,Griffith.2004,Segerstrom.1991}. Außerdem ist der Erfolg beider von Unsicherheit geprägt. Dies ist der Neuheit des Produktionsprozesses geschuldet, unabhängig davon, ob es sich um die Entwicklung eines vollkommen neuen Gutes bzw. Prozesses handelt, oder ob ein  für das Unternehmen neues Gut oder Prozess nachgeahmt wird \cite[S. 826]{Segerstrom.1991}.\\
%
Die Imitation als technischer Fortschritt kann auch als Technologieübertragung gesehen werden \cite[S. 70]{Cohen.1989,Griffith.2004,Nelson.1966}. Die Technologiediffusion beschränkt sich dabei nicht notwendigerweise auf die Verbreitung innerhalb einer Volkswirtschaft, sondern der Kerngedanke kann auch länderübergreifend übernommen werden. Dann wird wie bei \cite{Nelson.1966} Wissen durch Imitation in ein anderes Land übertragen. \\
%
Wissen nimmt auf zwei Arten zu: Zum einen durch die Verbreitung bereits bekannter Güter und Verfahren, zum anderen durch die Entwicklung neuer Güter und Verfahren. Im ersten Fall handelt es sich um Wissensdiffusion, die durch Imitationen umgesetzt wird. Bei dem zweiten Fall steigt der Wissensstock durch innovierende Tätigkeiten an \cite{Schmookler.1966}.
Als Technologiediffusion oder auch Technologietransfer wird die Verbreitung von technischem Wissen bzw. Technologien bezeichnet. Dies kann durch verschiedene Kanäle geschehen, wie beispielsweise durch Fachzeitschriften, ausländische Direktinvestitionen oder aber auch durch die Migration qualifizierter Arbeitskräfte. In dieser Arbeit liegt der Schwerpunkt auf dem internationalen Handel als Diffusionskanal von technischem Wissen und berücksichtigt die verschiedenen Absichten, Technologiediffusion gezielt hervorzurufen.\\
%
Eine Technologie ist erst dann diffundiert, wenn sie adaptiert wurde. Dabei kann es sich sowohl um die Diffusion von Wissen innerhalb eines Landes zwischen Unternehmen als auch um die grenzüberschreitende Diffusion zwischen Ländern handeln \cite[Kapitel 18, S. 611]{Acemoglu.2009}.\\
%
Aus welchem Grund Technologiediffusion letztendlich beabsichtigt wird, hängt im Wesentlichen von der Perspektive ab. \cite{Arrow.1969} sieht die Motivation für die Übertragung von technischem Wissen in dem Anreiz der Gewinnerzielungsabsichten und beschreibt dabei eher die mikroökonomische Perspektive. Makroökonomisch liegt der Grund des Technologietransfers vielmehr in einem möglichen Entwicklungspotential, das daraus resultieren kann.\\
%
Die Bedeutung des Technologietranfers für den Entwicklungsprozess eines Landes wird erstmals von \cite{Gerschenkron.1962} beschrieben. Dabei unterscheidet er zwischen horizontalem und vertikalem Technologietransfer. Bei der Übertragung und Implementierung technischer Neuerungen vom Forschungs- und Entwicklungsbereich in den Bereich praktischer Anwendung handelt es sich um den vertikalen Technologietransfer. Verlässt man die mikroökonomische Perspektive, dann ist der horizontale Technologietransfer auf der makroökonomischer Ebene zu finden. Dieser wiederum beschreibt die Übertragung von technischem Wissen und Produktionsfertigkeiten über Ländergrenzen hinweg.\\ In dieser Arbeit liegt der Fokus auf dem horizontalen Transfer und steht in einem engen Zusammenhang mit dem catching-up Effekt, dem Aufholprozess einer Volkswirtschaft. Zahlreiche Beispiele zu Zeiten der industriellen Revolution im 19. Jahrhundert untermauern den von \cite{Gerschenkron.1962} und \cite{Veblen.1915} beschriebenen Aufholprozess. So gelang es Deutschland durch Technologietransfer, an das Pionier-Land Großbritannien aufzuschließen. Der Technologie- und Wissenstransfer im 19. Jahrhundert erfolgte durch Kundschafterreisen von Unternehmern und Ingenieuren nach Großbritannien, dem Anwerben britischer Fachkräfte in das eigene Land sowie durch Akademien, wissenschaftliche Gesellschaften und Fachzeitschriften. Die technische Lücke konnte geschlossen werden und liefert Anhaltspunkte, dass dieser sogenannte Velben-Gerschenkron-Effekt auch auf die heutige Zeit und die Problematik der Entwicklungspolitik übertragen werden kann. Dieser Effekt beschreibt den Aufholprozess Deutschlands und Österreichs während der Industrialisierung und hebt dabei unter anderem Bildung, Staatseingriffe und Technologietransfer als wichtige Wachstumsfaktoren hervor \cite[S. 18-19]{Peri.2004}. \\
%
Ein Merkmal von Entwicklungsländern ist der große Abstand zur Welttechnologiegrenze und der damit einhergehende eingeschränkte Zugang zu sowie die Verfügbarkeit von technischem Wissen. Kann das bereits vorhandene Wissen genutzt werden und zusätzlich neues Wissen angeeignet werden, führt dies zum catching-up Prozess. Neben dem Beispiel Deutschlands während der Industrialisierung dienen für die neuere Zeit Japan und die "`Tigerstaaten"' als Musterbeispiele, die heute zu den führenden Industrienationen zählen. Die Ursache für diese Aufholprozesse sieht Gerschenkron in der anfänglichen Rückständigkeit eines Landes. Je rückständiger ein Land entwickelte ist, desto höher ist sein Entwicklungspotenzial. \cite{Nelson.1966}  schränken die These Gerschenkrons ein und halten die Fähigkeiten der Arbeiter im Land für einen weiteren bedeutenden Faktor. Die Rückständigkeit allein helfe einem Land ohne Humankapital nicht die Lücke zum technologisch führenden Land zu schließen. Für \cite{Nelson.1966} gilt, dass je besser ein Land mit adaptiven Fähigkeiten in der Bevölkerung ausgestattet ist, desto schneller findet der Entwicklungsprozess statt. Der Technologietransfer und die imitativen Fähigkeiten im Land können gemäß \cite{Abramovitz.1986} auch als Absorptionsfähigkeit bezeichnet werden, dessen Güte durch die strukturellen Voraussetzungen im Land bedingt wird. Ähneln sich die Strukturen der beiden interagierenden Länder des Technologietransfers, dann unterstützt dies den catching-up Prozess. Jedoch ist zu erwähnen, dass Gerschenkron selbst die Quantifizierung der strukturellen Konstellationen und der Absorptionsfähigkeit als kritisch bewertet. \\
%
Zusammenfassend lässt sich festhalten, dass jede Innovation einen Wissens- und Technologietransfer mit sich bringt, da ein uneingeschränkter Zugang zu Wissen und Ideen besteht und somit jegliche Ideen der Welt mit in den Entstehungsprozess einfließen \cite{Gerschenkron.1962}.
%
\paragraph{Diffusion durch Handel}
Die Wirkung und Intensität des Technologietransfers kann von außen durch die politische Förderung des Bildungssektor, des Forschungssektors oder auch durch den Außenhandel beeinflusst werden.\\ Die Bedeutung des Forschungssektors betonen \cite{Griffith.2004} in ihrer empirischen Arbeit über den Einfluss von Forschung und Entwicklung auf das Wachstum eines Landes. Dabei verdeutlichen sie gleichzeitig den Einfluss der Offenheit eines Landes durch die damit verbundene Technologiediffusion auf das Wirtschaftswachstum. Denn die Forschung wirkt nur dann über beide Kanäle, wenn das tangierte Land bereits Außenhandel aufgenommen hat. Zum einen steigt direkt die Innovationsrate und langfristig mit ihr auch die Wachstumsrate. Zum anderen kommt es zu einem indirekten Effekt auf die Wachstumsrate anderer Länder durch den nun möglichen Technologietransfer, jedoch nur in offenen Volkswirtschaften. Ihre Untersuchung bezieht sich auf die Erhöhung der Intensität des Technologietransfers, wenn Länder ihren Forschungssektor fördern. Demzufolge ist es unabhängig vom technologischen Entwicklungsstand immer angebracht, Investitionen in Forschung und Entwicklung zu tätigen. Dieser Einfluss verstärkt sich erneut  durch die Offenheit eines Landes. Laut \cite{Griffith.2004}  fördert der Ausbau des Forschungssektors sowohl den Aufholprozess durch imitative Aktivitäten, als auch den Entwicklungsprozess von Innovationen.\\
%
Hier soll gezeigt werden, welchen Einfluss der Bildungssektor auf die Technologiediffusion hat und inwieweit der Handel diese anregt.
%
Das weite Feld des "`Brain Drains"', die Abwanderung hochqualifizierter Arbeitskräfte und Wissenschaftler, wird vernachlässigt, da in der folgenden Analyse von Migration abgesehen wird, da diese keinen Schwerpunkt dieser Arbeit darstellt. Demzufolge findet ein Wissenstransfer nicht durch die Übertragung in Form von Zu- oder Abwanderung statt. Diesem Teilbereich der Wachstumstheorie widmen sich  Wissenschaftler wie \cite{Agrawal.2011,Docquier.Sept} und \cite{ONeil.WashingtonDC} mit dem Ergebnis, dass eine Abwanderung sehr gut ausgebildeter Arbeiter nicht den Wissensbestand einer Volkswirtschaft mindert oder sogar erschöpft.  \cite{Docquier.Sept}  belegen in ihrer Untersuchung positive Einflussfaktoren bedingt durch den "`Brain Drain"', da beispielsweise neue Kontakte entstehen und ein Netzwerk aufgebaut werden kann. Ein optimales Einwanderungslevel qualifizierter Arbeiter und Wissenschaftler berechnen \cite{Docquier.Sept} für weniger weit entwickelte Länder.\\
%
Das Modell von \cite{Grossman.1990c} geht von einem aktiven Informationsfluss zwischen Volkswirtschaften aus. Die Mehrheit der Handelsmodelle setzt gemeinhin voraus, dass mit der Öffnung eines Landes allen Wirtschaftsteilnehmern das gesamte Wissen des Weltmarktes zu Verfügung steht, ohne dies zwingend zu fokussieren. \cite{Grossman.1990c} formulieren den Wissenstransfer als bewussten Prozess, der durch das Zusammentreffen von beispielsweise Wissenschaftlern oder Handelsvertretern, die als Bindeglied zwischen den Märkten fungieren, zu Stande kommt.\\
%
Findet Handel statt und werden Technologien oder humankapitalreichere Güter in das Land importiert, dann führt dies nicht zwingend zu einem technischen Fortschritt. Es ist durchaus denkbar, dass der Import zu diesem Land nicht "`passt"' und demzufolge keine Produktivitätssteigerung hervorruft. So verhelfen neue Verfahrenstechniken der Pharmaindustrie einem Land ohne Pharmawesen nicht weiter, der Import ist demzufolge nicht zweckmäßig. Denn ob eine Imitation erfolgreich ist, hängt im Wesentlichen davon ab, ob ausreichend und vor allem angemessen qualifizierte Arbeitskräfte vorhanden sind, die den Nachahmungsprozess durchführen. Auch das kann dazu führen, dass bestimmte Güter oder Prozesse für eine Volkswirtschaft "`noch"' nicht geeignet sind, jedoch in einem späteren Entwicklungsstadium mit einem reformierten und angepassten Bildungssystem die Importe der selben Innovation die Produktivität steigern.
%
In dieser Arbeit wird klar zwischen Innovation und Imitation unterschieden. Als Imitationen werden implementierte ausländische Technologien verstanden. Es wird hier nicht nur graduell zwischen beidem unterschieden, sondern klar differenziert anhand des eingesetzten Humankapitals \cite[S.883]{Cohen.1989,Griffith.2004}.
%