\chapter{Konvergenz}
Ein weiterer, sehr bedeutender Bereich der Wachstumstheorie, die Konvergenztheorie, wurde bisher vernachlässigt und wird im folgenden kurz erläutert. Bei dieser steht nicht die Ergründung von Wachstum im Vordergrund, sondern die Erklärung der Entwicklung verschiedener Wachstumspfade.\\
%
Konvergenz beschreibt die Annäherung des Pro-Kopf-Einkommens bzw. der Wachstumsrate an einen Referenzwert, meist den der übrigen Welt oder den ähnlicher Volkswirtschaften. So zeigte die Entwicklung des Pro-Kopf Einkommens von 13 Ländern zwischen den Jahren 1870-1989, dass viele Länder zu parallelen Wachstumspfaden konvergieren \cite{Evans.1996}. Für die Konvergenz sind zwei Erklärungsansätze denkbar. Konvergenz kann entweder ein Ergebnis abnehmender Erträge der Kapitalakkumulation sein oder aufgrund internationaler Wissens-Spillover-Effekte entstehen. \\
%
Wird von abnehmenden Erträgen der Kapitalakkumulation ausgegangen, führt dies zur \textbf{absoluten Konvergenz} der Wachstumsraten. Empirisch belegt wurde diese These von \cite{SalaiMartin.2002}. Er zeigt, dass während des Beobachtungszeitraums von 1970-2000 die Einkommensungleichheit abnahm und die betrachteten Länder zueinander konvergierten. Die Arbeiten von \cite{Mankiw.1992} oder \cite{Barro.1997} bestätigen ebenfalls, dass relativ weniger weit entwickelte Länder schneller wachsen und gegen die führende Technologie, also die Welttechnologiegrenze, konvergieren.
Aber vergleicht man die ökonomischen Daten der ärmsten und reichsten Länder der Welt miteinander, dann fällt auf, dass eine hohe Ungleichheit zwischen beiden Extremen besteht und diese noch weiter voneinander divergieren \cite{Maddison.2001}. Das Phänomen der \textbf{Großen Divergenz} beschreibt die Ausweitung des Abstandes des Lebensstandards um ein fünffaches zwischen den ärmsten und reichsten Ländern von 1870  bis zum Jahr 1990 \cite{Pritchett.1997}. \cite{Helpman.2004} führt einen bedeutenden Teil der Einkommensunterschiede zwischen den Ländern auf die verschiedenen totalen Faktorproduktivitäten zurück.\\
%
Auf die Beobachtungen einer divergierenden Welt stützte sich \cite{MayerFoulkes.2006} und kategorisierte zunächst seine Daten, indem er fünf Ländergruppen bildete. Dabei stellt er fest, dass die Ungleichheit innerhalb einer Gruppe zwar über den Zeitraum hinweg abgenommen hat, dass aber das Einkommen zwischen den Gruppen divergiert. Die Wachstumsraten vieler armer weniger weit entwickelter Länder divergieren und die relative Divergenzlücke zwischen den Pro-Kopf-Einkommen der ärmsten und reichsten Konvergenzgruppen nahm vom Jahr 1960 bis 1995 um den Faktor 2,6 zu \cite{MayerFoulkes.2006}. Dabei handelt es sich hier um die sogenannte \textbf{bedingte Konvergenz}. Die Wachstumsraten bzw. die Pro-Kopf-Einkommen innerhalb einer Ländergruppe näher sich an, die Konvergenzclubs\footnote{Ein Konvergenzclub besteht aus Volkswirtschaften mit zueinander konvergierenden Wachstumsraten. Dies ist in der Regel dann der Fall, wenn Innovationen entwickelt werden und somit ein Technologietransfer stattfindet. Der damit einhergehende Wissenstransfer führt für weniger weit entwickelte Volkswirtschaften zu einer Eingliederung in den Konvergenzclub, sofern diesem entsprechende Ressourcen für innovierende Tätigkeiten bereitgestellt werden. So besteht bspw. eine Konvergenz zwischen Ländern, die innovativ ausgerichtet sind und dadurch mit der gleichen Rate wachsen. Dies bedeutet dann gleichzeitig, dass mit der Einstellung innovativer Tätigkeiten das Wachstum der Volkswirtschaft langfristig stagniert \cite{Aghion.2015}.} an sich entfernen sich aber von einander \cite{Quah.1993,Howitt.2000,Howitt.2005}. Abhängig von dem Entwicklungsstand eines Landes besteht die Möglichkeit, dass einige Länder frühzeitig stagnieren und das hohe Niveau an der WTG nicht erreichen \cite{Aghion.1992,Barro.1997,Howitt.2005}.\\
%
Der Schwerpunkt dieser Arbeit liegt auf den internationalen Spillover-Effekten durch Außenhandel.
Die Vertreter dieses zweiten Erklärungsansatzes folgen dem schumpetrianischem Ansatz der Wachstumstheorien. Freihandel begünstigt die Möglichkeit der weniger weit entwickelten Länder sich den Industrienationen anzuschließen. Technologisches Wissen passiert die Grenzen und alle beteiligten Länder können davon profitieren. 
Ein Technologietransfer führt zu einer Anpassung der Produktivitäten bei einander ähnlichen Volkswirtschaften, den beschriebenen Konvergenzclubs \cite{Durlauf.1995,Quah.1993,Quah.1997}.\\
%
Die Möglichkeit zu den führenden Ländern aufzuschließen ist als "`catching up"' Prozess bekannt und durch den Rückstand eines Landes hinsichtlich des Entwicklungsstandes bedingt. Zu dieser Erkenntnis des Vorteils des Rückstands kommen auch \cite{Barro.1990,Barro.1991,Barro.1992}. Sie implizieren ebenfalls, dass dadurch die meisten Länder zu parallelen Wachstumspfaden konvergieren. Um ihre These zu belegen untersuchen sie zunächst anhand der Daten von 48 US-Bundesstaaten den Einkommenszuwachs seit 1840 und stellen fest, dass relativ ärmere Bundesstaaten schneller wachsen als reichere und es zu einer Konvergenz aller kommt. In einer folgenden Arbeit, gemeinsam mit \cite{Blanchard.1989}, erweitern sie ihre Analyse auf selbstständige Staaten und vergleichen einerseits das Wachstum relativ weniger weit entwickelter Länder mit relativ weiter entwickelten Ländern und andererseits das Wachstum unterschiedlicher Regionen, wie beispielsweise die Annäherung Süditaliens an Norditalien \cite{Barro.1992}.\\
%
Der Aufholprozess weniger weit entwickelter Länder zur WTG kann auch auf wettbewerbseinschränkende Staatseingriffe zurückgeführt werden. Im 19ten Jahrhundert gelang es den relativ wenig entwickelten Ländern wie Deutschland, Frankreich und Russland durch die Adaption bestehender Produktionsprozesse und einem damit verbundenen hohen Investitionsaufwand die Lücke zu den weiter entwickelten Ländern zu schließen \cite{Gerschenkron.1962}. Dabei hängt der Einfluss von Staatsausgaben auf den Konvergenzprozess entscheidend von dem Entwicklungsstand eines Landes ab. Je weniger weit entwickelt eine Volkswirtschaft ist, desto höher ist die Konvergenzgeschwindigkeit durch die Staatsausgaben. In relativ weit entwickelten Volkswirtschaften ist die Lücke zur Welttechnologiegrenze per se nicht so groß und dementsprechend der Aufholprozess relativ langsamer \cite{Ott.2011}. 
Beziehen sich die Staatsausgaben auf die Finanzierung eines öffentlichen Bildungssystems, fördert dies den Anpassungsprozess weniger weit entwickelter Länder \cite{Glomm.1992}. Volkswirtschaften entscheiden sich für ein öffentliches Bildungssystem, deren Bevölkerung größtenteils unterhalb des durchschnittlichen Einkommens liegt und somit tendenziell weniger weit entwickelt ist.
Untersucht man Humankapital und die unterschiedliche Wirkung öffentlich und privat finanzierter Bildungssysteme, zeigt sich, dass zwar in Ländern mit einem öffentlichen Bildungssystem die Einkommensungleichheit schneller zurück geht, jedoch bei privater Bildung ein höheres Pro-Kopf-Einkommen erzielt wird, sofern die anfänglichen Einkommensunterschiede nicht erheblich waren \cite{Glomm.1992}.\\
%
Neben Staatsausgaben wird der Konvergenzprozess zusätzlich beschleunigt durch die globale Integration handelsliberalisierter Länder. Denn im Vergleich zu geschlossenen Volkswirtschaften  besteht für geöffnete Länder dieser Vorteil des Rückstands, der zu einem catching up Prozess führen kann. Denn Handel bedingt ein schnelleres Aufholen weniger weit entwickelter Länder \cite{Sachs.1995}. Dieser Rückstand erklärt auch das starke Wachstum exportorientierter osteuropäischer Staaten \cite{Ventura.1997}.\\
%
Der Technologietransfer zwischen Ländern führt jedoch nur zu einer Konvergenz der Wachstumspfade, sofern sich ein Land im geschlossenen Zustand ebenfalls gemäß einer positiven Wachstumsrate entwickelt hat. Ist dies nicht der Fall wird die Volkswirtschaft stagnieren. Dadurch wird verdeutlicht, dass die Offenheit eines Landes keinen wesentlichen Einfluss auf die Konvergenz einer Volkswirtschaft hat \cite{Howitt.2000}. 
Es sei denn, internationaler Handel ist durch Produktivitätsunterschiede der Volkswirtschaften bedingt, dann kann dies zu einer einheitlichen weltweiten Einkommensverteilung führen \cite{Howitt.2000,Acemoglu.2002,Eaton.2001}.\\
%
Anderer Meinung sind \cite{Galor.2006,Galor.2008}. Sie verdeutlichen, dass die Auswirkungen der Außenhandelsöffnung stark den Entwicklungsstand eines Landes beeinflussen kann. Sie führen die Divergenz zwischen den industrialisierten und den nicht-industrialisierten Ländern darauf zurück, dass die entsprechenden Ländergruppen unterschiedlich mit ihren Handelsgewinnen umgegangen sind und jeweils eine andere Strategie verfolgt haben. So lag der Schwerpunkt der heute weniger weit entwickelten, nicht-industrialisierten, Länder in der Drosselung des Bevölkerungswachstums. Wohingegen die heutigen industrialisierten Länder bestrebt waren den Pro-Kopf-Output zu erhöhen, indem sie beispielsweise den Bildungssektor förderten \cite{Galor.2006}.\\
%
Zusammenfassen lässt sich festhalten, dass Konvergenz durch den internationalen Wissenstransfer begünstigt wird und eine Handelsoffenheit zu einem catching up Prozess führen kann, jedoch ist dies von der Entwicklungsstrategie einer Volkswirtschaft abhängig.