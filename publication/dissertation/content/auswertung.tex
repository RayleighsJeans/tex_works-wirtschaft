\chapter{Auswertung}\label{Auswertung}
%
In Bezug auf die in Kapitel \ref{Einleitung} aufgestellten Thesen zeigt die vorangegangene Analyse, dass Außenhandel die Bedeutung des Humankapitals für den Wachstumsprozess betont. In einem weiteren Schritt wurde gezeigt, dass nicht nur Humankapital per se wichtig ist, sondern auch seine Zusammensetzung,  bezogen auf heterogene Fähigkeit, der imitierenden und innovierenden Art. Außenhandel stimuliert die Innovationsfähigkeit von Volkswirtschaften und bestätigt zudem, dass abhängig vom Entwicklungsstand auch die Imitationsstrategie den Entwicklungsprozess voranbringt. Es wurde gezeigt, dass Freihandel durch eine entsprechende Strategiewahl die Entwicklung aller Länder begünstigt, unabhängig vom Entwicklungsstand und der Beschaffenheit der Welttechnologiegröße. Technologisch kleine sowie technologisch große Länder profitieren vom Wissenstransfer und ebnen damit die Möglichkeit für dauerhaftes Wachstum.\\
%
Es konnte gezeigt werden, dass weniger weit entwickelte Länder vom Handel mit relativ humankapitalreich produzierten Gütern profitieren und sich dadurch ihr eigenes Bildungswesen verbessert. Diese Anhebung des Bildungsniveaus eines Landes führt zu einem höheren Wachstumspfad, durch die Weiterentwicklung des technologischen Entwicklungsstands. So bewirkt Außenhandel mit humankapitalreichen Gütern, dass relativ weiter entwickelte Länder die Innovationsstrategie verfolgen und relativ weniger weit entwickelte die Imitationsstrategie.\footnote{Je nach Nähe zur WTG könnte jedoch ein Wechsel angeraten werden.} Unabhängig von der technologischen Größe des Landes resultiert das gleiche Ergebnis. Demzufolge wird auch bei technologisch großen Ländern, die eine endogene WTG bedingen, mit abnehmender Distanz zur WTG die Innovationsstrategie präferiert. Der Abstand zur WTG wird sich zwar ausweiten, jedoch führt Außenhandel zu einem absolut gesehen zukünftig höheren Entwicklungsstand.\footnote{In der folgenden Auswertung wird nicht mehr zwischen den Ergebnissen bei einer endogenen und exogenen WTG unterschieden, da dies keinen wesentlichen Einfluss auf die strategische Entscheidung eines Landes hat.}\\
%
Differenziert man zusätzlich zwischen den Sektoren, was aus der exportunterstützenden Politik folgte\footnote{Diese wurde in Kapitel \ref{Papier1} modelliert.}, dann zeigt dies ebenfalls, dass bei einem sehr hohen Abstand zur WTG durchaus die Imitationsstrategie im Importsektor zu präferieren ist. Dies sollte einer Volkswirtschaft dazu verhelfen eine Basis an technologischem Wissen und Humankapital aufzubauen, indem zunächst Wissen importiert wird, das nun nachgeahmt werden kann, bevor ein Wechsel zur Innovationsstrategie sinnvoll ist. Dieser Zusammenhang wurde bereits von \cite{Glass.1999} gezeigt und kann hier auf andere Art und Weise bestätigt werden. Der Schwerpunkt des Importsektors liegt sowohl bei den weniger weit entwickelten, als auch bei den relativ weiter entwickelten Volkswirtschaften auf der Imitationsstrategie. Der Exportsektor hingegen stellt sich besser bzw. es ist profitabler der Innovationsstrategie zu folgen.\\ 
%
Weiterhin konnte gezeigt werden, dass der Schwerpunkt eines weniger weit entwickelten Landes im Importsektor auf der humankapitalintensiven Produktion liegt und es folgt damit die Imitationsstrategie. Durch die Öffnung des Landes verschlechtern sich die Entwicklungschancen des Sektors. 
Außerdem lässt sich daraus schlussfolgern, dass Volkswirtschaften, die der Imitationsstrategie folgen und einen relativ geringen Entwicklungsstand vorweisen tatsächlich ein höheres Wachstum erzielen, indem sie, unabhängig von einem existierenden Bildungssektor, mehr physisches Kapital in die Konsumgüterproduktion investieren. Wird nun das fehlende Kapital noch aus der Humankapitalakkumulation bezogen, wie es in der hier angeführten Modellwelt aus Kapitel \ref{Papier2} vorgesehen ist, ist zu erwarten, dass dies den Effekt verstärkt und eine noch geringere Wachstumsrate folgen könnte. Entwicklungsstrategisch wird damit die Annahme bestätigt, dass weniger weit entwickelte Länder in den Bildungssektor nur Humankapital investieren sollten.\\
%
Die gewonnenen Erkenntnisse bestätigen eine Arbeit von \cite{Mies.2013}, die den Ansatz verfolgte den Humankapitaleinsatz bei der Produktion von Imitationen hinsichtlich ihrer Intensität zu unterscheiden und danach eine Strategie zu wählen. Sie kommt zu dem Ergebnis, dass in relativ weniger weit entwickelten Ländern bei einem hohen Einsatz von Humankapital im Herstellungsprozess adaptierter Güter ein Wachstumspfad erreicht wird, der in einem geringen gleichgewichtigen Einkommen mündet. Wohingegen ein geringer Einsatz von Humankapital bei der Produktion zu einem höheren Einkommen führen kann. Die Wahl der Produktionsstrategie im adaptierenden Sektor hängt demzufolge vom Entwicklungsstand des Landes ab. Je weiter entwickelt ein Land ist, desto mehr Humankapital sollte in den Produktionsprozess eingehen und desto weiter entwickelte Technologien können angewendet werden \cite{Mies.2013}.\\
%
Ferner zeigt eine zusammenfassende Analyse der Handelseffekte, dass auch diese in den hier angeführten und sehr verschiedenen Wachstumsmodellen bestätigt werden konnten.\\
%
Der \textbf{Wettbewerbseffekt} beschreibt die gestiegene Rivalität der Anbieter durch den Zusammenschluss des heimischen mit dem ausländischen Markt. Der Wettbewerbsdruck veranlasst die Produzenten zu geringeren Grenzkosten zu produzieren, indem Ineffizienzen behoben werden, oder aber neue Güter zu entwickeln und sich somit von den Mitstreitern abzusetzen. Beides geht mit Innovationen einher. Demzufolge bedingt Freihandel einen höheren Innovationsanreiz und es folgt eine höhere Innovationsrate. \\
%
Bei erfolglos innovierenden Unternehmen geht das Risiko einher den bisherigen Absatz an ausländische erfolgreichere Anbieter zu verlieren. Es folgt der sogenannte Flucht-Eintritts-Effekt, der ein Bestreben der Unternehmen das Risiko eines Marktaustritts zu mindern bewirkt. Das Risiko des Marktaustritts erhöht also die Innovationsrate. Hinzu kommt außerdem, dass es den Unternehmen nicht nur um eine bestehende Position am Markt geht, sondern nun auch die Möglichkeit existiert die ausländischen Anbieter zu verdrängen und zusätzliche Gewinne zu erwirtschaften. \\
%
Der Wettbewerbseffekt führt einerseits im Inland zum Ausscheiden unproduktiver Unternehmen und damit zu einem Anstieg der gesamten Wirtschaftsleistung eines Landes. Andererseits induziert der zusätzliche Wettbewerbsdruck eine steigende Innovationstätigkeit der Unternehmer. \\ 
%
Die vorangegangenen Untersuchungen haben gezeigt, dass es zu einem Anstieg der Innovationsrate per Außenhandel kommt. Das erörterte Modell in Kapitel \ref{Papier1} verdeutlicht dies durch einen grundsätzlich früheren Wechsel zur Innovationsstrategie, unabhängig vom technologischen Entwicklungsstand eines Landes. Für die Umsetzung dieser Strategie ist qualifizierte Arbeit notwendig. Der Anstieg der Nachfrage an ausgebildeter Arbeit wird, wie auch in Kapitel \ref{Papier2} gezeigt, durch den Außenhandel induzierten Anreiz befriedigt, der die Haushalte veranlasst tendenziell eher in Weiterbildung zu investieren.\footnote{Es wurde gezeigt, dass durch Außenhandel die Entscheidungsvariable im Gleichgewicht $u^*$ ansteigt.} Demzufolge konnte der Wettbewerbseffekt in dieser Arbeit bestätigt werden. \\
%
Der \textbf{Marktgrößeneffekt} spielt bei der Öffnung eines Landes ebenfalls eine Rolle. In einem ökonomisch kleinen Land besteht die Möglichkeit, dass sich die Durchführung einiger Innovationen nicht lohnen würde, da die Forschungs- und Entwicklungskosten den erwarteten Gewinn übersteigen. Der erwirtschaftete Gewinn einer Innovation ist aus beiden Märkten  deutlich höher, als wenn die Innovation nur in einem Markt eingeführt worden wäre. Demnach kann eine zuvor noch unrentable Innovation nun lohnend sein. Alle weiteren Innovationen die bei geringen Gewinnaussichten durchgeführt worden wären, führen bei steigender Marktgröße zu deutlich höheren Erträgen.
Auch hier steigt die Innovationstätigkeit an und hat abhängig von der ökonomischen Größe eines Landes unterschiedliche Wachstumswirkungen. Denn in ökonomisch großen Ländern ändert sich die Marktgröße nicht so stark wie ökonomisch kleine Länder, die sich dem Handel öffnen. Demzufolge ist auch der Wachstumseffekt in ökonomisch kleinen Ländern höher als in ökonomisch großen Volkswirtschaften.\footnote{In diesem Zusammenhang ist nicht der negative Wachstumseffekt bei einer Ausweitung der endogenen Welttechnologiegrenze die Rede, sondern von der zusätzlichen Entwicklung eines Landes, die sich in Wachstum äußert.}\\
%
Grundsätzlich bedingt auch dies wieder einen höheren Entwicklungsstand durch die Innovationsstrategie und ist nun auch für Länder mit einem relativ gesehen größeren Abstand zu WTG ratsam, als in der Autarkiesituation. Wie schon zuvor beschrieben wird mehr ausgebildete Arbeit nachgefragt, die auch tatsächlich vorhanden ist.\footnote{Zumindest in einem größeren Umfang als in geschlossenen Volkswirtschaften.} Jedoch wird im Hauptteil dieser Arbeit nicht zwischen ökonomischen Ländergrößen unterschieden. Demzufolge kann auch ein stärkerer Wachstumseffekt bei ökonomisch kleinen Ländern nicht nachgewiesen werden. Da hier nur ökonomisch kleine Länder betrachtet werden wird lediglich angenommen, dass der Marktgrößeneffekt deutlich spürbar sein müsste. Neben der Innovationstätigkeit bewirkt der Marktgrößeneffekt allgemein, dass grundsätzlich ökonomisch kleine Länder stärker  vom Handel profitieren als dies bei großen Ländern der Fall ist. Dies ist ebenfalls durch das Ausmaß der Veränderung der Marktgröße zu erklären, woraus sich auch andere Gewinnmöglichkeiten ergeben. Somit ist der Zugewinn eines kleinen Landes relativ höher, als der eines großen Landes, welches nur im geringen Maße von der Markt\-erweiterung profitiert. \\
%
Diesen Zusammenhang bestätigen auch \cite{Alesina.2005} in ihrer Regression von Wachstum auf die Handelsoffenheit. Die Autoren haben in ihren Untersuchungen einen negativen Koeffizienten zwischen der Offenheit eines Landes und der Landesgröße festgestellt. Dabei endogenisieren sie die Größe eines Landes und können den Einfluss vom Außenhandel auf die Ländergröße hinsichtlich des ökonomischen Wachstums beobachten.\\
%
Durch die Einführung einer Exportförderung können außerdem hinsichtlich der Sektorgröße Aussagen getroffen werden. Wie in Kapitel \ref{Papier1} anhand des Modells gezeigt wurde, führt dies zu einer Fokussierung auf den Exportsektor, dem damit tendenziell größere Projekte zugeteilt werden. Daraus resultieren unterschiedlich große Ex- und Importsektoren. Obwohl der Exportsektor aktiv unterstützt wird, profitiert der nun relativ kleinere Importsektor stärker von Außenhandel.\footnote{Dieser Zusammenhang konnte hier für ein technologisch kleines Land jedoch nicht bestätigt werden. Denn im Importsektor verschlechtern sich durch Handel die Entwicklungsmöglichkeiten, wohingegen sie sich im Exportsektor verbessern. In einem technologisch großen Land hingegen trifft diese Aussage zu und Außenhandel fördert den vorwiegend imitierenden kleineren Importsektor stärker als den Exportsektor.} Diesen Zusammenhang zeigten ebenfalls \cite{Aghion.2013} anhand der Daten Südafrikas. 
Dabei weisen sie eine Förderung des Produktivitätswachstums durch die stetige Öffnung des Landes in kleineren Sektoren nach.  Weil Südafrika von einer heterogenen Struktur der Sektoren geprägt ist, können sie sogar spezifisch zeigen, dass Handel in relativ kleinen Sektoren einen stärkeren positiven Effekt hat als in relativ großen Sektoren.\\
%
Bezieht man sich nun auf die verschiedenen Entwicklungsstände einer Volkswirtschaft 
wurde grundsätzlich gezeigt, dass weniger weit entwickelte Volkswirtschaften eher der Imitationsstrategie folgen sollten. Die Bedeutung von Innovationen nimmt also erst mit steigendem Abstand zur WTG ab. Denn es ist möglich, dass sich ein Land von seiner "`schlechten"' Position entmutigen lässt und somit Handel negative Innovationsanreize setzt. Dieser Entmutigungseffekt führt bei relativ rückständigen Volkswirtschaften zu dem Impuls sich von jeglichen Innovationstätigkeiten abzuwenden. Zwar regen erfolgreiche Innovationen den Aufholprozess an, dies erscheint jedoch in Anbetracht möglicher Imitationen als sehr ressourcenaufwendig und nicht wirtschaftlich.  
Dieser Effekt konnte durch die vorgenommene Analyse nachgewiesen werden. Es wird vielmehr verdeutlicht, dass die Unternehmen nicht nur entmutigt werden, sondern, dass es grundsätzlich auch rentabler ist, mit einem relativ geringen technischen Entwicklungsstand zu imitieren als zu innovieren.\footnote{Da in der vorgelagerten Untersuchung aus Kapitel \ref{Papier2} nicht zwischen innovierenden und imitierenden Tätigkeiten unterschieden wird, können auch zu diesem Punkt keine Aussagen getroffen werden.}\\
%
Die Wirkung vom Außenhandel ist auch vom Entwicklungsstand eines Landes abhängig. Die Öffnung eines Landes stimuliert das Wachstum, jedoch profitieren die weniger weit entwickelten Länder stärker von den sogenannten \textbf{Wissens-Spillover-Effekten} als die weiter entwickelten Länder, die das Wissen "`abgeben"' \cite{Sachs.1995,Grossman.1990b}. Hier ist das Ausmaß der Aufholmöglichkeit entscheidend. Je weiter ein Land entwickelt ist, desto geringer sind die zusätzlichen Gewinne, die durch die Einführung neuer Technologien generiert werden können. Ein relativ weniger weit entwickeltes Land hingegen kann hinsichtlich des technologischen Fortschritts deutlich stärker aufholen und profitiert somit mehr von handelsliberalisierenden Maßnahmen, als  ein Land, das weit entwickelt ist und somit relativ wenig Möglichkeiten hat neue Technologien einzuführen durch die es bereichert wird \cite{Keller.2004}. Handel verstärkt eindeutig diesen Effekt, weil beispielsweise ausländische Forschungsinvestitionen mit zunehmendem Offenheitsgrad zu inländischen Produktivitätseffekten führen \cite{Coe.1995}.\\ 
%
Die Berücksichtigung des Entwicklungsstandes in dieser Arbeit erlaubt es Aussagen über den Spillover-Effekt vom Handel treffen zu können. Er ist sogar Kern der Überlegung, dass ein weniger weit entwickeltes Land von dem Handel mit einem weiter entwickelten Land profitiert. So beeinflusst zwar einerseits der Wissenstransfer die Innovationsrate positiv, andererseits führt dies ebenfalls zu einer höheren Imitationstätigkeit, die ebenso den technologischen Wissensstand eines Landes erhöht. Dieser Effekt des Entscheidungsproblem basiert auf der Humankapitalakkumulation durch den Wissenstransfer und führt in weniger weit entwickelten Ländern zu einem Aufholprozess.\\
%
Es bleibt jedoch noch die Frage nach der hier entwickelten Entwicklungsstrategie zu klären. Bis in die 1970er Jahre war es in vielen Entwicklungsländern üblich die importierten Industriegüter durch heimische Produkte zu ersetzen und somit die Importe einzuschränken. Diese Importsubstitution und auch andere protektionistische Maßnahmen führten beispielsweise in Ländern Lateinamerikas wie Brasilien oder Mexiko zu einem zu starken wirtschaftlichen  Wachstum. Überholt wurden diese mittlerweile stagnierenden Länder durch noch stärker wachsende Volkswirtschaften wie HongKong oder Singapur, deren Wachstum durch noch stärker wettbewerbseinschränkende politische Maßnahmen stimuliert wurde. Auf andere Art und Weise, jedoch genauso erfolgreich, gelang es Ländern wie Japan und Korea ein hohes Wirtschaftswachstum zu generieren. Sie haben auf starke Wettbewerbseinschränkungen verzichtet und der Schwerpunkt wurde auf hohe Investitionstätigkeiten, staatliche Subventionen und Konglomerate gelegt. Dieser strategische Ansatz wurde auch in Kapitel \ref{Papier1} implementiert und verdeutlichte die Wirkung vom Außenhandel auf den technologischen Fortschritt eines Landes.\\
%
Die vorgelegte Arbeit zeigt, dass es in dem hier angeführten Zusammenhang nicht notwendig ist weniger weit entwickelte Länder durch protektionistische Maßnahmen zu schützen. Denn 
Länder die noch weit von der WTG entfernt sind, stellen sich mit hohen Markteintrittsbarrieren, wie beispielsweise Zölle oder Kontingente nicht zwingend besser. Staatliche Eingriffe, die den Freihandel unterstützen führen zu einer geeigneten strategischen Ausrichtung mit einem anhaltenden Wachstum. So wurde bereits gezeigt, dass durch gezielte Investitionen die Strategie gelenkt werden kann. Weil die Innovations- und Imitationstätigkeiten von verschieden ausgebildeten Arbeitern durchgeführt werden, kann man aus diesem Umstand eine gezielte Entwicklungsstrategie ableiten. Wird der Bildungsstand wie von \cite{Benhabib.1994} anhand der Bildungsausgaben charakterisiert, dann führen Bildungsausgaben in den Bildungsbereich, der eine solide Grundausbildung der Bevölkerung sichert, zu erfolgreichen Imitationen. Die Innovationstätigkeit eines Landes wird durch die Unterstützung des höheren Bildungsbereiches intensiviert. Wie in dieser Arbeit gezeigt wurde steigt mit der Nähe zur WTG die Bedeutung von Innovationen. Dann folgt daraus, dass auch Investitionen im höheren Bildungsbereich mit der Nähe zur WTG an Bedeutung zunehmen.\footnote{Auf politischer Ebene lässt sich laut \cite{Vandenbussche.2006} daraus herleiten, dass technologisch weniger weit entwickelte Länder besser durch Bildungsinvestitionen in die Grundausbildung unterstützt werden, wohingegen das Produktivitätswachstum relativ weit entwickelter Länder durch Investitionen in den höheren Bildungsbereich gefördert werden.}
Wohingegen in Ländern, die relativ weit von der WTG entfernt sind, eher von Bildungsausgaben profitieren, die Grundkenntnisse und einfache Fertigkeiten fördern.\\
%
Zusammenfassend und in Bezug auf die aufgestellten Thesen lässt sich festhalten, dass politische Handlungsempfehlungen von der Lage zur Welttechnologiegrenze abhängen. Wird zwischen Innovationen und Imitationen anhand des Abstandes zur Welttechnologiegrenze unterschieden, dann lassen sich diesen verschiedene Segmente des Bildungssystems zuordnen. Die Bedeutung von Investitionen in die Grundausbildung, welche vor allem die Imitationstätigkeit unterstützen, nimmt mit der Nähe zur WTG ab. Wohingegen die Rolle höherer Bildungsinvestitionen mit der Lage zur WTG zunimmt.\\
%
In der vorliegenden Arbeit wurde eine Entwicklungsstrategie vorgestellt, die zunächst ein Angebot an qualifizierter Arbeit bereitstellt, damit diese anschließend durch gezielte Investitionen den technologischen Entwicklungsstand eines Landes und somit letztendlich auch das Wachstum begünstigt. Begründet wird das Wachstum durch den technischen Fortschritt und die Humankapitalakkumulation, ausgelöst und begünstigt durch den Außenhandel und den damit einhergehenden Effekten.
