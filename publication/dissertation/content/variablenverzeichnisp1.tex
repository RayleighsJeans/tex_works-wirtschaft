\begin{longtable}{l|l} %Beginn Tabelle

	\textsc{Variable} & \textsc{Bedeutung}\\
	\hline
	\endhead
	
	\endfoot

		${A}_{t} $ & lokale Technologiegrenze; durchschnittliche Produktivität des Landes\\
		$A_{t}(\nu)$ & Produktivitätsparamter des Zwischeng\"utersektors\\
		$a_{r}$ & Grenzwert unter Berücksichtigung der Kosten\\
		$a_{tj}$ & Abstand zur Welttechnologiegrenze\\%kommt mit unterschiedlichen Indizes vor
		$\bar{A}_{t} $ & Welttechnologiegrenze\\
		$\bar{A}_{0}$ & anf\"angliche Welttechnologiegrenze\\
		$\tilde{a}$ & maximal erzielbarer Wissensstand \\
		$E_{t}V_{t,j}$ & erwarteter Nutzen einer Alternative\\
		$e$ & Alter des Ingenieurs\\
		%$e\in\left\{y,o\right\}$ & Ausprägungsformen des Alters\\
		$g$ & exogene Wachstumsrate der Welttechnologiegrenze\\
		$g_{j}$ & endogene Wachstumsrate der Welttechnologiegrenze\\
		$I$ & Innovation\\
		$i$& Zins\\
		$k_t$ & Kosten im Zwischengutsektor\\
		$L$ & Produktionsfaktor Arbeit\\
		$\mathcal{L}$ & Lagrangefunktion\\
		%$N$ & Arbeiter für Produktion des Endprodukts\\
		%$N_{t}$ & Menge an Arbeit\\
		$N_{j}$ & Arbeiter im Sektor $j$\\
		$N+1$ & Bev\"olkerungsgrö{\ss}e\\
		$o$ & erfahrener Ingenieur\\
		$p^*$ & Preis des Importgutes\\
		$p_{t,j}(\nu)$ & limitierender Preis eines Monopolisten\\
		%$R_{t,j}$ & Parameter zum Austausch von Unternehmer\\
		$RE_{t}(\nu|s,e,z)$ & Gewinnrücklagen eines Unternehmens\\
		$r$ & Zinssatz\\
		$\hat{{RE}}_{t}(\nu|s,e,z)$ & ausgeschütteter Gewinn\\
		$s_{t}(\nu)$ & Projektgr\"o{\ss}e\\
		%$s\in\left\{\sigma,1\right\}$ & ???\\
		$t$ & Zeit\\
		$V_{t,j}$ & Nutzen einer Alternative gro\ss{}er Projekte\\
		$w_{j}$ & Lohn\\
		%$w_{t}(\nu|s,e,z)$ & Unternehmergehalt\\
		$x(\nu)$ & Menge der Zwischeng\"uter $\nu$\\
		$y$ & junger Ingenieur\\
		$y_{j}$ & aggregiertes Einkommen/ Output \\
		$z$ & Qualifikation des Ingenieurs\\
		%$z\in\left\{L,H\right\}$ & ???\\
		$\alpha$ & Produktionselastizit\"at des Faktor Kapital\\
		$\gamma_{t}$ & hohe technische F\"ahigkeiten des Ingenieurs/ Innovationsintensität\\
		%$\gamma_{t}(\nu)$ & ???\\%Parameter bereits oben benutzt aber ohne Abhängigkeit; dort Fähigkeiten des Unternehmens
		$\delta_{j}$ & Indikator f\"ur Wettbewerbsdruck\\
		$\eta$ & Produktivitätssteigerung durch Nachahmung / Immitationsintensität\\
		$\kappa$ & Investitionskosten\\
		$\lambda $ & hohe Wahrscheinlichkeit f\"ur hohe technische F\"ahigkeit\\
		$(1-\lambda)$ & niedrige Wahrscheinlichkeit f\"ur hohe technische F\"ahigkeit\\
		$\mu$ & finanzieller Schaden an Unternehmen durch Ingenieur\\
		$(1-\mu)$ & Gewinnbeteiligung\\
		$\nu$ & Zwischengut\\
		$\pi_{t,j}(\nu)$ & Gewinn im Zwischensektor\\%Achtung auf Seite 14 steht j groß außerhalb des Index
		$\sigma_{j}$ & kleines Projekt\\
		$\phi$ & anteilige Investitionskosten\\
		$\chi_{j}$ & Limit Preis\\
		
\caption{Variablenverzeichnis zu Kapitel \ref{Papier1}}
\end{longtable}
