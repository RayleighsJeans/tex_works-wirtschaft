\chapter[Mathematischer Anhang zu Kapitel \ref{Papier2}]{Mathematischer Anhang\\zu Kapitel \ref{Papier2}}
\section{Autarkie}\label{AutarkieAPPENDIX}

Die Ableitungen nach der Zeit - die Bewegungsgleichungen - für physisches Kapital und Humankapital lauten:
\begin{equation}
\dot{k}(t)=A(v(t)k(t))^\alpha(u(t)h(t))^{1-\alpha}-c(t)
\end{equation}
\vspace{-0.7cm}
\begin{equation}
\dot{h}(t)=B((1-v(t))k(t))^{\eta}((1-u(t))h(t))^{1-\eta}
\end{equation}
Aus Gründen der Anschaulichkeit wird im folgenden die Abhängigkeit der Variablen gegenüber der Zeit $t$ vernachlässigt. Die Wachstumsrate des physischen Kapitals lautet: 
\begin{equation*}
\hat{k}=Av^\alpha k^{\alpha-1}(uh)^{1-\alpha}-\frac{c}{k}
\end{equation*}
mit $\chi=\frac{c}{k}$ ergibt sich
\begin{equation}
\hat{k}=Av^\alpha u^{1-\alpha}\left(\frac{k}{h}\right)^{\alpha-1}-\chi
\end{equation}
Durch die Substitution von $x_1=\frac{vk}{uh}$ lässt sich die Wachstumsrate in verkürzter Form darstellen. 
\begin{equation}
\boxed{\hat{k}=Ax_1^\alpha \frac{uh}{k}-\chi}
\end{equation}
Das Humankapital wächst in diesem Modell wie folgt: 
\begin{equation}
\hat{h}=B\left[(1-v)\frac{k}{h}\right]^{\eta}(1-u)^{1-\eta}
\end{equation}
Ebenfalls lässt sich die Wachstumsrate durch eine Substitution von $x_2=\frac{(1-v)k}{(1-u)h}$ verkürzt darstellen. 
\begin{equation}
\boxed{\hat{h}=Bx_2^\eta(1-u)}
\end{equation}

%Daraus lässt sich auch die Wachstumsrate des Verhältnisses beider herleiten.
%\begin{equation}
%\hat{\left(\frac{k}{h}\right)}=\hat{k}-\hat{h}
%\end{equation}
%\begin{equation}
%\hat{\left(\frac{k}{h}\right)}= Av^\alpha u^{1-\alpha}\left(\frac{k}{h}\right)a^{\alpha-1}-\chi-\chi_{ex}+p^*\chi_{im}-B\bar{B}\left[(1-v)\left(\frac{k}{h}\right)\right]^{\eta}(1-u)^{1-\eta}
%\end{equation}
%Im Außenhandelsgleichgewicht entspricht sich der Wert der Exportegütermenge mit dem der Importgütermenge, demnach gilt $\chi_{ex}=p^*\chi_{im}$.
%Die Kurzform lautet: 
%\begin{equation}
%\hat{\left(\frac{k}{h}\right)}=Ax_1^{\alpha-1}-\chi-B\bar{B}x_2^{\eta}\label{WachstumOmega}
%\end{equation}

Der Haushalt löst das Maximierungsproblem mit Hilfe der Hamiltonfunktion.\\
\begin{equation}
\begin{split}\mathbb{H}=&~e^{-\rho t}\frac{c^{1-\sigma}}{1-\sigma}\\
&+\gamma_1(A(vk)^\alpha(uh)^{1-\alpha}-c)\\
&+\gamma_2B[(1-v)k]^{\eta}[(1-u)h]^{1-\eta}\end{split}
\end{equation}
Die Bedingungen erster Ordnung werden bestimmt durch:\\
\begin{align}
&\frac{\partial\mathbb{H}}{\partial c}\overset{!}{=}~0\label{eq:foc1}\\
&\frac{\partial\mathbb{H}}{\partial v}\overset{!}{=}~0\label{eq:foc2}\\
&\frac{\partial\mathbb{H}}{\partial k}\overset{!}{=}-\dot{\gamma_1}\label{eq:foc3}\\
&\frac{\partial\mathbb{H}}{\partial u}\overset{!}{=}~0\label{eq:foc4}\\
&\frac{\partial\mathbb{H}}{\partial h}\overset{!}{=}-\dot{\gamma_2}\label{eq:foc5}\end{align}
Die Berechnung von Gleichung \eqref{eq:foc1} ergibt:\\
\begin{equation*}
\partial\mathbb{H}/\partial c\overset{!}{=}0
\end{equation*}
\begin{equation}
e^{-\rho t}c^{-\sigma}-\gamma_1 \overset{!}{=} 0\label{eq:foc1a}
\end{equation}
\vspace{-0.7cm}
\begin{equation}
\gamma_1 = e^{-\rho t}c^{-\sigma}\label{eq:foc1b}
\end{equation}
Die Bewegungsgleichung des Schattenpreises $\gamma_1$ wird später für die Herleitung der  Keynes-Ramsey-Regel benötigt. Somit ist die Ableitung nach der Zeit von Gleichung \eqref{eq:foc1b} zu bilden.
\begin{equation}\frac{\partial\gamma_1}{\partial t} = \dot{\gamma}_{1}\end{equation}
\begin{equation*}
		\dot{\gamma}_{1} = -e^{-\rho t}\rho c^{-\sigma}-e^{-\rho t}c^{-\sigma-1}\sigma\dot{c}
\end{equation*}
\vspace{-0.7cm}
\begin{equation}
\dot{\gamma}_{1} = -e^{-\rho t}c^{-\sigma}(\rho+\sigma\hat{c}) = -\gamma_1(\rho+\sigma\hat{c})\label{eq:foc1c}
\end{equation}
Dividiert man diesen Term durch $\gamma_1$ aus \eqref{eq:foc1b} dann lässt sich $\gamma_1$ kürzen und es ergibt sich die Wachstumsrate des Schattenpreises für Gut 1 in einer geschlossenen Volkswirtschaft.
\begin{equation}
\hat{\gamma_1}=-\rho-\sigma\hat{c}\label{eq:foc1d}
\end{equation}
Die Bedingung laut Gleichung \eqref{eq:foc2} bestimmt die optimale Aufteilung des physischen Kapitals der Wirtschaftssubjekte zwischen dem Produktions- und Bildungssektor, bei der  der Nutzen über die Zeit maximiert wird.
\begin{equation*}
\partial\mathbb{H}/\partial v\overset{!}{=}0
\end{equation*}
\vspace{-0.2cm}
\begin{equation}
\gamma_1A\alpha v^{\alpha-1}k^\alpha(uh)^{1-\alpha}-\gamma_2B\eta(1-v)^{\eta-1}k^\eta[(1-u)h]^{1-\eta}\overset{!}{=}~0
\end{equation}
\vspace{-0.7cm}
\begin{equation}
\gamma_1A\alpha v^{\alpha-1}k^\alpha(uh)^{1-\alpha}=\gamma_2B\eta(1-v)^{\eta-1}k^\eta[(1-u)h]^{1-\eta}
\end{equation}
Daraus lässt sich das Verhältnis der Schattenpreise beider Güter herleiten.
\begin{equation}
\frac{\gamma_2}{\gamma_1}=\frac{A\alpha v^{\alpha-1}k^\alpha(uh)^{1-\alpha}}{B\eta(1-v)^{\eta-1}k^\eta[(1-u)h]^{1-\eta}}\label{Verhaltnisherleitung1WM}
\end{equation}
\begin{equation}
\quad ~~=\frac{A\alpha \left(\frac{vk}{uh}\right)^{\alpha-1}}{B\eta\left(\frac{(1-v)k}{(1-u)h}\right)^{\eta-1}}=\frac{A\alpha x_1^{\alpha-1}}{B\eta x_2^{\eta-1}}\label{Verhaltnisherleitung1aWM}
\end{equation}
\begin{equation}
\gamma_2=\gamma_1\frac{A\alpha \left(\frac{vk}{uh}\right)^{\alpha-1}}{B\eta\left(\frac{(1-v)k}{(1-u)h}\right)^{\eta-1}}\Longleftrightarrow \gamma_1=\gamma_2\frac{B\eta\left(\frac{(1-v)k}{(1-u)h}\right)^{\eta-1}}{A\alpha \left(\frac{vk}{uh}\right)^{\alpha-1}}\label{Verhaltnisherleitung2WM}
\end{equation}\\
Aus der Ableitung der Hamiltonian nach dem physischen Kapital gemä{\ss} Gleichung \eqref{eq:foc3}, folgt die Wachstumsrate des Schattenpreises von Gut 1.\newline
\begin{equation*}
\partial\mathbb{H}/\partial k\overset{!}{=}-\dot{\gamma}_1
\end{equation*}
\begin{equation}
\gamma_{1}A v^{\alpha}k^{\alpha -1} \alpha(u h)^{1- \alpha} + \gamma_{2}B(1- v)^{\eta} k^{\eta -1} \eta \left [ (1-u)h \right ]^{1- \eta}\overset{!}{=} - \dot{\gamma}_{1}\label{BedingungFoc3WM}
\end{equation}
\begin{equation*}
 A \alpha v^{\alpha}k^{\alpha -1} (uh)^{1- \alpha} + \frac{\gamma_{2}}{\gamma_{1}}B\eta (1- v)^{\eta} k^{\eta -1} \left [(1-u)h \right ]^{1- \eta}= - \hat{\gamma}_{1}
\end{equation*}
Dabei wird das Verhältnis beider Schattenpreise $\gamma_2/\gamma_1$ aus Gleichung \eqref{Verhaltnisherleitung1aWM} eingesetzt.
\begin{equation*}
 A \alpha v^{\alpha} {u}^{1- \alpha} \left(\frac{k}{h}\right)^{\alpha -1}+ \frac{A\alpha \left(\frac{vk}{uh}\right)^{\alpha-1}}{B\eta\left(\frac{(1-v)k}{(1-u)h}\right)^{\eta-1}}B\eta (1- v)^{\eta} k^{\eta -1} \left [(1-u)h \right ]^{1- \eta}= - \hat{\gamma}_{1}
\end{equation*}
\begin{equation*}
 A \alpha \left(\frac{vk}{uh}\right)^{\alpha -1}(v+ (1- v))= -\hat{\gamma_{1}}
\end{equation*}
\begin{equation*}
 A \alpha \left(\frac{vk}{uh}\right)^{\alpha -1} = - \hat{\gamma_{1}}
\end{equation*}
\begin{equation}
\hat{\gamma}_1=-A\alpha \left(\frac{vk}{uh}\right)^{\alpha-1}\Longleftrightarrow \quad \hat{\gamma}_1=-A\alpha x_1^{\alpha-1}\label{foc3}
\end{equation}
Die Keynes-Ramsey-Regel folgt aus der Kombination von Gleichung \eqref{BedingungFoc3WM} mit $\gamma_2$ laut \eqref{Verhaltnisherleitung2WM} und $\dot{\gamma}_1$ aus \eqref{eq:foc1c}.
\begin{equation}
\gamma_{1}A \alpha v^{\alpha} \left(\frac{k}{h}\right)^{\alpha -1} u^{1- \alpha}+\gamma_1\frac{A\alpha \left(\frac{vk}{uh}\right)^{\alpha-1}}{B\eta\left(\frac{(1-v)k}{(1-u)h}\right)^{\eta-1}}{B\eta (1- v)^{\eta} k^{\eta-1} \left [ h(1-u) \right ]^{1- \eta}}\overset{!}{=} \gamma_{1}(\rho+\sigma\hat{c})\\
\end{equation}
\begin{equation*}
A \alpha \left(\frac{vk}{uh}\right)^{\alpha -1}(v+(1-v))= \rho+\sigma\hat{c}
\end{equation*}
\begin{equation*}
A \alpha \left(\frac{vk}{uh}\right)^{\alpha -1}- \rho = \sigma \hat{c}
\end{equation*}
\begin{equation}
\boxed{
\hat{c}=\frac{1}{\sigma}\left(A\alpha \left(\frac{vk}{uh}\right)^{\alpha -1}-\rho\right)}\label{eq:KRR}
\end{equation}\\
Gleichung \eqref{eq:foc4} bedingt die optimale Aufteilung des Humankapitals zwischen dem Produktions- und Bildungssektor. 
\begin{equation*}
\partial\mathbb{H}/\partial u\overset{!}{=}0
\end{equation*}
\begin{equation}
\gamma_1A(1-\alpha)(vk)^{\alpha}h^{1-\alpha}u^{-\alpha}-\gamma_2B(1-\eta)[(1-v)k]^\eta (1-u)^{-\eta} h^{1-\eta}\overset{!}{=}0
\end{equation}
\vspace{-0.5cm}
\begin{equation}
\gamma_1A(1-\alpha)(vk)^{\alpha}h^{1-\alpha}u^{-\alpha}=\gamma_2B(1-\eta)[(1-v)k]^\eta (1-u)^{-\eta} h^{1-\eta}\label{foc4}
\end{equation}
\begin{equation}
\frac{\gamma_2}{\gamma_1}=\frac{A(1-\alpha)(vk)^{\alpha}h^{1-\alpha}u^{-\alpha}}{B(1-\eta)[(1-v)k]^\eta (1-u)^{-\eta} h^{1-\eta}}
\end{equation}
Zunächst erhält man wieder ein Verhältnis beider Schattenpreise. 
\begin{equation}
\quad~=\frac{A(1-\alpha)\left(\frac{vk}{uh}\right)^{\alpha}}{B(1-\eta)\left(\frac{(1-v)k}{(1-u)h}\right)^\eta}=\frac{A(1-\alpha)x_1^{\alpha}}{B(1-\eta)x_2^\eta}\label{Verhaltnisgleichung3WM}
\end{equation}
\begin{equation}
\gamma_1=\gamma_2\frac{B(1-\eta)\left(\frac{(1-v)k}{(1-u)h}\right)^\eta}{A(1-\alpha)\left(\frac{vk}{uh}\right)^{\alpha}}\Longleftrightarrow \gamma_2=\gamma_1 \frac {A(1-\alpha)\left(\frac{vk}{uh}\right)^{\alpha}}{B(1-\eta)\left(\frac{(1-v)k}{(1-u)h}\right)^\eta} = \gamma_1 \frac {A(1-\alpha)x_1^{\alpha}}{B(1-\eta)x_2^\eta}\label{Verhaltnisgleichung3bWM}
\end{equation}
Daraus lässt sich die Wachstumsrate des Schattenpreises von Gut 2 herleiten.
\begin{equation}
\hat{\gamma}_{2} = \hat{\gamma}_{1}+\alpha\hat{x}_1-\eta\hat{x}_2 \label{WachstumGamma2WM}
\end{equation}
Es werden nun die aus Bedingung \eqref{eq:foc2} und \eqref{eq:foc4} berechneten Verhältnisse der Schattenpreise \eqref{Verhaltnisherleitung1aWM} und \eqref{Verhaltnisgleichung3WM} miteinander gleichgesetzt und es ergibt sich: 
\begin{equation}
\frac{A\alpha x_1^{\alpha-1}}{B\eta x_2^{\eta-1}}=\frac{A(1-\alpha)x_1^{\alpha}}{B(1-\eta)x_2^\eta}
\end{equation}
\begin{equation}
\boxed{\frac{1-\alpha}{\alpha}x_1=\frac{1-\eta}{\eta}x_2}
\end{equation}
Aus der letzten Bedingung erster Ordnung gemä{\ss} Gleichung \eqref{eq:foc5} wurde folgende Gleichgewichtsbedingung hergeleitet.
\begin{equation*}
\partial\mathbb{H}/\partial h\overset{!}{=}-\dot{\gamma}_2
\end{equation*}
\begin{equation}
\gamma_1A(1-\alpha)(vk)^\alpha u^{1-\alpha}h^{-\alpha}+\gamma_2 B(1-\eta)[(1-v)k]^{\eta}(1-u)^{1-\eta}h^{-\eta}\overset{!}{=}-\dot{\gamma}_2
\end{equation}
Es wird $\gamma_1$ durch \eqref{Verhaltnisgleichung3bWM} ersetzt. 
\begin{equation*}
\gamma_2\frac{B(1-\eta)\left(\frac{(1-v)k}{(1-u)h}\right)^\eta}{A(1-\alpha)\left(\frac{vk}{uh}\right)^{\alpha}}A(1-\alpha)\left(\frac{vk}{uh}\right)^{\alpha}u+\gamma_2 B(1-\eta)\left(\frac{(1-v)k}{(1-u)h}\right)^\eta(1-u)=-\dot{\gamma}_2
\end{equation*}
\begin{equation*}
B(1-\eta)\left(\frac{(1-v)k}{(1-u)h}\right)^\eta[u+1-u]=-\hat{\gamma}_2
\end{equation*}
\begin{equation}
\hat{\gamma}_2=-B(1-\eta)\left(\frac{(1-v)k}{(1-u)h}\right)^\eta\Longleftrightarrow \hat{\gamma}_2=-B(1-\eta)x_2^\eta\label{WachstumGamma2bWM}
\end{equation}
Anschlie{\ss}end werden in Gleichung \eqref{WachstumGamma2bWM} die in Gleichung \eqref{WachstumGamma2WM} berechnete Wachstumsrate des Schattenpreises von Gut 2 eingesetzt und es ergibt sich 
\begin{equation}
\hat{\gamma}_{1}+\alpha\hat{x}_1-\eta\hat{x}_2 =-B
(1-\eta)x_2^\eta
\end{equation} 
In einem weiteren Schritt wird nun das Wachstum des Schattenpreises von Gut 1 aus Gleichung \eqref{eq:foc1d} substituiert. 
\begin{equation}
\boxed{B(1-\eta)x_2^\eta=\rho-\sigma\hat{c}-\alpha\hat{x}_1+\eta\hat{x}_2}
\end{equation}
Demzufolge ergibt sich folgendes Gleichungssystem, welches das Gleichgewicht beschreibt. 
\begin{align}
&\hat{k}=Ax_1^\alpha \frac{uh}{k}-\chi\label{GG1WM}\\
&\hat{h}=Bx_2^\eta(1-u)\label{GG2WM}\\
& x_1(1-\alpha)/\alpha =x_2(1-\eta)/\eta\label{GG3WM}\\
&\hat{c}=\frac{1}{\sigma}\left(A\alpha x_1^{\alpha -1}-\rho\right)\label{GG4WM}\\
&B(1-\eta)x_2^\eta=\rho-\sigma\hat{c}-\alpha\hat{x}_1+\eta\hat{x}_2\label{GG5WM}
\end{align}

Die Gleichgewichtsbedingung \eqref{GG3WM} zeigt, wie sich die Relationen $x_1$ und $x_2$ langfristig verhalten, indem die Wachstumsrate von $x_1$ gebildet wird. Es zeigt sich, dass beide mit der gleichen Rate wachsen, denn es gilt:
\begin{equation}
\hat{x}_1=\hat{x}_2=0
\end{equation}
Des weiteren gilt im Steady State $\hat{c}=\hat{k}=\hat{h}$.
Aus Bedingung \eqref{GG5WM} lässt sich $x_2^*$ herleiten.
\begin{equation}
B(1-\eta)x_2^\eta=\rho-\sigma\hat{c}
\end{equation}
\begin{equation*}
x_2^\eta=\frac{1}{B(1-\eta)}(\rho+\sigma)
\end{equation*}
\begin{equation}
x_2^*=\left(\frac{\rho+\sigma\hat{c}}{B(1-\eta)}\right)^{1/\eta}
\end{equation}
Aus der Bedingung \eqref{GG3WM} wird $x_1^*$ berechnet, indem $x_2^*$ eingesetzt wird.
\begin{equation}
\frac{1-\alpha}{\alpha}x_1 =\frac{1-\eta}{\eta}\left(\frac{\rho+\sigma\hat{c}}{B(1-\eta)}\right)^{1/\eta}
\end{equation}
\begin{equation}
x_1^* =\frac{\alpha(1-\eta)}{\eta(1-\alpha)}\left(\frac{\rho+\sigma\hat{c}}{B(1-\eta)}\right)^{1/\eta}
\end{equation}
Aus dem gleichgewichtigen Wachstumspfad gemä{\ss} \eqref{GG4WM} ergibt sich: 
\begin{equation}
\hat{c}=\frac{1}{\sigma}\left(A\alpha \left[\frac{\alpha(1-\eta)}{\eta(1-\alpha)}\left(\frac{\rho+\sigma\hat{c}}{B1-\eta)}\right)^{1/\eta}\right]^{\alpha-1}-\rho\right)
\end{equation}
\begin{equation*}
\hat{c}\sigma+\rho=A\alpha\left(\frac{\alpha(1-\eta)}{\eta(1-\alpha)}\right)^{\alpha-1}\left(\frac{\rho+\sigma\hat{c}}{B(1-\eta)}\right)^{\frac{\alpha-1}{\eta}}
\end{equation*}
\begin{equation*}
(\hat{c}\sigma+\rho)^{1-\frac{\alpha-1}{\eta}}=A\alpha\left(\frac{\alpha(1-\eta)}{\eta(1-\alpha)}\right)^{\alpha-1}\left(B(1-\eta)\right)^{\frac{1-\alpha}{\eta}}
\end{equation*}
\begin{equation*}
\hat{c}\sigma+\rho=\left[A^\eta\alpha^{\alpha\eta}\left(\frac{1-\eta}{\eta(1-\alpha)}\right)^{(\alpha-1)\eta}\left(B(1-\eta)\right)^{1-\alpha}\right]^\frac{1}{1+\eta-\alpha}
\end{equation*}
\begin{equation}
\boxed{\hat{c}^*=\frac{1}{\sigma}\left(\left[A^\eta\alpha^{\alpha\eta}(1-\eta)^{(1-\eta)(1-\alpha)}(\eta(1-\alpha))^{\eta(1-\alpha)}(B)^{1-\alpha}\right]^\frac{1}{1+\eta-\alpha}-\rho\right)}
\end{equation}
Der Übersichtlichkeit halber wird für die weitere Berechnung des Gleichgewichts der Platzhalter $M=\left[A^\eta\alpha^{\alpha\eta}(1-\eta)^{(1-\eta)(1-\alpha)}(\eta(1-\alpha))^{\eta(1-\alpha)}(B)^{1-\alpha}\right]^\frac{1}{1+\eta-\alpha}$ für das Grenzprodukt verwendet.
Die gleichgewichtige Aufteilung des Humankapitals $u^*$ berechnet sich aus $\hat{c}=\hat{h}$ gemä{\ss} Gleichung \eqref{GG2WM} unter Einbeziehung von $x_2^*$ und $\hat{c}^*$. 
\begin{equation}
\frac{1}{\sigma} (M-\rho)=B\left(\left(\frac{\rho+\sigma\frac{1}{\sigma}(M-\rho)}{B(1-\eta)}\right)^{1/\eta}\right)^\alpha(1-u)
\end{equation}
\begin{equation*}
\frac{\frac{1}{\sigma}(1-\eta)(M-\rho)}{M}=(1-u)
\end{equation*}
\begin{equation}
\boxed{u^*=\frac{\sigma M-(1-\eta)(M-\rho)}{\sigma M}}
\end{equation}
Im Steady State gilt: $\hat{c}=\hat{k}$. Aus dieser Bedingung lässt sich das optimale Verhältnis von physischem Kapital zu Humankapital ableiten, indem man die entsprechenden Terme für $x_1^*$, $x_2^*$ und $\hat{c}^*$  in Gleichung \eqref{GG1WM} einsetzt. 

\begin{equation*}
\frac{1}{\sigma} (M-\rho)=A\left(\frac{\alpha(1-\eta)}{\eta(1-\alpha)}\left(\frac{\rho+\sigma\frac{1}{\sigma} (M-\rho)}{B(1-\eta)}\right)^{1/\eta}\right)^{\alpha}\frac{h}{k}\frac{\sigma M-(1-\eta)(M-\rho)}{\sigma M}-\chi
\end{equation*}
\begin{equation}
\boxed{\chi^*=\frac{1}{\sigma}\left(\frac{A\alpha \sigma[-\eta\rho+M(\eta+\sigma-1)+\rho] \left(\frac{\alpha  (\eta -1) \left(\frac{M}{B (1-\eta) }\right)^{1/\eta }}{(\alpha -1) \eta }\right)^{\alpha -1}}{\rho  (\alpha -\eta )+M (\alpha  (\sigma -1)+\eta )}-M+\rho\right)}
\end{equation}
Die allgemeine Gleichung $v=\frac{vk}{uh}\frac{uh}{k}$ wird gelöst, indem man auch hier wieder $x_1^*$ und $x_2^*$ einsetzt, sowie unter zu Hilfenahme von $\frac{k}{h}=u x_1+(1-u)x_2$.\footnote{Diese Form leitet sich aus der allgemeinen Aufteilung des physischen Kapitals zwischen dem Bildungs- und Produktionssektors $k=vk+(1-v)k$ ab. Die gesamte Gleichung wurde durch $h$ geteilt und anschlie{\ss}end um die Faktoren $u$ und $(1-u)$ erweitert. Daraus folgt $\frac{k}{h}=u\frac{vk}{uh}+(1-u)\frac{(1-v)k}{(1-u)h}$}
\begin{equation}
\begin{split}
v=\frac{\alpha(1-\eta)}{\eta(1-\alpha)} \left(\frac{\rho+\frac{1}{\sigma} \sigma(M-\rho)}{B (1-\eta )}\right)^{\frac{1}{\eta}} \frac{M \sigma -(1-\eta ) (M-\rho )}{ \sigma M}\\
\frac{1}{\frac{ \sigma M -(1-\eta ) (M-\rho )}{\sigma M }\frac{\alpha(1-\eta)}{\eta(1-\alpha)}\left(\frac{\rho+\frac{1}{\sigma}\sigma(M-\rho )}{B (1-\eta )}\right)^{1/\eta }+\left(1-\frac{ \sigma M-(1-\eta ) (M-\rho )}{ \sigma M}\right) \left(\frac{\rho+\frac{1}{\sigma}\sigma(M-\rho )}{B (1-\eta )}\right)^{1/\eta }}
\end{split}
\end{equation}
Daraus ergibt sich schlie{\ss}lich die optimale Aufteilung $v^*$ des physischen Kapitals auf die Sektoren:
\begin{equation}
\boxed{
v^*=\frac{\alpha  (1-\eta ) \left(\frac{M}{B C (1-\eta )}\right)^{1/\eta } (M \sigma -(1-\eta ) (M-\rho ))}{(1-\alpha ) \eta  M \sigma  \left(\frac{\alpha  (1-\eta ) \left(\frac{M}{B C (1-\eta )}\right)^{1/\eta } (M \sigma -(1-\eta ) (M-\rho ))}{(1-\alpha ) \eta  M \sigma }+\left(\frac{M}{B C (1-\eta )}\right)^{1/\eta } \left(1-\frac{M \sigma -(1-\eta ) (M-\rho )}{M \sigma }\right)\right)}}
\end{equation}



\section[Offenes relativ weniger weit entwickeltes Land]{Offenes relativ weniger weit entwickeltes Land \sectionmark{Offene weniger weit entwickelte VW}}\label{APPENDIXEL}
\sectionmark{Offene weniger weit entwickelte VW}
Die Bewegungsgleichungen für physisches Kapital und Humankapital lauten:
\begin{equation}
\dot{k}(t)=Ak(t)^\alpha(u(t)h(t))^{1-\alpha}-c(t)-c_{ex}(t)+p^*c_{im}(t)
\end{equation}
\vspace{-0.7cm}
\begin{equation}
\dot{h}(t)=B\bar{B}(1-u(t))h(t)
\end{equation}
Vernachlässigt man von nun an die Abhängigkeit der Variablen gegenüber der Zeit $t$, dann lautet die Wachstumsrate des physischen Kapitals:
\begin{equation*}
\hat{k}=A k^{\alpha-1}(uh)^{1-\alpha}-\frac{c}{k}-\frac{c_{ex}}{k}+p^*\frac{c_{im}}{k}
\end{equation*}
mit $\chi=\frac{c}{k}$, $\chi_{ex}=\frac{c_{ex}}{k}$ sowie $\chi_{im}=\frac{c_{im}}{k}$ ergibt sich
\begin{equation}
\boxed{\hat{k}=A u^{1-\alpha}\left(\frac{k}{h}\right)^{\alpha-1}-\chi-\chi_{ex}+p^*\chi_{im}}
\end{equation}
Die Wachstumsrate des Humankapitals ist: 
\begin{equation}
\hat{h}=B\bar{B}(1-u)
\end{equation}
%Daraus lässt sich auch die Wachstumsrate des Verhältnisses beider herleiten.
%\begin{equation}
%\hat{\left(\frac{k}{h}\right)}=\hat{k}-\hat{h}
%\end{equation}
%\begin{equation}
%\hat{\left(\frac{k}{h}\right)}= Av^\alpha u^{1-\alpha}\left(\frac{k}{h}\right)a^{\alpha-1}-\chi-\chi_{ex}+p^*\chi_{im}-B\bar{B}\left[(1-v)\left(\frac{k}{h}\right)\right]^{\eta}(1-u)^{1-\eta}
%\end{equation}
%Im Au{\ss}enhandelsgleichgewicht entspricht sich der Wert der Exportegütermenge mit dem der Importgütermenge, demnach gilt $\chi_{ex}=p^*\chi_{im}$.
%Die Kurzform lautet: 
%\begin{equation}
%\hat{\left(\frac{k}{h}\right)}=Ax_1^{\alpha-1}-\chi-B\bar{B}x_2^{\eta}\label{WachstumOmega}
%\end{equation}
Der Haushalt maximiert seinen Lebenszeitnutzen unter Anwendung der Hamiltonfunktion.\\
\begin{equation}
\begin{split}\mathbb{H}=&~e^{-\rho t}\frac{(c^\beta c_{im}^{1-\beta})^{1-\sigma}}{1-\sigma}\\
&+\gamma_1(Ak^\alpha(uh)^{1-\alpha}-c-c_{ex}+p^*c_{im})\\
&+\gamma_2B\bar{B}(1-u)h\end{split}
\end{equation}
Die Bedingungen erster Ordnung lauten:\\
\begin{align}
&\frac{\partial\mathbb{H}}{\partial c}\overset{!}{=}~0\label{eq:lfoc1EL}\\
&\frac{\partial\mathbb{H}}{\partial c_{im}}\overset{!}{=}~0\label{eq:lfoc1imEL}\\
&\frac{\partial\mathbb{H}}{\partial k}\overset{!}{=}-\dot{\gamma_1}\label{eq:lfoc3EL}\\
&\frac{\partial\mathbb{H}}{\partial k}\overset{!}{=}-\dot{\gamma}_{1im}\label{eq:lfoc3imEL}\\
&\frac{\partial\mathbb{H}}{\partial u}\overset{!}{=}~0\label{eq:lfoc4EL}\\
&\frac{\partial\mathbb{H}}{\partial h}\overset{!}{=}-\dot{\gamma_2}\label{eq:lfoc5EL}\end{align}
Aus Gleichung \eqref{eq:lfoc1EL} ergibt sich:\\
\begin{equation*}
\partial\mathbb{H}/\partial c\overset{!}{=}~0
\end{equation*}
\begin{equation}
e^{-\rho t}\beta c^{\beta-1}c_{im}^{1-\beta}(c^\beta c_{im}^{1-\beta})^{-\sigma}-\gamma_1\overset{!}{=}0\label{eq:lfoc1aEL}
\end{equation}
\vspace{-0.7cm}
\begin{equation}
\gamma_1=e^{-\rho t}\beta c^{\beta-1}c_{im}^{1-\beta}(c^\beta c_{im}^{1-\beta})^{-\sigma}\label{eq:lfoc1bEL}
\end{equation}
Die Ableitung nach der Zeit $t$ von Gleichung \eqref{eq:lfoc1bEL} ist:
\begin{equation}\frac{\partial\gamma_1}{\partial t} = \dot{\gamma}_{1}\end{equation}
\begin{equation*}
\begin{split}
		\dot{\gamma}_{1} = &-e^{-\rho t} \beta \rho c^{\beta -1} c_{im}^{1- \beta} (c^{\beta} c_{im}^{1- \beta})^{- \sigma} + e^{-\rho t} \beta (\beta -1 )c^{\beta -2}\dot{c}c_{im}^{1- \beta}(c^{\beta} c_{im}^{1- \beta})^{- \sigma}\\
		& + e^{- \rho t} \beta c^{\beta -1} (c^{\beta} c_{im}^{1- \beta})^{- \sigma}(1- \beta) c_{im}^{1- \beta -1} \dot{c}_{im}\\
		&- e^{- \rho t} \beta \sigma c^{\beta -1} c_{im}^{1- \beta} (c^{\beta} c_{im}^{1- \beta})^{-1- \sigma} (c_{im}^{1- \beta} \beta c^{\beta -1} \dot{c} + c^{\beta} c_{im}^{1- \beta -1}(1- \beta) \dot{c}_{im})
\end{split}
\end{equation*}

\begin{equation*}
\begin{split}
	~\quad = ~& e^{- \rho t} \beta c^{\beta -1} c_{im}^{1- \beta} (c^{\beta} c_{im}^{1- \beta})^{- \sigma} \\
		&\left [ - \rho + (\beta -1)\hat{c}+(1-\beta)\hat{c}_{im} - \sigma (c^{\beta} c_{im}^{1- \beta})^{-1} (c_{im}^{1- \beta} c^{\beta} (\beta \hat{c} + (1- \beta) \hat{c}_{im})) \right]\\
\end{split}
\end{equation*}


\begin{equation*}
		~\quad = - \rho + (\beta - 1) \hat{c} + (1- \beta) \hat{c}_{im} - \sigma \frac{c^{ \beta} c_{im}^{1- \beta}} {c^{\beta} c_{im}^{1- \beta}} (\beta \hat{c} + (1- \beta) \hat{c}_{im})\\
\end{equation*}
\begin{equation*}
\begin{split}
			\quad~ = &~ e^{- \rho t} \beta c^{\beta -1} c_{im}^{1- \beta} (c^{\beta} c_{im}^{1- \beta} )^{- \sigma}\\
		&\left [ - \rho + (\beta-1) \hat{c} + (1- \beta) \hat{c}_{im} - \sigma (\beta \hat{c} + (1- \beta) \hat{c}_{im}) \right ]\\
\end{split}
\end{equation*}

\begin{equation}
\dot{\gamma_1}=e^{-\rho t}\beta c^{\beta-1}c_{im}^{1-\beta}(c^\beta c_{im}^{1-\beta})^{-\sigma}[-\rho+\hat{c}(\beta-1-\sigma\beta)+\hat{c}_{im}(1-\beta+\sigma\beta-\sigma)]\label{eq:lfoc1cEL}
\end{equation}
vereinfacht und umformuliert ist dies äquivalent mit
\begin{equation*}
\dot{\gamma_1}=\gamma_1[-\rho+\hat{c}(\beta-1-\sigma\beta)+\hat{c}_{im}(1-\beta+\sigma\beta-\sigma)]\label{eq:foc1cEL}
\end{equation*}
und es ergibt sich die Wachstumsrate des Schattenpreises für Gut 1 gemä{\ss} 
\begin{equation}
\hat{\gamma_1}=-\rho+\hat{c}(\beta-1-\sigma\beta)+\hat{c}_{im}(1-\beta+\sigma\beta-\sigma)\label{eq:foc1dEL}
\end{equation}
Der Konsum ausländisch produzierter und somit importierter Güter bedingt die Nutzenmaximierung und folgt aus Gleichung \eqref{eq:lfoc1imEL} \\
\begin{equation*}
\partial\mathbb{H}/\partial c_{im}\overset{!}{=}~0
\end{equation*}
\begin{equation}
e^{-\rho t}(1-\beta)c^\beta c_{im}^{-\beta}(c^\beta c_{im}^{1-\beta})^{-\sigma}+p^*\gamma_{1}\overset{!}{=}0\label{eq:lfoc1aim}
\end{equation}
\vspace{-0.7cm}
\begin{equation}
p^*\gamma_{1}\hat{=} \gamma_{1im}=-e^{-\rho t}(1-\beta) c^{\beta}c_{im}^{-\beta}(c^\beta c_{im}^{1-\beta})^{-\sigma}\label{eq:lfoc1bim}\end{equation}
Der Schattenpreis des importierten Gutes ergibt sich aus dem bekannten Schattenpreis $\gamma_1$, der den zukünftigen Grenznutzen einer zusätzlichen Sachkapitaleinheit angibt bewertet mit dem Weltmarktpreis des Konsumgutes. Diese Gleichung \eqref{eq:lfoc1bim} wird ebenfalls nach der Zeit abgeleitet.
\begin{equation}\partial\gamma_{1im}/\partial t=\dot{\gamma}_{1im}\end{equation}
\begin{equation*}
\begin{split}
\dot{\gamma}_{1im} = &[- e^{- \rho t} (1- \beta) \rho c^{ \beta} ( c^{ \beta} c_{im}^{1- \beta})^{- \sigma} c_{im}^{- \beta} + e^{ - \rho t} (1- \beta)(c^{ \beta} c_{im}^{1- \beta})^{- \sigma} c_{im}^{-\beta} \beta c^{ \beta - 1} \dot{c}\\
& - e^{- \rho t} (1 - \beta) \sigma c^{ \beta} (c^{ \beta} c_{im}^{1- \beta})^{- \sigma - 1} c_{im}^{- \beta}(c_{im}^{1- \beta} \beta c^{ \beta -1} \dot{c} + c^{ \beta} c_{im}^{1- \beta -1} (1- \beta) \dot{c}_{im}) \\
& + e^{- \rho t} (1- \beta) c^{ \beta} ( c^{ \beta} c_{im}^{1- \beta})^{- \sigma} c_{im}^{- \beta -1}(- \beta) \dot{c}_{im} ] (-\frac{p}{p^{^*}})
\end{split}
\end{equation*}
\begin{equation*}
	\begin{split}
	\qquad=&~-\frac{p}{p^{^*}} e^{- \rho t} (1- \beta) c^{\beta}(c^{\beta} c_{im}^{\beta -1})^{- \sigma} c_{im}^{- \beta}\\
	&~\left[-\rho + \beta \hat{c} - \sigma(c^{\beta} c_{im}^{1- \beta})^{-1}(c_{im}^{1- \beta} c^{\beta}(\beta \hat{c} + (1- \beta) \hat{c}_{im}) - \beta \hat{c}_{im})\right]
\end{split}
\end{equation*}
\begin{equation}
\dot{\gamma}_{1im}=-\frac{1}{p^*}e^{-\rho t}(1-\beta) c^{\beta}c_{im}^{-\beta}(c^\beta c_{im}^{1-\beta})^{-\sigma}[-\rho+\hat{c}(\beta-\sigma\beta)-\hat{c}_{im}(\beta-\sigma\beta+\sigma)]\label{eq:lfoc1cim}
\end{equation}
\begin{equation}
\dot{\gamma}_{1im}=\gamma_{1im}[-\rho+\hat{c}(\beta-\sigma\beta)-\hat{c}_{im}(\beta-\sigma\beta+\sigma)]\label{eq:foc1cimEL}
\end{equation}
Wird die Hamiltonian nach dem physischen Kapital gemä{\ss} Gleichung \eqref{eq:lfoc3EL} abgeleitet, ergibt sich die Wachstumsrate des Schattenpreises von Gut 1.
\begin{equation*}
\partial\mathbb{H}/\partial k\overset{!}{=}-\dot{\gamma_1}
\end{equation*}
\begin{equation}
\gamma_{1}A k^{\alpha -1} \alpha(u h)^{1- \alpha}\overset{!}{=} - \dot{\gamma}_{1}\label{BedingungFoc3EL}
\end{equation}
durch $\gamma_1$ teilen und $\hat{\gamma_1}$ einsetzen aus \eqref{eq:foc1dEL}
\begin{equation*}
 A \alpha k^{\alpha -1} (uh)^{1- \alpha} =\rho-\hat{c}(\beta-1-\sigma\beta)-\hat{c}_{im}(1-\beta+\sigma\beta-\sigma)
\end{equation*}
\begin{equation*}
 A \alpha k^{\alpha -1} (uh)^{1- \alpha}-\rho+\hat{c}_{im}(1-\beta+\sigma\beta-\sigma)=-\hat{c}(\beta-1-\sigma\beta)
\end{equation*}
\begin{equation}
\boxed{
\hat{c}=\frac{1}{(1-\beta+\sigma\beta)}\left(A\alpha \left(\frac{k}{h}\right)^{\alpha -1}u^{1-\alpha}-\rho+\hat{c}_{im}(1-\beta+\sigma\beta-\sigma)\right)}\label{eq:lKRREL}
\end{equation}

Sie beschreibt das inländische Konsumwachstum der Volkswirtschaft. Für dieses Handelsmodell ist aber nicht nur der optimale Wachstumspfad der heimisch produzierten Güter interessant, sondern auch der der importierten und somit im Ausland produzierten Güter. Für die Herleitung der Konsumwachstumsrate importierter Güter wird zunächst die Hamiltonian nach dem physischen Kapital abgeleitet und gleich der Bewegungsgleichung des Schattenpreises importierter Güter gesetzt, gemä{\ss} Gleichung \eqref{eq:lfoc3imEL}.\\
\begin{equation*}
\partial\mathbb{H}/\partial k\overset{!}{=}-\dot{\gamma_{1im}}
\end{equation*}
\begin{equation}
\gamma_{1 im}A(hu)^{1- \alpha}\alpha k^{\alpha -1}\overset{!}{=} - \dot{\gamma}_{1im}\label{eq:impEL}
\end{equation}
Es wird $\dot{\gamma}_{1im}$  aus Gleichungen \eqref{eq:foc1cimEL} in Gleichung \eqref{eq:impEL} ersetzt. 
\begin{equation*}
\gamma_{1im}A(hu)^{1- \alpha}\alpha k^{\alpha -1} = - \dot{\gamma}_{1m}
\end{equation*}
\begin{equation*}
\gamma_{1im} A \alpha \left(\frac{k}{uh}\right)^{\alpha -1} = - \gamma_{1im} \left[- \rho + \hat{c} (\beta - \sigma\beta) - \hat{c}_{im} (\beta-\sigma\beta+ \sigma) \right]
\end{equation*}
\begin{equation*}
A \alpha \left(\frac{k}{uh}\right)^{\alpha -1} - \rho + \hat{c} (\beta - \sigma\beta) = \hat{c}_{im}(\beta-\sigma\beta+ \sigma)
\end{equation*}
\begin{equation}
\boxed{\hat{c}_{im}=\frac{1}{\beta-\sigma\beta+ \sigma}\left(A\alpha \left(\frac{k}{uh}\right)^{\alpha -1}-\rho+\hat{c}(\beta - \sigma\beta)\right)}\label{eq:lKRRim}
\end{equation}
Aus der letzten Bedingung laut Gleichung \eqref{eq:lfoc4EL} folgt die optimale Aufteilung des Humankapitals zwischen dem Produktions- und Bildungssektor. 
\begin{equation}
\partial\mathbb{H}/\partial u\overset{!}{=}~0
\label{eq:}
\end{equation}
\begin{equation}
\gamma_{1}Ak^{\alpha}h^{1- \alpha}u^{- \alpha} (1- \alpha) - \gamma_{2} B \bar{B} h\overset{!}{=}0
\end{equation}
\vspace{-0.7cm}
\begin{equation}
\gamma_{1}Ak^{\alpha}h^{1- \alpha}u^{- \alpha} (1- \alpha) = \gamma_{2} B \bar{B} h 
\label{foc4}
\end{equation}
Dieser Term wird nach $\gamma_2$ umgestellt, um anschlie{\ss}end die Wachstumsrate des Schattenpreises in Sektor zwei bestimmen zu können.
\begin{equation}
\gamma_{2} = \gamma_{1} \frac{Ak^{\alpha}h^{1- \alpha} u^{- \alpha} (1- \alpha)} {B \bar{B} h} 
\end{equation}
\vspace{-0.5cm}
\begin{equation}
\hat{\gamma_{2}} = \hat{\gamma_{1}} + \alpha \hat{k} - \alpha \hat{h} - \alpha \hat{u}\label{WachstumGamma2EL}
\end{equation}
Das Verhältnis beider Schattenpreise lautet:
\begin{equation}
\frac{\gamma_{1}}{\gamma_{2}}= \frac{B \bar{B} h}{A k^{\alpha} h^{1- \alpha} u^{- \alpha} (1- \alpha)} = \frac{B \bar{B} h^{\alpha} u^{\alpha}} {A k^{\alpha} (1- \alpha)}
\label{Verhaltnisgleichung3b}
\end{equation}
Aus Gleichung \eqref{eq:lfoc5EL} wird folgende Gleichgewichtsbedingung hergeleitet.
\begin{equation}
\partial\mathbb{H}/\partial h\overset{!}{=}-\dot{\gamma_2}
\end{equation}
\begin{equation}
\gamma_{1} A k^{\alpha} u^{1- \alpha} (1- \alpha) h^{- \alpha} + \gamma_{2} B \bar{B} (1- u) \overset{!}{=} - \dot{\gamma}_{2}
\end{equation}
\begin{equation}
\frac{\gamma_{1}}{\gamma_{2}} A k^{\alpha} u^{1- \alpha} (1- \alpha) h^{- \alpha} + B \bar{B} (1- u) = - \hat{\gamma}_{2}
\end{equation}
Es wird das Verhältnis $\frac{\gamma_1}{\gamma_2}$ aus \eqref{Verhaltnisgleichung3b} ersetzt, sowie $\hat{\gamma}_2$ aus Gleichung \eqref{WachstumGamma2EL}.
\begin{equation*}
\frac{B \bar{B} h}{A k^{\alpha} h^{1 - \alpha}u^{- \alpha} (1- \alpha) } A k^{\alpha} u^{1- \alpha} (1- \alpha) h^{- \alpha} + B \bar{B} (1- u) = - \hat{\gamma}_{1} - \alpha \hat{k} + \alpha \hat{h} + \alpha \hat{u}
\end{equation*}
\begin{equation}
B \bar{B} u + B \bar{B} (1- u) = A \alpha k^{\alpha -1} (uh)^{1- \alpha} + \alpha \hat{u} - \alpha (\hat{k} - \hat{h}) 
\end{equation}
\begin{equation}
B \bar{B} u + B \bar{B} (1- u) = A \alpha k^{\alpha -1} (uh)^{1- \alpha} + \alpha \hat{u} - \alpha (Ak^{\alpha -1} (uh)^{1-\alpha} - \chi - (1-u) B \bar{B}) 
\end{equation}
\begin{equation}
B \bar{B} (u +1 - u)= \alpha (1-u)B \bar{B} + \alpha \hat{u}+ \alpha \chi -\alpha (1-u)B \bar{B}
\end{equation}
\begin{equation}
\alpha \hat{u}=B \bar{B} -\alpha (1-u)B \bar{B} - \alpha \chi  
\end{equation}
\begin{equation}
\hat{u}=\frac{1}{\alpha}B \bar{B} - (1-u)B \bar{B} - \chi 
\end{equation}
\begin{equation}
\hat{u}  =  B \bar{B} (\frac{1}{\alpha}- 1+u)- \chi 
\end{equation}
\begin{equation}
\boxed{\hat{u} = B \bar{B} \left(\frac{1- \alpha}{\alpha}\right) + B \bar{B} u - \chi }
\end{equation}


Befindet sich die Volkswirtschaft im Gleichgewicht, dann entspricht dies in einem Handelsmodell dem Au{\ss}enhandelsgewicht. Es kann davon ausgegangen werden, dass der optimale Konsumpfad der importieren Güter dem der heimisch produzierten Güter entspricht. Demzufolge gilt, dass $\hat{c}=\hat{c}_{im}$ ist. Nachfolgend werden die beiden Gleichungen  \eqref{eq:lKRREL} und \eqref{eq:lKRRim} gleichgesetzt und es ergibt sich ein gleichgewichtiger optimaler Konsumpfad einer offenen Volkswirtschaft. 
\begin{equation}
\hat{c}=\hat{c}_{im}
\end{equation}



\begin{equation}
\begin{split}
\frac{1}{(1-\beta+\sigma\beta)}\left[A\alpha \left(\frac{k}{h}\right)^{\alpha-1}u^{1- \alpha}-\rho+\hat{c}_{im}(1-\beta+\sigma\beta-\sigma)\right]=\\
\frac{1}{\sigma(1-\beta)+\beta}\left[A\alpha \left(\frac{k}{h}\right)^{\alpha-1}u^{1- \alpha}-\rho+\hat{c}(\beta-\sigma\beta)\right]
\end{split}
\end{equation}
erneutes einsetzen von $\hat{c}_{im}$ und auflösen nach $\hat{c}$ führt zu: 
\begin{equation*}
\begin{split}
\frac{1}{(1-\beta+\sigma\beta)}\left[A\alpha \left(\frac{k}{h}\right)^{\alpha-1}u^{1- \alpha}-\rho+\frac{1-\beta+\sigma\beta-\sigma}{\sigma-\sigma\beta+\beta}\left[A\alpha \left(\frac{k}{h}\right)^{\alpha-1}u^{1- \alpha}-\rho+\hat{c}(\beta-\sigma\beta)\right]\right]=\\
\frac{1}{\sigma-\sigma\beta+\beta}\left[A\alpha \left(\frac{k}{h}\right)^{\alpha-1}u^{1- \alpha}-\rho+\hat{c}(\beta-\sigma\beta)\right]
\end{split}
\end{equation*}

\begin{equation*}
\begin{split}
A\alpha \left(\frac{k}{h}\right)^{\alpha-1}u^{1- \alpha}-\rho+\frac{1-\beta+\sigma\beta-\sigma}{\sigma-\sigma\beta+\beta}\left[A\alpha \left(\frac{k}{h}\right)^{\alpha-1}u^{1- \alpha}-\rho+\hat{c}(\beta-\sigma\beta)\right]=\\
\frac{1-\beta+\sigma\beta}{\sigma-\sigma\beta+\beta}\left[A\alpha \left(\frac{k}{h}\right)^{\alpha-1}u^{1- \alpha}-\rho+\hat{c}(\beta-\sigma\beta)\right]
\end{split}
\end{equation*}

\begin{equation*}
A\alpha \left(\frac{k}{h}\right)^{\alpha-1}u^{1- \alpha}-\rho=\frac{\sigma}{\sigma-\sigma\beta+\beta}\left[A\alpha \left(\frac{k}{h}\right)^{\alpha-1}u^{1- \alpha}-\rho+\hat{c}(\beta-\sigma\beta)\right]
\end{equation*}

\begin{equation*}
A\alpha \left(\frac{k}{h}\right)^{\alpha-1}u^{1- \alpha}-\rho=\frac{\beta-1+\sigma}{\beta}\left(A\alpha \left(\frac{k}{h}\right)^{\alpha-1}u^{1- \alpha}-\rho\right)+\frac{\beta-1+\sigma}{\beta}\hat{c}(\beta-\sigma\beta)
\end{equation*}

\begin{equation*}
\frac{A\alpha \left(\frac{k}{h}\right)^{\alpha-1}u^{1- \alpha}-\rho}{\beta-1+\sigma}=\frac{A\alpha \left(\frac{k}{h}\right)^{\alpha-1}u^{1- \alpha}-\rho}{\beta}+\hat{c}(1-\sigma)
\end{equation*}
\begin{equation*}
\left(A\alpha \left(\frac{k}{h}\right)^{\alpha-1}u^{1- \alpha}-\rho\right)\left(\frac{1}{\beta-1+\sigma}-\frac{1}{\beta}\right)=\hat{c}(1-\sigma)
\end{equation*}
\begin{equation*}
\left(A\alpha \left(\frac{k}{h}\right)^{\alpha-1}u^{1- \alpha}-\rho\right)\frac{1-\sigma}{\sigma}=\hat{c}(1-\sigma)
\end{equation*}
\begin{equation}
\boxed{\hat{c}=\frac{1}{\sigma}\left(A\alpha \left(\frac{k}{h}\right)^{\alpha-1}u^{1- \alpha}-\rho\right)}\label{KRRGG}
\end{equation}

Im Gleichgewicht gilt $\hat{u}=0$.
\begin{equation}
\frac{1- \alpha}{\alpha} B \bar{B} + B \bar{B} u - \chi =0
\end{equation}
\vspace{-0.7cm}
\begin{equation}
\chi = \frac{1- \alpha}{\alpha} B \bar{B} + B \bar{B} u\label{ci1EL}
\end{equation}
Des weiteren gilt im Steady State $\hat{c}=\hat{k}=\hat{h}$.
Somit gilt:
\begin{equation}
\hat{k} = \hat{h}
\end{equation}
\begin{equation}
Ak^{\alpha -1} (uh)^{1- \alpha} - \chi = (1- u) B\bar{B}
\end{equation}
\vspace{-0.5cm}
\begin{equation}
Ak^{\alpha -1} (uh)^{1- \alpha} - \frac{1- \alpha}{\alpha}B \bar{B} - B \bar{B}u  = (1- u) B\bar{B}
\end{equation}
Verkürzt durch $z^*=Ak^{\alpha -1} (uh)^{1- \alpha}$ lässt sich dies darstellen als:
\begin{equation}
z + \frac{1}{\alpha}B \bar{B} + B\bar{B}- B \bar{B}u  =  B\bar{B} - B \bar{B} u
\end{equation}
\begin{equation}
\boxed{z = \frac{1}{\alpha} B \bar{B}}\label{zEL}
\end{equation}
Au{\ss}erdem entsprechen sich im Gleichgewicht auch: 
\begin{equation}
\hat{c}= \hat{k}
\end{equation}
\begin{equation}
\frac{1}{\sigma}(A\alpha k^{\alpha -1} (hu)^{1- \alpha}- \rho) = Ak^{\alpha -1} (uh)^{1- \alpha}- \chi
\end{equation}
Da sich der Term $A\alpha k^{\alpha -1} (hu)^{1- \alpha}$ auch als $\alpha z$ und für $Ak^{\alpha -1} (uh)^{1- \alpha}$ auch $z$ schreiben lässt ergibt sich zunächst:
\begin{equation}
\frac{1}{\sigma}(\alpha z- \rho) = z - \chi
\end{equation}
Hier wiederum kann $z$ aus Gleichung \eqref{zEL} eingesetzt werden. 
\begin{equation}
(B \bar{B} - \rho) \frac{1}{\sigma} = \frac{B \bar{B}}{\alpha} - \chi
\end{equation}
\begin{equation}
\boxed{\chi^* = \frac{B \bar{B}}{\alpha}- \frac{B \bar{B}- \rho}{\sigma}\label{chi2EL}}
\end{equation}
Um die maximierende Aufteilung des physischen Kapitals $u$ auf die beiden Sektoren zu erhalten, werden Gleichung \eqref{ci1EL} und Gleichung \eqref{chi2EL} gleichgesetzt. 
\begin{equation}
B \bar{B} \frac{1- \alpha}{\alpha} +  B \bar{B} u = \frac{B \bar{B} \alpha}{\alpha} - \frac{B \bar{B} - \rho}{\sigma}
\end{equation}
\begin{equation*}
\frac{1- \alpha}{\alpha} + u = \frac{1}{\alpha} - \frac{1}{\sigma} + \frac{\rho}{ \sigma B \bar{B}}
\end{equation*}
\begin{equation*}
u= \frac{1}{\alpha} - \frac{1- \alpha}{\alpha} - \frac{1}{\sigma}\left(1- \frac{\rho}{B \bar{B}}\right)
\end{equation*}
\begin{equation}
\boxed{u^*= 1- \frac{1}{\sigma}\left(1-  \frac{\rho}{B \bar{B}}\right)}
\end{equation}
Es ergibt sich für das offenen Entwicklungsland der gleichgewichtige Wachstumspfad unter Berücksichtigung von \eqref{zEL}
\begin{equation}
\boxed{\hat{c}^*=\frac{1}{\sigma}\left(\frac{1}{\alpha} B\bar{B}-\rho\right)}
\end{equation}






\section[Offenes relativ weiter entwickeltes Land]{Offenes relativ weiter entwickeltes Land \sectionmark{Offene weiter entwickelte VW}}\label{APPENDIXIL}
\sectionmark{Offene weiter entwickelte VW}
Die Ableitungen nach der Zeit $t$ für physisches Kapital $k$ und Humankapital $h$ lauten:
\begin{equation}
\dot{k}(t)=A(v(t)k(t))^\alpha(u(t)h(t))^{1-\alpha}-c(t)-c_{ex}(t)+p^*c_{im}(t)
\end{equation}
\vspace{-0.7cm}
\begin{equation}
\dot{h}(t)=B\bar{B}((1-v(t))k(t))^{\eta}((1-u(t))h(t))^{1-\eta}
\end{equation}
Auch hier wird künftig wieder die Abhängigkeit der Variablen gegenüber der Zeit $t$ vernachlässigt. Die Wachstumsrate des physischen Kapitals lautet: 
\begin{equation*}
\hat{k}=Av^\alpha k^{\alpha-1}(uh)^{1-\alpha}-\frac{c}{k}-\frac{c_{ex}}{k}+p^*\frac{c_{im}}{k}
\end{equation*}
mit $\chi=\frac{c}{k}$, $\chi_{ex}=\frac{c_{ex}}{k}$ sowie $\chi_{im}=\frac{c_{im}}{k}$ ergibt sich
\begin{equation}
\hat{k}=Av^\alpha u^{1-\alpha}\left(\frac{k}{h}\right)^{\alpha-1}-\chi-\chi_{ex}+p^*\chi_{im}
\end{equation}
Durch die Substitution von $x_1=\frac{vk}{uh}$ lässt sich die Wachstumsrate in verkürzter Form darstellen. 
\begin{equation}
\boxed{\hat{k}=Ax_1^\alpha \frac{uh}{k}-\chi-\chi_{ex}+p^*\chi_{im}}
\end{equation}
Das Humankapital wächst im relativ weiter entwickeltem Land wie folgt: 
\begin{equation}
\hat{h}=B\bar{B}\left[(1-v)\frac{k}{h}\right]^{\eta}(1-u)^{1-\eta}
\end{equation}
Eine Substitution von $x_2=\frac{(1-v)k}{(1-u)h}$ führt zu: 
\begin{equation}
\boxed{\hat{h}=B\bar{B}x_2^\eta(1-u)}
\end{equation}
%Daraus lässt sich auch die Wachstumsrate des Verhältnisses beider herleiten.
%\begin{equation}
%\hat{\left(\frac{k}{h}\right)}=\hat{k}-\hat{h}
%\end{equation}
%\begin{equation}
%\hat{\left(\frac{k}{h}\right)}= Av^\alpha u^{1-\alpha}\left(\frac{k}{h}\right)a^{\alpha-1}-\chi-\chi_{ex}+p^*\chi_{im}-B\bar{B}\left[(1-v)\left(\frac{k}{h}\right)\right]^{\eta}(1-u)^{1-\eta}
%\end{equation}
%Im Au{\ss}enhandelsgleichgewicht entspricht sich der Wert der Exportegütermenge mit dem der Importgütermenge, demnach gilt $\chi_{ex}=p^*\chi_{im}$.
%Die Kurzform lautet: 
%\begin{equation}
%\hat{\left(\frac{k}{h}\right)}=Ax_1^{\alpha-1}-\chi-B\bar{B}x_2^{\eta}\label{WachstumOmega}
%\end{equation}
Durch die Hamiltonfunktion kann der Haushalt sein Maximierungsproblem lösen.\\
\begin{equation}
\begin{split}\mathbb{H}=&~e^{-\rho t}\frac{(c^\beta c_{im}^{1-\beta})^{1-\sigma}}{1-\sigma}\\
&+\gamma_1(A(vk)^\alpha(uh)^{1-\alpha}-c-c_{ex}+p^*c_{im})\\
&+\gamma_2B\bar{B}[(1-v)k]^{\eta}[(1-u)h]^{1-\eta}\end{split}
\end{equation}
Auch hier gelten die Bedingungen erster Ordnung:\\
\begin{align}
&\frac{\partial\mathbb{H}}{\partial c}\overset{!}{=}~0\label{eq:lfoc1}\\
&\frac{\partial\mathbb{H}}{\partial c_{im}}\overset{!}{=}~0\label{eq:lfoc1im}\\
&\frac{\partial\mathbb{H}}{\partial v}\overset{!}{=}~0\label{eq:lfoc2}\\
&\frac{\partial\mathbb{H}}{\partial k}\overset{!}{=}-\dot{\gamma_1}\label{eq:lfoc3}\\
&\frac{\partial\mathbb{H}}{\partial k}\overset{!}{=}-\dot{\gamma}_{1im}\label{eq:lfoc3im}\\
&\frac{\partial\mathbb{H}}{\partial u}\overset{!}{=}~0\label{eq:lfoc4}\\
&\frac{\partial\mathbb{H}}{\partial h}\overset{!}{=}-\dot{\gamma_2}\label{eq:lfoc5}\end{align}
Beginnend mit Gleichung \eqref{eq:lfoc1} ergibt sich:\\
\begin{equation*}
\partial\mathbb{H}/\partial c\overset{!}{=}~0
\end{equation*}
\begin{equation}
e^{-\rho t}\beta c^{\beta-1}c_{im}^{1-\beta}(c^\beta c_{im}^{1-\beta})^{-\sigma}-\gamma_1\overset{!}{=}0\label{eq:lfoc1a}
\end{equation}
\vspace{-0.7cm}
\begin{equation}
\gamma_1=e^{-\rho t}\beta c^{\beta-1}c_{im}^{1-\beta}(c^\beta c_{im}^{1-\beta})^{-\sigma}\label{eq:lfoc1b}
\end{equation}
Für die Berechnung der Keynes-Ramsey-Regel wird der Schattenpreise $\gamma_1^*$ nach der Zeit  aus Gleichung \eqref{eq:lfoc1b} abgeleitet.
\begin{equation}\frac{\partial\gamma_1}{\partial t} = \dot{\gamma}_{1}\end{equation}
\begin{equation*}
\begin{split}
		\dot{\gamma}_{1} = &-e^{-\rho t} \beta \rho c^{\beta -1} c_{im}^{1- \beta} (c^{\beta} c_{im}^{1- \beta})^{- \sigma} + e^{-\rho t} \beta (\beta -1 )c^{\beta -2}\dot{c}c_{im}^{1- \beta}(c^{\beta} c_{im}^{1- \beta})^{- \sigma}\\
		& + e^{- \rho t} \beta c^{\beta -1} (c^{\beta} c_{im}^{1- \beta})^{- \sigma}(1- \beta) c_{im}^{1- \beta -1} \dot{c}_{im}\\
		&- e^{- \rho t} \beta \sigma c^{\beta -1} c_{im}^{1- \beta} (c^{\beta} c_{im}^{1- \beta})^{-1- \sigma} (c_{im}^{1- \beta} \beta c^{\beta -1} \dot{c} + c^{\beta} c_{im}^{1- \beta -1}(1- \beta) \dot{c}_{im})
\end{split}
\end{equation*}

\begin{equation*}
\begin{split}
	~\quad = ~& e^{- \rho t} \beta c^{\beta -1} c_{im}^{1- \beta} (c^{\beta} c_{im}^{1- \beta})^{- \sigma} \\
		&\left [ - \rho + (\beta -1)\hat{c}+(1-\beta)\hat{c}_{im} - \sigma (c^{\beta} c_{im}^{1- \beta})^{-1} (c_{im}^{1- \beta} c^{\beta} (\beta \hat{c} + (1- \beta) \hat{c}_{im})) \right]\\
\end{split}
\end{equation*}


\begin{equation*}
		~\quad = - \rho + (\beta - 1) \hat{c} + (1- \beta) \hat{c}_{im} - \sigma \frac{c^{ \beta} c_{im}^{1- \beta}} {c^{\beta} c_{im}^{1- \beta}} (\beta \hat{c} + (1- \beta) \hat{c}_{im})\\
\end{equation*}
\begin{equation*}
\begin{split}
			\quad~ = &~ e^{- \rho t} \beta c^{\beta -1} c_{im}^{1- \beta} (c^{\beta} c_{im}^{1- \beta} )^{- \sigma}\\
		&\left [ - \rho + (\beta-1) \hat{c} + (1- \beta) \hat{c}_{im} - \sigma (\beta \hat{c} + (1- \beta) \hat{c}_{im}) \right ]\\
\end{split}
\end{equation*}

\begin{equation}
\dot{\gamma_1}=e^{-\rho t}\beta c^{\beta-1}c_{im}^{1-\beta}(c^\beta c_{im}^{1-\beta})^{-\sigma}[-\rho+\hat{c}(\beta-1-\sigma\beta)+\hat{c}_{im}(1-\beta+\sigma\beta-\sigma)]\label{eq:lfoc1c}
\end{equation}
umformuliert folgt:
\begin{equation*}
\dot{\gamma_1}=\gamma_1[-\rho+\hat{c}(\beta-1-\sigma\beta)+\hat{c}_{im}(1-\beta+\sigma\beta-\sigma)]\label{eq:foc1c}
\end{equation*}
Daraus resultiert die Wachstumsrate des Schattenpreises für Gut 1: 
\begin{equation}
\hat{\gamma_1}=-\rho+\hat{c}(\beta-1-\sigma\beta)+\hat{c}_{im}(1-\beta+\sigma\beta-\sigma)\label{eq:foc1dja}
\end{equation}
Der Konsum ausländisch produzierter und somit importierter Güter bedingt ebenfalls den Lebenszeitnutzen laut Gleichung \eqref{eq:lfoc1im}: \\
\begin{equation*}
\partial\mathbb{H}/\partial c_{im}\overset{!}{=}~0
\end{equation*}
\begin{equation}
e^{-\rho t}(1-\beta)c^\beta c_{im}^{-\beta}(c^\beta c_{im}^{1-\beta})^{-\sigma}+p^*\gamma_{1}\overset{!}{=}0\label{eq:lfoc1aim}
\end{equation}
\vspace{-0.7cm}
\begin{equation}
p^*\gamma_{1}\hat{=}\gamma_{1im}=-e^{-\rho t}(1-\beta) c^{\beta}c_{im}^{-\beta}(c^\beta c_{im}^{1-\beta})^{-\sigma}\label{eq:lfoc1bim}\end{equation}
Diese Gleichung \eqref{eq:lfoc1bim} wird wiederum nach der Zeit abgeleitet.
\begin{equation}\frac{\partial\gamma_{1im}}{\partial t}=\dot{\gamma}_{1im}\end{equation}
\begin{equation*}
\begin{split}
\dot{\gamma}_{1im} = &[- e^{- \rho t} (1- \beta) \rho c^{ \beta} ( c^{ \beta} c_{im}^{1- \beta})^{- \sigma} c_{im}^{- \beta} + e^{ - \rho t} (1- \beta)(c^{ \beta} c_{im}^{1- \beta})^{- \sigma} c_{im}^{-\beta} \beta c^{ \beta - 1} \dot{c}\\
& - e^{- \rho t} (1 - \beta) \sigma c^{ \beta} (c^{ \beta} c_{im}^{1- \beta})^{- \sigma - 1} c_{im}^{- \beta}(c_{im}^{1- \beta} \beta c^{ \beta -1} \dot{c} + c^{ \beta} c_{im}^{1- \beta -1} (1- \beta) \dot{c}_{im}) \\
& + e^{- \rho t} (1- \beta) c^{ \beta} ( c^{ \beta} c_{im}^{1- \beta})^{- \sigma} c_{im}^{- \beta -1}(- \beta) \dot{c}_{im} ] (-\frac{p}{p^{^*}})
\end{split}
\end{equation*}
\begin{equation*}
	\begin{split}
	\qquad=&~-\frac{p}{p^{^*}} e^{- \rho t} (1- \beta) c^{\beta}(c^{\beta} c_{im}^{\beta -1})^{- \sigma} c_{im}^{- \beta}\\
	&~\left[-\rho + \beta \hat{c} - \sigma(c^{\beta} c_{im}^{1- \beta})^{-1}(c_{im}^{1- \beta} c^{\beta}(\beta \hat{c} + (1- \beta) \hat{c}_{im}) - \beta \hat{c}_{im})\right]
\end{split}
\end{equation*}

\begin{equation}
\dot{\gamma}_{1im}=-\frac{1}{p^*}e^{-\rho t}(1-\beta) c^{\beta}c_{im}^{-\beta}(c^\beta c_{im}^{1-\beta})^{-\sigma}[-\rho+\hat{c}(\beta-\sigma\beta)-\hat{c}_{im}(\beta-\sigma\beta+\sigma)]\label{eq:lfoc1cim}
\end{equation}
\begin{equation}
\dot{\gamma}_{1im}=\gamma_{1im}[-\rho+\hat{c}(\beta-\sigma\beta)-\hat{c}_{im}(\beta-\sigma\beta+\sigma)]\label{eq:foc1cim}
\end{equation}
Die Bedingung laut Gleichung \eqref{eq:lfoc2} führt zu der resultierenden optimalen Aufteilung des physischen Kapitals der Wirtschaftssubjekte zwischen dem Produktions- und Bildungssektor $v^*$.
\begin{equation*}
\partial\mathbb{H}/\partial v\overset{!}{=}~0
\end{equation*}
\begin{equation}
\gamma_1A\alpha v^{\alpha-1}k^\alpha(uh)^{1-\alpha}-\gamma_2B\bar{B}\eta(1-v)^{\eta-1}k^\eta[(1-u)h]^{1-\eta}\overset{!}{=}~0
\end{equation}
\vspace{-0.7cm}
\begin{equation}
\gamma_1A\alpha v^{\alpha-1}k^\alpha(uh)^{1-\alpha}=\gamma_2B\bar{B}\eta(1-v)^{\eta-1}k^\eta[(1-u)h]^{1-\eta}
\end{equation}
Es lässt sich das Verhältnis der Schattenpreise beider Güter herleiten.
\begin{equation}
\frac{\gamma_2}{\gamma_1}=\frac{A\alpha v^{\alpha-1}k^\alpha(uh)^{1-\alpha}}{B\bar{B}\eta(1-v)^{\eta-1}k^\eta[(1-u)h]^{1-\eta}}\label{Verhaltnisherleitung1}
\end{equation}
\begin{equation}
\quad ~~=\frac{A\alpha \left(\frac{vk}{uh}\right)^{\alpha-1}}{B\bar{B}\eta\left(\frac{(1-v)k}{(1-u)h}\right)^{\eta-1}}=\frac{A\alpha x_1^{\alpha-1}}{B\bar{B}\eta x_2^{\eta-1}}\label{Verhaltnisherleitung1a}
\end{equation}
\begin{equation}
\gamma_2=\gamma_1\frac{A\alpha \left(\frac{vk}{uh}\right)^{\alpha-1}}{B\bar{B}\eta\left(\frac{(1-v)k}{(1-u)h}\right)^{\eta-1}}\Longleftrightarrow \gamma_1=\gamma_2\frac{B\bar{B}\eta\left(\frac{(1-v)k}{(1-u)h}\right)^{\eta-1}}{A\alpha \left(\frac{vk}{uh}\right)^{\alpha-1}}\label{Verhaltnisherleitung2}
\end{equation}

Die Ableitung der Hamiltonian nach dem physischen Kapital gemä{\ss} Gleichung \eqref{eq:lfoc3}, führt zu der Wachstumsrate des Schattenpreises von Gut 1.
\begin{equation*}
\partial\mathbb{H}/\partial k\overset{!}{=}-\dot{\gamma}_1
\end{equation*}
\vspace{-0.2cm}
\begin{equation}
\gamma_{1}A v^{\alpha}k^{\alpha -1} \alpha(u h)^{1- \alpha} + \gamma_{2}B\bar{B}(1- v)^{\eta} k^{\eta -1} \eta \left [ (1-u)h \right ]^{1- \eta}\overset{!}{=} - \dot{\gamma}_{1}\label{BedingungFoc3}
\end{equation}
\begin{equation*}
 A \alpha v^{\alpha}k^{\alpha -1} (uh)^{1- \alpha} + \frac{\gamma_{2}}{\gamma_{1}}B\bar{B}\eta (1- v)^{\eta} k^{\eta -1} \left [(1-u)h \right ]^{1- \eta}= - \hat{\gamma}_{1}
\end{equation*}
Es wird das Verhältnis beider Schattenpreise $\gamma_2/\gamma_1$ aus Gleichung \eqref{Verhaltnisherleitung1a} eingesetzt.
\begin{equation*}
 A \alpha v^{\alpha} {u}^{1- \alpha} \left(\frac{k}{h}\right)^{\alpha -1}+ \frac{A\alpha \left(\frac{vk}{uh}\right)^{\alpha-1}}{B\bar{B}\eta\left(\frac{(1-v)k}{(1-u)h}\right)^{\eta-1}}B\bar{B}\eta (1- v)^{\eta} k^{\eta -1} \left [(1-u)h \right ]^{1- \eta}= - \hat{\gamma}_{1}
\end{equation*}
\begin{equation*}
 A \alpha \left(\frac{vk}{uh}\right)^{\alpha -1}(v+ (1- v))= -\hat{\gamma_{1}}
\end{equation*}
\begin{equation*}
 A \alpha \left(\frac{vk}{uh}\right)^{\alpha -1} = - \hat{\gamma_{1}}
\end{equation*}
\begin{equation}
\hat{\gamma}_1=-A\alpha \left(\frac{vk}{uh}\right)^{\alpha-1}\Longleftrightarrow \hat{\gamma}_1=-A\alpha x_1^{\alpha-1}\label{foc3}
\end{equation}
Aus der Kombination von Gleichung \eqref{BedingungFoc3} mit $\gamma_2$ laut \eqref{Verhaltnisherleitung2} und $\dot{\gamma}_1$ aus \eqref{eq:lfoc1c} folgt die Keynes-Ramsey-Regel.
\begin{equation}
\begin{split}
&\gamma_{1}A \alpha v^{\alpha} \left(\frac{k}{h}\right)^{\alpha -1} u^{1- \alpha}+\gamma_1\frac{A\alpha \left(\frac{vk}{uh}\right)^{\alpha-1}}{B\bar{B}\eta\left(\frac{(1-v)k}{(1-u)h}\right)^{\eta-1}}{B\bar{B}\eta (1- v)^{\eta} k^{\eta-1} \left [ h(1-u) \right ]^{1- \eta}}\\
&\overset{!}{=} - \gamma_{1}\left [ - \rho + \hat{c} (\beta -1 - \sigma \beta)+\hat{c}_{im}(1- \beta + \sigma \beta - \sigma) \right ]
\end{split}
\end{equation}
\begin{equation*}
A \alpha \left(\frac{vk}{uh}\right)^{\alpha -1}(v+(1-v))= \rho - \hat{c} (\beta -1 - \sigma \beta)-\hat{c}_{im}(1- \beta + \sigma \beta - \sigma)
\end{equation*}
\begin{equation*}
A \alpha \left(\frac{vk}{uh}\right)^{\alpha -1}- \rho +\hat{c}_{im}(1- \beta + \sigma \beta - \sigma)=- \hat{c}  (\beta -1 - \sigma \beta) \overset{\wedge}{=} \hat{c}(1- \beta+ \sigma \beta)
\end{equation*}
\begin{equation}
\boxed{
\hat{c}=\frac{1}{(1-\beta+\sigma\beta)}\left(A\alpha \left(\frac{vk}{uh}\right)^{\alpha -1}-\rho+\hat{c}_{im}(1-\beta+\sigma\beta-\sigma)\right)}\label{eq:lKRR}
\end{equation}
\\
Das inländische Konsumwachstum der Volkswirtschaft hängt hier ebenfalls von dem optimalen Wachstumspfad importierten und somit im Ausland produzierten Gütern ab. Dafür wird zunächst die Konsumwachstumsrate importierter Güter hergeleitet, indem die Hamiltonian nach dem physischen Kapital abgeleitet und gleich der Bewegungsgleichung des Schattenpreises importierter Güter gemä{\ss} Gleichung \eqref{eq:lfoc3im} gesetzt wird.\\
\begin{equation*}
\partial\mathbb{H}/\partial k\overset{!}{=}-\dot{\gamma}_{1im}
\end{equation*}
\begin{equation}
\gamma_{1 im}A(hu)^{1- \alpha}\alpha v^{\alpha} k^{\alpha -1} + \gamma_{2}B\bar{B} [h(1-u)]^{1- \eta} \eta(1-v)^{\eta}k^{\eta -1} \overset{!}{=} - \dot{\gamma}_{1im}\label{eq:imp}
\end{equation}
Auch hier werden die folgenden Variablen, $\dot{\gamma}_{1im}$ und $\gamma_2$ durch die entsprechenden Gleichungen \eqref{eq:foc1cim} und \eqref{Verhaltnisherleitung2} in Gleichung \eqref{eq:imp} ersetzt. 
\begin{equation*}
\gamma_{1im}A(hu)^{1- \alpha}\alpha v^{\alpha} k^{\alpha -1} +\gamma_{1im} \frac{A\alpha \left(\frac{vk}{uh}\right)^{\alpha-1}}{B\bar{B}\eta\left(\frac{(1-v)k}{(1-u)h}\right)^{\eta-1}} B\bar{B} [h(1-u)]^{1- \eta} \eta(1-v)^{\eta}k^{\eta -1} = - \dot{\gamma}_{1m}
\end{equation*}
\begin{equation*}
\gamma_{1im} A \alpha \left(\frac{vk}{uh}\right)^{\alpha -1} (v+(1-v)) = - \gamma_{1im} \left [ - \rho + \hat{c} (\beta - \sigma\beta) - \hat{c}_{im} (\beta-\sigma\beta+ \sigma) \right ]
\end{equation*}
\begin{equation*}
A \alpha \left(\frac{vk}{uh}\right)^{\alpha -1} - \rho + \hat{c} (\beta - \sigma\beta) = \hat{c}_{im}(\beta-\sigma\beta+ \sigma)
\end{equation*}
\begin{equation}
\boxed{\hat{c}_{im}=\frac{1}{\beta-\sigma\beta+ \sigma}\left(A\alpha \left(\frac{vk}{uh}\right)^{\alpha -1}-\rho+\hat{c}(\beta - \sigma\beta)\right)}\label{eq:lKRRim}
\end{equation}
\newpage
Es folgt die optimale Aufteilung des Humankapitals zwischen dem Produktions- und Bildungssektor aus der Bedingung laut Gleichung \eqref{eq:lfoc4}.
\begin{equation*}
\partial\mathbb{H}/\partial u\overset{!}{=}0
\end{equation*}
\begin{equation}
\gamma_1A(1-\alpha)(vk)^{\alpha}h^{1-\alpha}u^{-\alpha}-\gamma_2B\bar{B}(1-\eta)[(1-v)k]^\eta (1-u)^{-\eta} h^{1-\eta}\overset{!}{=}0
\end{equation}
\vspace{-0.7cm}
\begin{equation}
\gamma_1A(1-\alpha)(vk)^{\alpha}h^{1-\alpha}u^{-\alpha}=\gamma_2B\bar{B}(1-\eta)[(1-v)k]^\eta (1-u)^{-\eta} h^{1-\eta}\label{foc4}
\end{equation}
\begin{equation}
\frac{\gamma_2}{\gamma_1}=\frac{A(1-\alpha)(vk)^{\alpha}h^{1-\alpha}u^{-\alpha}}{B\bar{B}(1-\eta)[(1-v)k]^\eta (1-u)^{-\eta} h^{1-\eta}}
\end{equation}
\\
Das Verhältnis beider Schattenpreise beträgt: 
\begin{equation}
\quad~=\frac{A(1-\alpha)\left(\frac{vk}{uh}\right)^{\alpha}}{B\bar{B}(1-\eta)\left(\frac{(1-v)k}{(1-u)h}\right)^\eta}=\frac{A(1-\alpha)x_1^{\alpha}}{B\bar{B}(1-\eta)x_2^\eta}\label{Verhaltnisgleichung3}
\end{equation}
\begin{equation}
\gamma_1=\gamma_2\frac{B\bar{B}(1-\eta)\left(\frac{(1-v)k}{(1-u)h}\right)^\eta}{A(1-\alpha)\left(\frac{vk}{uh}\right)^{\alpha}}\Longleftrightarrow \gamma_2=\gamma_1 \frac {A(1-\alpha)\left(\frac{vk}{uh}\right)^{\alpha}}{B\bar{B}(1-\eta)\left(\frac{(1-v)k}{(1-u)h}\right)^\eta} = \gamma_1 \frac {A(1-\alpha)x_1^{\alpha}}{B\bar{B}(1-\eta)x_2^\eta}\label{Verhaltnisgleichung3b}
\end{equation}
Daraus ergibt sich die Wachstumsrate des Schattenpreises von Gut 2.
\begin{equation}
\hat{\gamma}_{2} = \hat{\gamma}_{1}+\alpha\hat{x}_1-\eta\hat{x}_2 \label{WachstumGamma2}
\end{equation}
Anschlie{\ss}end werden die aus Bedingung \eqref{eq:lfoc2} und \eqref{eq:lfoc4} berechneten Verhältnisse der Schattenpreise \eqref{Verhaltnisherleitung1a} und \eqref{Verhaltnisgleichung3} gleichgesetzt: 
\begin{equation}
\frac{A\alpha x_1^{\alpha-1}}{B\bar{B}\eta x_2^{\eta-1}}=\frac{A(1-\alpha)x_1^{\alpha}}{B\bar{B}(1-\eta)x_2^\eta}
\end{equation}
\begin{equation}
\boxed{\frac{1-\alpha}{\alpha}x_1=\frac{1-\eta}{\eta}x_2}
\end{equation}
Die letzte Bedingung erster Ordnung gemä{\ss} Gleichung \eqref{eq:lfoc5} besagt:
\begin{equation*}
\partial\mathbb{H}/\partial h\overset{!}{=}-\dot{\gamma}_2
\end{equation*}
\begin{equation}
\gamma_1A(1-\alpha)(vk)^\alpha u^{1-\alpha}h^{-\alpha}+\gamma_2 B\bar{B}(1-\eta)[(1-v)k]^{\eta}(1-u)^{1-\eta}h^{-\eta}\overset{!}{=}-\dot{\gamma}_2
\end{equation}
Es wird zunächst $\gamma_1$ aus \eqref{Verhaltnisgleichung3b} ersetzt. 
\begin{equation*}
\gamma_2\frac{B\bar{B}(1-\eta)\left(\frac{(1-v)k}{(1-u)h}\right)^\eta}{A(1-\alpha)\left(\frac{vk}{uh}\right)^{\alpha}}A(1-\alpha)\left(\frac{vk}{uh}\right)^{\alpha}u+\gamma_2 B\bar{B}(1-\eta)\left(\frac{(1-v)k}{(1-u)h}\right)^\eta(1-u)=-\dot{\gamma}_2
\end{equation*}
\begin{equation*}
B\bar{B}(1-\eta)\left(\frac{(1-v)k}{(1-u)h}\right)^\eta[u+1-u]=-\hat{\gamma}_2
\end{equation*}
\begin{equation}
\hat{\gamma}_2=-B\bar{B}(1-\eta)\left(\frac{(1-v)k}{(1-u)h}\right)^\eta\Longleftrightarrow \hat{\gamma}_2=-B\bar{B}(1-\eta)x_2^\eta\label{WachstumGamma2b}
\end{equation}
Es wird in Gleichung \eqref{WachstumGamma2b} die in Gleichung \eqref{WachstumGamma2} berechnet Wachstumsrat des Schattenpreises von Gut 2 eingesetzt.
\begin{equation}
\hat{\gamma}_{1}+\alpha\hat{x}_1-\eta\hat{x}_2 =-B\bar{B}(1-\eta)x_2^\eta
\end{equation} 
Wird das Wachstum des Schattenpreises von Gut 1 aus Gleichung \eqref{eq:foc1dja} substituiert, folgt: 
\begin{equation}
\boxed{B\bar{B}(1-\eta)x_2^\eta=\rho-\hat{c}(\beta-1-\sigma\beta)-\hat{c}_{im}(1-\beta+\sigma\beta-\sigma)-\alpha\hat{x}_1+\eta\hat{x}_2}
\end{equation}
Es resultiert folgendes Gleichungssystem, welches das Gleichgewicht beschreibt. 
\begin{align}
&\hat{k}=Ax_1^\alpha \frac{uh}{k}-\chi-\chi_{ex}+p^*\chi_{im}\label{GG1}\\
&\hat{h}=B\bar{B}x_2^\eta(1-u)\label{GG2}\\
& x_1(1-\alpha)/\alpha =x_2(1-\eta)/\eta\label{GG3}\\
&\hat{c}=\frac{1}{(1-\beta+\sigma\beta)}\left(A\alpha x_1^{\alpha -1}-\rho+\hat{c}_{im}(1-\beta+\sigma\beta-\sigma)\right)\label{GG4}\\
&B\bar{B}(1-\eta)x_2^\eta=\rho-\hat{c}(\beta-1-\sigma\beta)-\hat{c}_{im}(1-\beta+\sigma\beta-\sigma)-\alpha\hat{x}_1+\eta\hat{x}_2\label{GG5}
\end{align}
Im Au{\ss}enhandelsgleichgewich entspricht der optimale Konsumpfad der importierten Güter  dem der heimisch produzierten Güter. So gilt $\hat{c}=\hat{c}_{im}$. Es werden die beiden Gleichungen  \eqref{eq:lKRR} und \eqref{eq:lKRRim} gleichgesetzt somit ergibt sich ein gleichgewichtiger optimaler Konsumpfad einer offenen Volkswirtschaft. 
\begin{equation*}
\hat{c}=\hat{c}_{im}
\end{equation*}
\begin{equation}
\frac{1}{(1-\beta+\sigma\beta)}[A\alpha x_1^{\alpha-1}-\rho+\hat{c}_{im}(1-\beta+\sigma\beta-\sigma)]=\frac{1}{\sigma(1-\beta)+\beta}[A\alpha x_1^{\alpha-1}-\rho+\hat{c}(\beta-\sigma\beta)]
\end{equation}
Wird $\hat{c}_{im}$ erneut eingesetzt und  nach $\hat{c}$ aufgelöst, dann folgt: 
\begin{equation*}
\begin{split}
&\frac{1}{(1-\beta+\sigma\beta)}\left[A\alpha x_1^{\alpha-1}-\rho+\frac{1-\beta+\sigma\beta-\sigma}{\sigma-\sigma\beta+\beta}[A\alpha x_1^{\alpha-1}-\rho+\hat{c}(\beta-\sigma\beta)]\right]\\
& =\frac{1}{\sigma-\sigma\beta+\beta}[A\alpha x_1^{\alpha-1}-\rho+\hat{c}(\beta-\sigma\beta)]
\end{split}
\end{equation*}
\begin{equation*}
A\alpha x_1^{\alpha-1}-\rho+\frac{1-\beta+\sigma\beta-\sigma}{\sigma-\sigma\beta+\beta}[A\alpha x_1^{\alpha-1}-\rho+\hat{c}(\beta-\sigma\beta)]=\frac{1-\beta+\sigma\beta}{\sigma-\sigma\beta+\beta}[A\alpha x_1^{\alpha-1}-\rho+\hat{c}(\beta-\sigma\beta)]
\end{equation*}
\begin{equation*}
A\alpha x_1^{\alpha-1}-\rho=\frac{\sigma}{\sigma-\sigma\beta+\beta}[A\alpha x_1^{\alpha-1}-\rho+\hat{c}(\beta-\sigma\beta)]
\end{equation*}
\begin{equation*}
A\alpha x_1^{\alpha-1}-\rho=\frac{\beta-1+\sigma}{\beta}(A\alpha x_1^{\alpha-1}-\rho)+\frac{\beta-1+\sigma}{\beta}\hat{c}(\beta-\sigma\beta)
\end{equation*}
\begin{equation*}
\frac{A\alpha x_1^{\alpha-1}-\rho}{\beta-1+\sigma}=\frac{A\alpha x_1^{\alpha-1}-\rho}{\beta}+\hat{c}(1-\sigma)
\end{equation*}
\begin{equation*}
(A\alpha x_1^{\alpha-1}-\rho)\left(\frac{1}{\beta-1+\sigma}-\frac{1}{\beta}\right)=\hat{c}(1-\sigma)
\end{equation*}
\begin{equation*}
(A\alpha x_1^{\alpha-1}-\rho)\frac{1-\sigma}{\sigma}=\hat{c}(1-\sigma)
\end{equation*}
\begin{equation}
\boxed{\hat{c}=\frac{1}{\sigma}(A\alpha x_1^{\alpha-1}-\rho)}\label{KRRGG}
\end{equation}
In Gleichung \eqref{GG3} wird gezeigt, wie sich die Relationen $x_1$ und $x_2$ langfristig verhalten. Es wird die Wachstumsrate von $x_1$ gebildet. Beide wachsen mit der gleichen Rate, so gilt:
\begin{equation}
\hat{x}_1=\hat{x}_2=0
\end{equation}
Zudem gilt im Steady State $\hat{c}=\hat{k}=\hat{h}$.
Aus Bedingung \eqref{GG5} wird $x_1^*$ berechnet mit $\hat{c}=\hat{c}_{im}$.
\begin{equation}
B(1+\bar{B})(1-\eta)x_2^\eta=\rho-\hat{c}(\beta-1-\sigma\beta)-\hat{c}(1-\beta+\sigma\beta-\sigma)
\end{equation}
\begin{equation*}
x_2^\eta=\frac{1}{B(1+\bar{B})(1-\eta)}\left[\rho+\hat{c}(-\beta+1+\sigma\beta+\beta-\sigma\beta+\sigma)\right]
\end{equation*}
\begin{equation}
x_2^*=\left(\frac{\rho+\sigma\hat{c}}{B(1+\bar{B})(1-\eta)}\right)^{1/\eta}
\end{equation}
Mit Hilfe von Gleichung \eqref{GG3} wird $x_1^*$ berechnet, indem $x_2^*$ eingesetzt wird.
\begin{equation}
\frac{1-\alpha}{\alpha}x_1 =\frac{1-\eta}{\eta}\left(\frac{\rho+\sigma\hat{c}}{B(1+\bar{B})(1-\eta)}\right)^{1/\eta}
\end{equation}
\begin{equation}
x_1^* =\frac{\alpha(1-\eta)}{\eta(1-\alpha)}\left(\frac{\rho+\sigma\hat{c}}{B(1+\bar{B})(1-\eta)}\right)^{1/\eta}
\end{equation}
Gleichung \eqref{KRRGG} gibt den gleichgewichtigen Wachstumspfad wieder und es ergibt sich: 
\begin{equation}
\hat{c}=\frac{1}{\sigma}\left(A\alpha \left[\frac{\alpha(1-\eta)}{\eta(1-\alpha)}\left(\frac{\rho+\sigma\hat{c}}{B(1+\bar{B})(1-\eta)}\right)^{1/\eta}\right]^{\alpha-1}-\rho\right)
\end{equation}
\begin{equation*}
\hat{c}\sigma+\rho=A\alpha\left(\frac{\alpha(1-\eta)}{\eta(1-\alpha)}\right)^{\alpha-1}\left(\frac{\rho+\sigma\hat{c}}{B(1+\bar{B})(1-\eta)}\right)^{\frac{\alpha-1}{\eta}}
\end{equation*}
\begin{equation*}
(\hat{c}\sigma+\rho)^{1-\frac{\alpha-1}{\eta}}=A\alpha\left(\frac{\alpha(1-\eta)}{\eta(1-\alpha)}\right)^{\alpha-1}\left(B(1+\bar{B})(1-\eta)\right)^{\frac{1-\alpha}{\eta}}
\end{equation*}
\begin{equation*}
\hat{c}\sigma+\rho=\left[A^\eta\alpha^{\alpha\eta}\left(\frac{1-\eta}{\eta(1-\alpha)}\right)^{(\alpha-1)\eta}\left(B(1+\bar{B})(1-\eta)\right)^{1-\alpha}\right]^\frac{1}{1+\eta-\alpha}
\end{equation*}
\begin{equation}
\boxed{\hat{c}^*=\frac{1}{\sigma}\left(\left[A^\eta\alpha^{\alpha\eta}(1-\eta)^{(1-\eta)(1-\alpha)}(\eta(1-\alpha))^{\eta(1-\alpha)}(B(1+\bar{B}))^{1-\alpha}\right]^\frac{1}{1+\eta-\alpha}-\rho\right)}
\end{equation}
Für die weitere Berechnung des Gleichgewichts wird wieder ein Platzhalter\\ $M=\left[A^\eta\alpha^{\alpha\eta}(1-\eta)^{(1-\eta)(1-\alpha)}(\eta(1-\alpha))^{\eta(1-\alpha)}(B(1+\bar{B}))^{1-\alpha}\right]^\frac{1}{1+\eta-\alpha}$ für das Grenzprodukt verwendet.
Im Gleichgewicht wird das Humankapital aufgeteilt gemä{\ss} $u$ und berechnet sich aus $\hat{c}=\hat{h}$ gemä{\ss} Gleichung \eqref{GG2} unter Berücksichtigung von $x_2^*$ und $\hat{c}^*$. 
\begin{equation}
\frac{1}{\sigma} (M-\rho)=B(1+\bar{B})\left(\left(\frac{\rho+\sigma\frac{1}{\sigma}(M-\rho)}{B(1+\bar{B})(1-\eta)}\right)^{1/\eta}\right)^\alpha(1-u)
\end{equation}
\begin{equation*}
\frac{\frac{1}{\sigma}(1-\eta)(M-\rho)}{M}=(1-u)
\end{equation*}
\begin{equation}
\boxed{u^*=\frac{\sigma M-(1-\eta)(M-\rho)}{\sigma M}}
\end{equation}
Im Steady State gilt: $\hat{c}=\hat{k}$. Aus dieser Bedingung lässt sich das optimale Verhältnis von physischem Kapital zu Humankapital ableiten, indem man die entsprechenden Terme für $x_1^*$, $x_2^*$ und $\hat{c}^*$  in Gleichung \eqref{GG1} einsetzt. 
\begin{equation}
\boxed{\chi^*=\frac{1}{\sigma}\left(\frac{A\alpha \sigma[-\eta\rho+M(\eta+\sigma-1)+\rho] \left(\frac{\alpha  (\eta -1) \left(\frac{M}{B (1+\bar{B})(1-\eta) }\right)^{1/\eta }}{(\alpha -1) \eta }\right)^{\alpha -1}}{\rho  (\alpha -\eta )+M (\alpha  (\sigma -1)+\eta )}-M+\rho\right)}
\end{equation}
Durch das Einsetzen von $x_1^*$, $x_2^*$ und mit $\frac{k}{h}=u x_1+(1-u)x_2$ in die allgemeine Gleichung $v=\frac{vk}{uh}\frac{uh}{k}$ kann diese gelöst werden.\footnote{Auch hier wird dies wieder aus der allgemeinen Aufteilung des physischen Kapitals zwischen dem Bildungs- und Produktionssektors $k=vk+(1-v)k$ hergeleitet. Die gesamte Gleichung wurde durch $h$ geteilt und anschlie{\ss}end um die Faktoren $u$ und $(1-u)$ erweitert. Es folgt erneut $\frac{k}{h}=u\frac{vk}{uh}+(1-u)\frac{(1-v)k}{(1-u)h}$.}
\begin{equation}
\begin{split}
v=\frac{\alpha(1-\eta)}{\eta(1-\alpha)} \left(\frac{\rho+\frac{1}{\sigma} \sigma(M-\rho)}{B (1+\bar{B}) (1-\eta )}\right)^{\frac{1}{\eta}} \frac{M \sigma -(1-\eta ) (M-\rho )}{ \sigma M}\\
\frac{1}{\frac{ \sigma M -(1-\eta ) (M-\rho )}{\sigma M }\frac{\alpha(1-\eta)}{\eta(1-\alpha)}\left(\frac{\rho+\frac{1}{\sigma}\sigma(M-\rho )}{B (1+\bar{B}) (1-\eta )}\right)^{1/\eta }+\left(1-\frac{ \sigma M-(1-\eta ) (M-\rho )}{ \sigma M}\right) \left(\frac{\rho+\frac{1}{\sigma}\sigma(M-\rho )}{B (1+\bar{B}) (1-\eta )}\right)^{1/\eta }}
\end{split}
\end{equation}
Es ergibt sich somit auch hier wieder die optimale Aufteilung $v^*$ des physischen Kapitals auf die beiden Sektoren:
\begin{equation}
\boxed{
v^*=\frac{\alpha  (1-\eta ) \left(\frac{M}{B (1+\bar{B}) (1-\eta )}\right)^{1/\eta } (M \sigma -(1-\eta ) (M-\rho ))}{(1-\alpha ) \eta  M \sigma  \left(\frac{\alpha  (1-\eta ) \left(\frac{M}{B (1+\bar{B}) (1-\eta )}\right)^{1/\eta } (M \sigma -(1-\eta ) (M-\rho ))}{(1-\alpha ) \eta  M \sigma }+\left(\frac{M}{B (1+\bar{B}) (1-\eta )}\right)^{1/\eta } \left(1-\frac{M \sigma -(1-\eta ) (M-\rho )}{M \sigma }\right)\right)}}
\end{equation}
